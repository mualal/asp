\documentclass[main.tex]{subfiles}

\begin{document}

\linksection{Лекция 01.04.2024 (Шипунова О.Д.)}

\sublinksection{\,\,Развитие теоретической биологии в XX веке}

\insertlectureslide{1}{12}

\sublinksection{\,\,Содержание}

\insertlectureslide{2}{12}

\sublinksection{\,\,Основные направления биологии в XX в.}

\insertlectureslide{3}{12}

\sublinksection{\,\,Предыстория генетики}

\insertlectureslide{4}{12}

\sublinksection{\,\,Гипотезы о природе наследственности}

\insertlectureslide{5}{12}

\sublinksection{\,\,Гипотеза идиоплазмы К.Нечели}

\insertlectureslide{6}{12}

\sublinksection{\,\,Законы наследственности Г.Менделя}

\insertlectureslide{7}{12}

\sublinksection{\,\,Концептуальная основа классической генетики}

\insertlectureslide{8}{12}

\sublinksection{\,\,Краткая история генетики}

\insertlectureslide{9}{12}

\sublinksection{\,\,Концепции классической генетики}

\insertlectureslide{10}{12}

\sublinksection{\,\,Хромосомная теория наследственности}

\insertlectureslide{11}{12}

\sublinksection{\,\,Хромосомная теория наследственности Т.Х.Моргана}

\insertlectureslide{12}{12}

\sublinksection{\,\,Неоклассический период в развитии генетики}

\insertlectureslide{13}{12}

\sublinksection{\,\,Неоклассический период -- популяционная генетика С.С. Четвериков}

\insertlectureslide{14}{12}

\sublinksection{\,\,Основные положения популяционной генетики}

\insertlectureslide{15}{12}

\sublinksection{\,\,Теория мутаций}

\insertlectureslide{16}{12}

\sublinksection{\,\,Типы мутаций}

\insertlectureslide{17}{12}

\sublinksection{\,\,Проблема структуры гена}

\insertlectureslide{18}{12}

\sublinksection{\,\,Современный этап в развитии генетики}

\insertlectureslide{19}{12}

\sublinksection{\,\,Молекулярная биология}

\insertlectureslide{20}{12}

\sublinksection{\,\,Молекулярный механизм передачи наследственной информации}

\insertlectureslide{21}{12}

\sublinksection{\,\,Формы естественного отбора}

\insertlectureslide{22}{12}

\sublinksection{\,\,Системные принципы объяснения биологической эволюции}

\insertlectureslide{23}{12}

\sublinksection{\,\,Классическая популяционно-генетическая модель биологической эволюции}

\insertlectureslide{24}{12}

\sublinksection{\,\,Эпигенетический подход в объяснении биологической эволюции}

\insertlectureslide{25}{12}

\insertlectureslide{26}{12}

\sublinksection{\,\,Системные принципы объяснения биологической эволюции}

\insertlectureslide{27}{12}

\sublinksection{\,\,Неуглеродная форма жизни}

\insertlectureslide{28}{12}



\end{document}
