\documentclass[main.tex]{subfiles}

\begin{document}

\section{Лекция 14.12.2023 (Шипунова О.Д.)}

\insertlectureslide{1}{06}

\subsection{Междисциплинарные взаимодействия в истории науки XX в.}

\insertlectureslide{2}{06}

\subsection{Развитие системных представлений в генетике начала XX в}

\insertlectureslide{3}{06}

\subsection{Проблема синтеза классической генетики и эволюционной теории в биологии}

\insertlectureslide{4}{06}

\subsection{Междисциплинарный статус кибернетики в системе современной науки}

\insertlectureslide{5}{06}

\subsection{Науки о сложных системах}

\insertlectureslide{6}{06}

\subsection{Основные понятия кибернетики}

\insertlectureslide{7}{06}

\insertlectureslide{8}{06}

\insertlectureslide{9}{06}

\subsection{Законы кибернетики}

\insertlectureslide{10}{06}

\subsection{Функциональный подход. Концептуальные основания}

\insertlectureslide{11}{06}

\subsection{Теория функциональной системы академика П.К. Анохина}

\insertlectureslide{12}{06}

\subsection{Функциональный подход. Информация и управление}

\insertlectureslide{13}{06}

\subsection{Информационная парадигма}

\insertlectureslide{14}{06}

\subsection{Информационные модели причинно-следственной связи}

\insertlectureslide{15}{06}

\subsection{Ключевые понятия информационной парадигмы}

\insertlectureslide{16}{06}

\subsection{Развитие системного подхода}

\insertlectureslide{17}{06}

\subsection{Основные понятия синергетики}

\insertlectureslide{18}{06}

\subsection{Нелинейность}

\insertlectureslide{19}{06}

\insertlectureslide{20}{06}

\subsection{Принципы синергетики}

\insertlectureslide{21}{06}

\subsection{Картина эволюции системы}

\insertlectureslide{22}{06}

\insertlectureslide{23}{06}

\insertlectureslide{24}{06}

\subsection{Синергетика. Нелинейные структуры}

\insertlectureslide{25}{06}



\end{document}
