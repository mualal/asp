\documentclass[main.tex]{subfiles}

\begin{document}

\section{Лекция 14.12.2023 (Шипунова О.Д.)}

\insertlectureslide[0]{1}{06}

Сегодня лекция очень напряжённая, так как она посвящена новой области в системе науки, которая оформляется в середине XX века (во второй половине XX века) и получила название междисциплинарной области.

Основной смысл этой области в том, что вводятся такие концепты (понятия) и такой статус исследуемого объекта, которых раньше не было в науке, и всё это опирается на новое представление о самом изучаемом объекте как некой системе, которая имеет внутреннюю организацию и внешние взаимодействия.

\subsection{Междисциплинарные взаимодействия в истории науки XX в.}

\insertlectureslide[0]{2}{06}

Само явление систематизации в каком-либо виде особенно было развита в области зоологии, биологии и так далее.
То есть тематика живых существ.
К середине XIX века весь этот материал начинает систематизироваться в определённых концепциях.
Первая концепция, которая концептуализирует весь уже систематизированный материал, представлена в химии в виде периодического закона для свойств элементов.
Тогда было известно всего лишь 61 элемент.
Менделеев располагает их таким образом, чтобы выявить некоторую закономерность в химических свойствах, которая позволяет глубже исследовать сами элементы, а более того открывать новые химические элементы, которые в естественном режиме было трудно бы найти.

Именно систематизация и рассмотрение закономерности свойств в истории науки породила такие замечательные открытия как открытие радиоактивности и целый класс тяжёлых элементов.
Это результат систематизации фактического материала, который наблюдается, например, и в области биологии (систематика насекомых и так далее).

Строго говоря, систематика живых существ в определённые виды по определённым критериям тоже легла в основании уже эволюционной теории в биологии, которая тоже считается одним из концептуальных оснований биологии.

Если брать именно первые науки, которые формулируются именно на взаимосвязи разных наук, то это, конечно, геохимические концепции (разные миграции).

Объясняется естественная эволюция физического мира Земли, которая привязывается к таким периодам как безатмосферный период, период первоначальной атмосферы, в которой содержалось больше углекисилоты, и современная кислородная атмосфера.

\subsection{Развитие системных представлений в генетике начала XX в}

\insertlectureslide[0]{3}{06}

В генетике тоже развивались системные представления.
Начало генетики совпадает с началом XX века (1900 год).
Генетика целенаправленно изучает механизмы наследственной передачи и памяти наследственных признаков.

Морган указывает на то, что недостаточно одного гена (рассматривает набор генов в хромосоме, которые передаются пучком).

С одной стороны Морган выступил против эволюционной теории Дарвина, утверждая, что устойчивость наследственных признаков настолько очевидна, что вообще исключает какие-либо признаки естественного отбора и вообще никакой эволюции не предполагает.
По его мнению, наследственный признак никак не может измениться.

\subsection{Проблема синтеза классической генетики и эволюционной теории в биологии}

\insertlectureslide[0]{4}{06}

Следующее учение, которое выдвигается уже в 20-х годах XX века Сергеем Четвериковым получило название популяционной генетики, в которой вводится уже такое понятие как генотипическая среда.
Это учение призвано объяснить роль естественного отбора.

В эволюционном плане изменения вида активная роль естественного отбора заключается в создании благоприятной генотипической среды.

Генотип (в классической генетике) -- это некая абстракция в наборе признаков, которая характеризует наследование определённых структур и морфологии (так сказать, усреднённая форма морфологии вида, которая реализуется в каждом индивидууме).

Фенотип (в классической генетике) -- это внешнее проявление генотипа, которое связано с окружающей средой (чёрные лебеди/белые лебеди).

Естественный отбор по системе Четверикова связан с тем, что поскольку гены в процессе отбора по хромосомной теории взаимодействуют со всеми системами организма, то собственно говоря, если в какой-то среде отбирается или фиксируется один признак (через отбор, естественно, выживание), то в конечном счёте именно для отбора этого признака в системе складывается определённая генетическая среда, которая либо блокирует этот признак, либо наоборот развивает.

Концепция Четверикова позволила соединить классическую генетику и эволюционную теорию.

\subsection{Междисциплинарный статус кибернетики в системе современной науки}

\insertlectureslide[0]{5}{06}

Самая главная наука, с которой начинается междисциплинарная область науки -- это кибернетика, которая связана с введением в научный оборот трёх понятий (система, управление и информация) в абстрактном виде (без привязки к какой-то структуре).

Объектом кибернетики заявлены любые системы (технические, живые и так далее) и их поведение.

Основной акцент кибернетики был связан с автоматизацией и управлением автоматическими системами.

\subsection{Науки о сложных системах. История кибернетики}

\insertlectureslide[0]{6}{06}

Отец кибернетики, Норберт Винер, был инженером, который занимался управлением самонаводящихся автоматических систем.
Но в системе науки Норберт Винер связан не только с термином кибернетика, который означает "<искусство управления">, но и тем, что он написал свою замечательную книгу "<Кибернетика или управление в животном и машине">, которая определила новый стиль технического и научного мышления.
Уже в названии было заявлено, что и животные, и машины, и вообще любая система подчиняются неким одним и тем же законам.

Дальше разворачивается новое исследование таких систем через новые термины, которых раньше в науке не было.

Ещё один основатель кибернетики, Джон фон Нейман, создаёт теорию логических автоматов.
Формулирует принципы математики и участвует в создании первых вычислительных машин. 

\subsection{Основные понятия кибернетики}

\insertlectureslide[0]{7}{06}

Кибернетика вводит некий концептуальный аппарат общенаучного плана, который в дальнейшем распространяется в самые разные научные области.
Прежде всего это система -- именно как абстрактная система.

Система -- это некая целостность (взаимосвязанные элементы или структуры), которая имеет свойство, отличающееся от свойств тех элементов, которые составляют систему.

Свойства самой системы определяют или могут изменять то, что происходит у неё внутри.

Системное свойство -- это свойство, которое относится именно к самой системе, а не к её составляющим.
Такие свойства с классической точки зрения нельзя отнести к какой-либо материальной структуре.
Например, понятие инстинкт мы не можем отнести к какому-либо органу, хотя это физиологическая система.

Первая классификация систем: простые системы и сложные системы.

Новация кибернетики в том, что она выделяет сложные системы, взаимодействие между частицами которых настолько серьёзно, что пренебречь ими нельзя.
То есть сложные системы характеризуются как динамические системы, то есть постоянно изменяющиеся и содержащие внутри тоже такого же типа динамические системы.
Для характеристики поведения сложных систем вводятся совершенно новые параметры такие как целесообразность (которая указывает характер причины и следствия в её поведении) и организованность (которая связана как со структурными характеристиками системы, так и с её поведением).

\insertlectureslide[0]{8}{06}

Следующее понятие, которое вводится в кибернетике -- это информация, которая тоже вводится в абстрактном понимании.
В абстракции от всего вплоть до того, что первое определение информации по Винеру: информация -- это не материя и не энергия.

Поэтому в дальнейшем народ пытается этот феномен информации (которая не материя и не энергия) определить (дать ей определение как некому явлению).
Шеннон предложил количественный способ измерения потока информации (поток сообщений).
Этот поток сообщений абсолютно абстрагирован от всего (в том числе от смысла самой информации).
Речь идёт только о количестве.
Винер знаменит тем, что он как раз показал, что это не просто количество сообщений, а это некий феномен, который до сих пор рассматривается наряду с такими универсальными понятиями, которые характеризуют материальный мир (материя и энергия).

Материя, энергия и информация -- взаимосвязанная триада.

Попытки определить этот феномен (не материя и не энергия) в физическом подходе был представлен Бриллюэном, который представляет информацию как негэнтропию (или отрицательную энтропию).

В отечественной философии конца 80-х годов информация понимается как отражённое разнообразие или как функциональное отражение, то есть то отражение внешних взаимодействий, которое функционально для организации сложной системы.

\insertlectureslide[0]{9}{06}

Третье понятие кибернетики (управление) тоже формулируется в очень абстрактном варианте.

Основной целью техники управления является обеспечение целенаправленного воздействия и обратной связи.

Обратная связь тоже вводится в достаточно абстрактном варианте и применима практически везде.

Цель определяет установку, которая двигает систему.

\subsection{Законы кибернетики}

\insertlectureslide[0]{10}{06}

В системе кибернетики (науки об управлении) формулируются универсальные законы, которые связаны с эффективностью управления (или надёжностью управления).

\subsection{Функциональный подход. Концептуальные основания}

\insertlectureslide[0]{11}{06}

Следствие развития кибернетики -- это распространение функционального подхода как определённой стратегии познавательного исследования.
Известен как принцип "<чёрного ящика">.

Функциональный подход применим к любым системам, так как здесь не фиксируется структура системы.
Хотя, конечно, есть идея о том, что пока мы изучаем реакции системы на воздействия, то ящик становится менее чёрным.

Существует взаимосвязь целесообразности и управления в организации действия системы.

\subsection{Теория функциональной системы академика П.К. Анохина}

\insertlectureslide[0]{12}{06}

Одна из первых теорий функциональных систем была предложена академиком Петром Анохиным.
Он был физиолог и исследовал взаимосвязи разных уровней живого.

Понятие опережающее отражение потом заменилось на понятие информации, функционального отражения.
Когда отражённое воздействие начинало работать на адаптацию.

Подход Анохина получил название структурно-функциональный подход.

\subsection{Функциональный подход. Информация и управление}

\insertlectureslide[0]{13}{06}

Развитие функционального подхода связано с тем, что здесь связываются два фундаментальных понятия кибернетики -- информация и управление, поскольку в любой органической системе элементарный процесс управления предполагает цель, а целесообразное поведение, так или иначе, управляемо.

Получается, что информация управляет, а управление информирует.

\subsection{Информационная парадигма}

\insertlectureslide[0]{14}{06}

Далее в системе науки складывается то, что называется информационной парадигмой (такой достаточно широкой установкой мировоззренческой и исследовательской).
Само понятие информации тоже развивается, но связано прежде всего с функциональной ролью результата взаимосвязи в действии системы и прогнозировании её поведения.

Информационная парадигма связана с тем, что выдвигаются универсальные мировоззренческие положения, которые характеризуют современную науку.

Универсальность информационных процессов наравне с материальными, энергетическими и так далее.

\subsection{Информационные модели причинно-следственной связи}

\insertlectureslide[0]{15}{06}

Что ещё характеризует информационную парадигму как некие универсальные установки?

Формулируется такое понятие как информационная причинность, под которой понимается закономерность действия системных требований.

Идея результата как системообразующего фактора.

Информационная причинность = системная причинность.

Суть информативного кода.
Как действет система причин?
Через нормирование некоторого потенциального жизненного пространства системы.

Именно в областях теоретической биологии, биохимии, биофизике пытаются формулировать ключевые понятия информационной парадигмы.

\subsection{Ключевые понятия информационной парадигмы}

\insertlectureslide{16}{06}

\subsection{Развитие системного подхода}

\insertlectureslide[0]{17}{06}

Системная методология складывается на базе абстрактных понятий системы, управления и информации.
И выливается в конечном счёте в то, что называется информационной парадигмой, и на этой базе уже развивается сам системный подход.
Следующее развитие, которое было названо синергетикой, связано не с тем, что оно отрицает предыдущие понятия и информационную парадигму, а с тем, что оно начинает рассматривать особого рода системы и особые механизмы эволюции этой системы.

Кибернетика была ориентирована на создание автоматических замкнутых циклов и не рассматривала эволюционные скачкообразные изменения, то новое развитие системных наук связано с тем, что тут исследуется новый объект.

Согласованность некого коллективного действия.

Здесь объектом становится не просто динамическая система, а система эволюционирующая и прогнозирующая.
Очень сложно, да?
Поэтому для неё пытаются создать определённые теории.

На слайде представлены теоретические источники, которые исследуют закономерности согласованного поведения и скачкообразные переходы сложных систем от одного состояния к другому (фазовые переходы).

Процессы самоорганизации и фазовые переходы в сложных системах.

Герман Хакен ввёл в оборот термин "<Синергетика"> и организовал первый институт синергетики.

\subsection{Основные понятия синергетики}

\insertlectureslide[0]{18}{06}

Основные понятия, которые вводит эта новая система знаний (синергетика).

Процесс самоорганизации.

В относительно устойчивом равновесии самоорганизующаяся система всеми силами пытается сохранить это равновесие.

\subsection{Нелинейность}

\insertlectureslide[0]{19}{06}

В синергетике вводится понятие нелинейность.

В обыденной речи нелинейная система сама выбирает один из возможных путей эволюции.

Важно то, что общее понятие эволюции сложной системы плавно переходит из области биологии в такую достаточно универсальную область, поскольку она характеризует развитие (вводит фактор времени) в поведение сложной самоорганизующейся системы, поэтому каждая такая система эволюционирует.

Про термин "<эволюция"> мы, может быть, позже ещё поговорим.

Изначально сам термин "<эволюция"> ещё в XVII веке предполагал разворачивание сложного организма из одной клетки (из одного зёрнышка).

В теории Дарвина и Ламарка уже термин "<эволюция"> имеет совсем другое содержание.
Эволюция в биологии -- это постепенное развитие и усложнение живых видов организмов.

А в синергетике эволюция характеризует вообще поведение самоорганизующейся системы и её жизненный цикл.
То есть термин "<эволюция"> имеет абстрактный характер.

\insertlectureslide[0]{20}{06}

Синергетика вводит некий алгоритм, который позволяет объяснять и отчасти давать некие вероятностные прогнозы в отношении поведения такой системы, благодаря тому, что выделяются две фазы жизни такой системы (линейная фаза и нелинейная фаза).

Бифуркация как раздвоение решения.

Шарик, расположенный на вершине горы, когда-нибудь скатится.
Есть поле возможных состояний.
Но точно сказать, куда скатится шарик, практически невозможно.

Формируется представление о причинно-следственных связях, которое сейчас названо вероятностным детерминизмом.

\subsection{Принципы синергетики}

\insertlectureslide[0]{21}{06}

Представлены принципы синергетики, которые получают универсальное звучание.

\subsection{Картина эволюции системы}

\insertlectureslide{22}{06}

\insertlectureslide[0]{23}{06}

Показан аттрактор осциллятора.

Показаны возможные фазовые траектории в окрестностях особых точек.

\insertlectureslide{24}{06}

\subsection{Синергетика. Нелинейные структуры}

\insertlectureslide[0]{25}{06}

Просмотр фильма о синергетике и нелинейных структурах.

\end{document}
