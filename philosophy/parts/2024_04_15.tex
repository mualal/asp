\documentclass[main.tex]{subfiles}

\begin{document}

\linksection{Лекция 15.04.2024 (Шипунова О.Д.)}

\sublinksection{\,\,Концепции о природе и эволюции человека в естествознании}

\insertlectureslide{1}{14}

\sublinksection{\,\,Генетические основания происхождения человека}

\insertlectureslide{2}{14}

\sublinksection{\,\,Физиология человека как результат биологической эволюции}

\insertlectureslide{3}{14}

\sublinksection{\,\,Систематическое положение человека}

\insertlectureslide{4}{14}

\sublinksection{\,\,Почему именно обезьяны?}

\insertlectureslide{5}{14}

\sublinksection{\,\,Этологические доказательства}

\insertlectureslide{6}{14}

\sublinksection{\,\,Эволюция приматов}

\insertlectureslide{7}{14}

\insertlectureslide{8}{14}

\insertlectureslide{9}{14}

\insertlectureslide{10}{14}

\insertlectureslide{11}{14}

\sublinksection{\,\,Эволюция вида Homo Sapiens}

\insertlectureslide{12}{14}

\insertlectureslide{13}{14}

\sublinksection{\,\,Концепции о происхождении человека в современном естествознании}

\insertlectureslide{14}{14}

\sublinksection{\,\,Эволюция мозга}

\insertlectureslide{15}{14}

\insertlectureslide{16}{14}

\insertlectureslide{17}{14}

\sublinksection{\,\,Эволюционное изменение мозга у палеоантропа -- увеличение лобных долей}

\insertlectureslide{18}{14}

\sublinksection{\,\,Современная биология о строении и эволюции мозга}

\insertlectureslide{19}{14}

\sublinksection{\,\,Нейроны головного мозга}

\insertlectureslide{20}{14}

\sublinksection{\,\,Так выглядит нейрон}

\insertlectureslide{21}{14}

\insertlectureslide{22}{14}

\sublinksection{\,\,Иллюзии моделирования мозга}

\insertlectureslide{23}{14}

\insertlectureslide{24}{14}

\sublinksection{\,\,Защита мозга}

\insertlectureslide{25}{14}

\insertlectureslide{26}{14}

\insertlectureslide{27}{14}

\insertlectureslide{28}{14}

\sublinksection{\,\,Мозг и Нервная система организма}

\insertlectureslide{29}{14}

\sublinksection{\,\,Психофизиология человека. И.П.Павлов}

\insertlectureslide{30}{14}

\sublinksection{\,\,Первая и вторая сигнальные системы и их взаимодействие}

\insertlectureslide{31}{14}

\insertlectureslide{32}{14}

\sublinksection{\,\,Развитие головного мозга}

\insertlectureslide{33}{14}



\end{document}
