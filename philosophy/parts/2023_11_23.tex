\documentclass[main.tex]{subfiles}

\begin{document}

\section{Лекция 23.11.2023 (Шипунова О.Д.)}

\insertlectureslide{1}{03}

\subsection{Критическая философия Иммануила Канта}

\insertlectureslide{2}{03}

\subsection{Учение Канта об априорных формах}

\insertlectureslide{3}{03}

\subsection{Учение Канта об идеях и антиномиях разума}

\insertlectureslide{4}{03}

\subsection{Антиномии разума, неизбежные в познании мира}

\insertlectureslide{5}{03}

\subsection{Георг Вильгельм Фридрих Гегель}

\insertlectureslide{6}{03}

\subsection{Логика развития и диалектический метод познания}

\insertlectureslide{7}{03}

\subsection{Понимание движения в диалектической логике Гегеля}

\insertlectureslide{8}{03}

\subsection{Развитие теории познания эмпиризма в XVIII в.}

\insertlectureslide{9}{03}

\subsection{Философия науки в XIX в.}

\insertlectureslide{10}{03}

\subsection{Становление философии позитивизма в середине XIX века}

\insertlectureslide{11}{03}

\subsection{Первый позитивизм (классический)}

\insertlectureslide{12}{03}

\subsection{Огюст Конт}

\insertlectureslide{13}{03}

\subsection{Джон Стюарт Милль}

\insertlectureslide{14}{03}

\subsection{Герберт Спенсер}

\insertlectureslide{15}{03}

\subsection{Второй позитивизм (эмпириокритицизм)}

\insertlectureslide{16}{03}

\subsection{Э. Мах}

\insertlectureslide{17}{03}

\subsection{Р. Авенариус}

\insertlectureslide{18}{03}

\subsection{Третий позитивизм (неопозитивизм)}

\insertlectureslide{19}{03}

\insertlectureslide{20}{03}

\insertlectureslide{21}{03}

\insertlectureslide{22}{03}

\insertlectureslide{23}{03}

\insertlectureslide{24}{03}

\insertlectureslide{25}{03}

\subsection{Язык -- знаковая система}

\insertlectureslide{26}{03}

\insertlectureslide{27}{03}

\insertlectureslide{28}{03}

\subsection{Проблемы семантики}

\insertlectureslide{29}{03}



\end{document}
