\documentclass[main.tex]{subfiles}

\begin{document}

\section{Лекция 23.11.2023 (Шипунова О.Д.)}

\insertlectureslide[0]{1}{03}

XVII век считается началом не только научной революции, но и началом отдельной научной философской дисциплины -- теории познания, которая рассматривает проблемы метода науки и источника.

Эта проблема остаётся и в дальнейшем, и она очень усиленно обсуждается и в самой науке, и в философии науки.

Как традиции эмпиризма и рационализма отражаются в XVIII-XIX веках.

\subsection{Критическая философия Иммануила Канта}

\insertlectureslide[0]{2}{03}

Продолжателем традиции рационализма, родоначальником которой выступает Рене Декарт (у него обозначена проблема дуализма знания и опыта, психофизического дуализма), является Иммануил Кант и его критическая философия.

Остаётся тот же самый вопрос: как возможно познание мира и есть ли у него границы?

Пытаясь выяснить как познаёт мир человек, Кант обнаруживает границы познания.
И они обозначены у него жёстким разделением сущности вещей (или вещь-в-себе он это называет) и явления.
Познавать мы можем только то, что мы наблюдаем (что является нам в опыте), но сама сущность (которая фиксирует качество вещи по Аристотелю) остаётся скрытой.
Как бы разум не пытался познать (особенно в универсальном варианте), он оказывается ограничен и попадает в противоречие (антиномию).

Вещь-в-себе Канта = наука не может познать сущности вещей, а познаёт только то: что является в опыте.
В этом отношении Кант как бы наследует и эмпиризм, и рационализм.

В априорном (внеопытном) знании Кант следует точно постановке Декарта.

\subsection{Учение Канта об априорных формах}

\insertlectureslide[0]{3}{03}

Чувственность -- способность восприятия с использованием пяти органов чувств (или пяти каналов) поступления информации.

Рассудок -- способность рассуждать, придавать объективность восприятиям, которая выражается на уровне категорических суждений.
На уровне рассудка мир познаётся через категории (количества, отношения, качества и модальности).

Разум пытается осмыслить мир в целостности явлений и явления.

Важно то, как понимает пространство и время Кант: это определённые формы интеграции восприятий, которые человеку свойственны.
Они не относятся к органу чувств, но тем не менее человек их воспринимает.
Усвоение пространства и времени через интеграцию тех восприятий, которые дают пять органов чувств (зрение, осязание, обоняние, слух, вкус).
Но по отдельности эти пять органов чувств не дают информации о пространстве и времени.

У Ньютона пространство и время -- это отдельные субстанции, а по Канту -- это формы восприятия.

Самое замечательное утверждение: для возникновения знания необходимо объединение априорных форм чувственности и априорных форм рассудка.

Для того, чтобы получить следующую ступень знания нужно ещё что-то, грубо выражаясь, интеллектуальное усилие.

\subsection{Учение Канта об идеях и антиномиях разума}

\insertlectureslide[0]{4}{03}

Кант понимает уровень разума по способности понимать мир в его целостности.

Действие рассудка интегрирует чувственный опыт и эффективна в рамках конечного опыта, а разум старается выйти за пределы опыта и понять мир в его скрытых причинах (сущностях).

Разум продуцирует идеи, которые не могут быть даны в опыте.

Антиномия -- логическое противоречие, когда 2 противоположных утверждения могут быть равно обоснованы.

\subsection{Антиномии разума, неизбежные в познании мира}

\insertlectureslide[0]{5}{03}

Агностицизм -- мир непознаваем.
И у Канта фиксируются границы познания.
Но при этом Кант обосновывает возможность научного познания, хотя и считает его не безграничным.

Грубо говоря, Кант оказывается агностиком.

\subsection{Георг Вильгельм Фридрих Гегель}

\insertlectureslide[0]{6}{03}

С Гегелем связывают завершающий этап в развитии европейского рационализма.

Система философии Гегеля получила название панлогизма.
Всё развитие мира он выводит из абсолютной идеи, которая выступает основанием всеобщей взаимосвязи, источником развёртывания всего природного и человеческого мира и его конечным результатом.

Гегель выступает против Канта и формулирует основные позиции, преодолевающие дуализм познающего субъекта и вещи-в-себе.

\subsection{Логика развития и диалектический метод познания}

\insertlectureslide[0]{7}{03}

Что приносит Гегелевская философия в теорию познания?
Во-первых, он вводит понимание развития мира.
Как можно мыслить мир в развитии (не просто в движении механическом, а в развитии)?
Как можно мыслить источник движения, не обращаясь к Богу?
Основной вклад -- логика развития абстрактного целого.

Диалектика -- новый рациональный метод познания, который позволяет мыслить целое и постигать развивающееся единство.

Диалектическая логика Гегеля не отменяет традиционную логику Аристотеля, а дополняет её новыми схемами.

\subsection{Понимание движения в диалектической логике Гегеля}

\insertlectureslide[0]{8}{03}

Логику развития Гегель даёт в абстрактной форме, поэтому она до сих пор сохраняется в виде уважаемой формы спекулятивного философствования.

Гегель создаёт рациональную модель познания, в которой для того, чтобы происходило изменение в движении, не обязательно должна действовать внешняя сила.
Противоречие рассматривается как источник движения.
Самодвижение.
Это отчасти напоминает целевую (целесообразную) причину Аристотеля.

Всё, что сказано по Канту и Гегелю, остаётся в рамках философии.
Формируется такая отдельная область -- формулировка теории познания. 

\subsection{Развитие теории познания эмпиризма в XVIII в.}

\insertlectureslide[0]{9}{03}

В XVIII веке также развивается эмпиризм.
Он ближе к конкретным исследованиям.
В отношении теории познания эмпиризм далеко расходится с рационализмом.

Традиции эмпиризма связаны с именами Томаса Гоббса и Джона Локка, согласно которым в основе мышления и познания мира лежат ощущения, которые интегрируются.
Момент ощущения и восприятия мира являются источником идей, которыми далее оперирует мышление.

Позиция Джона Локка, которая привязывает источник любого знания к ощущениям, получила название сенсуализм.
Нет ничего в человеческом мышлении, чего нет в ощущениях.
Такая жёсткая позиция, которая потом по разному интерпретируется.

Джон Локк ещё известен своей теорией "<чистой доски"> (tabula rasa).
Когда рождается ребёнок, то только потом (в процессе его жизни) опыт и цивилизация "<пишут"> на нём свои письмена.
Другими словами, опыт знания не даётся правом рождения.

Как интерпритируется идея чувственного ощущения как источника знания у Джорджа Беркли?
Таким образом, что весь мир человека замкнут на ощущениях и если человек умирает, то и мир, в котором он живёт, схлопывается.
Так называемая позиция солипсизма, которая замыкает внутренний мир человека только на ощущениях.

Сам факт ощущения имеет две стороны: с одной стороны мы ощущаем те вещи, которые от нас не зависят, но существуют, а с другой стороны мы фиксируем и интегрируем только те ощущения, которые наши.
Таким образом, каждый человек -- это своя молекула, которая не связана с другими.
У каждого свой мир.

Давид Юм считается основоположником скептицизма и агностицизма в сфере философии науки.
Он наследует идеи эмпиризма, но он трактует знание, получаемое из эмпирического опыта, всегда вероятным и ограниченным.
Человек не может только из эмпирических наблюдений явлений установить некие универсальные законы.
Опыт с точки зрения Юма -- это достаточно узкий и ограниченный источник, и человек к нему привязан (если нет опыта, то и нет знания).

\subsection{Философия науки в XIX в.}

\insertlectureslide[0]{10}{03}

Философия науки (как особая область, имеющая свой предмет) складывается в XIX веке под влиянием различных философских воззрений.
С одной стороны, научной картины мира, а с другой стороны большое влияние имеет эмпирическая традиция в теории познания, так как в дальнейшем философии науки жёстко стоит на эмпирическом основании, полагая что опыт является главным источником знания.

\subsection{Становление философии позитивизма в середине XIX века}

\insertlectureslide[0]{11}{03}

В первом позитивизме фиксируется проблема, которая получила название демаркации (разделения) науки и философии.
Формулируется понятие позитивной науки, с которой связывается понятие рациональное знание.

Рациональное знание в духе позитивизма -- это только то, которое вырабатывает наука и которое приложимо в социальной жизни.

Универсальные (очень абстрактные) законы философии (которые далеки от социальной жизни) уходят в разряд метафизики.

\subsection{Первый позитивизм (классический)}

\insertlectureslide[0]{12}{03}

В дальнейшем будем отмечать, что последующие этапы развития позитивизма так или иначе начинает рассматривать и фиксировать по содержанию опыта науки.
Опыт науки не укладывается в ощущения.

Но в первом позитивизме содержание должно быть сведено к ощущениям.

Экономное мышление предполагает стимул для наиболее удобного описания ощущений субъекта познания.

Специфическая логика познания в системе позитивизма связывает наращивание позитивного знания через усовершенствование метода индукции.

В этой системе логика познания обогащается методами установления причинно-следственных связей.

\subsection{Огюст Конт}

\insertlectureslide[0]{13}{03}

Спенсер дополнительно "<социальной физике"> накладывает позицию "<социального дарвинизма">.

\subsection{Джон Стюарт Милль}

\insertlectureslide[0]{14}{03}

Положения Джона Стюарта Милля в философии науки -- это прежде всего задача индуктивной логики (как систематизировать знание, как разработать методологические процедуры позволяющие выявить наиболее перспективные гипотезы и направления в науке).

Выделяет 4 метода установления причинных связей.
Всё равно эти методы говорят только о том, что можно построить некоторые вероятные умозаключения или гипотезы. 

\subsection{Герберт Спенсер}

\insertlectureslide[0]{15}{03}

Понимают, что индуктивного метода не достаточно.

Поэтому вводят понятие позитивного научного метода, который является эмпирическим и гипотетико-дедуктивным.
Ставится эксперимент, формулируется гипотеза и дедукций формулируются следствия гипотезы, которые подтверждаются далее в эксперименте.

У Спенсера идея социальной физики опирается не только на физику, но и на биологию.

\subsection{Второй позитивизм (эмпириокритицизм)}

\insertlectureslide[0]{16}{03}

Второй позитивизм в конечном счёте определяется неясностью самого опыта науки.

Первый позитивизм: что такое чистая наука, что такое позитивная наука?

Второй позитивизм: что такое опыт?
Эмпириокритицизм (дословно переводится как критика опыта) строится (крутится) вокруг этого самого опыта, который является источником научного знания.
Этот кризис усиливается кризисом физики на рубеже XX века (оказалось, что материя может превратиться в нечто, что не имеет массы).

Опыт первичен, а всё остальное (материя и так далее) вторичны.

\subsection{Э. Мах}

\insertlectureslide[0]{17}{03}

Здесь теории -- это просто способ упорядочивания данных (а не отражение универсальных связей).

Теория решается из достоверности, объективности и обязательности.

Выступает против механистической картины мира, указывает на то, что она не достаточна.
Пытается создать новую модель реальности, которая опирается не на материю, пространство, время, а на некоторые функциональные отношения между элементами мира.

\subsection{Р. Авенариус}

\insertlectureslide[0]{18}{03}

Авенариусу принадлежит идея принципиальной координации, которая подчёркивает, что опыт сам по себе, представляя некую изначальную реальность, тем не менее существует только в координации (слово координация заменяет слово ощущение).

Если в классическом варианте познавательные отношения субъекта и объекта сводились к зеркальному отражению (тождеству бытия и мышления как у Гегеля), то здесь эти отношения критикуются. 

\subsection{Третий позитивизм (неопозитивизм)}

\insertlectureslide[0]{19}{03}

Неопозитивизм -- следующее развитие позитивистской традиции и оно тоже опирается на проблематичность самого опыта.
В основе науки по-прежнему лежат эмпирические данные и ощущения.
Неопозитивизм выступает альтернативой эмпириокритицизму, потому что он акцентирует тот момент опыта, который связан со знанием.
Опыт знания науки и языка науки.
Т.е. в системе эмпирического опыта науки мы должны учитывать не только непосредственно эмпирические данные, которые получаем через ощущения (приборы и так далее), но и то знание, которое мы получаем.
И это знание передаётся через язык.

Главная задача неопозитивизма в философии науки -- это изучение языка науки и языковых моделей, которые отождествляются с моделями реальности.

\insertlectureslide{20}{03}

\insertlectureslide{21}{03}

\insertlectureslide[0]{22}{03}

Научный факт -- это не то, о чём говорится в предложении, а то, что делает это предложение истинным.

Мы видим только тот мир, который мы понимаем.

Есть язык опыта, а есть язык науки.

Научных революций быть не может, а каждое знание развивается как наращивание и расширение.

Принципы логического позитивизма: принцип верификации, конвенционализм, физикализм.

\subsection{Методологический принцип верификации}

\insertlectureslide[0]{23}{03}

Принцип верификации в практическом виде -- это сведение утверждений науки (утверждений фактов) к проверяемым протокольным суждениям, которые можно проверить в опыте.

Очень жёсткий принцип верификации.
С одной стороны он следует принципам эмпиризма, а с другой стороны он не позволяет проверить очень общие универсальные утверждения, которых в науке гораздо больше, чем протокольных.

\subsection{Методологический принцип конвенционализма}

\insertlectureslide[0]{24}{03}

Конвенционализм: положения, которые не выводимы из опыта, а просто постулируются.
Так как не выводятся из опыта, то поэтому требуют соглашения.
Далеко не все положения можно сразу безоговорочно признать.

\subsection{Методологический принцип физикализма}

\insertlectureslide[0]{25}{03}

Поставленные неопозитивизмом проблемы.

Интерсубъективность говорит о том, что некоторые предложения становятся общезначимыми.
Не потому что они истинны или доказаны, а потому что они признаны в виде аксиом или постулатов.
Этот принцип интерсубъективности подчёркивает различие объективности той или иной теории и её признания научным сообществом.

Также неопозитивисты акцентируют продуктивную природу теоретического знания и мышления.
Включают опыт мышления и логического построения в опыт науки.

Одна из главных проблем: проблема семантики, которая связана с языком как знаковой системой.
То есть та система, которая как раз имеет отношение к наращиванию опытом знанию.

\subsection{Язык -- знаковая система}

\insertlectureslide[0]{26}{03}

С другой стороны, язык науки развивается по своим законам.
Поэтому разные уровни информации вписаны в сам язык науки.

\insertlectureslide[0]{27}{03}

Семантический треугольник присущ любому термину.
Любое слово как термин имеет предметное значение и смысловой контекст.
И этот контекст оказывается разным.
Смысловой контекст может трактоваться по разным каналам.

Семантический аспект = смысловой аспект.

\insertlectureslide{28}{03}

\subsection{Проблемы семантики}

\insertlectureslide[0]{29}{03}

Если мы берём более широкий контекст системы понимания в философии, то условия понимания помимо простых (языковой традиции) рассматриваются ещё на рефлексивном и экзистенциальном уровнях.

Неопозитивизм вознаградил не только науку, но и открылись совсем новые дисциплины, такие как семантика, символика и так далее.

\end{document}
