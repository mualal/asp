\documentclass[main.tex]{subfiles}

\begin{document}

\section{Лекция 23.11.2023 (Шипунова О.Д.)}



\subsection{Традиция рационализма в XVIII-XIX вв. Критическая философия Иммануила Канта (1724-1804).}

{\parindent0pt
Основной вопрос критической философии по Канту.
\begin{itemize}[nosep,leftmargin=0.5cm]
\item Как возможно познание мира человеком, и если человеческое знание, как и сам человек, не безгранично, то где границы его познания.
\item Границы познания, по Канту, связаны с соотношением сущности вещей и её проявлением, которое доступно в чувственном опыте.
\begin{itemize}[nosep,leftmargin=0.6cm]
\item Граница между вещью-в-себе и явлением неустранима.
\item Душа как субстанция, мир как целое -- это вещи-в-себе, т.е. объекты, которые не могут быть никогда даны в чувственном познании, а следовательно, в отношении них не может быть достоверного знания.
\end{itemize}
\end{itemize}
}
\ 

{\parindent0pt
Кант различает два вида знания по способу его получения субъектом.
\begin{itemize}[nosep,leftmargin=0.5cm]
\item Апостериорное, опытное знание, основанное на эмпирических данных, т.е. (дословно -- после опыта).
\item Априорное, внеопытное знание (дословно -- до опыта) имеет всеобщий и необходимый характер, апостериорное знание -- не обладает таким свойством.
\item Априорными знаниями являются научные законы, постулаты, относящиеся к целому классу вещей, явлений, состояний, например, все тела протяжёны.
\begin{itemize}[nosep,leftmargin=0.6cm]
\item Из опыта мы знаем, что то или это тело протяжённо, но когда мы утверждаем, что все тела протяжёны, мы совершаем внеопытный скачок мысли -- она переходит в ту сферу, которая опытом не обусловлена.
\item Всеобщее знание (априорное) не может быть выведено из опыта, так как опыт никогда не завершён -- исследователь никогда не увидит всех тел, но изучив лишь некоторые из них выводит закон природы.
\item Таким образом, получается, что при высказывании всеобщих и необходимых теоретических суждений мы мыслим иначе, чем при простом обобщении опытных данных.
\end{itemize}
\end{itemize}
}



\subsection{Учение Канта об априорных формах}

{\parindent0pt
Кант выделяет уровни познания мира человеком в соответствии со способностями его души.
\begin{itemize}[nosep,leftmargin=0.5cm]
\item Чувственность (способность восприятия).
\item Рассудок (способность категорических суждений).
\item Разум (способность умозаключений, доходящих до понятия о целостности явления).
\end{itemize}
}
\ 

{\parindent0pt
Априорные формы чувственности: пространство и время -- образуют необходимое основание для форм рассуждения (мышления).
\begin{itemize}[nosep,leftmargin=0.5cm]
\item Пять каналов поступления опытного знания (зрение, осязание, вкус, обоняние, слух) лежат в основании способности получать впечатления от предметов, которую Кант связывает с чувственным восприятием (созерцанием, перцепцией)
\item По Канту, пространство и время -- не характеристики вещи, а способы, в которых мы созерцаем вещь, способы интеграции ощущений в определённые структуры, которые придают очевидность, непосредственность и достоверность воспринимаемым явлениям.
\end{itemize}
}
\ 

{\parindent0pt
Априорные формы рассудка представлены категориями: количества (единичное, множество, целое), качества (реальность, отрицание, ограничение), отношения (причина-следствие) и модальности (возможность, необходимость, случайность).
\begin{itemize}[nosep,leftmargin=0.5cm]
\item Под эти категории Кант подводит всякое содержание знания, поставляемое чувственностью.
Категории придают объективную значимость знаниям.
\item Однако сами по себе, взятые по отдельности, ни априорные формы чувственности, ни априорные понятия рассудка не дают знания.
Для возникновения знания необходимо их объединение -- синтез чувственного созерцания и понятий рассудка.
\item Только взаимосвязь чувственности и рассудка может привести к появлению знания: "<Мысли без содержания пусты, а наглядные представления без понятий слепы ... рассудок не может ничего наглядно представить, а чувства не могут ничего мыслить">.
\end{itemize}
}
\ 

{\parindent0pt
Сознание предварительно должно привести себя в единство, а потом лишь познавать.
Кант фиксирует две формы единства сознания: 1) трансцендентальное единство апперцепции ("<самосознание учёного">), 2) продуктивная сила воображения.
}



\subsection{Учение Канта об идеях и антиномиях разума}

{\parindent0pt
Познание формируется под влиянием двух факторов:
\begin{itemize}[nosep,leftmargin=0.5cm]
\item объективно существующего предметного мира (вещи-в-себе)
\item нашего сознания, которое активно синтезирует, оформляет, организует ощущения в целостный образ
\end{itemize}
}
\

{\parindent0pt
Деятельность Рассудка эффективна в рамках конечного опыта.
}
\ \\

{\parindent0pt
Идеи разума не могут быть даны в опыте, они могут лишь мыслиться нами как безусловные понятия, ориентирующие деятельность рассудка.
\begin{itemize}[nosep,leftmargin=0.5cm]
\item "<Под идеей я разумею такое необходимое понятие разума, для которого в чувствах не может быть дан никакой адекватный предмет">, -- пишет Кант.
Примеры трансцендентальных идей: идея добра как такового, идея совершенного государственного устройства.
\end{itemize}
}
\ 

{\parindent0pt
Разум пытается выйти за пределы конечного опыта, познать сверхопытным путём вещи-в-себе, охватить явления в их целостности и подвижности, при этом неизбежно приходит к противоречиям с самим собой.
}
\ \\

{\parindent0pt
Идеи разума априорны и являются подлинной границей разума, отделяя познаваемый мир (мир для нас) от непознаваемого мира, мира вещей-в-себе.
}



\subsection{Антиномии разума, неизбежные в познании мира}

{\parindent0pt
Первая антиномия:
\begin{itemize}[nosep,leftmargin=0.5cm]
\item Мир имеет начало во времени и ограничен в пространстве. -- Мир не имеет начала во времени и границ в пространстве, он бесконечен.
\end{itemize}
}
\ 

{\parindent0pt
Вторая антиномия:
\begin{itemize}[nosep,leftmargin=0.5cm]
\item Всякая сложная субстанция в мире состоит из простых частей и вообще существует только простое и то, что сложено из простого. -- Ни одна сложная вещь в мире не состоит из простых частей и, вообще, в мире нет ничего простого.
\end{itemize}
}
\ 

{\parindent0pt
Третья антиномия:
\begin{itemize}[nosep,leftmargin=0.5cm]
\item Причинность согласно законам природы -- не единственная причинность, из которой могут быть выведены все явления в мире. Для их объяснения необходимо ещё допустить также свободную причинность. -- Не существует никакой свободы, но всё совершается в мире согласно законам природы.
\end{itemize}
}
\ 

{\parindent0pt
Четвёртая антиномия:
\begin{itemize}[nosep,leftmargin=0.5cm]
\item Безусловно необходимое существо принадлежит к миру или часть его, или как его необходимая причина. -- Нет никакого абсолютно необходимого существа ни в мире, ни вне его
\end{itemize}
}
\ 

{\parindent0pt
Антиномичный характер идей разума связан с тем, что эти идеи не обеспечены чувственным созерцанием: они только мыслимы.
}
\ \\

{\parindent0pt
То обстоятельство, что Кант считает познание вещей-в-себе невозможным, дало повод для упрёка Канта в агностицизме, что представляется проблематичным, поскольку Кант обосновывает возможность научного познания, хотя и считает его не безграничным.
}



\subsection{Георг Вильгельм Фридрих Гегель (1770-1831 гг.) создал систему, которая считается завершающим звеном в развитии европейского рационализма.}

{\parindent0pt
Главные характеристики его философии: объективный идеализм в учении о бытии, диалектический метод в учении о познании, диалектика духа в учении о человеке и общественной истории.
}
\ \\

{\parindent0pt
Обозначив систему взглядов Канта как субъективный идеализм, Гегель формулирует основные позиции, преодолевающие дуализм познающего субъекта и вещи-в-себе.
\begin{itemize}[nosep,leftmargin=0.5cm]
\item Природа существует независимо от человека и его сознания.
\item Человеческое познание обладает объективным содержанием.
\item Нет абсолютной границы между сущностью и явлением, подвижность этой границы Гегель фиксирует в афоризме: "<Сущность является, явление существенно">
\item В природе вещей нет непреодолимых преград для человеческого познания
\begin{itemize}[nosep,leftmargin=0.6cm]
\item По Гегелю, познание является познанием сущности, так как видимость не есть нечто субъективное, она есть проявление сущности.
\end{itemize}
\end{itemize}
}
\ 

{\parindent0pt
Главная мировоззренческая установка Гегеля определена тем, что ни природа, ни общество не могут выведены из человеческого "<Я">.
Наоборот, само человеческое сознание должно быть понятно как внешнее проявление и выражение внутренней сущности, первоосновы всего существующего.
В качестве такой первоосновы, обеспечивающей принципа единства мира, выступает Абсолютная идея -- онтологическое абсолютное тождество бытия и мышления.
\begin{itemize}[nosep,leftmargin=0.5cm]
\item Абсолютная идея -- метафора, фиксирующая не только единства мира, но и его внутреннюю связность.
При этом мышление в системе Гегеля оказывается не только вне психологии человека, но и вне мира, с одной стороны, но с другой -- присутствует в нём как его внутреннее содержание.
Мир познаваем, поскольку в основе своей логичен: законы мира и законы мышления тождественны.
\end{itemize}
}



\subsection{Логика развития и Диалектический метод познания}

{\parindent0pt
Гегель считал себя пантеистом, однако в истории философии его система получила название панлогизма.
}
\ \\

{\parindent0pt
В абсолютной идее он выразил главным образом необходимую закономерность развития мира как целого.
}
\ \\

{\parindent0pt
Логика развития абстрактного целого, развёртывания его внутреннего содержания присутствует изначально в мире и определяет его эволюцию.
\begin{itemize}[nosep,leftmargin=0.5cm]
\item В философии Гегеля Абсолютная идея выступает основанием всеобщей взаимосвязи, источником развёртывания всего природного и человеческого мира и его конечным результатом.
\end{itemize}
}
\ 

{\parindent0pt
Гегель разрабатывает диалектику как новый рациональный метод познания, позволяющий мыслить целое и постигать развивающееся единство.
}
\ \\

{\parindent0pt
Три закона диалектики: закон противоречия, закон меры (перехода количества в качество) и закон двойного отрицания -- формулируются Гегелем в крайне абстрактной форме как законы развития понятия - абстрактного целого и становятся основанием диалектической логики.
}



\subsection{Понимание движения в Диалектической логике Гегеля}

{\parindent0pt
Для современников Гегеля диалектика становится непонятной, но уважаемой формой спекулятивного философствования.
}
\ \\

{\parindent0pt
Однако даже в абстрактной форме законы диалектики представляют новую логику, позволяющую постигать развивающееся целое через противоречие.
\begin{itemize}[nosep,leftmargin=0.5cm]
\item Диалектическая логика Гегеля вводит в новую схему рассуждения, отличную от традиционной формальной логики, на которую опирался в своём анализе познавательных способностей Кант.
\item Традиционно закон противоречия регламентирует познание жёстким выбором ("<или-или">)
\item В диалектической логике закон противоречия касается отношения противоположностей в рамках некоторого единства.
Эти отношения могут развиваться от сходства к различию, затем к напряженному противостоянию (противоречию), которое неизбежно разрешается.
В результате возникает качественно новое единство, и всё снова повторяется.
\end{itemize}
}
\ 

{\parindent0pt
Логика развития приводит Гегеля к утверждению, что противоречие движет миром.
\begin{itemize}[nosep,leftmargin=0.5cm]
\item Действительно, любой элемент мира можно рассматривать как некоторое единство, содержащее внутри противоречие, постоянно созревающее и разрешающееся.
Поэтому жизненно то, что содержит противоречие, а постоянное движение, наблюдаемое в мире, есть всегда самодвижение.
\end{itemize}
}



\subsection{Развитие теории познания эмпиризма в XVIII в.}

{\parindent0pt

\textbf{Материалистический}

Томас Гоббс (1588-1679) защищал опытно-экспериментальный метод познания: "<Нет ни одного понятия в человеческом уме, которое не было бы порождено первоначально, целиком или частично, в органах ощущения">.
Мышление, по Гоббсу, оказывается лишь счётной операцией над знаками -- словами.
Выступал с позиций атеизма в защиту научного знания.


Джон Локк (1632-1704) даёт фундаментальное обоснование эмпиризма, показывая как из простых чувственных данных образуются сложные идеи.
В борьбе против теории врождённых идей защищал понимание человеческого сознания как tabula rasa (чистой доски), на которой опыт пишет свои письмена.
\\

\textbf{Идеалистический}

Джордж Беркли (1685-1753) утверждал, что существование вещей и их восприятие тождественны: "<Esse ost percipi"> -- существовать значит быть воспринимаемым.
Следовательно, нет ничего, кроме ощущений, а их совокупность и есть то, что называется вещами.
Тем самым разработал основные идеи солипсизма.

Давид Юм (1711-1776) исходил из того, что человеку даны лишь его чувственные впечатления, которые связаны психологическими ассоциативными связями.
Развил последовательный агностицизм, ибо источник впечатлений принципиально непознаваем.
Человек может выбирать между ложным знанием и отказом от знания.
}



\subsection{Философия науки в XIX в.}

{\parindent0pt
\textbf{Представление о науке} формируется под влиянием различных философских воззрений.
\begin{itemize}[nosep,leftmargin=0.5cm]
\item Значительную роль играют те конкретные научные области, которые формируют базовые представления, определяющие научную картину мира.
\item Вместе с решением фундаментальных вопросов о том, что представляет собой наука и каким образом она развивается, складывается определённый образ науки.
\end{itemize}
}
\ 

{\parindent0pt
\textbf{Философия науки} сформировалась в XIX в. как область знания, обращённая к разработке методолгических и мировоззренческих проблем науки.
\begin{itemize}[nosep,leftmargin=0.5cm]
\item Термин "<философия науки"> предложил немецкий философ Е. Дюринг, который поставил задачу разработать логику познания с опорой на достижения науки.
\end{itemize}
}
\ 

{\parindent0pt
В XX веке философия науки превратилась в специализированную область исследований.
\begin{itemize}[nosep,leftmargin=0.5cm]
\item Научно-философское мировоззрение требует от человека значительных интеллектуальных усилий и времени на его освоение.
\begin{itemize}[nosep,leftmargin=0.6cm]
\item Пользоваться этим способом освоения мира на сегодняшний день способны не более 20\% населения планеты.
Остальные предпочитают жить в рамках мифологического или религиозного мировоззрения.
\end{itemize}
\end{itemize}
}



\subsection{Становление философии позитивизма в середине XIX века}

{\parindent0pt
\textbf{Общие предпосылки}.
\begin{itemize}[nosep,leftmargin=0.5cm]
\item Возрастание роли точных и конкретных знаний в широкой области социальной жизни.
\item Рациональность и обоснованность действия противостоит общим словам, не подкреплённым и непроверенным опытом рекомендациям, которые развенчиваются как нецелесообразные и даже опасные.
\end{itemize}
}
\ 

{\parindent0pt
\textbf{Позитивизм} -- одно из проявлений влияния на философию стандартов мышления, сложившихся в науках о природе, в математике и логике.
\begin{itemize}[nosep,leftmargin=0.5cm]
\item К середине XIX в. каждая научная дисциплина стала развивить свои представления об исследуемой реальности и свои методы.
\item Рациональное знание, которое имеет сферу приложения в социальной жизни стали обозначать понятием позитивная наука.
\item Под сомнением оказалась универсальность механической картины мира, идеалов и методов механистического объяснения.
\end{itemize}
}
\ 

{\parindent0pt
\textbf{Исходная идея позитивизма} -- проведение разграничительной линии между наукой и всеми остальными формами духовной деятельности.
\begin{itemize}[nosep,leftmargin=0.5cm]
\item Борьба позитивистов с метафизикой не была самоцелью.
\item Эта борьба рассматривалась как средство защиты и обоснования рационального знания в противовес иррациональному и демагогии.
\end{itemize}
}



\subsection{Первый позитивизм (Классический) 30-гг. XIX в.}

{\parindent0pt
Огюст Конт (1798-1857) -- во Франции

Джон Стюарт Милль (1806-1873), Герберт Спенсер (1820-1903) -- в Англии.
}
\\

{\parindent0pt
Ключевые термины.
\begin{itemize}[topsep=0pt,itemsep=0.1cm,leftmargin=0.5cm]
\item \textbf{Позитивное знание} удовлетворяет утилитарному критерию.
\begin{itemize}[nosep,leftmargin=0.6cm]
\item его содержание должно быть сведено к непосредственно "<данному"> (ощущениям).
\end{itemize}
\item \textbf{Научная рациональность} самоценна и самодостаточна.
\begin{itemize}[nosep,leftmargin=0.6cm]
\item Утверждается автономия науки от культуры, религии, философии, морали, истории.
\end{itemize}
\item \textbf{Чистая наука} -- знание, свободное от философской интерпретации.
\item \textbf{Научная социология} -- "<Социальная физика">.
\begin{itemize}[nosep,leftmargin=0.6cm]
\item Принцип естественнонаучного познания распространялся на область биологии и социологии.
\end{itemize}
\item \textbf{Экономное мышление} -- основа позитивной науки.
\begin{itemize}[nosep,leftmargin=0.6cm]
\item Средство удобного описания ощущений субъекта познания.
\end{itemize}
\item \textbf{Логика познания} -- разработка методов получения нового позитивного знания, индуктивных методов установления причинных связей.
\begin{itemize}[nosep,leftmargin=0.6cm]
\item Метод сходства, различия, остатков и сопутствующих изменений (Дж.С.Милль).
\end{itemize}
\end{itemize}
}



\subsubsection{Основные принципы и положения О. Конт}

{\parindent0pt
\textbf{Принцип демаркации позитивной науки и метафизики}
\begin{itemize}[nosep,leftmargin=0.5cm]
\item Отказ от поисков первопричин.
Эти поиски -- бесплодная "<метафизика">.
Стремление к построению, которое должно опираться исключительно на "<факты">, полезное для применения на практике -- позитивное знание.
\end{itemize}
}
\ 

{\parindent0pt
\textbf{Идея чистой науки.} "<Наука сама себе философия">
}
\ \\

{\parindent0pt
\textbf{Принцип экономии мышления}
\begin{itemize}[nosep,leftmargin=0.5cm]
\item Научное знание -- "<экономное"> обозрения многообразия ощущений субъекта и ориентации в будущих ощущениях.
\item Наука и её законы отвечают не на вопрос "<почему">, а только на вопрос "<как">. (О.Конт)
\end{itemize}
}
\ 

{\parindent0pt
\textbf{Идея научной социологии как позитивной науки}
}
\ \\

{\parindent0pt
\textbf{Идея "<социальной физики">} как фундамента изучения и реорганизации общества на принципах естествознания (О.Конт, Г.Спенсер).
}



\subsubsection{Основные принципы и положения Дж.С.Милль}

{\parindent0pt
\textbf{Задачи философии науки}
\begin{itemize}[nosep,leftmargin=0.5cm]
\item систематизация специального знания
\item разработка методологических процедур, позволяющих выявлять наиболее перспективные гипотезы и направления в науке
\end{itemize}
}
\ 

{\parindent0pt
\textbf{Идеализация науки} как единственного эффективного средства решения всех проблем человечества
\begin{itemize}[nosep,leftmargin=0.5cm]
\item Оптимизм в отношении будущего, связанный с неограниченными возможностями прогресса науки -- основание сциентизма
\end{itemize}
}
\ 

{\parindent0pt
\textbf{Индуктивная логика} -- главное средство науки в установлении причинных связей
\begin{itemize}[nosep,leftmargin=0.5cm]
\item \textbf{Метод сходства} -- во всех сопутствующих исследуемому явлению обстоятельствах выделяется общее
\begin{itemize}[nosep,leftmargin=0.6cm]
\item Вывод: это обстоятельство -- причина явления, тем более вероятен, чем больше рассмотрено случаев и чем более они разнообразны.
\end{itemize}
\item \textbf{Метод различия} -- сравниваются два максимально похожих случая, в одном из которых имеет место данное явление, а в другом отсутствует.
\begin{itemize}[nosep,leftmargin=0.6cm]
\item То обстоятельство, которым различаются сравниваемые случаи, будет, вероятно, причиной явления.
\end{itemize}
\item \textbf{Метод сопутствующих изменений} применяется в тех случаях, когда изменение одного явления сопутствует изменению другого.
\begin{itemize}[nosep,leftmargin=0.6cm]
\item Если при этом другие обстоятельства остаются прежними, делается заключение, что одно из изменяющихся явлений -- причина другого.
\end{itemize}
\item \textbf{Метод остатков} применяется в случае, когда известно, что явление вызывается или может быть вызвано комплексом причин.
\begin{itemize}[nosep,leftmargin=0.6cm]
\item Если известно также то действие, которое оказывают некоторые причины из комплекса, то можно сказать, что остаток действия вызывается остатком причины.
\end{itemize}
\end{itemize}
}



\subsubsection{Основные принципы и положения Г.Спенсер}

{\parindent0pt
Идеал научной рациональности и объективности (впоследствии обозначался как классический идеал рациональности).
\begin{itemize}[nosep,leftmargin=0.5cm]
\item Построение научного знания -- по образцу механики, математики и физики
\item Позитивный научный метод -- эмпирический, гипотетико-дедуктивный
\item Принципы классификации наук
\end{itemize}
}
\ 

{\parindent0pt
Идея научной социологии как позитивной науки, в задачу которой входит создание "<социальной физики">.
}
\ \\

{\parindent0pt
Принципы физики распространяются на процесс познания в области биологии и социологии
}
\ \\

{\parindent0pt
Разработка идеи эволюционизма в её широком применении.
\begin{itemize}[nosep,leftmargin=0.5cm]
\item Основание социал-дарвинизма.
\item Положение об эволюции Вселенной как закономерном и необходимом процессе.
\end{itemize}
}



\subsection{Второй позитивизм (эмпириокритицизм, или махизм)}

{\parindent0pt
Конец XIX в.

Р.Авенариус (1843-1896),
Э.Мах (1838-1916).
\\

Далее представлены ключевые термины.
\\
}

{\parindent0pt
\textbf{Эмпириокритицизм} -- критика опыта как главного источника научного знания на фоне "<кризиса в физике"> на рубеже XX в.
\begin{itemize}[nosep,leftmargin=0.5cm]
\item Попытка интерпретировать явление радиоактивности, приводила к выводу о том, что вещество, т.е. материя (как тогда считали), может превратиться в нечто, не имеющее массы, а это не материя.
\item На первый план вышли чисто мировоззренческие вопросы: что мы изучаем, каково соотношение наших знаний об этом мире с самим этим миром?
\end{itemize}
}
\ 

{\parindent0pt
\textbf{Чистый опыт} -- "<ничей"> опыт -- единство субъективного и объективного, физического и психического.
\begin{itemize}[nosep,leftmargin=0.5cm]
\item Опыт состоит из элементов, он первичен, материя и дух -- вторичны.
\end{itemize}
}
\ 

{\parindent0pt
\textbf{Нейтральные элементы опыта}
\begin{itemize}[nosep,leftmargin=0.5cm]
\item "<Ничьи"> ощущения, "<ничей"> опыт -- таким должно быть естественное представление о мире
\item Вместо "<исчезнувшей материи"> остаётся вечный и неизменный комплекс элементов опыта, которые в зависимости от рассмотрения могут трактоваться как физические или психические.
\item Опыт представляет собой изначальную реальность, в которой нет расщепления на субъект и объект.
Такое расщепление возникает в результате некритического восприятия индивидами чужого опыта.
\end{itemize}
}
\ 

{\parindent0pt
\textbf{Принципиальная координация} -- интегральное единство субъективного и объективного, физического и психического, "<Я и мира">.
\begin{itemize}[nosep,leftmargin=0.5cm]
\item Организм в своём поведении постоянно трансформирует внешнее во внутреннее, а внутреннее во внешнее.
Акты поведения выступают одновременно актами понимания мира
\end{itemize}
}
\ 

{\parindent0pt
\textbf{Интроекция} (от лат. intro -- внутрь, lacere -- бросать) -- усвоение опыта других людей как своеобразное вкладывание (вбрасывание) чужих ощущений и восприятий в мою душу и тело.
\begin{itemize}[nosep,leftmargin=0.5cm]
\item Позднее этот термин стал применяться в психоанализе, обозначив включение в психику индивида взглядов, мотивов, образов, установок других людей.
\end{itemize}
}



\subsubsection{Основные принципы и положения Э.Мах}

{\parindent0pt
Исходное положение эмпириокритицизма: "<существует только опыт">.
\begin{itemize}[nosep,leftmargin=0.5cm]
\item Наш опыт ощущений -- это и есть мир, в котором мы живём.
\begin{itemize}[nosep,leftmargin=0.6cm]
\item Ощущения существуют сами по себе как нейтральные элементы мира
\item Отказ от признания какой бы то ни было объективности в опыте
\end{itemize}
\end{itemize}
}
\ 

{\parindent0pt
Цель научного познания -- накопление опытных данных и наиболее экономное описание элементов опыта.
\begin{itemize}[nosep,leftmargin=0.5cm]
\item Опытные факты представлены в науке прямыми описаниями, непосредственно фиксирующими наблюдения
\item Научные законы -- экономный способ описания ощущений, представляющих данные наблюдения, которые и есть элементы чистого опыта (без метафизики).
\begin{itemize}[nosep,leftmargin=0.6cm]
\item Мах отмечал невозможность свести к механическим движениям все изучаемые наукой процессы
\item Представления об атомистическом строении вещества -- мифологема
\end{itemize}
\end{itemize}
}
\ 

{\parindent0pt
Теории -- косвенные описания многообразия наблюдений для удержания в памяти.
\begin{itemize}[nosep,leftmargin=0.5cm]
\item Теоретические законы, представления и понятия -- сжатая сводка опытных данных, способ их упорядочивания
\begin{itemize}[nosep,leftmargin=0.6cm]
\item По мере расширения опыта происходит смена теорий.
Прежние теории отбрасываются и заменяются новыми, более экономно описывающими опыт.
\end{itemize}
\end{itemize}
}
\ 


{\parindent0pt
Новая модель реальности вместо механицизма.
\begin{itemize}[nosep,leftmargin=0.5cm]
\item Элементы опыта (ощущения) и их функциональные отношения представляют собой единственную реальность, которую можно допустить.
\item Функциональные отношения между элементами мира позволяют сконструировать два типа процессов -- физические и психические.
\begin{itemize}[nosep,leftmargin=0.6cm]
\item Поскольку оба этих типа порождают комбинации одних и тех же элементов, постольку сами элементы не являются ни физическими, ни психологическими.
Они нейтральны.
\end{itemize}
\end{itemize}
}



\subsubsection{Основные принципы и положения Р.Авенариус}

{\parindent0pt
Познание -- особый аспект жизнедеятельности.
\begin{itemize}[nosep,leftmargin=0.5cm]
\item Жизнь -- процесс накопления и расходования энергии.
Стратегия выживания связана со стремлением организмов минимизировать затраты энергии в процессе адаптации к среде, экономно расходовать свои энергетические запасы
\end{itemize}
}
\ 

{\parindent0pt
Идея принципиальной координации подчёркивала, что опыт представляет собой изначальную реальность, в которой нет расщепления на субъект и объект.
Такое расщепление возникает в результате некритического восприятия индивидами чужого опыта
}
\ \\

{\parindent0pt
Организм в своём поведении постоянно трансформирует внешнее во внутренне, а внутреннее во внешнее, интегрирует то, что связано с внешней средой, и то, что связано с человеческой активностью.
}
\ \\

{\parindent0pt
Опыт любого индивида не ограничивается только личным чувственным опытом, он расширяется за счёт научения, восприятия опыта других людей.
\begin{itemize}[nosep,leftmargin=0.5cm]
\item Но в этом процессе чужой опыт, который выступает таким же единством внутреннего и внешнего, как и собственный, воспринимается и оценивается как нечто внешнее
\item В результате возникает представление о внешнем объективном и внутреннем субъективном, которые затем преобразуются в противопоставление субъекта и объекта, души и тела, материи и сознания
\item Чувственный опыт начинает рассматриваться как состояние души, как психическое
\item Усвоение опыта других людей истолковывается как своеобразное вкладывание (вбрасывание = интроекция) чужих ощущений и восприятий в мою душу и тело
\end{itemize}
}
\ 

{\parindent0pt
Критика представления о сознании как функции мозга как недопустимого проявления интроекции, порождающей противопоставление духовного и телесного.
}
\ \\

{\parindent0pt
Критика познавательного отношения субъекта к объекту как зеркального отражения в сознании свойств, связей и отношений внешних вещей, благодаря чему человек может адекватно ориентироваться во внешнем мире.
}



\subsection{Третий позитивизм (неопозитивизм) 20-30 гг. XX в.}

{\parindent0pt

Основные представители: М.Шлик (1882-1936), Л.Витгенштейн (1889-1936), Р.Карнап (1891-1970), Г.Райхенбах (1891-1953), Б. Рассел (1872-1970).
\\

Терминология:
логический атомизм;
интерсубъективность;
конвенционализм;
верифицируемость;
физикализм;
редукция;
формализм;
языковая игра;
формализованные системы;
атомарные и молекулярные высказывания;
кумулятивизм;
протокольные предложения;
язык наблюдения, язык теории и метатеории;
значение, смысл, семантический треугольник.
\\

Альтернатива эмпириокритицизму -- Логический позитивизм.
\\


Отказ от признания "<чувственных данных"> субстанциальной основой мира
}
\ \\

{\parindent0pt
Исходные предпосылки всякого познания в неопозитивизме -- "<события"> и "<факты">, находящихся в сфере сознания субъекта.
\begin{itemize}[nosep,leftmargin=0.5cm]
\item Признаётся различие между "<голыми"> ощущениями и результатами их рациональной обработки.
Ощущения -- "<материал познания">, исходная данность, с которой в процессе познания можно манипулировать.
\end{itemize}
}
\ 

{\parindent0pt
Предмет философии науки -- язык науки как способ выражения знания.
\begin{itemize}[nosep,leftmargin=0.5cm]
\item Задача философии -- устранение псевдо-проблем, возникающих из-за неправильного употребления языка
\item Построение идеальных моделей осмысленного рассуждения с использованием аппарата математической логики.
\item Представление о творческой роли мышления, которое преобразует и расширяет опыт.
\end{itemize}
}
\ 


{\parindent0pt
В учении неопозитивизма о логических (теоретических) конструкциях подчёркивалось тождество: объекта и теории объекта; "<объективного факта"> (существующего независимо от процесса познания) и "<научного факта"> (зафикисированного в системе науки с помощью знаковых средств).
\begin{itemize}[nosep,leftmargin=0.5cm]
\item Приоритет в системе науки логики и математики как дедуктивных построений, опирающихся на произвольные соглашения (конвенции).
\item Разрыв формального языка науки (в лице логики и математики) и сферы чувственного опыта.
\item Превращение формального начала и вообще языка в главный объект философии науки, почему это течение было названо логической разновидностью позитивизма.
\item Логический позитивизм сложился в 1923 г. в Венском университете под руководством Морица Шлика (1882-1936).
\end{itemize}
}
 


\subsubsection{Концепция логического атомизма Б.Рассела}

{\parindent0pt
Истоки: проблема обоснования математики, парадоксы теории множеств.
}
\ \\

{\parindent0pt
Парадоксы возникают в результате смешения уровней абстракции, когда один термин может обозначать абстракции разного уровня
}
\ \\

{\parindent0pt
Эта идея была положена в основу теории типов, которая требовала чётко разделять абстракции разных уровней и налагала запреты на их смешение.
\begin{itemize}[nosep,leftmargin=0.5cm]
\item Она требовала различать язык, который говорит о признаках некоторого класса объектов, и метаязык, который говорит о классе классов.
\item Парадоксы теории множеств, согласно Расселу, являются результатом смешения языка и метаязыка.
\end{itemize}
}
\ 

{\parindent0pt
Простые высказывания, из которых образуются сложные, Рассел называл атомарными, а сложные -- молекулярными.
Он придал им гносеологическую трактовку.

}
\ 

{\parindent0pt
\textbf{Атомарные высказывания} непосредственно фиксируют реальное "<положение дел">, присущие реальным предметам свойства или отношения.
}
\ \\

{\parindent0pt
\textbf{Молекулярные высказывания} опосредованно описывают реальность, положение дел.
Их истинность обосновывается редукцией к атомарным.
}
\ \\



\subsubsection{Логический позитивизм Л.Витгенштейн}

{\parindent0pt
Л.Витгенштейн истолковал язык пропозициональной логики как модель мира, находящуюся к нему в отношении отображения.
}
\ 

{\parindent0pt
Принцип однозначного соответствия между структурой языка и структурой мира.
\begin{itemize}[nosep,leftmargin=0.5cm]
\item Предложение выступает как образ факта, как его изображение.
Оно по своей логической структуре должно быть картиной факта
\item Факт -- это то, о чём говорится в предложении, это то, что делает предложение истинным.
\item Логический анализ, проясняющий логическую структуру языка, выявляющий её природу как повествование о фактах.
В этом случае язык будет показывать структуру мира.
Он не описывает эту структуру, но демонстрирует её.
\item Границы языка и есть границы мира.
\end{itemize}
}
\ 

{\parindent0pt
Противопоставление теоретического знания, как обоснованного, эмпирическому -- необоснованному (изменчивому).
\begin{itemize}[nosep,leftmargin=0.5cm]
\item В науке существует "<словарь опыта"> и "<словарь теории">.
\item Язык наблюдения независим от языка теории.
Опыт и теория говорят на разных языках.
\end{itemize}
}
\ 

{\parindent0pt
Протокольное предложение (высказывание) фиксирует данные чистого опыта, обладающего безусловной достоверностью и нейтральностью по отношению к теоретическому знанию.
}
\ \\

{\parindent0pt
Выдвинули кумулятивную модель развития научного знания.
\begin{itemize}[nosep,leftmargin=0.5cm]
\item Научных революций не может быть, поскольку новая теория включает в себя старую как частный случай, её можно дедуцировать с помощью логических правил из новой теории.
Понятия старой теории не изменяют своего значения при переходе к новой теории.
\item Образец научности -- теории математической физики.
Всё научное знание должно иметь аксиоматический и гипотетико-дедуктивный строй.
\end{itemize}
}
\ 

{\parindent0pt
Философия науки -- сама является наукой (аналитическая философия).
\begin{itemize}[nosep,leftmargin=0.5cm]
\item В ней должна быть одна общезначимая, признанная методологическая концепция.
\item Логический позитивизм как философию науки характеризуют три методологических принципа: принцип верификации, конвенционализм, физикализм.
\end{itemize}
}



\subsubsection{Логческий позитивизм. Методологический принцип верификации}


{\parindent0pt
Идея совершенного языка науки в отличие от обыденного (Бертран Рассел).
}
\ \\

{\parindent0pt
Принцип верификации (лат. versus - истинный; facere - делать).
\begin{itemize}[nosep,leftmargin=0.5cm]
\item Призван осуществить "<демаркацию"> (разграничение) между утверждениями, имеющими смысл для науки, и утверждениями, лишёнными научного смысла.
Понятия старой теории не изменяют своего значения при переходе к новой теории.
\item Поскольку обыденная речь из-за своей многозначности и неопределённости для целей науки малопригодна.
\end{itemize}
}
\ 

{\parindent0pt
Принцип верификации гласит: только то предложение имеет смысл, которое хотя бы в принципе, прямым или косвенным образом, допускает сведение к предложениям, обозначающим непосредственный чувственный опыт индивида или протокольным предложениям учёного (фиксация опыта в предложении).
}



\subsubsection{Методологический принцип конвекционализма}

{\parindent0pt
\textbf{Конвекционализм} постулирует существование в науке произвольных соглашений, действующих в виде исходных (аксиоматических) положений логической структуры науки.
\begin{itemize}[nosep,leftmargin=0.5cm]
\item Развивая этот принцип, Р.Карнап в 1934 году предложил "<принцип терпимости">, согласно которому можно выбирать ("<можно терпеть">) любую избранную решением субъекта непротиворечивую логическую систему.
\item Это привело к вопросу о мотивах выбора тех или иных конвенций.
\item Основатель логического позитивизма М.Шлик считал, что при выборе аксиом надо стремиться к простой формулировке законов природы.
\item Принцип конвенционализма был связан с признанием "<свободы"> в формальных построениях, но, в сущности, отрицал, что сама эта "<свобода"> обусловлена многообразием мира, существующим независимо от субъекта.
\end{itemize}
}
\ 

{\parindent0pt
Из сочетания принципов верификации и конвенционализма вытекало понимание строения науки как совокупности условных теоретических конструкций, создаваемых при помощи условных логических средств на базе эмпирических (фактуальных) констатаций ("<протоколов">) и поддающихся затем сведению (редукции) к эти протоколам.
}
\ \\

{\parindent0pt
Индуктивные обобщения играли среди этих средств значительную роль.
Связанные с ними трудности вызвали среди неопозитивистов разногласия, в результате сложилась гипотетико-дедуктивная схема построения науки.
}



\subsubsection{Методолгический принцип физикализма}

{\parindent0pt
\textbf{Физикализм} - требование адекватного перевода предложений всех наук, содержащих описание предметов в терминах наблюдения, на предложения, состоящие исключительно из терминов, которые употребляются в физике.
\begin{itemize}[nosep,leftmargin=0.5cm]
\item Философы в XVII в. пытались уложить все науки в прокрустово ложе механики, а неопозитивисты -- математической физики, по её состоянию на 30-е годы XX в.
\item В первой половине 30-х годов физикализм пережил полосу бурного расцвета, а затем началось его быстрое падение.
\end{itemize}
}
\ 

{\parindent0pt
Распад физикализма привёл к обеднению неопозитивистской доктрины, чему также способствовало "<ослабление"> принципов верификации и конвенционализма.
\begin{itemize}[nosep,leftmargin=0.5cm]
\item Первый из них был сведён к общему пожеланию о подкреплении утверждений опытом, а второй к не уточнённому далее отрицанию опытного происхождения законов логики и математики.
\end{itemize}
}
\ 

{\parindent0pt
Поставленные неопозитивизмом проблемы.
}
\ \\

{\parindent0pt
Интерсубъективности предложений науки, продуктивности теоретического знания и мышления.
}
\ \\

{\parindent0pt
Проблема значения и смысла -- проблемы семантики.
}
\ \\

{\parindent0pt
Задача логического анализа языка науки.
}
\ \\

{\parindent0pt
Единство (унификация) науки и методологию, которая бы обеспечила прогрессивный рост научного знания.
}



\subsubsection{Язык -- знаковая система. Значение, смысл, семантический треугольник}

{\parindent0pt
Всякий знак имеет два значения -- предметное и смысловое.
}
\ \\

{\parindent0pt
\textbf{Предметное значение} -- это объект, который представлен знаком (обозначен).
Предметное значение знака определяет его содержание.
}
\ \\

{\parindent0pt
\textbf{Смысловое значение}, или смысл знака -- это характеристика объекта, представителем которого выступает знак.
\begin{itemize}[nosep,leftmargin=0.5cm]
\item Смысловое значение всегда предполагает тот или иной план выражения (или контекст).
\item Информация об этом обозначенном предмете (объекте, явлении) может подаваться в разном контексте...
\end{itemize}
}
\ 

{\parindent0pt
Таким образом, знак в пространстве коммуникации порождает некоторое смысловое поле вокруг обозначенного предмета, допуская различные способы его характеристики.
Что, в свою очередь, порождает проблему взаимопонимания.
}
\ \\

{\parindent0pt
Язык -- знаковая символическая система, которая выступает наиболее эффективным средством коммуникации в человеческом обществе.
\begin{itemize}[nosep,leftmargin=0.5cm]
\item Языковые знаки -- слова, которые имеют всегда предметное значение.
Смысл написанных или произнесённых слов позволяет выделить различные уровни информации о предмете, явлении, событии.
\end{itemize}
}



\subsubsection{Семантический треугольник образован тремя вершинами: слово-значение-смысл}

{\parindent0pt

Слово: "<Байкал">.

Предметное значение: озеро.

Смысловой контекст (план выражения):
1) эстетический -- самое красивое озеро в Сибири;
2) географический -- озеро, имеющее много притоков и только один исток;
3) экологический -- ... и т.д.
\\

Главная цель использования знаков -- эффективная передача информации.

Всякая система знаков имеет три аспекта: синтаксический, семантический, прагматический.
\\

\textbf{Синтаксический аспект знаковой системы} представлен правилами образования и преобразования выражений, имеющих смысл.

\textbf{Семантический аспект знаковой системы} представлен устойчивыми значениями и смыслом знаков в данной системе.
Главный вопрос семантики: что стоит за знаком?

\textbf{Прагматический аспект знаковой системы} выделяет отношение человека к знаковой системе (как она воспринимается, доступна или нет, вызывает ли отрицательные последствия).
\\

\textbf{Пример:} светофор как знаковая система.

Синтаксический: правила подачи сигналов, регулирующие движение пешеходов и машин.

Семантический: смысл красного цвета -- "<стой!">; смысл жёлтого цвета -- "<жди!">; смысл зелёного цвета -- "<иди!">.

Прагматический: всегда ли человек хорошо воспринимает сигналы (знаки) светофора;
всем ли понятно их значение;
если цвета меняются слишком быстро, восприятие знаков затрудняется.
\\

Естественные системы знаков складываются стихийно в процессе жизнедеятельности людей и животных.
К естественным знакам относятся: естественные языки (русский, английский, испанский и др.), запахи, позы у животных.
Специфика знаковой деятельности у животных выражается в том, что она непосредственно привязана к чувственным инстинктивным формам проявления потребностей и ограничена естественными знаками в виде запахов, поз, криков.
Искусственные системы знаков создаются человеком сознательно с определённой целью.
Например, дорожные знаки, нотные знаки, математические знаки, искусственные языки (языки программирования).

}



\subsubsection{Проблемы семантики}

{\parindent0pt
Проблема понимания может возникнуть на каждом из трёх уровней действия знаковой системы:
\begin{itemize}[nosep,leftmargin=0.5cm]
\item на синтаксическом -- из-за незнания правил образования выражений или их пренебрежением (известный пример из сказки: "<Казнить нельзя помиловать">, -- показывает как легко изменить смысл утверждения на противоположный в зависимости от постановки запятой);
\item на семантическом -- из-за различия в интерпретации знаков и выражений, примеры многозначности слов даёт омонимия, полисемия;
\item на прагматическом -- из-за психического барьера в восприятии речи или знака.
\end{itemize}
}
\ 

{\parindent0pt
В философии подчёркивается, что условия понимания располагаются на трёх уровнях: семантическом, рефлексивном, экзистенциальном.
}
\ \\

{\parindent0pt
Реальность, на которую направлено мышление субъекта в процессе понимания, всегда определённым способом освоенная реальность.
}
\ \\

{\parindent0pt
В мире человеческой культуры это, прежде всего, языковая реальность.
}
\ \\

{\parindent0pt
События "<проговариваются"> во внутренней речи и предстают для субъекта как текст, требующий интерпретации, которая никогда не может быть завершена и не может быть отделена от самопонимания интерпретатора.
}
\ \\




\end{document}
