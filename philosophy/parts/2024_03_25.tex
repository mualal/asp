\documentclass[main.tex]{subfiles}

\begin{document}

\linksection{Лекция 25.03.2024 (Шипунова О.Д.)}

\sublinksection{\,\,Система естествознания в XX веке}

\insertlectureslide{1}{11}

\sublinksection{\,\,Содержание лекции}

\insertlectureslide{2}{11}

\sublinksection{\,\,Общая характеристика развития естествознания в XX веке}

\insertlectureslide{3}{11}

\sublinksection{\,\,Стандарт логически строгой теории}

\insertlectureslide{4}{11}

\sublinksection{\,\,Принципы теоретического естествознания}

\insertlectureslide{5}{11}

\sublinksection{\,\,Теоретическая физика создаёт новые абстракции}

\insertlectureslide{6}{11}

\sublinksection{\,\,Специальная теория относительности (СТО)}

\insertlectureslide{7}{11}

\sublinksection{\,\,Постулаты теории относительности}

\insertlectureslide{8}{11}

\sublinksection{\,\,Следствия специальной теории относительности}

\insertlectureslide{9}{11}

\sublinksection{\,\,Релятивистская динамика}

\insertlectureslide{10}{11}

\sublinksection{\,\,Общая теория относительности (ОТО)}

\insertlectureslide{11}{11}

\insertlectureslide{12}{11}

\sublinksection{\,\,Теория строения атома}

\insertlectureslide{13}{11}

\sublinksection{\,\,Физика высоких энергий}

\insertlectureslide{14}{11}

\sublinksection{\,\,Теория радиоактивного распада}

\insertlectureslide{15}{11}

\sublinksection{\,\,Закон радиоактивного распада}

\insertlectureslide{16}{11}

\sublinksection{\,\,Физика элементарных частиц}

\insertlectureslide{17}{11}

\sublinksection{\,\,Антивещество}

\insertlectureslide{18}{11}

\sublinksection{\,\,Физика микромира}

\insertlectureslide{19}{11}

\sublinksection{\,\,Стандартная модель строения микромира}

\insertlectureslide{20}{11}

\sublinksection{\,\,Квантовая теория}

\insertlectureslide{21}{11}

\sublinksection{\,\,Квантовая теория поля}

\insertlectureslide{22}{11}



\end{document}
