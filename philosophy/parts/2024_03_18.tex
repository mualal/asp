\documentclass[main.tex]{subfiles}

\begin{document}

\linksection{Лекция 18.03.2024 (Шипунова О.Д.)}

\sublinksection{\,\,История классической системы естествознания}

\insertlectureslide{1}{10}

\insertlectureslide{2}{10}

\sublinksection{\,\,Краткая история естественнонаучных дисциплин}

\insertlectureslide{3}{10}

\sublinksection{\,\,Натурфилософия -- первая система естествознания}

\insertlectureslide{4}{10}

\sublinksection{\,\,Краткая история натурфилософии как системы естествознания}

\insertlectureslide{5}{10}

\sublinksection{\,\,Направления научной мысли в XVII в.}

\insertlectureslide{6}{10}

\sublinksection{\,\,Разработка математических методов описания физических движений}

\insertlectureslide{7}{10}

\sublinksection{\,\,Натурфилософия Рене Декарта -- Картезианская физика}

\insertlectureslide{8}{10}

\sublinksection{\,\,Принцип психофизического дуализма Декарта}

\insertlectureslide{9}{10}

\sublinksection{\,\,Натурфилософия И. Ньютона}

\insertlectureslide{10}{10}

\sublinksection{\,\,Классическая механика И. Ньютона}

\insertlectureslide{11}{10}

\sublinksection{\,\,Два принципа взаимодействия в описании движений -- Ньютоновская и Картезианская физика}

\insertlectureslide{12}{10}

\sublinksection{\,\,Мировоззренческие принципы точного экспериментального естествознания}

\insertlectureslide{13}{10}

\sublinksection{\,\,Методологические принципы точного экспериментального естествознания}

\insertlectureslide{14}{10}

\sublinksection{\,\,Принцип механической редукции}

\insertlectureslide{15}{10}

\sublinksection{\,\,Принцип детерминизма}

\insertlectureslide{16}{10}

\sublinksection{\,\,Оформление классической системы естествознания в XIII в.}

\insertlectureslide{17}{10}

\sublinksection{\,\,Математика и естествознание в XVIII в.}

\insertlectureslide{18}{10}

\sublinksection{\,\,Вариационный принцип Лагранжа-Гамильтона}

\insertlectureslide{19}{10}

\sublinksection{\,\,Математика и астрономия XIII в.}

\insertlectureslide{20}{10}

\sublinksection{\,\,Проблемы в естествознании XVIII в.}

\insertlectureslide{21}{10}

\sublinksection{\,\,Открытия в естествознании XVIII в.}

\insertlectureslide{22}{10}

\sublinksection{\,\,Гипотезы о природе электрических явлений в XVIII в.}

\insertlectureslide{23}{10}

\sublinksection{\,\,Теория электростатики и электродинамики в XVIII в.}

\insertlectureslide{24}{10}

\sublinksection{\,\,Основания химии в XVIII в. -- Теория горения вещества}

\insertlectureslide{25}{10}

\sublinksection{\,\,Корпускулярная теория строения вещества}

\insertlectureslide{26}{10}

\sublinksection{\,\,Развитие теоретических представлений в химии XIX-XX вв.}

\insertlectureslide{27}{10}

\sublinksection{\,\,Основные тенденции в развитии теоретической химии}

\insertlectureslide{28}{10}

\sublinksection{\,\,Концепции химической кинетики}

\insertlectureslide{29}{10}

\sublinksection{\,\,Представление о химической эволюции в естествознании}

\insertlectureslide{30}{10}

\sublinksection{\,\,Биохимия}

\insertlectureslide{31}{10}

\sublinksection{\,\,Молекулярная биология}

\insertlectureslide{32}{10}

\sublinksection{\,\,Биохимия живой клетки}

\insertlectureslide{33}{10}

\sublinksection{\,\,Роль воды в биохимических процессах}

\insertlectureslide{34}{10}

\sublinksection{\,\,Нейрохимия}

\insertlectureslide{35}{10}

\sublinksection{\,\,Нейрохимия в объяснении биологической эволюции}

\insertlectureslide{36}{10}


\end{document}