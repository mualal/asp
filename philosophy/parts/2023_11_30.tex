\documentclass[main.tex]{subfiles}

\begin{document}

\section{Лекция 30.11.2023 (Шипунова О.Д.)}

\insertlectureslide[0]{1}{04}

Постпозитивизм выступает альтернативой предыдущим течениям.
Постпозитивизм пытается создать такие системы, которые в известной степени отрицают ту систему философии науки, которая сложилась до этого (и получила название логический позитивизм).

Общая традиция в постпозитивизме связана с тем, что он вообще выступает против исходных положений логического позитивизма (которые связаны с тем, что основной метод познания -- индуктивный).

Постпозитивизм выступает своего рода критической философией.

\subsection{Постпозитивистские концепции в философии науки}

\insertlectureslide[0]{2}{04}

Первый критик, который выступает в этом контексте -- это Карл Поппер (создатель системы критического рационализма).

Он критикует два главных устоя (или истока) предыдущего течения.
В прошлый раз обсуждали, что логический позитивизм формулирует 3 основных методологических принципа в системе науки: принцип верификации, принцип конвенционализма и принцип физикализма.
Так вот критики (в частности критическая философия Карла Поппера) направлена против принципа верификации и принципа конвенционализма.

\subsection{Критический рационализм Карла Поппера}

\insertlectureslide[0]{3}{04}

Здесь с одной стороны фиксируется принцип верификации так, как он был сформулирован в предыдущем течении.
Он предполагает сведение любых утверждений в науке к протокольным суждениям и проверке на опыте и личных ощущениях.
Это очень узкая позиция сенсуализма.
А с другой стороны, индуктивный метод (от фактов к обобщению) всегда подвержен критике, потому что индуктивное рассуждение не может привести к достоверному заключению, а приводит к вероятному.

Таким образом, принцип верификации либо требует соглашения, либо он требует фальсификации (как говорит Поппер).
Когда говорят о принципе фальсификации Поппера, то он опирается на критику индуктивной логики.

Поппер требует заменить принцип верификации принципом фальсификации.

\insertlectureslide[0]{4}{04}

Следующее достижение (нововведение), которое Карл Поппер вводит в систему философии науки -- это идея третьего мира знания.
Он, строго говоря, рассматривает структуру реальности через 3 уровня (через 3 мира).

Третий мир -- это мир знания.

Важно, что этот мир знания существует отдельно и независимо от каждого индивидуального субъекта познания.
И этот мир знания -- то, с чем работает наука.

\insertlectureslide[0]{5}{04}

Поппер строит определённую эволюцию научного знания.
В дальнейшем появляется термин эволюционная эпистемология (или эволюционная теория познания), где он предлагает модель смены научных теорий.

Если брать модель развития научного знания Поппера (абстрактную, которая не завязана на сознании человека, а завязана только на росте научного знания, как растёт знание в третьем мире), то она представлена следующим образом: сначала есть некая исходная проблема, дальше её предположительное решение или "<пробная теория">, далее эту гипотезу либо поддерживают экспериментальные результаты, либо опровергают, и затем формулируется новая проблема.
Т.е. метод фальсификации Поппера работает в этом случае, когда проводится решающий эксперимент и отбрасываются гипотезы или переформулируются в новую проблему. 

Важно то, что Поппер подчёркивает, что в процессе выдвижения гипотез участвуют не только собственно научные представления, но и другие идеи, поскольку его исходная позиция -- это взаимосвязь трёх миров (мира природы, мира сознания человека и мира знаний), то соответсвенно мировоззренческие идеи и так далее тоже участвуют в процессе выдвижения гипотез.

\subsection{Научное знание}

\insertlectureslide[0]{6}{04}

На этом слайде проиллюстрирована идея о том, что любая теория (или теоретическое знание) погружена в некий более широкий контекст или какие-то социокультурные представления, которые раньше в науке устранялись (через принцип демаркации всех нерациональных не позитивных наук).

Здесь можно проследить такую взаимосвязь: как мировоззрение влияет на принцип познания причин и следствий и те теоретические установки, которые формулируются в данную эпоху.

Это иллюстрация к тому, что в данном случае произошло в системе постпозитивизма, когда в объяснение развития науки вводится совсем другой некий элемент, который назван миром знания (третим миром), и как он вписывается в разные эпохи (его структура в разные эпохи).
Мир знания очень широк и включает в себя не только науку, но и другие слои знания (явно или неявно), которые в обществе существуют.

\subsection{Концепция исследовательских программ Имре Лакатос}

\insertlectureslide[0]{7}{04}

Следующий представитель постпозитивизма -- Имре Лакатос.

Выступает против метода фальсификации, поскольку этот метод не позволяет обосновать истинное знание и построить более-менее фундаментальные теории, поэтому его основная идея -- как всё-таки развивается наука и что обеспечивает устойчивость научного знания?
Эта его идея выражена в понятии исследовательская программа.
В данном случае исследовательская программа предполагает комплекс взаимодействующих и развивающихся теорий.
Не просто программа, а именно программа для развития теорий, которая имеет определённую структуру.

Гипотезы предохранительного пояса могут быть взяты чисто интуитивно, но формулируются именно для того, чтобы сохранить исходную позицию ядра.
Пример: работы Уильяма Гарвина в XVII веке; он известен как автор теории кровообращения в человеческом теле; в то время, когда он работал, уже обнаружили, что кровь в венах и кровь в артериях имеет разный состав, но не могли объяснить, как это происходит; Гарвин, чтобы создать теорию на основании естественного объяснения (без привлечения божественных сил) выдвигает идею, что есть тоненькие сосуды (капилляры), которые соединяют эти два круга; никто не поверил; но эта идея была выдвинута именно как защитная гипотеза теории кровообращения (потому что круг кровообращения тогда ещё не наблюдали); как говорится, угадал.

Негативная и позитивная эвристики предполагают некие исследования или поиск таких факторов, которые либо опровергают, либо поддерживают эту программу (ядро).

"<Логика открытия"> предполагает ряд правил, согласно которым можно сформулировать новые идеи и новые теории.
Именно "<логика открытия"> связана с понятием эвристики.

\subsection{Эвристика}

\insertlectureslide[0]{8}{04}

Эвристическую логику открытия Лакатос формулирует против (или в альтернативу) метода проб и ошибок.

Задача эвристики распространяется не только на действия людей (методологию научной деятельности), но и лежит в основании разработки машинных эвристических программ, которые построены на правдоподобных рассуждениях и развиваются в области интеллектуальных технологий.

\subsection{"<Историческая школа"> в постпозитивизме. Томас Кун}

\insertlectureslide[0]{9}{04}

Следующая позиция в постпозитивизме получила название "<историческая школа">.
Её основатель -- Томас Кун, который написал всем известную работу "<Структура научных революций">.

Томас Кун знаменит тем, что он вводит в оборот конкретно-исторический субъект познания, который мы сейчас знаем под термином "<научное сообщество">.
Это достаточно пионерское введение, поскольку до этого момента субъект познания ассоциировался с индивидом (одним человеком и его сознанием).
А здесь получается, что субъект познания уже абстрактный субъект, называемый "<научным сообществом">.
Из этого следует, что каждое научное сообщество принимает собственные стандарты рациональности.

Кун вводит понятия парадигмы и доктрины.
Парадигма как то основание, которое даёт стандарт научной рациональности в том или ином научном сообществе.

Не случайно Томас Кун относится к тому движению, которое критикует преемственность научного знания (согласно Куну история науки предстаёт как совокупность разобщённых и не понимающих друг друга научных сообществ).

\insertlectureslide[0]{10}{04}

Ещё одна характеристика рациональности в концепции Куна связана с понятием нормальная наука, то есть отличительным признаком науки в данном случае является не сама по себе какая-то рациональность, а некие признаки или совокупность признаков, которыми характеризуется нормальная наука.

Нормальная наука последовательно развивает интерпретации и методы исследования мира из какой-то одной признанной парадигмы.

Когда начинаем рассуждать о научном знании, то если мы берём систему Куна, то у него всё привязано к нормальной науке, которая фиксирует через парадигму некие знания в научном сообществе.
А всё остальное попадает во вненаучную рациональность.

Проводится граница между наукой и здравым смыслом (или обыденным знанием; повседневным рациональным действиям).

Переход от одной парадигмы к другой невозможен как последовательная плавная смена и наращивание знания.
А переход осуществляется через скачок (или как гештальтпереключение).
Поэтому периоды нормальной науки сменяются научными революциями, то есть процесс развития научного знания дискретный.

\insertlectureslide[0]{11}{04}

Возникает проблема: как же всё-таки формируется новая парадигма? На каком основании?
Этот вопрос остаётся открытым.
Хотя Томас Кун пишет, что в этой смене (переходе) участвуют не только чисто внутринаучные факторы, но и вненаучные (философские, эстетические, религиозные и вообще любые).

Плюрализм = разные точки зрения.

\subsection{Эпистемологический анархизм П. Фейерабенда}

\insertlectureslide[0]{12}{04}

На почве момента перехода в развитии научного знания (как развивается научное знание, если есть совершенно разные несовместимые парадигмы) развивает свою идею Пол Фейерабенд.
Его позиция получила название методологический анархизм (или эпистемологический анархизм).

Фейерабенд пытается противопоставить концепцию исторического релятивизма обычной концепции научной рациональности.
То есть у Фейерабенда нет концепции научной рациональности, а есть концепция исторического релятивизма, по которой стандарты рациональности меняются от эпохи к эпохе, от учёного к учёному, от одной научной школы к другой научной школе.

Фактически Фейерабенд очень ярко представляет ту модель развития научного знания, которая получила название антикумулятивная модель.

\insertlectureslide[0]{13}{04}

Чтобы решить проблему движения научного знания, Пол Фейерабенд выдвигает принцип размножения теорий.
Этот принцип призван обосновать плюрализм в методологии научного познания.

Появляется проблема отбора теорий (как из множества теорий выбрать; какой критерий отбора).
Таким образом, из тезиса несоизмеримости конкурирующих теорий невозможно построить общую методологическую концепцию.

\subsection{Философские концепции динамики науки}

\insertlectureslide[0]{14}{04}

К чему мы приходим в позитивизме и в постпозитивизме?
То, что говорил Лакатос, Поппер, то, что говорили их последующие современники, выливается в две тенденции (или в две модели) развития научного знания.
Проблема того, как развивается научное знание вообще в абстрактном виде (здесь нет речи о структурах или ещё чём-то).
Как возможно развитие знания в абстрактном виде в современной ситуации?
Кумулятивизм и антикумулятивизм.

\subsection{Проблема инноваций и преемственности в развитии науки. Джеральд Холтон}

\insertlectureslide[0]{15}{04}

Следующие идеи, которые связаны с развитием постпозитивизма в более поздние эпохи, как раз озабочены проблемой инноваций, поскольку проблема нового знания никуда не ушла.
Эту проблему пытается решить Холтон, который создаёт концепцию (идею) эволюции тематических структур.
Его работа называется "<Тематический анализ науки">.
Он считает, что исторически развивающаяся система (наука) может быть понятой (или объяснена), если мы найдём некие сквозные тематические структуры, которые в науке сохраняются и выступают объединяющим основанием самых разных теорий.

То есть идея Холтона о тематической структуре (как о своеобразной исторической траектории науки) обосновывается в самой истории научной мысли.

\subsection{Концепция тематических структур (траекторий) Дж. Холтон}

\insertlectureslide[0]{16}{04}

Третий аспект роста знания -- пространство жизни науки (широкий социокультурный контекст, выступающий в качестве среды, в которой живёт и развивается наука) -- тоже рассматривается/изучается в философии науки.

\insertlectureslide[0]{17}{04}

Примечательно то, что в тематическом подходе Холтона изменения и новации увязываются с преемственностью.
Это как раз была та проблема, которую остро обозначили Кун в революции и Фейерабенд.

Язык математики и логическая структура, которые были акцентированы в логическом позитивизме, составляют особую тематическую структуру в развитии науки.

\subsection{Эволюционистская концепция науки. Стивен Тулмин}

\insertlectureslide[0]{18}{04}

Ещё один представитель позднего постпозитивизма -- Стивен Тулмин -- формулирует ещё одну идею (модель) развития науки, которая получила название эволюционной теории познания (или эволюционной эпистемологии).
Здесь термин эволюционный указывает на то, что он накладывает на развитие научного знания (и науки как особого субъекта в системе социальных отношений) некие биологические эволюционные идеи.

\subsection{Эволюция концептуальных структур}

\insertlectureslide[0]{19}{04}

Тулмин выдвигает идею не тематических структур, а эволюцию концептуальных структур и описывает их в терминах популяций (мутаций и естественного отбора).

\subsection{Концепция личностного знания}

\insertlectureslide[0]{20}{04}

Ещё один известный представитель, который пытается решить проблему развития знания -- это Макс Полани.
Его идея связана с понятием неявного знания (оно не проговорено, не зафиксировано явным способом, но присутствует как фоновое знание).

\insertlectureslide[0]{21}{04}

В каком смысловом поле оказывается человек, пытающийся решать те или иные задачи?
6 (как минимум) векторов плюс методы рациональной деятельности.

Проблема выяснения структур неявного знания и влияния неявного знания на человека приводит к тому, что очень актуальным становится сама психология научной деятельности.
Психологические аспекты научной деятельности становятся очень актуальными (но не на первом плане).

\subsection{Психологические аспекты научной деятельности}

\insertlectureslide[0]{22}{04}

Далее небольшой экскурс в характеристику основных психологических идей.

Первое, что даёт нам XX век -- это попытки построить рационалистическую психологию, потому что сама психология личности оказалась очень размытой областью.

На слайде зафиксировано, какие формы и какие отдельные движения в этом направлении есть.

\insertlectureslide[0]{23}{04}

Основные попытки построить рационалистические концепции в самой психологии как раз связаны с тем, чтобы рационально представить структуру неявной базы или неявного знания, если брать личностные аспекты.

Здесь мы берём наиболее популярную в XX веке систему Фрейда.
Это одна из первых попыток построить именно рациональную психологию (подобно естественно-научным структурам).

Фрейд выводит структуру из трёх составляющих: "<Сверх-Я">, "<Оно">, "<Я">.
Показывает, что сама суть деятельности психики состоит из конфликта между принципом удовольствия (бессознательное) и принципом реальности (предсознательное).

Но он вводит ещё одну структуру (так называемое "<Сверх-Я">) -- внутренний центр.
Фактически это идентификация индивидуума с обществом (с нормами) через некий идеал, соответствующий тем или иным нормам культуры.
Некий внутренний центр, который диктует, что можно, а что нельзя.

Поэтому социальные чувства и социальные основания действий ориентируются (или опираются) именно на эту структуру психики.

\insertlectureslide[0]{24}{04}

Помимо внутриличностного конфликта одна из его идей -- это механизмы защиты от фрустраций.
Здесь он выдвигает свой самый главный метод психологической защиты -- это рационализация.

Половина выпускных клапанов негативной энергии (проекция, вытеснение, подавление и так далее) идут через самосознание.
\\

\insertlectureslide[0]{25}{04}

Его ученик и потом оппонент -- это Густав Юнг.
О создаёт именно аналитическую психологию, где как раз показывает, что источники психической активности (в том числе и познавательной) находятся не внутри человека (не в его инстинктивном начале), а в социуме.

Юнг рассматривает 4 системы психики, которые привязаны к тому, что получило название коллективное бессознательное, которое соотносится с тем основанием личностной бессознательной структуры, которую он называет архетипы.

Личное бессознательные опирается на коллективное бессознательное, которое человек воспринимает через символы (не просто через символику -- нарисованные картинки, а через смысл символов).

Эта идея Юнга столкнулась с физиологией Павлова (второй сигнальной системой), которая говорит о том, что человек реагирует не просто на знаки, а на смысл воспринимаемых знаков (и прежде всего словесный).

\insertlectureslide[0]{26}{04}

Ещё одна рационалистическая традиция в психологии получила название гуманистическая.

Здесь речь уже не идёт о сознательных/бессознательных структурах, а речь идёт о том, что человек неизлечимо социален.

Фактически он (и другие авторы тоже) пытается построить структуру психики личности, используя идею биосоциальности, скажем так.
То есть не чисто биологическая.
По Роджерсу психика человека глубоко социальная.
И соответсвенно поведение личности и мотивы (в том числе познавательная деятельность) определяются актуализацией и общением.

\insertlectureslide[0]{27}{04}

Последний ознакомительный аспект -- это когнитивная психология.
Когнитивная психология очень популярна.
Во-первых, это та часть психологии, которая формировалась под влиянием практических задач, взаимодействия человека с техническими системами.

На слайде перечислены когнитивные процессы, но надо сказать, что в данном случае само понятие когнитивного процесса сейчас более расширено, поскольку сюда попадает не только то, что делает человек (восприятие и так далее), но выходит уже за рамки человеческой деятельности (технические системы тоже имеют память, способны обучаться и так далее).

Сам когнитивный процесс рассматривается в более широком плане: не только как психологический процесс, но и как любой процесс, который оперирует системами знания.
В таком абстрактном виде это уже переводит эти все процессы в область интеллектуальных технологий.

\end{document}
