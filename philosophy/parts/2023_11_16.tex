\documentclass[main.tex]{subfiles}

\begin{document}

\linksection{Лекция 16.11.2023 (Шипунова О.Д.)}

\insertlectureslide[0]{1}{02}

На самом деле тема большая, потому что Античная эпоха и Новое время -- это более тысячелетия истории науки.
Точнее предыстории науки, поскольку современное понятие науки относится к Новому времени (Modern).

Но на самом деле Новое время и развитие науки связывается с первой научной революцией, когда оформляются истоки и стиль мышления, которые характеризуют классическую науку.
Её институты оформляются уже позже -- только в XVIII веке.

А предыстория у неё очень большая.

\sublinksection{Предпосылки философии науки в Античную эпоху и новое время}

\insertlectureslide{2}{02}

\sublinksection{Становление теоретической мысли в Античности}

\insertlectureslide[0]{3}{02}

Главное достижение Античной науки -- создание умозрительной системы познания на основе двух методов (метод сомнения и метод рассуждения).

\sublinksection{Проблема основания единства мира в Античной философии}

\insertlectureslide[0]{4}{02}

Как объяснить единство мира, найдя его скрытые основания?

Платон и Аристотель -- наследники Элейской школы.

Материалистическая традиция понимания единства мира противостоит пифагорейской математической традиции (вплоть до XVIII века).

Фалес -- основатель Милетской школы.

\sublinksection{Развитие представлений о первоначале в милетской школе}

\insertlectureslide{5}{02}

\sublinksection{Первые концепции Античной науки}

\insertlectureslide[0]{6}{02}

Первые концепции о строении космоса.

Одной из наиболее влиятельных учений в Античности -- это элементаризм.

Эмпедокл был одним из первых основателей школы красноречия.
Одним из первых высказал идею эволюции материи, в том числе человека.

Противоположная концепция -- атомистическое учение, развиваемое в Милетской школе.

Пустота воспринимается очень плохо (трудно вообразить).

\sublinksection{Физика Аристотеля}

\insertlectureslide[0]{7}{02}

Аристотель жил в III веке до нашей эры.
Его авторитет поддерживало то, что он был учителем Александра Македонского.
А сами греки не очень то и любили Аристотеля.
Аристотель был учеником Платона.
Аристотель был систематизатором знаний и создал некую классификацию наук.
Вводит несколько понятий, которые оказываются базовыми и для развития натурфилософии.

В эпоху Античности существовал парадокс движения: мы можем наблюдать движение, но мы его не можем мыслить.
Как только пытаемся мыслить движение, то получаем просто совокупность неподвижных точек в пространстве.
В этом ключе: Ахиллес не догонит черепаху, стрела не достигнет цели (бесконечное деление -- всегда остаётся промежуток).
Это решается в системе интегрального исчисления.

У Аристотеля: физика -- это наука о внешней причине.

\sublinksection{Вклад Аристотеля в философию науки}

\insertlectureslide{8}{02}

\sublinksection{Классификация наук Аристотеля}

\insertlectureslide[0]{9}{02}

Очень важна категория сущности, которая по Аристотелю, может связывать противоположности.

У Аристотеля наука о методах получения знания называется аналитикой.
А примерно через 100 лет эта система получает название логика.
Именно Аристотель формулирует 3 закона логики.

\sublinksection{Научное знание в эпоху поздней Античности}

\insertlectureslide[0]{10}{02}

Эпоха поздней Античности = эпоха Эллинизма.
Её начало связано с завоеваниями Александра Македонского.
В Египте создаётся новый город Александрия.
Там формируется центр медицинских, научных исследований, библиотека и так далее.

Символические фигуры: Евклид, Архимед, Птолемей, Аристарх.

\sublinksection{Вклад в теоретическое знание Средневековой науки}

\insertlectureslide[0]{11}{02}

Понятие пустоты наделяется статусом реальности, потому что Бог творит мир из ничего.

Ранее было сказано, что в Средние века (особенно в позднем Средневековье) складывается понятие об опытной науке и возникает идея (которая далее выражается в эмпиризме) о том, что любое знание нам даётся опытом.
А сам мир, который мы наблюдаем, это некий мир взаимодействий (о которых уже говорил и Аристотель) и движений (мы их можем рассчитать), но причина этих движений не материальна.

Важно, что в этой системе складывается представление о машине мира, о механике мира.
И это уже в Средние века, т.е. раньше Ньютона.

\sublinksection{Натурфилософские концепции о строении мира}

\insertlectureslide[0]{12}{02}

Натурфилософские концепции о строении мира включают в себя постепенно переход к Средневековью.

\insertlectureslide{13}{02}

\insertlectureslide{14}{02}

\sublinksection{Натурфилософия Платона}

\insertlectureslide[0]{15}{02}

Платон частично следует атомистическому учению и частично следует элементаризму.

\sublinksection{Постулаты геоцентризма в поздней Античности}

\insertlectureslide{16}{02}

\sublinksection{Натурфилософия Христианского запада}

\insertlectureslide{17}{02}

\sublinksection{Натурфилософия и наука в Средние века}

\insertlectureslide{18}{02}

\insertlectureslide{19}{02}

\insertlectureslide{20}{02}

\sublinksection{Восточная наука в Средние века (3-10 вв)}

\insertlectureslide{21}{02}

\sublinksection{Натурфилософия эпохи Возрождения}

\insertlectureslide{22}{02}

\sublinksection{Предпосылки научной революции на рубеже XVII века}

\insertlectureslide{23}{02}

\sublinksection{Галилео Галилей (1564-1642)}

\insertlectureslide{24}{02}

\sublinksection{Новый стиль познания природных явлений}

\insertlectureslide[0]{25}{02}

Не случайно Галилею приписывают такое выражение: "<Если факт не укладывается в мою теорию, тем хуже для факта"> (значит эксперимент плохо поставлен и не соотносится с мысленным идеальным экспериментом).

\sublinksection{Разработка математических методов описания физических движений}

\insertlectureslide[0]{26}{02}

Своей натурфилософией Декарт пытался отождествить математическое и физическое пространства через протяжённость (или непрерывность по Аристотелю).

\sublinksection{Натурфилософия И. Ньютона}

\insertlectureslide[0]{27}{02}

Натурфилософия Ньютона оказывается таким основанием последующего развития классической науки.

В своей работе (Математические начала натуральной философии) Ньютон (прежде чем давать какую-то схему расчёта) фиксирует исходные онтологические категории (пространство, время и материальные тела).

Существует три разные субстанции: пространство, время и материальные тела.
Существуют и не влияют друг на друга.
На этой базе формируется аппарат расчёта движений.

\sublinksection{Эмпиризм и рационализм нового времени}

\insertlectureslide[0]{28}{02}

Эмпиризм и рационализм Нового времени -- это своеобразная философия науки, которая сосредотачивается на поиске источника истинного знания и метода.

Эмпиризм -- из чувственных ощущений.

Рационализм -- из разума.

\sublinksection{Истинный метод науки в эмпиризме}

\insertlectureslide{29}{02}

\sublinksection{Метод индукции}

\insertlectureslide{30}{02}

\sublinksection{Истинный метод науки в рационализме}

\insertlectureslide{31}{02}

\sublinksection{Дедуктивный метод науки в рационализме}

\insertlectureslide{32}{02}

\sublinksection{Классическая наука}

\insertlectureslide[0]{33}{02}

Социальный статус науки как особой сферы деятельности и не просто сферы познавательной деятельности, но и сферы профессиональной и академической деятельности.

Основой для классической науки стало точное экспериментальное познание (исследование) природы.

\end{document}
