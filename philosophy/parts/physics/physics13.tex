\documentclass[exam_answers.tex]{subfiles}

\fontsize{14pt}{14pt}\selectfont

\begin{document}

\renewcommand{\baselinestretch}{\blch}
\sublinksectionold{\normalsize Эрвин Шрёдингер. "<Что такое жизнь с точки зрения физики?">}

Как могут физика и химия объяснить те явления в пространстве и времени, которые имеют место внутри живого организма?

Явная неспособность современной физики и химии объяснить такие явления совершенно не даёт никаких оснований сомневаться в том, что они могут быть объяснены этими науками.

Неспособность физики и химии объяснима, так как ранее не приходилось работать с такого рода объектами как живые существа. А всё известное о структуре живого вещества заставляет ожидать, что деятельность живого вещества нельзя свести к обычным законам физики. Дело не в "<новой силе">, а в том, что структура отличается от всего изученного до сих пор в физической лаборатории.

Шрёдингер задаётся вопросом, почему мы состоим из такого большого количества атомов? Акцентирует внимание, что работа организма требует точных физических законов и упорядоченности, которая возможна только при большом количестве атомов. Точность всех физических законов (например, в парамагнетизме или при изучении броуновского движения) основана на большом количестве участвующих атомов. Если бы живой организм состоял из небольшого количества атомов, то он не мог бы мыслить (так как мышление – это сложный упорядоченный процесс).

Шрёдингер приводит важное количественное положение, касающееся степени статистической неточности, которую надо ожидать в любом физическом законе (закон $\sqrt{n}$). Из этого закона тоже следует, что организм должен иметь сравнительно массивную структуру для того, чтобы протекающие в нём процессы подчинялись вполне точным законам.

Однако невероятно маленькие группы атомов, слишком малые, чтобы они могли проявлять точные статистические законы, играют главенствующую роль в весьма упорядоченных и закономерных явлениях внутри живого организма и управляют видимыми признаками большого масштаба (механизм наследственности).

И мутации (изменения), проведённые с этими группами атомов, могут привести к весьма значительным видимым изменениям признаков большого масштаба.

Однако наследственные черты передаются из поколения в поколение с удивительным постоянством, необъяснимым классической физикой. Квантовая теория объясняет это тем, что для перехода в другое состояние (квантовый скачок) должно быть затрачено достаточно большое количество энергии (Гейтлер-Лондоновское представление о связи).

Неживые системы довольно быстро переходят в состояние термодинамического равновесия или "<максимальной энтропии">.
Живые же организмы избегают перехода к равновесию (благодаря еде, питью, дыханию).
Живой организм непрерывно увеличивает свою энтропию (производит положительную энтропию) и приближается к опасному состоянию максимальной энтропии, которое представляет собой смерть.
Однако организм избегает этого состояния путём постоянного извлечения из окружающей среды отрицательной энтропии (в метаболизме организму удаётся освобождать себя от всей той энтропии, которую он вынужден производить, пока жив).

Высказывается мнение, что жизнь основана на чистом механизме, на принципе "<часового механизма">, на принципе "<порядка из порядка"> (а не на принципе "<порядка из беспорядка">).

\end{document}
