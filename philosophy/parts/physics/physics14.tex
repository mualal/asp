\documentclass[exam_answers.tex]{subfiles}

\fontsize{14pt}{14pt}\selectfont

\begin{document}

\renewcommand{\baselinestretch}{\blch}
\sublinksectionold{\normalsize Стивен Хокинг. "<Краткая история времени">}

Хокинг ставит множество вопросов о происхождении и развитии Вселенной, о природе пространства и времени, чёрных дырах.

Для ответа на эти вопросы принимаются некие картины мира. Однако ни одна из этих картин (теорий) не может претендовать на полную правильность.

Согласно доктрине научного детерминизма Лапласа должна существовать система законов, точно определяющих, как будет развиваться Вселенная, по её состоянию в один какой-нибудь момент времени.
Сейчас мы знаем, что это не так. В силу квантово-механического принципа неопределённости некоторые пары величин, например, положение частицы и её скорость, нельзя одновременно абсолютно точно предсказать.
Но даже если ставим задачу найти законы, которые позволили бы предсказывать события с точностью, допускаемой принципом неопределённости, то всё равно остаётся без ответа вопрос: как и почему производился выбор законов и начального состояния Вселенной?
Хокинг особо выделил законы, которым подчиняется гравитация, так как именно под действием гравитации формируется крупномасштабная структура Вселенной (при этом гравитационные силы самые слабые из существующих четырёх типов сил).

Утверждается, что законы гравитации несовместимы с точкой зрения, что Вселенная не изменяется со временем, так как из того, что гравитационные силы всегда являются силами притяжения, вытекает, что Вселенная должна либо расширяться, либо сжиматься.
Большой взрыв и большой хлопок – состояния с бесконечной плотностью.

Хокинг отмечает, что совсем не просто сразу строить полную единую теорию всего, что происходит во Вселенной, поэтому создаются частные теории, описывающие какую-то ограниченную область событий.
На вопрос, может ли единая теория реально существовать, Хокинг отвечает, что возможно 3 варианта:

1) полная единая теория действительно существует, и мы её когда-нибудь откроем, если постараемся;

2) окончательной теории Вселенной нет, а есть просто бесконечная последовательность теорий, которые дают всё более и более точное описание Вселенной;

3) теории Вселенной не существует: события не могут быть предсказаны далее некоторого предела и происходят произвольным образом и беспорядочно.

Сейчас считается, что наиболее полная единая теория -- это теория струн.

Но даже если бы удалось открыть окончательную теорию Вселенной, мы никогда бы не могли бы быть уверенными в том, что найденная теория действительно верна, потому что никакую теорию нельзя доказать.

Также если нам действительно удастся открыть полную единую теорию, то это не будет означать, что мы сможем предсказывать события вообще, так как наши предсказательные возможности ограничены квантово-механическим принципом неопределённости и мы не умеем (если не считать очень простых случаев) находить точные решения уравнений, описывающих теорию.
Например, мы не в состоянии точно решить даже уравнения движения трёх тел в ньютоновской теории гравитации, а с ростом числа тел и усложнением теории трудности ещё больше увеличиваются.


\end{document}
