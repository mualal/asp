\documentclass[exam_answers.tex]{subfiles}

\fontsize{14pt}{14pt}\selectfont

\begin{document}

\renewcommand{\baselinestretch}{\blch}
\sublinksectionold{\normalsize Фридрих Энгельс. "<Диалектика природы">}

Вторая половина XIX века ознаменовалась бурным развитием наук о природе.
Философская обработка открытий естествознания принадлежит Фридриху Энгельсу.

В "<Диалектике природы"> Энгельс решает очень большую общую задачу -- истолковать диалектику Гегеля материалистически и применить её для самых разных сфер.
В "<Диалектике природы"> дано диалектически-материалистическое обобщение важнейших достижений естественных наук середины XIX века.

Энгельс стремился, анализируя достижения и проблемы современных ему наук о природе, показать, что диалектические закономерности столь же типичны для природы, как и для общества.

В "<Диалектике природы"> Энгельс обосновывает мысль о том, что развитие естественных наук, начиная с эпохи Возрождения, шло таким путем, что к середине XIX века наука сама собой, не осознавая этого, подходит к диалектическому пониманию природы.

Свидетельством и доказательством этого Фридрих Энгельс считает 3 великих открытия в естественных науках XIX века:

1) открытие органической клетки;

2) закон сохранения и превращения энергии;

3) эволюционную теорию Ч. Дарвина.

Исходя из двоякой области применения диалектики, а именно природы и человеческой истории, Фридрих Энгельс делает важный вывод относительно человеческого мышления и познания.
По его мнению, великое открытие Гегеля состояло в том, что мир является совокупностью не готовых вещей, а процессов.
И как для природы, так и для истории справедливо, что они являются процессами или совокупностью процессов.
Из этого следует, что человеческое познание как зеркало этой двоякой действительности само является процессом, не достигающим и не могущим достичь неизменной и абсолютной истины.

Энгельс нападает на представление об "<абсолютных истинах">.
Он считает необходимым допустить существование истин, в которых нельзя сомневаться, не навлекая подозрение в сумасшествии, к примеру, что Париж находится во Франции или что человек, который ничего не ест, умирает от голода.

Энгельс попытался классифицировать формы движения материи (и, соответственно, науки, изучающие определённые формы движения).
Энгельс высказал гипотезу общей связи и развития материального мира и попытался нарисовать схематичный эскиз общей картины природы.
Здесь Энгельс использовал испытанный диалектический метод построения подобных гипотез -- движение от низшего к высшему, где всякая низшая форма посредством "<скачка"> преобразуется в высшую.
В результате получается иерархическая система, в которой всякое "<высшее"> содержит в себе, как подчиненный и частный момент, "<низшее">, но к нему уже не сводится.

Высшая форма движения материи, по Энгельсу, – мышление.
Низшая -- простое пространственное перемещение.
Каждая из форм движения исследуется определенной естественной наукой -- механическая, физическая, химическая и биологическая.
Переход от биологической формы движения материи к социальной, то есть от живой природы к человеческому обществу, Энгельс объясняет в трудовой теории происхождения человека.

"<Диалектика природы"> является лучшим свидетельством того, как знание законов материалистической диалектики, соединённое с глубоким знанием специальных областей науки, позволяет находить правильное решение принципиальных (философских, методологических) вопросов, выдвигаемых естествознанием.



\end{document}
