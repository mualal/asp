\documentclass[exam_answers.tex]{subfiles}

\fontsize{14pt}{14pt}\selectfont

\begin{document}

\renewcommand{\baselinestretch}{\blch}
\sublinksectionold{\normalsize Пол Девис. "<Поиск единой теории природы">}

Пол Девис рассказывает о перспективах объединения четырёх фундаментальных взаимодействий в природе (гравитационного, электромагнитного, слабого ядерного и сильного ядерного) в рамках одной суперсилы с помощью рассмотрения двух диаметрально противоположных разделов современной физики: c одной стороны -- физики микромира, и с другой стороны -- космологии.

Успешное описание электричества и магнетизма в рамках электромагнитного поля (в работах Фарадея и Максвелла) делает заманчивым распространить процесс объединения связав электромагнитное поле с другими силовыми полями (например, гравитационным), что может привести к необыкновенным результатам.
Также и Эйнштейн мечтал о создании единой теории поля, в которой все силы природы сливались бы воедино.
Создание универсальной всеобъемлющей теории тесно связано с вопросами происхождения Вселенной, нашем местом в пространстве и во времени, составом материи и характером взаимодействия частиц материи.
Иногда при описании взаимодействий частиц материи недостаточно экстраполяции привычных представлений и возникает необходимость вводить абстрактные, лишённые всякой наглядности понятия, допускающие только математическое описание.
Под влиянием квантовой физики и теории относительности на наши традиционные представления о пространстве и времени мир приобрёл неопределённость и субъективность, противоречащие его повседневной реальности.

 Исторической вехой на пути к созданию универсальной теории (суперсилы) было объединение электромагнитного и слабого взаимодействий в 60-е годы XX века.
 А в 1973г. была опубликована первая теория Великого объединения, в которой слабое взаимодействие сливалось с сильным в единое взаимодействие.
 Таким образом, уже в 70-е годы XX века существовала теория, объединяющая три из четырёх фундаментальных взаимодействий.
 Включение эту теорию гравитации оказалось ещё сложнее поскольку гравитационное взаимодействие имеет заметные отличия от других взаимодействий и при её квантовом описании возникают серьёзные трудности. Одним из способов преодоление этих трудностей явилось введении дополнительных "<ненаблюдаемых"> пространств (измерений).
 
Единая теория поля, построенная на основе геометрии (введении дополнительных измерений) рассматривает гравитацию, ка проявление структуры пространство-время. В соответствии с этой теорией гравитация обусловлена кривизной четырёхмерного пространства-времени, тогда как остальные силы обусловлены кривизной пространства другой размерности.

Ещё одной из универсальных (всеобъемлющих) теорией является теория суперструн, которой некоторые физики предсказывают ведущую роль в будущем.
Однако на данный момент математические проблемы, связанные с переходом от струн в десяти измерениях к свойствам частиц в четырёх измерениях, кажутся непреодолимыми.
Важным моментом универсальной теории является объяснение физического механизма процесса рождения Вселенной, поскольку с точки зрения физики внезапное возникновения Вселенной в результате гигантского взрыва представляется парадоксальным.  Из четырёх управляющих миром взаимодействий только гравитация проявляется в космическом масштабе, и она имеет характер притяжения.
Однако для большого взрыва и последующего расширения нужна сила отталкивания огромной величины.
Таким образом, хотя на данный момент не создано универсальной всеобъемлющей теории (объединяющей все четыре типа фундаментальных взаимодействий), но автор уверен, что такая теория будет создана и "<не один физический объект, ни одна систем не выпадут из сферы воздействия небольшого числа фундаментальных принципов">.



\end{document}
