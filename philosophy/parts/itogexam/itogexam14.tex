\documentclass[exam_answers.tex]{subfiles}

\fontsize{14pt}{14pt}\selectfont

\begin{document}

\renewcommand{\baselinestretch}{\blch}
\sublinksectionold{\normalsize (14) Эволюционная биология -- проблема естественного отбора и механизмов биоэволюции.}

Теоретическая разработка принципа естественного отбора и его форм в
эволюции органического мира связана с именем российского ученого Ивана
Ивановича Шмальгаузена (1884-1963), автора концепции стабилизирующего
отбора, применившим принцип действия авторегулируемых систем в
объяснении эволюционных процессов, в частности, в интеграции
закономерностей формообразования (морфогенеза) в индивидуальном развитии
(онтогенезе) и истории вида (филогенезе). Что положило начало новому
междисциплинарному направлению естествознания – биокибернетике (1967).

И.И.Шмальгаузен исходил из того, что естественный отбор меняет свои
формы и направление в зависимости от условий и форм борьбы за
существование. Рассматривая динамику исторической изменяемости
популяций, он выделил три основные формы естественного отбора: 1)
положительный (ведущий) отбор в направлении нарастания, усложнения
признака; 2) отрицательный (ведущий) отбор в направлении упрощения или
уменьшения признака; 3) нейтральный (стабилизирующий) отбор,
поддерживающий установившуюся нормальную величину и строение признака.
В результате происходит стабилизация формообразования, которая выражается
в развитии регуляторных механизмов, обеспечивающих независимость
индивидуального развития организма (онтогенеза).

Стабилизирующий отбор ведет к повышению устойчивости
существующей или устанавливающейся нормы формообразования (через
устранение случайных отклонений). В результате возникает регуляторный
аппарат, защищающий нормальное формообразование от возможных
нарушений со стороны случайных уклонений в факторах внешней среды, а
также со стороны небольших уклонений во внутренних факторах.
Эволюционная роль стабилизирующего отбора – охрана нормы. В
процессе эволюции наиболее существенные адаптивные нормы в известной
мере стабилизируются благодаря развитию регуляторных механизмов,
защищающих эти приспособительные реакции от возможных нарушений со
стороны случайных внешних влияний. Эволюция организмов, живущих в
меняющихся условиях, не ограничивается выработкой одной нормы.
Преимущества в борьбе за существование будут на стороне некоторой средней
нормы в типичных условиях. Но в других реальных условиях преимущества
будут на стороне тех или иных уклонений от этой главной нормы. Например, у
растений-амфибий вырабатываются две-три адаптивные нормы для жизни в
воде, на болоте, на суше. Они имеют целостный характер и осуществляются
при посредстве внутреннего авторегуляторного механизма развития. Благодаря
такому механизму стабилизирующего отбора, популяция сохраняет свой
нормальный фенотип, несмотря на непрерывное накопление мутаций и,
следовательно, непрерывную перестройку генотипа.

Во второй половине XX века в эволюционной биологии развиваются два
подхода: генетический и эпигенетический. Первый характерен для
синтетической теории эволюции неодарвинизма, согласно которой под
действием естественного отбора происходит процесс изменения частот генов в
генофонде популяции. На уровне генотипов особей накапливаются полезные
наследственные уклонения, определяющие фенотипические признаки и
адаптивные возможности. В неодарвинизме развивается классическая
популяционно-генетическая модель биологической эволюции, в которой
видообразующую роль играет мутация в структурных генах. Положение:
приобретенные признаки не наследуются, - составляет базовую аксиому
генетического подхода. Однако в классическую схему не укладывается факт
независимости морфологической эволюции (макроэволюции) от эволюции
структурных генов (микроэволюции), выявленный современными
исследованиями.

Эпигенетический подход к биологической эволюции выделяет роль
негенетических информационных потоков в жизни организма и вида.
Микроэволюционный процесс обусловливает постепенное изменение и
выживание наиболее приспособленных вследствие преимущества в данных
условиях. Макроэволюционный процесс связан с изменениями системными,
возникающими, например, в результате комбинирования комплексов свойств.
Наличие гетерогенной (генетической и эпигенетической) информации
отмечается исследователями в области генетики, экологии, эмбриологии.
Несовпадение информационных потоков обусловлено связью между
структурными генами и регуляцией их количества и продукта, то есть
процессом самоорганизации, в котором возникает дополнительная информация,
не закодированная в геноме и не поступившая из окружающей среды, а
обязанная своим происхождением пространственной организации в процессе
морфогенеза. Динамической единицей памяти (в отличие от генетической
единицы) выступает «эпиген». Благодаря наличию и несовпадению 
разнопорядковой информации, на популяционном уровне фенотипическая
однородность противопоставляется генотипической неоднородности.

Генетический подход и эпигенетический имеют разные сферы приложения
в вопросе об отборе и эволюции. Первый представляет наиболее общую
познавательную стратегию исследования, согласно которой материалом
эволюции служит неопределенная изменчивость, включающая как генетически
обусловленные нормы реакции фенотипов, так и вариации фенотипа,
обусловленные различием условий жизни. При этом расширение нормы
реакции сначала выражается в возникновении спектра возможных состояний
фенотипа. Второй обращен к закономерности развития (вида и особи), которая
определяется целым. В контексте этого подхода развивается идея
эпигенетической эволюции. Основанием служит факт автономности
индивидуального развития (онтогенеза) от генотипа, выделенный в
исследованиях И.И.Шмальгаузена. В эволюции фенотипа, как такового,
видообразующее значение приобретает феномен преадаптации, в котором
критерий выживаемости вида определен широтой нормы реакции.
Эволюционное развитие связывается с изменением нормы реакции
(биохимической, инстинктивно-физиологической), несущей главную
эпигенетическую информацию.

В учении о макроэволюции выделяется эволюционная роль формирования
потенциальной нормы, определяющей некий коридор индивидуальных
возможностей самоорганизации в конкретных условиях жизни. Этот
нормативный коридор обеспечивает различие индивидуальных норм реакции
на условия. Благодаря чему, вид способен переносить катастрофические
условия, ставящие на грань вымирания среднестатистическую массу особей.
При этом в благоприятной среде, когда действует стабилизирующий отбор,
особи с более широкой нормой реакции выглядят ненужным излишеством
природы и подавляются, не имея преимущества.

В конце века на основе принципа системности формулируется
недарвиновская концепция адаптивной эволюции, согласно которой
формообразующим фактором выступают длительные модификации,
возникающие в органических формах под стрессовым давлением среды с
последующим закреплением на генетическом уровне за 5-7 и более поколений.
В механизме наследования таких модификаций определенную роль играют
ретровирусы и мобильные генетические элементы, участвующие в переносе
генетической и эпигенетической информации. Линию эволюции представляют
как экспансию прогрессивных фрагментов вещества, которая сопровождается
конкуренцией материальных структур. Самосборка нового, более сложного
фрагмента осуществляется по принципу минимакса (максимизация функции
полезности при одновременной минимизации функции затрат). Вектор
прогрессивной эволюции конкретизируется («тематизируется» - С.Д.Хайтун)
давлением всей системы взаимодействий, включая давление среды. Например,
в водной среде будет иное направление самосборки, чем на суше.

Эволюционное усложнение происходит в результате наращивания все новых
этажей структурности материи (не только вещества, но и полей
взаимодействия), причем новые «этажи» не отменяют старые. Так, в
современном мире органические структуры не отменили неорганические, а
социальные – органические.

Традиционный для дарвинизма подход связан с выяснением
происхождения видов на основе субвидовых подразделений (популяций) по
принципу «снизу вверх» - от событий, происходящих с особью или вообще с
элементами, к их совокупности. Опыт биологических исследований в XXв.
привел к убеждению, что понять целое можно только через знание
существенных для целого свойств его частей.
 Объяснительные возможности
новых теорий в биологии должны быть связаны с историей происхождения
специфических качеств и функций высших организмов. В контексте
макроэволюции исходную базу такого исторического исследования составляет
фактический материал взаимодействий в геосферно-биосферной системе.
Выживаемость вида в этой системе определена возможностью потенциального
приспособления, которое получило название преадаптации.

Если адаптация – это способность к саморегуляции в соответствии с
наличными условиями, которая обычно выражается в увеличении численности
вида, то преадаптация предполагает не просто умножение числа особей в
соответствии со средой существования, а способность переносить
катастрофические условия, которые могут оказаться длительными. В
привычных условиях принцип «минимакса» действует как стабилизирующий
отбор (на основе конкуренции), оставляя случайным образом минимум особей,
потенциально способных к преадаптации. В катастрофических условиях
именно этот минимум спасает вид от вымирания. При этом функциональные (и
морфологические) изменения, поддерживающие способность к преадаптации
(существованию на грани вымирания) противоречат актуальной
приспособленности в рамках механизма конкуренции, поскольку предполагают
расходование энергии на «ненужные» вещи. Отрицание узкой утилитарности,
формирование «излишков» (или запасов?) в структурно-функциональном плане
выступает системной макроэволюционной закономерностью, которая не имеет
явного выражения в актуальной прагматической жизни вида, где преобладает
борьба, оставляя жизнь сильнейшим и наиболее приспособленным.

Недарвиновское направление в эволюционной биологии, в частности
учение о макроэволюции опирается на биосферную концепцию
В.И.Вернадского, в которой живое вещество рассматривается как
естественный компонент земной коры, наряду с минералами и горными
породами. Масса (вес), геохимическая энергия и химический состав живого
вещества в совокупности определяют интенсивность его важнейших
геологических функций (газовую, концентрационную, окислительновосстановительную, метаболическую).

Основные формы существования живого вещества, согласно Вернадскому,
представляют собой системные объекты:

пленки - в океане (например, планктонная и донная);

сгущения - в атмосфере, гидросфере и в пограничных областях (области
приливов и отливов, прибрежные морские и океанические территории, а также
озера, пруды, реки, грунтовые воды, болота, торфяники, леса, степи, луга);

разрежения - в атмосфере (воздушное пространство в горах), в гидросфере
(нижние слои некоторых морей, ледяные покровы) и в литосфере (пустыни
различных типов, ледники, пески, скалистые обнажения).

Разрежения разбросаны среди сгущений живой природы и
взаимодействуют с ними. Сгущения одного типа переходят в другие (лес/степь)
или происходит видоизменение сгущений (хвойный лес/лиственный лес).

Понятие живого вещества, введенное Вернадским, не отменяло
традиционную в биологии классификацию видов живой природы, а дополняло
ее новым системным содержанием. Если традиционная систематика
основывалась на единстве клеточной структуры живого, и строилась
структурно (начиная с одноклеточных), то у Вернадского систематизация
живого строится на биогеохимической основе, а его единство обеспечивается
обменными процессами в биосфере. Живое вещество проявляет себя на всех
уровнях биологической организации и в пределе охватывает всю живую
материю Земли. Введение новых функциональных систем в виде обменных
циклов (биогеоценозов), позволяло рассматривать биосферное единство в его
внутренних и внешних взаимосвязях.

Предметом исследования в естествознании становится биосфера как
целостная эволюционирующая и поддерживающая себя система, которая
характеризуется устойчивостью, взаимосвязью систем и состояний разного
уровня, качества и состава. Биосферное единство в его феноменальной
устойчивости характеризуют взаимодополняющие трофические (пищевые)
связи и круговорот живого вещества. 



\end{document}
