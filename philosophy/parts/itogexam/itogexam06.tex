\documentclass[exam_answers.tex]{subfiles}

\fontsize{14pt}{14pt}\selectfont

\begin{document}

\renewcommand{\baselinestretch}{\blch}
\sublinksectionold{\normalsize (6) Мировоззренческие и методологические принципы классического естествознания. Динамический детерминизм. Выявление границ механического объяснения на рубеже 20в.}

Мировоззренческие принципы точного экспериментального естествознания:

1) субстанциальная концепция пространства и времени.

Пространство, время и материя, состоящая из корпускул (т.е. имеющая
дискретную, атомарную природу), существуют как независимые, не влияющие
друг на друга субстанции.

Пространство понимается в абсолютном значении - как вместилище мира и в
относительном - как реальное трехмерное пространство, которое можно
измерить и представить формально (математически) в декартовых
координатах.

Время также понимается двояко: в абсолютном значении - как абсолютное
начало (чистая длительность) и в реальном значении - как течение событий.

Свойствами времени выступают: длительность, непрерывность, однородность
(время везде одинаково), необратимость (как однозначность и направленность
причинной связи).

žРеальное пространство и реальное время обладают определенной
размерностью.
\\

Методологические принципы точного экспериментального естествознания (позволяют объяснить и построить теорию):

1) принцип инвариантности законов природы;

2) принцип симметрии законов природы (равноправие всех точек и направлений);

3) универсальность принципа дальнодействия.
\\

Механистический детерминизм -- главный методологический принцип точного
экспериментального естествознания.

žВ классической форме механистический детерминизм был развит
французским ученым П. Лапласом.

Лапласовский детерминизм = Динамический (силовой) детерминизм.

žСамой сложной, неподдающейся формальным средствам классической
механики проблемой в физике XVIII столетия выступает объяснение
природы тепла и механизма теплопередачи.

\end{document}
