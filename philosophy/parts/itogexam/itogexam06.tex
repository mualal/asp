\documentclass[exam_answers.tex]{subfiles}

\fontsize{14pt}{14pt}\selectfont

\begin{document}

\renewcommand{\baselinestretch}{\blch}
\sublinksectionold{\normalsize (6) Мировоззренческие и методологические принципы классического естествознания. Динамический детерминизм. Выявление границ механического объяснения на рубеже 20в.}

Мировоззренческие принципы точного экспериментального естествознания:

1) субстанциальная концепция пространства и времени.

2) Пространство, время и материя, состоящая из корпускул (т.е. имеющая
дискретную, атомарную природу), существуют как независимые, не влияющие
друг на друга субстанции.

3) Пространство понимается в абсолютном значении - как вместилище мира и в
относительном - как реальное трехмерное пространство, которое можно
измерить и представить формально (математически) в декартовых
координатах.

4) Время также понимается двояко: в абсолютном значении - как абсолютное
начало (чистая длительность) и в реальном значении - как течение событий.

Свойствами времени выступают: длительность, непрерывность, однородность
(время везде одинаково), необратимость (как однозначность и направленность
причинной связи).

5) Реальное пространство и реальное время обладают определенной
размерностью.
\\

Методологические принципы точного экспериментального естествознания (позволяют объяснить и построить теорию):

1) принцип инвариантности законов природы;

2) принцип симметрии законов природы (равноправие всех точек и направлений);

3) универсальность принципа дальнодействия.
\\

Механистический детерминизм -- главный методологический принцип точного
экспериментального естествознания.

В классической форме механистический детерминизм был развит
французским ученым П. Лапласом.

Лапласовский детерминизм = Динамический (силовой) детерминизм.

Самой сложной, неподдающейся формальным средствам классической
механики проблемой в физике XVIII столетия выступает объяснение
природы тепла и механизма теплопередачи.

Границы механической (динамической) модели объяснения обозначились
в классической физике уже во второй половине 19в. В термодинамике при
исследовании поведения больших масс газа формируется представление о
статистических закономерностях. До сих пор физика оперировала только
понятием динамического закона, сформулированного в классической механике.
Исследования массы газа и жидкости в рамках молекулярно-кинетической
теории показали, что для совокупности частиц нельзя определить точное
движение одной частицы, но можно установить диапазон ее возможного
движения, который выражается законом распределения. Один из первых
статистических законов распределения молекул газа по скоростям получил
Джеймс Клерк Максвелл (1831-1867) для азота при температурах 20С и 500С. В 
дальнейшем представление о статистическом законе обобщается, под ним
понимается описание поведения большой массы частиц в целом.

Статистический закон в отличие от необходимости динамического
закона, только приписывает определенную вероятность каждому из возможных
видов случайного поведения частиц, составляющих большую массу.
Основания статистической термодинамики заложены в трудах
австрийского физика Людвига Больцмана (1844-1906), которому столь часто
приходилось отражать нападки со стороны противников молекулярно-кинетической теории, что одну из своих статей он завершил словами в
отношении молекул "<И все-таки они движутся">, перефразировав знаменитую
фразу Галилея. Сегодня не вызывает сомнений, что тепловая, внутренняя
энергия тела пропорциональна температуре, а температура, характеризующая
состояние движения частиц должна быть пропорциональна средней
кинетической энергии одной частицы. Коэффициент пропорциональности,
связывающий среднюю энергию одной частицы с температурой тела,
называется постоянной Больцмана (k=1,38*10\^{-23}Вт*сек/К). 

В соответствии с главным методологическим принципом
механистического детерминизма ведутся исследования в проблемной области
электромагнитных явлений. Важнейшее достижение физики XIXв. - создание
теории электромагнитных взаимодействий. Ее предпосылки связаны с именем
Анри Ампера, который свел все наблюдаемые электрические и магнитные
явления к единой причине – взаимодействию двух элементов тока, объяснив
эффект, обнаруженный Эрстедом (отклонение магнитной стрелки вблизи
проводника с током). В 1820г. на заседаниях Парижской Академии наук Андре
Мари Ампер прочитал серию докладов по электромагнетизму, где провел
различие между статическим электричеством, которое не влияет на магнитную
стрелку, и электричеством в движении, обозначил новый круг
электромагнитных явлений, ввел понятие электродинамических сил. Его
слушали молодые физики Био, Савара и семидесятилетний Лаплас. 

Установив связь между различными видами электричества и магнетизма,
Ампер выдвинул идею универсального механизма передачи электромагнитных
взаимодействий посредством поля, полагая, что в основе электрических и
магнитных явлений лежат не заряды и частицы, а пространство между ними.
Введение в систему науки понятия электромагнитного поля А.Эйнштейн
считал самым важным открытием со времен Ньютона.

Запись знаменитого опыта о возникновении электрической волны при
движении магнита появилась в дневнике Майкла Фарадея 17 октября 1831г. Так
было открыто явление электромагнитной индукции, а железное кольцо с двумя
обмотками стало прообразом будущих трансформаторов. Поставив обратную
задачу: получить ток из обыкновенного магнита и мотка проволоки, он создает
новый вид источников тока. Установив между полюсами большого магнита
Королевского общества вращающийся медный диск и соединив его
скользящими контактами с гальванометром, Фарадей получил источник
переменного тока, создав первую динамо-машину, или первый генератор
переменного электрического тока. Открытие Фарадея послужили началом
новой области – электротехники, которая и законами, и материалами сильно
отличалась от механики.

В 1834г. молодой профессор Петербургского университета Эмиль
Христианович Ленц после блестящих экспериментов сформулировал
обобщенный закон индукции, который в современном виде звучит так:
индуцированное напряжение равно скорости изменения магнитного потока.

Попытку обобщить опытные данные и создать математический фундамент
теории электромагнитных явлений в середине века предприняли сразу
несколько ученых: Франц Нейман, Густав Теодор Фехнер, Вильгельм Эдуард
Вебер. Но удалось это только Джеймсу Клерку Максвеллу.

В теории Максвелла представлена геометрическая модель электрических и
магнитных сил, учитывающая направление этих сил. Основными элементами
выступают не частицы или заряды, а напряженности магнитного и
электрического полей, которые представлены функциями четырех независимых
переменных: трех координат и времени. Теоретическое предсказание
Максвелла о распространении электромагнитных волн экспериментально было
подтверждено в 1888г. Генрихом Герцем, создавшим первый колебательный
контур с антенной. Максвелл ввел понятие тока смещения, равного
производной по времени от индукции электрического поля. Считая ток
смещения такой же реальностью, как и ток проводимости, Максвелл полагал,
что именно токи смещения создают магнитное поле.

Таким образом, к концу XIXв. физика оформляется как область
экспериментальных и теоретических исследований материальных процессов, в
основании которых лежат разнообразные взаимодействия: механические,
тепловые, гравитационные, электромагнитные. Было установлено, что
электромагнитное поле является носителем энергии. Связь между
электрическими и тепловыми явлениями демонстрировал закон Джоуля -
Ленца. Диапазон электромагнитных излучений охватывал видимый свет, а
также невидимые инфракрасные (тепловые), ультрафиолетовые и радиоволны.

Целью физической науки становится создание единой теории
наблюдаемых взаимодействий. При этом механические силы отходят на задний
план, фундаментальным становится понятие электродинамические силы.
Максвелл формулирует главную цель точной науки – "<свести проблемы
естествознания к определению величин при помощи действий над числами">.

Однако, несмотря на выдающиеся достижения, именно в области
электродинамики границы механического объяснения обозначились настолько
остро, что привели впоследствии к пересмотру универсальности динамического
закона и базовых понятий механической картины мира – материи, пространства
и времени.
Не укладывался в границы классического механического объяснения
феномен излучения, который проявлялся в исследованиях самым неожиданным
образом и привел к новым проблемам и открытиям.

Загадка атмосферного электричества (молнии) издавна занимала умы
ученых. В середине XVIIIв. складывается представление о молнии как
электрической искре огромных размеров. С тех пор газовые электрические
разряды становятся объектом физических исследований. Экспериментальное
изучение газоразрядных процессов в трубке, в частности катодных лучей,
природу которых не удавалось объяснить, привело к открытию электрона
(Дж.Дж. Томсон – 1897г.) и Х-лучей, обладающих сильной проникающей
способностью (К.Рентген – 1895г.).

В ходе изучения катодных лучей было осознано, что атомы не являются
неделимыми частицами материи, а имеют сложную структуру. Первоначально
Дж.Дж. Томсон назвал обнаруженную в катодных лучах частицу (в 1836 раз
легче водорода, несущую отрицательный заряд, равный заряду
электролитических ионов) корпускулой. Позднее ее стали называть электрон,
благодаря теоретическим исследованиям голландского физика Хендрика
Антона Лоренца (1853-1928) - создателя электродинамики движущихся сред.

Лоренц строил свою теорию, исходя из существования эфира -
заполняющей пространство неподвижной среды, в которой движутся атомы,
состоящие из элементарных электрических зарядов, при этом по его
предположению заряды могут существовать отдельно от атомов в виде
свободных электронов. Природу электрона Лоренц связывал с деформацией
эфира.

Лоренц показал, что ток проводимости, наблюдаемый при электролизе и в
газоразрядных процессах, не является самостоятельным, т.к. в его основе лежит
конвекционный ток (движение ионов в электролитах и электронов в металлах).
В электронной концепции Лоренца диэлектрические и магнитные свойства тел
сводились к поляризации и молекулярным токам, переставая быть первичными
характеристиками среды. Лоренц придал этим характеристикам статистический
характер, вычисляя их как статистически усредненные величины большого
числа электрических и магнитных дипольных моментов. На основе
представления о зарядах, движущихся в неподвижном эфире, была создана
электродинамика движущихся сред. Сила, действующая на элементарный заряд
(элементарную частицу) в электромагнитном поле, названа силой Лоренца.
Теоретические выкладки Лоренца при расчете движений частиц со скоростью,
сравнимой со скоростью света в эфире, вошли в современную физику под
названием "<преобразования Лоренца">.

Факт постоянства скорости света, с которой отождествляется скорость
распространения электромагнитных волн, противоречил классическому закону
сложения скоростей. В электродинамике Максвелла не выполнялся
классический принцип относительности. В обосновании теории Максвелла и
объяснении постоянства скорости света ученые опираются на гипотезу эфира
как светоносной субстанции (реального вещества) и абсолютной системы.
Отрицательный результат опыта Альберта Майкельсона (1852-1931),
поставленного с целью обнаружения эфирного ветра, стал той чертой, за
которой эфир перестал существовать в качестве физической реальности.
Гипотеза эфира была последней попыткой объяснить все происходящее в
природе на основе механики. 



\end{document}
