\documentclass[exam_answers.tex]{subfiles}

\fontsize{14pt}{14pt}\selectfont

\begin{document}

\renewcommand{\baselinestretch}{\blch}
\sublinksectionold{\normalsize (2) Онтологические проблемы естествознания.}

Онтологические проблемы группируются вокруг "<бытийственных"> вопросов: что и как существует (что представляет собой исследуемый объект)?
Для выбора конкретных научных исследовательских стратегий важны два аспекта онтологической проблематики.

Первый, собственно онтологический аспект определен вопросом о статусе
самой реальности, в которой укоренен объект исследования: с чем
исследователь имеет дело – с реальным физическим объектом, с феноменом,
идеальной конструкцией?

Объективная реальность - существующая независимо от наблюдателя,
регистрируемого явления, регистрирующего прибора, мышления. Этот уровень
реальности в истории естествознания соотносится с абстракцией физической
реальности. Выявление объективного статуса существования исследуемого
объекта придает ему статус реального физического объекта. Например, Солнце
существует объективно как звезда, атом - как мельчайшая частица химического
элемента, сохраняющая его свойства. В образовательной и научной практике
этот философский аспект естествознания представлен выявлением физического
смысла в операции определения понятия, которая далеко не проста. Построить
определение не всегда удается. Так, в математике проблемным остается вопрос:
что такое число, в физике - сила, поле. В последнем случае просто
демонстрируется феномен (как сила / поле действует).

Феноменологическая реальность - существование наблюдаемого или
регистрируемого явления, причины которого могут быть скрытыми и
неизвестными, а физический смысл не ясен. Например, солнечное затмение,
циклон, регистрируемое отклонение характеристик. Статус феномена относят
также к уникальному явлению, которое не повторяется и не укладывается ни в
какую закономерность.

Идеальный уровень реальности - существование мысленных конструкций
как самодостаточных и самостоятельных объектов соотносится с абстрактными
сущностями (например, число, множество - в математике). Абстракции
(абстрактные объекты) – важный инструмент научного познания природы. В
качестве примера можно привести такие понятия как идеальный газ, абсолютно
черное тело. Идеализация позволяет рассматривать виртуально разные статусы
исследуемого объекта в зависимости от угла зрения.

Абстрактный статус объектов в математических дисциплинах соотносится
с математической реальностью. Новые философские проблемы в
естествознании появляются с распространением представлений об
информационной реальности, виртуальной реальности.

Субъективный статус, который подчеркивает зависимость исследуемого
объекта от сознания наблюдателя, традиционно не рассматривается в области
естествознания. Зависимость научных построений от группового сознания,
научного сообщества, школы, исторического контекста эпохи, конвенции в той
или иной научной дисциплине указывает на коммуникативную природу
научного знания и научного обоснования.

Второй, структурно-функциональный аспект существования объекта
исследования связан с выяснением вопроса, что исследуется – структура,
функция, свойство, отношение? Определение онтологического статуса объекта
– важный момент осмысления исходных позиций и определения
познавательной стратегии и исследовательской программы.

Онтологический статус структура (или субстрат) может выступать в
разных вариантах: как целое или система, как часть или элемент, как
фундаментальная бесструктурная единица. Поиск такой единицы в истории
естествознания остается актуальной познавательной стратегий до сих пор,
определяя горизонт поиска в физике элементарных частиц. Заявлена эта
исследовательская установка в античной натурфилософии проблемой
первоначального элемента в строении мира (Милетская школа). Наиболее
конструктивной естественнонаучной гипотезой оказался атом Демокрита –
изначально неделимая единица бытия. В современной системе знания все
сложное многообразие структурной Вселенной сводится к фундаментальным
бесструктурным микрочастицам (в частности, кваркам и лептонам).

Статус свойства позволяет выделить качественные уровни исследуемого
через анализ:

- свойств единичного (элемента, структуры);

- свойств отношения (функциональные, информационные);

- свойств целого, или системы (системные свойства, структурнофункциональные).

Статус отношения указывает на взаимодействие, его виды (например, 4
вида фундаментальных физических взаимодействий, новые информационные
взаимодействия) и принципы (например, принцип дальнодействия и
близкодействия в физике, принцип обратной связи в кибернетике).
Философская категория "<отношение"> выделяет функциональный аспект в
существовании объекта, в котором принципиально важен характер
системообразующих связей:

- тотальная связность (сеть),

- направленная связь (эволюция),

- соотносительная (синхронизм, коэволюция),

- информационная, семантическая связь.

Функция в современной науке может рассматриваться в качестве
самостоятельного объекта (обратной связи, координации, отражения,
управления, корреляции, цикла), что выражается в формировании
представления о новом объекте науки - функциональной системе.

Круг онтологических вопросов задает проблематику философии
естествознания в конкретных дисциплинах. В области физики – это проблема
пространства и времени, проблема структурного строения и единства мира,
природа фундаментальных взаимодействий, природа и статус физического
закона, проблема существования и строения Вселенной.

В области химии круг онтологических вопросов связан с системностью и
сложностью химических объектов, существованием и самоорганизацией
химических систем.

В области биологии – с проблемой сущности и происхождения жизни,
существованием и происхождением генетической информационной системы
(генетического кода, ДНК, переходных структур), с проблемой эволюции
генетических систем и эволюции видов организмов, проблемой объяснения
психики.



\end{document}
