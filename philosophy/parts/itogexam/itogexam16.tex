\documentclass[exam_answers.tex]{subfiles}

\fontsize{14pt}{14pt}\selectfont

\begin{document}

\renewcommand{\baselinestretch}{\blch}
\sublinksectionold{\normalsize (16) Синергетическая парадигма: основные понятия и принципы. Теория самоорганизации.}

Предмет синергетики -- закономерности согласованного поведения систем различной природы.

Самоорганизация -- способность системы изменять свою структуру и функции в ответ на
внешние воздействия; возникновение упорядоченных структур и форм движения из
первоначально неупорядоченных, нерегулируемых форм без
специальных, упорядочивающих внешних воздействий.

Самоорганизующаяся система -- система, находящаяся в состоянии
постоянного обмена веществом, энергией и информацией с
окружающей средой, в относительно устойчивом равновесии.

Характеристики самоорганизующейся системы:

-- Способность активно взаимодействовать со средой, изменять ее в
своих целях

-- Гибкая структура, способность к адаптации в среде

-- Непредсказуемость поведения

-- Способность учитывать прошлый опыт

Адаптация -- принцип жизни самоорганизующейся системы.

Самоорганизующаяся система – это система открытая, адаптивная, когнитивная, прогнозирующая.



\end{document}
