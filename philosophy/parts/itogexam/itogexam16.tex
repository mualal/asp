\documentclass[exam_answers.tex]{subfiles}

\fontsize{14pt}{14pt}\selectfont

\begin{document}

\renewcommand{\baselinestretch}{\blch}
\sublinksectionold{\normalsize (16) Синергетическая парадигма: основные понятия и принципы. Теория самоорганизации.}

Предмет синергетики -- закономерности согласованного поведения систем различной природы.

Самоорганизация -- способность системы изменять свою структуру и функции в ответ на
внешние воздействия; возникновение упорядоченных структур и форм движения из
первоначально неупорядоченных, нерегулируемых форм без
специальных, упорядочивающих внешних воздействий.

Самоорганизующаяся система -- система, находящаяся в состоянии
постоянного обмена веществом, энергией и информацией с
окружающей средой, в относительно устойчивом равновесии.

Характеристики самоорганизующейся системы:

-- Способность активно взаимодействовать со средой, изменять ее в
своих целях

-- Гибкая структура, способность к адаптации в среде

-- Непредсказуемость поведения

-- Способность учитывать прошлый опыт

Адаптация -- принцип жизни самоорганизующейся системы.

Самоорганизующаяся система – это система открытая, адаптивная, когнитивная, прогнозирующая.

Начиная с 60-х гг. ХХ века внимание ученых различных отраслей
естествознания привлекают наблюдаемые, но не объясненные процессы
самоорганизации в сложных системах. Подобные процессы ученые
зафиксировали не только в живой природе, но также на уровне химическом и
физическом (в виде самопроизвольно возникающих структур и периодических
процессов - автоколебаний). Долго оставалась необъясненной химическая
реакция, открытая в 1951 г. советским химиком Б.П.Белоусовым, который
установил особые закономерности в автокаталитических химических реакциях:
строгую периодичность смены цвета в процессе определенной
окислительно-восстановительной реакции, которую можно было проверять по часам.
Периодичность изменения цвета, говорила о периодическом чередовании
промежуточных продуктов реакции. В 60-х гг. молодой биофизик
А.М.Жаботинский объяснил механизм реакции Белоусова, исследовав сходные
химические реакции. Периодичность возникновения промежуточных
продуктов химических реакций указывала на сходство протекания таких
химических реакций с автоколебаниями, характерными для различных
физических (механических, электромагнитных) систем и биологических
ритмов.

Теория автоколебательных процессов разрабатывалась в отечественной
науке в середине века школами академика Л.И.Мандельштама (1873-1944) и
академика А.А.Андронова (1901-1952). Развивая эту традицию, академик
Р.В.Хохлов (1926-1977) ввел в оборот понятие «автоволны», обозначавшее
особый род волн, автоматически поддерживающих свои физические параметры
за счет среды, в которой они распространяются.

Теория автоколебаний нашла применение в нейрофизиологии. В
частности, нервный импульс, который бежит без затухания по длинному (до 1,5
м) тонкому нервному волокну (диаметром менее 0,025 мм), представляет собой
пример автоволны. По такому же принципу работают сердце и головной мозг.
Обработка информации в коре головного мозга происходит на уровне
взаимодействия между автоволнами возбуждения и торможения, которые
охватывают обширные участки головного мозга. Работа сердца также
регулируется волной возбуждения, которая с периодичностью в секунду
распространяется по сердцу, вызывая сокращение сердечной мышцы. Волна
возбуждения связана с временным уменьшением разности электрических
потенциалов между наружной и внутренней сторонами мембраны сердечных
клеток, которая регистрируется на электрокардиограмме в виде периодического
всплеска.

В 60-х гг. выдвигается еще одна концепция самоорганизации в области
химии (А.П. Руденко), объясняющая способность катализаторов к
собственному структурному совершенствованию в ходе химической реакции.
Это оказывается возможным за счет энергии базовой химической реакции в
случае открытой системы. При своевременном отводе отработанной энергии и
усвоении свежей энергии базовой химической реакции каталитическая система
поэтапно совершенствуется (эволюционирует).

Исследуя поведение органических макромолекул на уровне неживых,
доклеточных структур, микробиолог М.Эйген установил закономерности
усложнения организации макромолекул на предбиологическом уровне, к
которым применимо понятие естественного отбора и применил термин
самоорганизации в описании наблюдаемых процессов.

Одной из предпосылок возникновения нового направления в исследовании
сложных систем, несомненно, послужили работы в области кибернетики, где
еще в 50-х гг. была поставлена задача создания самосовершенствующихся
автоматов. Найти решение тогда не удалось, но начало исследованию
проблемы самоорганизации в широком междисциплинарном контексте было
положено.

Ученые, работавшие в области кибернетики, были не только
математиками, но хорошо разбирались и в других областях естествознания и
техники. Исследуя диффузионные процессы, Н.Винер совместно с биологом
А.Розенблютом рассмотрел задачу о радиальном несимметричном
распределении концентрации в сфере. Английский математик А. Тьюринг
предложил модель структурообразования (морфогенеза) в виде системы двух
уравнений диффузии с дополнением, которое описывало реакции между
возникающими структурами («морфогенами»). А.Тьюринг показал, что в
реактивной диффузионной системе (обменивающейся со средой энергией)
может существовать неоднородное распределение концентраций, которое
периодически меняется в определенные промежутки времени. Непрерывная
модель самовоспроизведения автоматов Дж. фон Неймана также
основывалась на нелинейных дифференциальных уравнениях в частных
производных, описывающих диффузионные процессы в жидкости.

В области физики процессы самоорганизации сначала исследовались в
связи с изучением турбулентности и созданием новой лазерной техники. Союз
математиков и физиков в отечественной науке опирался на достижения первой
половины века в развитии математических методов нелинейной динамики
(А.М.Ляпунов, Н.Н.Боголюбов). К проблеме самоорганизации приводили
исследования неравновесных структур плазмы в термоядерном синтезе, 
разработка теории активных сред, биофизические исследования. В 60-х гг.
процессы самоорганизации исследовались в рамках отдельных дисциплин
(химии, биологии, физики), между которыми ученые не видели связей. В 60-70
гг. была создана теория турбулентности (А.Н.Колмгоров, Ю.Л.Климонтович).
За теорию генерации лазера группа ученых: Г.Б.Басов, А.М.Прохоров, Ч.Таунс,
- получила Нобелевскую премию.

В следующем десятилетии предметом анализа становится аналогия
процессов самоорганизации в системах различной природы. Шаг к
концептуальному обобщению в объяснении процессов самоорганизации был
сделан в начале 70-х гг. Группа бельгийских ученых во главе с И. Пригожиным
сопоставила реакцию Белоусова - Жаботинского с абстрактной моделью
самоорганизации английского математика и кибернетика А.Тьюринга и
выдвинула собственную теоретическую модель самоорганизации физических и
химических систем. Источник процесса самоорганизации И.Пригожин связал
со случайными неоднородностями (флуктуациями, микрочастицами,
микросредами), которые до некоторых пор гасятся силами внутренней инерции.
Нарастание случайных микрофлуктуаций ведет к состоянию внутреннего хаоса
в системе. Но когда в систему с хаотическим состоянием поступает достаточно
большое количество внешней энергии, то возникают определенные
макроскопические конфигурации (или моды), представляющие собой
коллективные формы поведения множества микрочастиц. Среди возникающих
мод происходит отбор наиболее устойчивых.

Следующий и самый решительный шаг в становлении общей науки о
самоорганизации сделал немецкий физик Герман Хакен (р.1927г.), выделивший
особое значение коллективных процессов в организации поведения всех
сложных систем. Общность и значение этих процессов для самоорганизации
сложной системы он и подчеркнул введенным термином «синергетика».
Синергия – с греческого переводится как согласование (συνεργέτυκός – греч.
совместный, согласованно действующий). В Штутгартском Институте
синергетики и теоретической физики Профессор Г.Хакен объединил усилия
большой международной группы ученых, создавших серию книг по
синергетике.

Исследуя согласованные процессы в различных физических и химических
системах, Г.Хакен подчеркнул фундаментальную роль коллективного
поведения подсистем в процессе самоорганизации – возникновении новой
устойчивой неравновесной структуры. Переход системы от неупорядоченного
(хаотичного) состояния к упорядоченному, по мнению Г.Хакена, происходит за
счет совместного, синхронного действия многих образующих ее элементов.

С этого времени синергетика ассоциируется с теорией совместного
действия и теорией самоорганизации. Под самоорганизацией понимается
возникновение упорядоченных структур и форм движения из первоначально
неупорядоченных, нерегулируемых форм без специальных, упорядочивающих
внешних воздействий.

Проблема самоорганизации, перехода от хаоса к порядку, которая
приобрела особую остроту в 80-х гг., до настоящего времени привлекает
внимание исследователей самых разных областей науки.

Теоретические и экспериментальные основания синергетики.

Новое направление в естествознании, возникшее в 80 – 90-е гг. XX в., и
получившее название синергетика, в качестве основного предмета
исследования выделила поиск общих закономерностей согласованного
поведения сложных систем различной природы.

Системный подход, ставший к этому времени традиционным, претерпевает
существенные изменения по сравнению с кибернетикой, исследующей
саморегуляцию в равновесных сохраняющихся системах на основе
отрицательной обратной связи. В новом направлении главный акцент ставится
на положительной обратной связи, выводящей систему из состояния
равновесия, и механизмах возникновения нового упорядоченного состояния. В
современной литературе синергетику часто определяют как науку о
самоорганизации в системах, далеких от равновесия. Такие системы
характеризуются нелинейностью (процессы в них описываются
математическими уравнениями второй и третьей степени), открытостью
(способностью за счет обмена энергией удерживать состояние вне
термодинамического равновесия).

В конце века синергетика как общая теория самоорганизации становится
популярным научным направлением, ориентированным на исследование связей
между структурными элементами, которые образуются в открытых системах
(биологических, физико-химических и др.) благодаря интенсивному обмену
веществом и энергией с окружающей средой в неравновесных условиях.
Особый понятийный аппарат синергетики разрабатывается на базе физической
химии и термодинамики, математической теории случайных процессов,
нелинейных колебаний и волн. В современной литературе синергетика
определяется как одна из фундаментальных теорий постнеклассической науки,
изучающая поведение сложных нелинейных систем.

Источниками синергетики - как общей теории самоорганизации,
изучающей единый алгоритм перехода от менее сложных и неупорядоченных
состояний к более сложным и упорядоченным - стали работы в области
математической теории катастроф (Р.Том, В.И.Арнольд), неравновесной
термодинамики (И. Пригожин), согласованных (когерентных) процессов в
физике (Г. Хакен).

Математическая теория катастроф, которая была сформулирована в 70-
х гг. XX в., по своему вилянию на умы сравнивалась с переворотом, вызванным
введением дифференциального исчисления. В три последних десятилетия века
теория катастроф с успехом применялась в естествознании, технике,
экономике, лингвистике, психологии, социологии. Наиболее эффективно - в
обосновании хлопков упругих конструкций, в теории опрокидывания кораблей. 

Основной предмет теории катастроф – ситуации, когда небольшие
постепенные изменения ведут к неожиданному резкому, непредсказуемому
поведению системы. Термин «катастрофа» связывается именно с такими
скачкообразными изменениями, возникающими при плавно меняющихся
параметрах. В теории катастроф разрабатываются методы факторного анализа.
Математические модели критических ситуаций, которые были построены на
этой основе, выявили зависимость поведения системы в критических ситуациях
от ее предыстории (это явление получило название «гистерезиса»). Факторный
анализ поведения системы, позволил также выявить основные факторы,
влияющие на неожиданно возникающий беспорядок в системе. В частности
применение теории катастроф в исследовании динамики поведения в тюрьме,
выявил в качестве таких факторов напряженность (вызванную чувством
безысходности) и разобщенность (взаимное отчуждение, разбиение на два
лагеря).

Теория катастроф выделила нелинейность в качестве фундаментальной
характеристики поведения сложной системы в критической ситуации, ввела в
оборот понятие бифуркации (bifurcus - лат. раздвоенный). Содержание этого
понятия в математике определено изменением числа (или устойчивости)
решений определенного типа для модели, описывающей систему при
изменении управляющих параметров. В точке бифуркации система имеет
разные ветви решений, и как бы совершает выбор, который определяет ее
дальнейшую эволюцию. Но этот выбор не является чисто субъективным, а
зависит от случайных, непредсказуемых факторов. Представители
естественных наук к термину «бифуркация» относятся осторожно, полагая, что
в физических, химических, биологических системах точек бифуркации не так
уж много. Типичным для естественных систем является устойчивое состояние и
устойчивое развитие.

Теория неравновесных процессов в термодинамике сформулирована
бельгийским ученым Ильей Романовичем Пригожиным (1917-2003),
Нобелевским лауреатом 1977 в области физической химии. И. Пригожин с
группой сотрудников исследовал процессы в незамкнутых системах,
обменивающихся с окружающей средой веществом и энергией. Его теория
сформулирована на экспериментальном материале исследования фазовых
переходов. Отправным пунктом в исследованиях Пригожина стала
чувствительность неравновесных фазовых переходов к конечным размерам
образца, форме границ и другим факторам, в отличие от обычных фазовых
переходов.

Само представление о равновесии сложной системы в физике конца века
претерпело изменение. С точки зрения молекулярно-кинетической теории в
замкнутой изолированной системе положению равновесия отвечает состояние с
высокой энтропией, равнозначное состоянию максимального хаоса (в смысле
броуновского движения частиц). Сложная система, двигаясь к так понимаемому
равновесию (состоянию с максимальной энтропией), не всегда его достигает из-за
ограничивающих условий, которые могут быть постоянными, а могут
изменяться. Если ограничения постоянны (например, определенная
температура на границах), то переменные состояния системы стремятся к
независимым от времени величинам, достигая квазистационарного или
стационарного состояния. Такие состояния сложной системы Л. фон
Берталанфи назвал текущим равновесием.

В сложной системе процессам, нарушающим текущее равновесие,
противостоит внутренняя релаксация (восстанавливающий, возвратный
процесс). Если возмущающие процессы менее интенсивны, чем
релаксационные, то говорят о локальном равновесии (существующем в малом
объеме), которое может возникать независимо от состояний других частей
системы. Идею локального равновесия И.Пригожин иллюстрировал на примере
газа, находящего между плоскостями, нагретыми до 100 С и 0 С. Поскольку
процесс теплопередачи происходит медленно, газ находится в неравновесном
состоянии, но где-то найдется малая область локального равновесия газа.

Равновесное и неравновесное состояние тел в термодинамике
характеризуется количеством энтропии. В 1947 г. И.Пригожин сформулировал
теорему о минимуме производства энтропии в стационарном состоянии (в
состоянии текущего равновесия), которое отвечает небольшим значениям
температурных градиентов. Если граничные условия не позволяют системе
прийти в устойчивое равновесие, в котором производство (прирост) энтропии
равно нулю, то система придет в состояние с минимальным производством
энтропии. Устойчивость стационарных состояний с минимальным
производством энтропии получила название устойчивого неравновесного
состояния. Эта идея Пригожина перекликалась с принципом Ле Шателье,
сформулированным в 1884 г.: если в системе, находящейся в равновесии
изменить один из факторов равновесия, то происходит реакция,
компенсирующая это изменения и возвращающая систему в состояние
равновесия. Способность возвращаться в исходное состояние – свойство
саморегулирующихся систем, которые в природе встречаются довольно часто.
Этот принцип известен в физике как принцип наименьшего действия, в
биологии – как закон выживания, в экономике – как закон спроса и
предложения. Общее для всех этих случаев состоит в том, что система
стремится выйти из преобразований с наименьшими потерями.

Принцип локального равновесия и теорема о минимуме производства
энтропии в стационарных состояниях были положены И.Пригожиным в основу
термодинамики необратимых процессов, которая, по его мысли, должна
преодолеть разрыв двух картин мира:

-- физической (структурной и стационарной, описывающей
обратимые процессы, происходящие в абстрактном геометрическом мире -
события предстают траекториями в неизменном трехмерном евклидовом
пространстве) и

- биологической (эволюционной, описывающей необратимые
процессы, происходящие в функциональном мире, локализованном во времени
и пространстве). 

\end{document}
