\documentclass[exam_answers.tex]{subfiles}

\fontsize{14pt}{14pt}\selectfont

\begin{document}

\renewcommand{\baselinestretch}{\blch}
\sublinksectionold{\normalsize (5) Первая система естествознания -- натурфилософия: познавательная установка, метод, круг проблем.}

Натурфилософия возникла в античном мире (VII в. до н.э.) как система знаний о естественных причинах природных явлений.
Исследовать мир в его единстве и многообразии, опираясь только на силу человеческого разума и разработку так называемых умозрительных методов познания природы.

От практических знаний, которые в те времена давала математика,
астрономия (астрология), знахарство, ее отличало умозрительное
толкование природы, в котором на основании положения о строении
мира подчеркивалось единство явлений природы и ее целостность.

Проблема структурных уровней материального мира, которую выявила
натурфилософия, характерна для фундаментальной науки и в наше
время.

Не различаются какие-то особые науки, а просто проводится попытка логически представить систему мира и его закономерности.

Существует до Исаака Ньютона и Декарта.

Если до сих пор картина мира формировалась в единой натурфилософии, то,
начиная с XVII-XVIII вв., научная и философская картины мира не совпадают.

Первая научная картина мира строится на основе классической
механики Ньютона и сложившегося в XVIII в. точного
экспериментального естествознания, которое вводит новые методы
получения знания.

Гюйгенс -> теория математического маятника -> создание часового механизма и морского хронометра -> век часов XVIII.



\end{document}
