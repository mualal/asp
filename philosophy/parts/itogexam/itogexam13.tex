\documentclass[exam_answers.tex]{subfiles}

\fontsize{14pt}{14pt}\selectfont

\begin{document}

\renewcommand{\baselinestretch}{\blch}
\sublinksectionold{\normalsize (13) Проблемы концептуального синтеза генетики и теории эволюции.}

Генетика, убедительно обосновала механизм саморепродукции живого
одноклеточного и многоклеточного организма, но проблема происхождения
гена осталась в стороне. Именно эта проблема создала общий
мировоззренческий фон противостояния дарвинизма и генетики в развитии
теоретической биологии.

В начале 20 века эволюционисты враждебно отнеслись к законам Менделя
и генетике. Одна из попыток представить схему эволюции на базе генетики,
предпринятая Лотси (1916), была неудачной, поскольку на тот момент доказать
самый факт возникновения новых генов в природе было невозможно. Опыт
селекции и гибридизации подводил к выводу, что наблюдаемые
наследственные изменения – результат искусственного вмешательства
(одомашнивание или лабораторный эксперимент по выращиванию). Порочный
круг, в который попадала генетика, был связан с тем, что для доказательства
факта природного наследственного изменения необходимо провести
генетический анализ в двух поколениях, т.е. создать искусственные условия.
Поэтому в первой четверти века утвердилось представление, что большинство
возникающих природных изменений является «уродством», которое не влияет
на эволюционный процесс. Мутации (геновариации) только портят сложный,
совершенный механизм адаптации. В жестокой борьбе за существование среди
нормальных особей «уродцы» должны гибнуть очень быстро, не оставляя
потомства. Эволюционный процесс превращения видов организмов в другие
виды не мог идти геновариационным путем, поскольку при всех геновариациях
(мутациях) муха остается мухой, а крыса остается крысой, не давая уклонения в
сторону собаки. Таким образом, генетика опровергала эволюционный принцип
видообразования. Противостояние генетики и дарвинизма обостряло и
философскую проблему происхождения и сущности жизни.

Расхождение эволюционизма и генетики проявилось в трактовке роли
естественного отбора в поступательном эволюционном процессе. В дарвинизме
естественный отбор – главный движущий фактор эволюции биологического
вида. В генетике сформировалась противоположная точка зрения.
В.Л.Иоганнсен и Т.Х.Морган, включая его учеников, исходили из
представления о неизменяемости генов и поддерживали точку зрения о
пассивной роли отбора в эволюции, полагая, что он выступает средством, лишь
устраняющим менее пригодные гены. Сами гены не изменяются и не зависят от
внешних условий, что подтверждалось опытами Иоганнсена (1913) и рядом
аналогичных опытов, которые показали невозможность изменения признака
путем отбора в генетически однородной среде.

Однако Морган, не подозревая о том, сам создал почву для иной точки
зрения в учении о множественном (плейотропном) действии генов. Согласно
этому учению, каждый ген может воздействовать не только на
соответствующий специфический признак, но и на ряд других и вообще на всю
сому. Сами гены качественно независимы друг от друга, но их проявления, т.е.
признаки, являются уже сложным результатом многообразного взаимодействия
всех генов, входящих в генотип. Такое представление меняло сложившуюся на
базе генетики традиционную мозаичную картину строения организма (из
отдельных независимых признаков). Каждый ген контролирует определенный
признак, но индивидуальное выражение этого признака зависит от всего
генотипа. Наследственная структура каждой клетки определяется комплексом
генов. Проявление признаков получало вероятностное, статистическое
толкование и было развито в концепции популяционной генетики
С.С.Четвериковым (1926).

Поставив перед собой задачу - соединить эволюционизм с аппаратом
генетики, С.С.Четвериков выделил три линии возможной конструктивной
взаимосвязи через: 1) анализ возникновения мутаций (геновариаций) в природе,
2) анализ влияния свободного скрещивания (согласно менделевским законам)
на генотип и изменчивость, 3) соотнесение этих факторов в жизни популяции с
эволюционной ролью естественного отбора.

Изучая генетический состав природных популяций плодовой мушки
(дрозофилы), Сергей Сергеевич Четвериков (1880-1959) показал, что даже
фенотипически однородная популяция неоднородна на уровне генетических
признаков. Четвериков подчеркнул независимость появления геновариаций от
искусственной обстановки исследования, поскольку до сих пор человек не
может влиять на частоту появления геновариаций, тем более вызывать
желаемые. Даже применение таких сильных воздействий, как ионизирующее
или рентгеновское облучение, алкоголь, эфир, ненормальное давление,
гибридизация не приводили пока к желаемым результатам.

В дополнение к наследственной, генотипической изменчивости
Четвериков ввел понятие геновариационной изменчивости, подразумевая
разнообразие признаков и амплитуду их отклонения в результате накапливания
мутаций в видовом сообществе. Генотипическая (наследственная)
изменчивость связана с возникновением новых признаков и происходит на
основе мутаций, которые могут быть соматическими (закрепляющими признак
только при бесполом размножении – митозе) и генеративными
(закрепляющими мутации клеток зародышевого пути при половом
размножении - мейозе).

Любая мутация вызывает целый спектр изменений. На мутационный
процесс влияет весь набор генов, содержащихся в генотипе, поэтому
мутационная (геновариационная) изменчивость носит всегда внутривидовой
характер: муха всегда остается мухой. Мутации в большей или меньшей
степени снижают адаптивные возможности организма, часто оказываются
летальными. Однако мутации не влияют на численность их несущих особей,
поэтому не исчезают, а передаются по наследству и таким образом
накапливаются за счет мутирования других генов. Мутация конкретного гена –
событие редкое, в среднем один на миллион, но в генотипе не менее 106
– 107
генов, а число особей в популяции от десятков до миллиардов.

Явление накапливания мутаций было названо Четвериковым
геновариационной изменчивостью. Возникающие мутации, как правило, не
являются доминантным (часто встречающимся) признаком. Свободное
скрещивание поглощает геновариации. Каждая вновь возникающая
рецессивная геновариация при скрещивании с нормальной формой как бы
растворяется в ней и не обнаруживается в морфологии организма. Подметить
геновариацию в естественных условиях можно только в самый момент
зарождения, пока она не уничтожена отбором.

В природе происходят два противоположных процесса: накапливание
геновариаций и их устранение. Эти процессы лежат в основании различия
геновариационной и генотипической изменчивости. Видовое сообщество
постоянно, подобно губке, впитывает все новые и новые геновариации,
оставаясь внешне однотипным. По мере накопления внутри вида большого
числа геновариаций, та или другая начинает обнаруживаться, тогда и внешне
вид начинает проявлять все большую генотипическую изменчивость. Чем
старее вид, тем он больше внешне изменчив. При равенстве прочих условий
генотипическая изменчивость растет пропорционально возрасту вида. 

Геновариационная изменчивость, согласно Четверикову, - основной путь
медленный эволюции органического мира. Резкие и глубокие изменения
организма возможны только путем длительного накопления геновариационных
изменений, продолжительного напластования одних отклонений на другие.

Развивая учение Моргана о множественном действии генов, Четвериков
ввел понятие «генотипической среды». Эволюционное значение
генотипической среды – наследственные колебания признаков. В комбинации с
одним генотипом данный признак, обусловленный одним геном, будет
выражен сильнее, в комбинации с другим – слабее. Учение о генотипической
среде объясняло непонятное различие между качественной и количественной
изменчивостью, а также открывало новые возможности в понимании
эволюционной роли и механизма естественного отбора.

В популяционной генетике активная роль естественного отбора
раскрывается посредством создания благоприятной генотипической среды.

Действие естественного отбора простирается на весь комплекс генов, на
всю генетическую среду, в обстановке которой данный ген себя проявляет. В
процессе естественного отбора косвенно определяется наиболее благоприятная
для проявлений данного признака генотипическая среда. Устраняя, таким
косвенным образом, неблагоприятные комбинации генов, отбор способствует
образованию благоприятной генотипической среды и ведет к усилению
признака.

Каждый ген действует не изолировано, он проявляет себя внутри генотипа
и в связи с ним. Каждый признак в своем выражении зависит от строения всего
генотипа, является реакцией на определенные внутренние взаимодействия.
Генетическая структура вида состоит из громадного числа более или менее
отличных другу от друга генотипов. Один и тот же ген в различных
генотипических комбинациях попадает в различную «генотипическую среду»,
следовательно, каждый раз его внешнее проявление будет наследственно
видоизменяться, его проявление будет наследственно колебаться,
наследственно «флуктуировать».

Генетический анализ эволюционного процесса опирается на принцип
множественного (плейотропного) действия генов. В основании закономерного
процесса эволюции лежит случайное появление геновариаций, поэтому
эволюционные закономерности имеют вероятностный характер и могут быть
описаны статистически на основании законов больших чисел.

Аппарат популяционной генетики позволил органично соединить
эволюционную концепцию дарвинизма с молекулярной биологией,
раскрывающей механизмы жизненных процессов в клетках, а также с
генетикой, выявившей материальные структуры, обеспечивающие передачу и
изменение наследственных признаков. Теоретическая система в биологии,
сложившаяся в 50-х гг. XXв., получила название «неодарвинизма», или
синтетической теории эволюции.

Большой вклад в становление синтетической теории эволюции внес
американский зоолог-систематик Эрнст Майр, работы которого посвящены
структуре вида, факторам и механизмам видообразования. Майр теоретически
совместил выводы зоогеографии, генетики и экологии, сформулировав
фундаментальное положение о принципиальном единстве микро- и
макроэволюции.

Работы российского ученого А.Н.Северцова послужили фундаментом, для
современного представления о механизмах эволюции. Опираясь на
экспериментальные исследования и современный информационный подход,
А.Н.Северцов выделили три механизма эволюции: генетический (на основе
рекомбинации структурных генов), селективный (на основе отбора),
эпигенетический (на основе динамических информационных потоков
негенетического характера).



\end{document}
