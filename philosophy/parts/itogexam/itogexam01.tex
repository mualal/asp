\documentclass[exam_answers.tex]{subfiles}

\fontsize{14pt}{14pt}\selectfont

\begin{document}

\renewcommand{\baselinestretch}{\blch}
\sublinksectionold{\normalsize (1) Предмет философии естествознания.}

 К XX веку чётко оформляются три сферы научного познания, которые отличаются целью, структурой и объектом исследования.
 Прежде всего объектом.
 Знание о человеке.
 Знание о природе.
 Знание об обществе.
 
 В естествознании объектом выступают именно природные явления.
 Физика -- на уровне неживой природы.
 Биология -- явления органического мира (в основании -- клетка).
 Геология -- мир природы, но в других масштабах.

Отличие естествознания -- формирование теоретических моделей объяснения природных явлений на основе выявления причинно-следственных связей.
Двоякая роль объяснения: как объяснение уже открытых закономерностей и как эвристическая процедура.

В предмете философии естествознания речь идёт об очень общих проблемах.
Общие проблемы, закономерности и тенденции развития корпуса естественных наук,
а также особого рода познавательной деятельности, главная цель которой -- истинное, объективное знание о законах природы.
\\

Задачи философии естествознания:

1) исследование способов формирования нового естественнонаучного знания, а также механизмов воздействия социокультурных факторов на этот процесс;

2) анализ структуры и динамики знания в конкретных естественнонаучных дисциплинах (физики, химии, биологии);

3) сравнение естественнонаучных дисциплин и выявление общих проблем и общих закономерностей в их развитии;

4) концептуальный анализ эволюции конкретной области естественнонаучного знания на материале истории конкретных наук (физики, химии, биологии);

5) формирование моделей развития естественнонаучного знания, проверка их на соответствующем историческом материале.
\\

Предмет философии естествознания:

1) онтологические (бытийственные) проблемы (единство мира и так далее);

2) теоретико-познавательные (гносеологические) проблемы (центральный вопрос гносеологической проблематики -- истинность и объективность естественнонаучного знания);

3) методологические проблемы (проблема метода исследования, адекватного современному уровню развития научного знания; научная картина мира; исторические типы научной рациональности).
\\

Ключевые понятия, указывающие на предмет философии естествознания:
онтологический статус объекта,
уровни реальности,
истинность и объективность естественнонаучного знания,
принцип единства мира,
принцип причинности,
детерминизм и его формы,
релятивизм (относительность), механизм, индетерминизм (отрицание наблюдаемых причинно-следственных связей), фатализм (предопределённость; судьба), финализм (описание поведения через стремление к некоторой конечной причине, которая и объясняет движение), холизм (=макродетерминизм; характер причинно-следственной связи, который определяется целым; системный характер связи; параметр порядка).


Современная философия трактует науку как социокультурный феномен,
специфика которого связана с производством и ростом объективного знания о
мире. В сложившейся системе науки разграничиваются сферы знания: о
человеке – о природе – об обществе. Отличие естествознания – формирование
теоретических моделей объяснения природных явлений на основе выявления
фундаментальных структур и причинно-следственных связей.

Предмет философии естествознания – общие проблемы, закономерности
и тенденции развития корпуса естественных наук, а также особого рода
познавательной деятельности, главная цель которой – истинное, объективное
знание о законах природы.

Философия естествознания имеет трансдисциплинарный
(сверхдисциплинарный) статус в системе наук, что обусловлено ее основными
функциями:

- анализом проблем формирования общей научной картины мира,
концептуальным согласованием теоретических моделей и научного метода с
мировоззрением культурно-исторической эпохи, с категориальным строем
научного и обыденного сознания (мировоззренческая функция);

- обоснованием постулатов, познавательных принципов и методов
естествознания (методологическая функция);

- выявлением междисциплинарных проблем, а также проблем конкретной
естественнонаучной дисциплины и естествознания в целом (эвристическая
функция);

- сохранением новых идей, не имеющих достаточного обоснования,
отвергнутых научным сообществом из-за жестких авторитетных оценок в
конкретно-исторических условиях (защитная функция);

- продвижением новых научных идей, распространением и
популяризацией построений науки в широких интеллектуальных культурных
слоях (коммуникативная функция).

В задачи философии естествознания входит:

- исследование способов формирования нового естественнонаучного
знания, а также механизмов воздействия социокультурных факторов на этот
процесс;

- анализ структуры и динамики знания в конкретных естественнонаучных
дисциплинах (физики, химии, биологии); 

- сравнение естественнонаучных дисциплин и выявление общих проблем и
общих закономерностей в их развитии;

- концептуальный анализ эволюции конкретной области
естественнонаучного знания на материале истории конкретных наук (физики,
химии, биологии);

- формирование моделей развития естественнонаучного знания, проверка
их на соответствующем историческом материале.

Философская проблематика естествознания в целом, а также конкретных
дисциплин наиболее явно представлена возникающими в той или иной области
научного познания вопросами, которые выходят за пределы признанных теорий
и методов решения конкретной дисциплины. Эвристическая роль философии
естествознания заключается как раз в осмыслении такого рода проблем и
познавательных ситуаций, требующих поиска новой исследовательской
стратегии и оценки ее перспектив. Традиционно выделяют три круга
философских проблем естествознания, которые связаны с онтологическими,
гносеологическими и методологическими аспектами научного познания.



\end{document}
