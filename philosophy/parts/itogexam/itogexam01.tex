\documentclass[exam_answers.tex]{subfiles}

\fontsize{14pt}{14pt}\selectfont

\begin{document}

\renewcommand{\baselinestretch}{\blch}
\sublinksectionold{\normalsize (1) Предмет философии естествознания.}

 К XX веку чётко оформляются три сферы научного познания, которые отличаются целью, структурой и объектом исследования.
 Прежде всего объектом.
 Знание о человеке.
 Знание о природе.
 Знание об обществе.
 
 В естествознании объектом выступают именно природные явления.
 Физика -- на уровне неживой природы.
 Биология -- явления органического мира (в основании -- клетка).
 Геология -- мир природы, но в других масштабах.

Отличие естествознания -- формирование теоретических моделей объяснения природных явлений на основе выявления причинно-следственных связей.
Двоякая роль объяснения: как объяснение уже открытых закономерностей и как эвристическая процедура.

В предмете философии естествознания речь идёт об очень общих проблемах.
Общие проблемы, закономерности и тенденции развития корпуса естественных наук,
а также особого рода познавательной деятельности, главная цель которой -- истинное, объективное знание о законах природы.
\\

Задачи философии естествознания:

1) исследование способов формирования нового естественнонаучного знания, а также механизмов воздействия социокультурных факторов на этот процесс;

2) анализ структуры и динамики знания в конкретных естественнонаучных дисциплинах (физики, химии, биологии);

3) сравнение естественнонаучных дисциплин и выявление общих проблем и общих закономерностей в их развитии;

4) концептуальный анализ эволюции конкретной области естественнонаучного знания на материале истории конкретных наук (физики, химии, биологии);

5) формирование моделей развития естественнонаучного знания, проверка их на соответствующем историческом материале.
\\

Предмет философии естествознания:

1) онтологические (бытийственные)проблемы (единство мира и так далее);

2) теоретико-познавательные (гносеологические) проблемы (центральный вопрос гносеологической проблематики -- истинность и объективность естественнонаучного знания);

3) методологические проблемы (проблема метода исследования, адекватного современному уровню развития научного знания; научная картина мира; исторические типы научной рациональности).
\\

Ключевые понятия, указывающие на предмет философии естествознания:
онтологический статус объекта,
уровни реальности,
истинность и объективность естественнонаучного знания,
принцип единства мира,
принцип причинности,
детерминизм и его формы,
релятивизм (относительность), механизм, индетерминизм (отрицание наблюдаемых причинно-следственных связей), фатализм (предопределённость; судьба), финализм (описание поведения через стремление к некоторой конечной причине, которая и объясняет движение), холизм (=макродетерминизм; характер причинно-следственной связи, который определяется целым; системный характер связи; параметр порядка).



\end{document}
