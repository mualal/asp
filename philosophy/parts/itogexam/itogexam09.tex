\documentclass[exam_answers.tex]{subfiles}

\fontsize{14pt}{14pt}\selectfont

\begin{document}

\renewcommand{\baselinestretch}{\blch}
\sublinksectionold{\normalsize (9) Методологические установки в создании теоретической физики. СТО и становление релятивистской физики.}

В конце XIX открываются и радиоактивное излучение, и разного рода лучи, ...
Формируется новая система естествознания -- теоретическое естествознание.

Теоретическая физика создаёт новые абстракции.

Эйнштейн, Пуанкаре и Минковский совершают переворот в научной картине мира.
Благодаря им формулируется новая концепция пространства-времени.
Третья научная революция связана не только с новыми методами, с новой системой теоретического естествознания, но и с новой картиной мира.
Фундаментальные понятия материи, пространства и времени допускают иную трактовку, чем классические.

СТО.
Предпосылкой и стимулом развития новой физической теории выступила теория Максвелла, породившая проблему реального носителя электромагнитных излучений.
До начала XX века в качестве такого носителя
признавали эфир (светоносное вещество, субстанция).
Опыт Майкельсона (идея была предложена Максвеллом) для обнаружения эфира.

В 1905 г. Альберт Эйнштейн публикует положения специальной теории
относительности, изменившей классические представления о пространстве и времени.
Главная мировоззренческая новация была связана с введением в
систему физического знания 4-х мерного континуума, в котором совершаются мировые события.


Идея о том, что время можно рассматривать как 4-е измерение, равноправное по
отношению к координатам, была выдвинута немецким математиком Германом
Минковским (1864-1909), который полагал, что время связано с пространством
функциональной зависимостью, не существует отдельно и не может рассматриваться
как самостоятельная сущность.

Право на самостоятельное существование, по мысли Минковского, получает только
"<определенная форма их совместного союза"> (пространства и времени).
Он предложил понятие мировой линии 4-х мерного пространства, которое стали называть
пространством Минковского.

В основании специальной теории относительности лежат два положения:

1) инвариантность физических законов (в том числе электромагнитных);

2) постоянство скорости света.

Эти положения и следствия специальной теории относительности легли в
основание новой релятивистской физики XX в.

Следствия специальной теории относительности:

1) принцип относительности одновременности;

2) парадокс близнецов;

3) линейная метрика не является абсолютной величиной, а зависит от скорости движения тела относительно данной системы отсчета;

4) релятивистская динамика (в зависимости от масштаба скоростей преобладает или гравитационная масса, или инертная масса).



\end{document}
