\documentclass[exam_answers.tex]{subfiles}

\fontsize{14pt}{14pt}\selectfont

\begin{document}

\renewcommand{\baselinestretch}{\blch}
\sublinksectionold{\normalsize (19) Междисциплинарные принципы в формировании естественнонаучной картины мира (системность и самоорганизация).}

Принцип самоорганизации в формировании естественнонаучной картины
мира опирается на два положения синергетики:

1) Мир состоит из разномасштабных открытых систем, развитие которых
протекает по единому алгоритму, имеющему две фазы: линейную и
нелинейную.

Линейная фаза представляет собой однонаправленное изменение, которое
обнаруживает четкую закономерность, ее можно точно рассчитать и на этой
основе дать прогноз будущих состояний системы.

Нелинейная фаза представляет собой кризисное состояние, которое
характеризуется возможностью вероятностного прогноза некоторого
множества будущих возможных состояний.

Упорядоченность возникает через флуктуации, устойчивость – через
неустойчивости. Хаотическое состояние содержит в себе неопределенность,
которая конкретизируются понятиями информации и энтропии.
Фундаментальным в описания природных процессов признается принцип
вероятности.

2) Эволюция структурных уровней материи определяется
фундаментальной способностью материи к самоорганизации. При этом чётко
различается равновесное и неравновесное состояние, а также равновесные и
неравновесные структуры.

С точки зрения синергетики, в природе преобладают открытые системы,
обменивающиеся веществом, энергией, информацией с окружающим миром,
абсолютно замкнутых систем нет. В неживой природе рассеивание и
преобразование системой поступающей энергии может приводить к
упорядоченным структурам. В живой природе обмен веществом, энергией и
информацией со средой обитания позволяет эволюционировать системам от
простого к сложному, разворачивать программу роста организма из клетки-зародыша.

Подчеркивается относительность микро- и макроуровней
самоорганизующейся системы. Взаимосвязь уровней играет решающую роль в
эволюции системы. Рождение порядка трактуется как рождение коллективных
макродвижений (и новых макростепеней свободы) из хаотических движений 
микроуровня, трансформация которых и выливается в новый порядок.
Развивается идея создания теоретической картины эволюционно-исторического
развития мирового единства (от Большого Взрыва до образования химических
элементов, звезд и планет, и далее - до сложных органических соединений,
клетки, экосистем живой природы, вплоть до человека и социума).

В становлении современной картины мира решающее значении сыграло
учение В.И.Вернадского о биосфере, в котором ключевое положение занимает
трактовка живого вещества как единой системы всех растительных и
животных организмов планеты, естественного компонента земной коры,
наряду с минералами и горными породами.

Согласно системному биокосмическому принципу Вернадского
необходимо рассматривать живую природу Земли как целостную систему,
взаимодействующую с вещественно-энергетическими процессами,
протекающими в земных, околоземных и отдаленных пространствах Космоса.
Такое обобщение, вводя новые функциональные системы в виде обменных
циклов (биогеоценозов), позволяло рассматривать биосферное единство в его
внутренних и внешних взаимосвязях. 

Мировоззренческим расширением биосферного учения выступает
системно-генетический принцип, который подчеркивает реальность скрытых
системных условий, закономерно направляющих динамику
самоорганизующейся системы, и их роль в рождении нового порядка.
Жизненное пространство, образующее макроуровень жизни системы, очерчено
единством системных условий, которые с точки зрения элементов самой
системы (микроуровня) воспринимаются как априорные ограничения.
Изменение системных макроусловий оказывается эволюционным фактором,
меняющим потенциальную норму жизни системы, что вызывает ее
кардинальную перестройку. Новая структура и ее новые свойства вроде бы не
имеют видимых оснований. Такой характер возникновения специфических для
новой целостности свойств получил название эмерджентной эволюции
(наглядный пример - принцип действия калейдоскопа). В этом же ключе
развиваются представления о системной детерминации в современной
биологии. 

Системная методология лежит в основании различных исследовательских
стратегий. В современном постнеклассическом естествознании системно
генетический принцип представлен в научно-мировоззренческих позициях:
эмерждентного материализма
(отрицающего физикализм в объяснении
человека и его сознания), системогенеза (или социогенетики),
универсального эволюционизма.
Их общую философскую основу составляет принцип
макродетерминации, утверждающий равноправие двух типов причинения:
структурных (механических, например) и функциональных. Традиционная
схема детерминации целого его структурными элементами (их природой и
взаимодействиями) дополняется достаточно жесткой детерминацией «сверху» -
от возникающего на высоком уровне сложности системного качества. Таким
образом, динамика сложной иерархически организованной целостности
оказывается дважды детерминированной: структурно («снизу вверх») и
функционально («сверху вниз»).
 Хорошо проясняет суть макродетерминации
аналогия с компьютерной программой, формирующей изображения на экране,
но не влияющей на базовый физический уровень процессов в системе.

В рамках макродетерминизма распространяется идея информационной
(функциональной) причинности, определенной взаимодействием тел и
структур. Информационная причинность имеет системный кодовый характер и
осуществляется через запуск иерархически построенных программ действия.
Наглядным примером служит генетический код, а также наличие
инстинктивных программ поведения, отработанных в филогенезе.

Междисциплинарный принцип системности и принцип самоорганизации
выступают концептуальным основанием в формировании мировоззренческой
позиции глобального эволюционизма, утверждающей всеобщий характер
эволюции во Вселенной. 



\end{document}
