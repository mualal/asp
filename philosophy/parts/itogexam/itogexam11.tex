\documentclass[exam_answers.tex]{subfiles}

\fontsize{14pt}{14pt}\selectfont

\begin{document}

\renewcommand{\baselinestretch}{\blch}
\sublinksectionold{\normalsize (11) Философские аспекты квантовой теории. Проблема индетерминизма.}

Начало квантовой теории связано с проблемой теплового излучения,
сформулированной в конце XIXв. Попытки теоретического анализа этой
проблемы на основании представления об абсолютно черном теле столкнулись
с большими трудностями. Несмотря на максимальное поглощение светового
излучения, такое тело испускает в пространство непрерывный спектр волн,
определяемый температурой тела. Согласно теории электромагнитного
излучения, нагретое тело непрерывно теряет энергию и должно охладиться до
абсолютного нуля. Тепловое равновесие между веществом и излучением
невозможно с классической точки зрения из-за противоположности вещества
(структуры) и излучения (непрерывного, волнового процесса). Но
повседневный опыт показывает, что нагретое тело не расходует всю свою
энергию на излучение. Чтобы снять возникшее противоречие между теорией и
опытом немецкий физик Макс Планк49 в 1900г. выдвинул гипотезу о том, что
энергия излучения состоит из очень маленьких порций (квантов).

Планк выдвинул свою гипотезу только для объяснения теплового
излучения. Распространение понятия квант связано с именем А.Эйнштейна,
который в 1905г. опубликовал три знаменитые работы. Две работы были
посвящены специальной теории относительности и молекулярному движению,
третья – явлению внешнего фотоэлектрического эффекта, которое он
убедительно объяснил на основе квантовой гипотезы.

Явление фотоэффекта (вырывание электронов из вещества под
воздействием света) было открыто Г.Герцем, и тщательно исследовано русским
физиком Александром Григорьевичем Столетовым (1839-1896). Объяснить
фотоэффект на основе электродинамики Максвелла, согласно которой свет - это
электромагнитная волна, непрерывно распределенная в пространстве, - не
удавалось. Опыты показали, что кинетическая энергия вырываемых светом
электронов зависит только от частоты света и не зависит от интенсивности
освещения. Если же частота света меньше определенного порога, то явление
фотоэффекта не наблюдается.

Философский аспект в объяснении Эйнштейна был связан с утверждением
дискретной энергетической природы света. Энергия каждой порции светового
излучения пропорциональна частоте: $E=h\nu$ (где $h = 6,62517\cdot 10^{–27}$ эрг*с -
постоянная Планка).

Из гипотезы Планка о порциях излучения еще не вытекало представление
о прерывистой структуре самого света. Дождь, например, выпадает на землю
тоже каплями, но отсюда не следует, что вода состоит из неделимых частей –
капель. Но явление фотоэффекта показало, что свет излучается порциями, эта
порция индивидуальна и сохраняется в дальнейшем распространении света.
Поглощается только вся порция целиком. По Эйнштейну, интенсивность света
пропорциональна числу квантов (порций) энергии в световом пучке. Объясняя
фотоэффект, он ввел понятие работы выхода электрона – определенное
количество энергии светового кванта, необходимое для сообщения электрону
такой энергии, чтобы он покинул металл.

Световой квант Эйнштейн назвал фотоном, подчеркивая, что порция света
похожа на частицу. Но фотон не имеет массы покоя, не существует в состоянии
покоя, а при самом своем рождении приобретает скорость равную с.
Фактически Эйнштейн открыл первую элементарную частицу квантовой
природы. Сейчас современная наука насчитывает множество таких частиц.

Следствием теории фотоэффекта стало представление о двойственной
природе света. При распространении света проявляются его волновые свойства,
а при взаимодействии с веществом (при излучении и поглощении) –
корпускулярные. В последствие физики обнаружили такие противоречивые
свойства у электрона и других элементарных частиц.

Поведение и свойства микрообъектов становятся предметом квантовой
механики. Первоначально в работах Н.Бора, Э.Шредингера, В.Гейзенберга и
других ученых первой половины 20 века квантовая механика являлась в
основном теорией атомных спектров. Обобщение ее до теории, описывающей
поведение всех микрообъектов в микромире, оказалось возможным благодаря
синтезу квантовой механики и специальной теории относительности. В
результате возникала релятивистская квантовая механика. К концу века на
этой базе развивается разветвленная квантовая теория, которая включает
квантовую статистику, квантовую теорию поля, теорию атомного ядра и
физику высоких энергий.

Во второй половине XXв. квантовая теория поля, заложенная в трудах
Дирака, Паули, Гейзенберга, становится фундаментальным основанием
современной физики. В ней развивается общий подход ко всем известным
типам взаимодействий (гравитационным, электромагнитным, ядерным -
слабым и сильным), а также представление о физическом вакууме, насыщенном
флуктуациями различных полей.

К первоэлементам Вселенной стали относить физический вакуум,
порождающий вещество Вселенной (главным образом протоны, электроны и
нейтроны) и антивещество (антипротоны и позитроны). К фундаментальным
процессам образования и преобразования материи - взаимное превращение
элементарных частиц и процесс аннигиляции (взаимное уничтожение) частиц и
античастиц, освобождающий колоссальную энергию в виде излучения.

Фундаментальным в описании взаимодействий микрообъектов становится
принцип целостности, который неявно представлен в физике законами
сохранения. Значение этого общего принципа для физической теории стало
расширяться в связи исследованием физических полей и характеристик
элементарных частиц. Были обнаружены эффекты, говорящие о связи
определенных состояний микрочастиц с определенными состояниями
физического вакуума. На принцип целостности в физическом описании
явлений микромира указывал также открытый Эйнштейном в совместной
работе с Розеном и Подольским необычный эффект несиловой корреляции
фундаментальных (спиновых) характеристик элементарных частиц, который
получил название парадокса ЭПР (Эйнштейна – Подольского – Розена).

Одна из проблем квантовой теории связана с физической (и
мировоззренческой) интерпретацией волновой функции, которая имеет
значение основного параметра квантового поля и элементарной частицы.
Австрийский физик Эрвин Шредингер, создав основное уравнение квантовой
механики, не смог разъяснить физический смысл этой функции, которая
выступает дополнительной (по отношению к импульсу) характеристикой в
квантовом описании поведения микрочастиц.

Копенгагенская интерпретация квантовой механики, предложенная
Максом Борном и Нильсом Бором, провозглашая принцип дополнительности,
подчеркивала отказ от классического принципа детерминизма, утверждавшего
однозначную причинную связь событий. Соотношение неопределенности
Гейзенберга фиксировало границы применимости классической механики к
описанию квантовых объектов (микрочастиц). Согласно принципу
неопределенности, в мире квантовых явлений нельзя пренебречь
взаимодействием между измерительным прибором и изучаемым явлением.

Процесс измерения в микромире породил проблему онтологического
статуса микрочастицы, поставил под сомнение классическую познавательную
схему, в которой субъект и объект познания не влияют друг на друга. Принцип
неопределенности вводил сознание наблюдателя в качестве необходимого
параметра исследования квантовых явлений, ставил под сомнение и
объективность микромира, и объективность физической теории.

В «копенгагенской интерпретации» квантовой механики была предпринята
попытка устранения сознания наблюдателя из исследуемой ситуации. Был
введен постулат о редукции состояния (коллапсе волновой функции), согласно
которому при соприкосновении квантовой системы (микросистемы) с
прибором (макросистемой) происходит отбрасывание всех альтернативных
исходов, возможных с точки зрения квантовой механики.

Наиболее парадоксальная интерпретация квантовых взаимодействий,
получившая название многомировой, была предложена Х.Эвереттом, согласно
гипотезе, которого кроме реальной Вселенной существуют множество ее
параллельных двойников – теневых миров, где обитают наши «дублеры». Эти
двойники никак себя не проявляют за исключением квантового уровня. В
случае прохождения электрона сквозь щели, электрон и его двойник
взаимодействуют, снимая неопределенность. Именно этот странный мир
взаимодействий, где порогом той или иной реальности выступает очень узкое
место – щель, и описывает квантовая механика.



\end{document}
