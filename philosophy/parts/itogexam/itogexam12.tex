\documentclass[exam_answers.tex]{subfiles}

\fontsize{14pt}{14pt}\selectfont

\begin{document}

\renewcommand{\baselinestretch}{\blch}
\sublinksectionold{\normalsize (12) Философские проблемы теоретической биологии. Принципы наследственности и изменчивости в становлении генетики.}

Основная тенденция развития биологии в течение XXв. – стремление к
теоретическому обобщению фактического материала, накопленного с
появлением новых дисциплин, исследующих различные уровни организации
живого (клетку и ее составные части, органы, зародыши, популяции).
Клеточная теория строения живого, давая концептуальную базу биологическим
исследованиям, обостряет вопрос о закономерностях воспроизводства живых
организмов и самой клетки.

В дарвиновской эволюционной концепции представления о
наследственности весьма расплывчаты, основная проблема связана с
принципами видообразования, вопросами изменчивости, борьбы за
существование, естественным отбором. «Временная гипотеза пангенезиса»,
выдвинутая в последней главе труда Ч.Дарвина «Изменение домашних
животных и культурных растении» (1868), предполагала образование в каждой
клетке любого организма особых частиц - геммул, которые обладают
способностью распространяться по организму и собираться в клетках,
служащих для полового или вегетативного размножения. Геммулы отдельных
клеток могут изменяться в ходе онтогенеза каждого индивидуума и давать
начало измененным потомкам. Предположение Дарвина о наследовании
приобретенных признаков было экспериментально опровергнуто Ф. Гальтоном
(1871).

Еще одна гипотеза о природе наследственности была предложена
ботаником К.Нечели в работе «Механико-физиологическая теория эволюции»
(1884). Нечели предположил, что наследственные задатки передаются лишь
частью вещества клетки, названного им идиоплазмой. Остальная часть
(стереоплазма) наследственных признаков не несет. Он предположил, что
идиоплазма состоит из молекул, соединенных друг с другом в крупные
нитевидные структуры - мицеллы, группирующиеся в пучки и образующие
сеть, пронизывающую все клетки организма. Гипотеза Нечели подготовила
биологов к мысли о сложной структуре материальных носителей
наследственности.

Наука о механизмах наследственности и изменчивости живых организмов
получила развитие в начале XX в. Название генетика (γενεσις, genesis – греч.
рождение, происхождение) было предложено английским ученым У.Бэтсоном в
1906г. Концептуальную основу генетики составили: теория гена, хромосомная
теория наследственности, теория мутаций.

В становлении генетики выделяют три этапа: классический (1900-1930гг.),
неоклассический (1930-1953гг.), синтетический (по настоящее время).

Классический этап в развитии генетики начинается с переоткрытия
законов Менделя, который еще в 1865г., анализируя потомство, полученное от
скрещивания контрастных сортов гороха, сформулировал законы
наследственности. Мендель показал, что наследуемые задатки не смешиваются,
а передаются от родителей к потомкам в виде обособленных единиц,
сформулировал принципы независимости комбинирования этих элементарных
единиц при скрещивании.

Исследователями классического периода развития генетики были
выяснены основные закономерности наследования и доказано, что
наследственные факторы сосредоточены в хромосомах. В первые десятилетия
XXв. представление о дискретных наследуемых задатках получило
подтверждение на основании громадного числа опытов с растениями,
животными, микроорганизмами, а также в наблюдениях за наследственностью
человека. Большая заслуга в становлении классического этапа генетики
принадлежит английскому ученому У.Бэтсону (1861-1926), показавшему, что
законы, сформулированные Менделем, свойственны не только растениям, но и
животным. В 1909г. датский ученый Вильгельм Иоганнсен (1857-1927) ввел
понятие «ген» для обозначения дискретной единицы, ответственной за
наследование определенного признака (задатка).
В 1912г. Т.Х.Морган показал, что гены расположены в хромосомах.

Важнейшее свойство генов – сочетание их высокой устойчивости
(неизменяемости в ряду поколений) со способностью к наследуемым
изменениям, служащим основой изменчивости организмов, дающей материал
для естественного отбора. Совокупность всех генов (или задатков) организма -
сложно взаимодействующая система, которая получила название генотип.

Иоганнсен представил естественный отбор в качестве главного фактора,
преобразующего генотип на основе наследственной изменчивости при
формирующей роли среды. Складывается учение о фенотипе и генотипе
организма. Под фенотипом понимается совокупность всех признаков,
которыми обладает организм, под генотипом – генетический состав, которым
определяются эти признаки. На основании этого учения, в частности выделения
генотипических линий популяции растений и животных, формируются
аналитические методы селекции.

Улучшение оптических качеств микроскопов в конце XIXв. позволило
вести экспериментальные гибридологические и цитологические исследования.
В 1875г. Гертвиг обратил внимание, что при оплодотворении яиц морского ежа
происходит слияние двух ядер (ядра спермия и ядра яйцеклетки). Флемминг в
1882г. описал поведение особых структур ядра во время митоза (бесполого
размножения посредством деления клетки). Для обозначения этих особых,
хорошо наблюдаемых структур ядра, играющих определенную роль в делении
клетки, В.Вальдейер в 1888г. предложил термин хромосома.

Идея о неравном наследственном делении ядер клеток развивающегося
зародыша была высказана В.Ру в 1883г. В это же время А.Вейсман пришел к
выводу о существовании в организме двух четко разграниченных видов клеток
- зародышевых и соматических. Первые, обеспечивая непрерывность передачи
наследственной информации, «потенциально бессмертны» и способны дать
начало новому организму. Вторые такими свойствами не обладают. Выделение
двух категорий клеток имело большое значение для последующего развития
генетики. Предположение о линейном расположении наследственных факторов
(хромативных зерен - по Ру, ид - по Вейсману) и их продольном расщеплении
во время митоза предвосхитили хромосомную теорию наследственности.

В 70-80-х годах XIXв. были описаны митоз и поведение хромосом во
время деления клетки. Это привело к утверждению об ответственности этих
структур за передачу наследственных потенций от материнской клетки
дочерним. Деление материала хромосом на две равные частицы
свидетельствовало в пользу гипотезы, что именно в хромосомах сосредоточена
генетическая память. Изучение хромосом у животных и растений привело к
выводу, что каждый вид животных существ характеризуется строго
определенным числом хромосом.

В начале XX века ученые, исследовавшие живые клетки, обнаружили в
них материальные структуры, роль и поведение которых могли быть
однозначно связаны с закономерностями, выявленными Менделем (В.Сэттон -
1903). Гипотетические представления о наследственных факторах, о наличии
одинарного набора факторов в гаметах (половых клетках), и двойного - в
зиготах (оплодотворенных клетках) получили экспериментальное обоснование.
В начале XXв. Т.Бовери (1902) продемонстрировал важную роль ядра в
регуляции развития наследственных признаков организма, представил
доказательства в пользу участия хромосом в процессе наследственной
передачи, показав, что нормальное развитие морского ежа возможно только при
наличии всех хромосом.

Установлением факта, что именно хромосомы несут наследственную
информацию, В.Сэттон и Т.Бовери положили начало новому
экспериментальному направлению в биологии - исследованию хромосом на
основе гибридологического и цитологического анализа. Экспериментальные
факты, полученные в цитологических исследованиях, подтверждали
дискретность фактора, несущего наследственный материал. Сложилось
представление, что единица наследственности (ген) отвечает за развитие
одного признака и передается при скрещиваниях как неделимое целое.

В формулировании и обосновании хромосомной теории
наследственности большая заслуга принадлежит Томасу Ханту Моргану
(1866-1945). Согласно хромосомной теории, каждая хромосома несет по
одному фактору, каждая пара факторов локализована в паре гомологичных
хромосом. Поскольку число признаков у любого организма во много раз
больше числа хромосом, видимых в микроскоп, каждая хромосома должна
содержать множество факторов.

В начале XXв. Морган сформулировал положение о сцеплении генов в
хромосомах. Он экспериментально доказал, что гены, находящиеся в одной
хромосоме, передаются при скрещивании совместно. Число групп сцепления
соответствует числу пар хромосом. Проследив за поведением генов в потомстве
определенных самцов и самок, Морган получил убедительное подтверждение
предположения о сцеплении генов.

С помощью светового микроскопа в 1934г. были обнаружены гигантские
хромосомы, в которых чередовались темные и светлые поперечные полосы.
Причем искусственным путем можно было вызвать различные фенотипические
аномалии, которые сопровождаются определенными изменениями в рисунке
поперечных полос. Наиболее явным примером того, что фенотипические
признаки организма связаны со строением хромосом, служит различие между
полами. Гены, находящиеся в половых хромосомах, назвали сцепленными с
полом. Эта особая форма сцепления позволила объяснить, в частности
наследование таких признаков как раннее облысение и гемофилия, которые
присущи определенному полу.

Основы теории гена сложились к началу 30-х годов XXв. Обнаруженное
Морганом нарушение сцепления генов в результате обмена участками между
хромосомами (явление кроссинговера) подтверждало неделимость генов. В
результате обобщения всех данных ген стали понимать как элементарную
единицу наследственности, которая характеризуется вполне определенной
функцией - изменяется во время кроссинговера как целое. Как единица обмена
(между участками хромосом) ген получил новый статус единицы
наследственной изменчивости.

Под изменчивостью в биологии понимают всю совокупность различий по
тому или иному признаку между организмами, принадлежащими к одной
популяции или виду. Морфологическое разнообразие особей в пределах любого
вида поразило в свое время Дарвина и Уоллеса, послужила толчком в
исследованиях Менделя, показавшего предсказуемый, закономерный характер
передачи различий в поколениях. В наблюдаемых фенотипических различиях
различают две формы изменчивости: дискретную (качественную) и
непрерывную (количественную). Фенотипические различия, которые
характеризуют дискретную изменчивость, четко выражены и между ними
отсутствуют промежуточные формы. Например, пол у животных и растений,
группа крови у человека, длина крыльев у дрозофилы. Признаки, связанные с
дискретной изменчивостью, представлены ограниченным числом вариантов и
контролируются одним или двумя главными генами, которые могут иметь
несколько аллей (генов, расположенных в том же месте хромосомы). Внешние
условия мало влияют на такие признаки различия внутри вида. Например,
климатические условия и катаклизмы не влияют на группу крови человека.

Количественная (непрерывная) изменчивость определяет наблюдаемые
различия признаков в популяции (рост, вес, форма, окраска). Большинство
индивидов попадает в среднюю часть статистической кривой, описывающей
распределение непрерывной изменчивости в популяции по некоторому
признаку (средний рост, средний вес и т.п.). Крайними различиями обладают
малое количество особей. Признаки, характерные для непрерывной
(количественной) изменчивости обусловлены совместным действием многих
генов и факторов среды. Главный фактор, определяющий любой
фенотипический признак – генотип, который закладывается в момент
оплодотворения. Последующая реализация генетического потенциала в
значительной мере зависят от внешних условий развития организма. Но среда
никогда не может вывести фенотип за пределы, определенные генотипом.

Неоклассический этап в развитии генетики (30-50 гг. XXв.) связан с
молекулярными и биохимическими исследованиями механизма наследственной
изменчивости. Решающим событием в этот период было открытие мутаций –
внезапно возникающих изменений, которые могут передаваться по наследству.
Систематическому изучению мутаций положили начало работы голландского
ученого Хуго де Фриза, который предложил термин «мутация» в 1901г.

Крупнейшим достижением было обнаружение возможности искусственно
вызывать мутации при помощи разнообразных физических и химических
агентов. Было установлено, что любое ионизированное облучение вызывает
мутации.

В генетике появилось учение о системе репарирующих ферментов,
исправляющих повреждения генетических структур, вызванные облучением
или обработкой химическими агентами.
Затем была обнаружена возможность
искусственно вызывать мутации при помощи разнообразных физических и
химических агентов. За открытие искусственного мутагенеза Г.Меллеру
была присуждена в 1946 г. Нобелевская премия

В середине 30-х годов формулируется теория, описывающая кинетические
зависимости активирующего и мутагенного эффекта ионизирующих излучений
(«теория мишени»). Важнейшие эксперименты, ставшие основой этой теории,
были проведены в период 1931-1937гг. Н.В.Тимофеевым-Ресовским,
М.Дельбрюком, Р.Цимером и другими исследователями.

В ходе исследования химических факторов в процессе мутации было
открыто мощное мутагенное действие некоторых химических веществ,
оформилось новое направление генетики – химический мутагенез.
В настоящее время известно большое количество веществ, усиливающих
мутационный процесс. Разработана теория действия мутагенных соединений на
наследственные структуры, интенсивно разрабатываются проблемы
специфичности действия мутагенов.

Большой материал, накопившийся в области изучения изменчивости,
позволил создать классификацию типов мутаций. Было установлено
существование трех видов мутации - генных, хромосомных и геномных.
К первому классу относятся изменения, затрагивающие лишь один ген. В этом
случае либо полностью нарушается работа гена и, организм теряет одну
функцию, либо изменяется его функция. Хромосомные мутации - изменение в
структуре хромосом, которые могут иметь разные следствия. Может произойти
удвоение, утроение отдельных участков хромосомы (дупликация), в другом
случае оторвавшийся кусок хромосомы может остаться в той же хромосоме, но
окажется в перевернутом виде, при этом порядок расположения генов в
хромосоме изменяется (инверсия). Если утрачивается участок хромосомы,
говорят о делеции, или нехватке. Все типы хромосомных перестроек
объединяют под общим термином - хромосомные аберрации. В геномных
мутациях изменяется число хромосом.

В неоклассический период развития генетики появляются факты,
вызывающие сомнение в неделимости гена. В 1928г. Н.П.Дубинин (работая в
лаборатории А.С.Серебровского при Биологическом институте им.
К.А.Тимирязева) обнаружил необычную мутацию, свидетельствующую о том,
что ген не является неделимой структурой, а представляет собой область
хромосомы, отдельные участки которой могут мутировать независимо друг от
друга. Это явление было названо ступенчатым аллеломорфизмом (аллели –
гены, расположенные в одном и том же месте хромосомы). Экспериментально
подтвердить мутационную дробимость гена удалось только в 1938г.
Окончательное решение этот вопрос получил в работах М.Грина (1949),
Э.Льюиса (1951) и Г.Понтекорво (1952), убедительно показавших, что считать
ген неделимым неправильно. Актуальной становится проблема структуры гена,
с которой связано рождение новой области биологических исследований.

Современный период в становлении генетики начинается в 50-х гг. с
оформлением молекулярной биологии. Термин «молекулярная биология» ввел
У.Астбери, которому принадлежат основополагающие работы в исследовании
белков. В 40-х гг. господствует представление, что гены – особый тип
белковых молекул. В 1944г., однако, было показано, что генетические функции
в клетке выполняет не белок, а особые макромолекулы
дезоксирибонуклеиновой кислоты (ДНК). Установление роли нуклеиновых
кислот в передаче наследственных признаков положило начало новой области –
молекулярной биологии. В 1953г. Ф.Крик (Англия) и Д.Уотсон (США) выявили
пространственную структуру ДНК и создали ее модель в виде двойной
спирали, элементы которой повторяются в строгой последовательности.

За сравнительно короткий срок были установлены природа гена и
основные принципы его организации, воспроизведения и функционирования,
расшифрован генетический код, выявлены и исследованы механизмы и главные
пути образования белка в клетке, в которой фундаментальную роль играет
пространственно ориентированная полипептидная цепь. Молекулярная
биология установила принципы организации разных субклеточных частиц,
вирусов, путь их биогенеза в клетке.

На базе молекулярной биологии в 70-х гг. развивается методы генной
инженерии (внедрение в клетку желаемой информации), а также методы
выделения в чистом виде фрагментов ДНК – молекулярная генетика. В 80-х гг.
процесс выделения генов и получения из них различных цепей
автоматизируется. Генная инженерия в сочетании с микроэлектроникой
открывает новые перспективы исследования и управления законами живой
материи. В прессе активно обсуждаются возможности рождения ребенка «из
пробирки» (проблема гомункулуса). Большой общественный резонанс
получили опыты по клонированию. Первый опыт с овечкой Долли был
проведен в 1997г.

Одно из наиболее существенных достижений молекулярной генетики
заключается в установлении минимальных размеров участка гена,
передающихся при кроссинговере (в молекулярной генетике -«рекомбинация»),
подвергающихся мутации и осуществляющих одну функцию. Среди различных
внутригенных мутаций С.Бензер выделил два класса: точечные мутации
(мутации минимальной протяженности) и делеции (мутации, занимающие
достаточно широкую область гена). Установив факт существования точечных
мутаций, Бензер задался целью определить минимальную длину участка ДНК,
передаваемую при рекомбинации. Оказалось, что эта величина, названная
реконом, составляет не более нескольких нуклеотидов. Далее он установил
минимальную длину участка, изменения которого достаточно для
возникновения мутации, назвав его мутоном. По мнению Бензера, эта величина
равна нескольким нуклеотидам. В дальнейшем было выявлено, что длина
одного мутона не превышает размер одного нуклеотида

Следующим важным шагом в изучении генетического материала было
подразделение всех генов на два типа: регуляторные гены, дающие
информацию о строении регуляторных белков и структурные гены,
кодирующие строение остальных полипептидных цепей. Экспериментальное
доказательство этой идеи было дано Ф.Жакобом и Ж.Моно (1961).

Определение основной функции гена как хранителя информации о
строении определенной полипептидной цепи столкнулось с проблемой записи
генетической информации и механизма ее переноса от генетических структур
(ДНК) к морфологическим структурам в клетке.

Согласно модели Уотсона - Крика, генетическую информацию в ДНК
несет последовательность расположения оснований. Таким образом, в ДНК
заключены четыре элемента генетической информации. В тоже время в белках
было обнаружено 20 основных аминокислот. Необходимо было выяснить, как
язык четырехбуквенной записи в ДНК может быть переведен на язык
двадцатибуквенной записи в белках. Физик Г.Гамов предположил, что для
кодирования одной аминокислоты используется сочетание из трех нуклеотидов
ДНК.
Эта элементарная единица наследственного материала, кодирующая
одну аминокислоту, получила название кодон.

Предположение Гамова о трехнуклеотидном составе кодона долгое
время не удавалось доказать экспериментально. Только в конце 1961г. была
опубликована работа кембриджской группы исследователей во главе с
Ф.Криком, выяснивших тип кода и установивших его общую природу. Они
доказали, что в каждом гене есть строго фиксированная начальная точка, с
которой фермент, синтезирующий РНК, начинает «прочтение» гена, причем
читает его в одном направлении и непрерывно. Авторы так же доказали, что
размер кодона действительно равен трем нуклеотидам и что наследственная
информация, записанная в ДНК, читается от начальной точки гена «без запятых
и промежутков».



\end{document}
