\documentclass[exam_answers.tex]{subfiles}

\fontsize{14pt}{14pt}\selectfont

\begin{document}

\renewcommand{\baselinestretch}{\blch}
\sublinksectionold{\normalsize (20) Глобальный эволюционизм -- новая натурфилософская позиция в системе современного естествознания. Картина мира в глобальном эволюционизме.}

Главный тезис глобального эволюционизма: все познанная история
Вселенной как самоорганизующейся системы от Большого взрыва до
возникновения человечества представляет собой единый процесс развития,
который характеризуется преемственностью механизмов космической,
химической, биологической и социальной эволюции.

Естественнонаучные основания глобального эволюционизма составили:
эволюционная биология, учение о живом веществе и биосфере, эволюционные
теории в космологии, в частности теория Большого взрыва и ее подтверждения
(красного смещения, реликтовое излучение), теория самоорганизации.

Философскими основаниями глобального эволюционизма выступили:
принцип детерминизма в современной интерпретации вероятностного
детерминизма и макродетерминизма, идея развития мира и всеобщей
взаимосвязи явлений (выдвинутая в XIXв. Гегелем в немецкой классической
философии и в отношении природы - в диалектическом материализме).

Эволюционное развитие понимается как закономерно направленный
процесс необратимых качественных изменений мирового единства. В отличие
от эволюционной теории в биологии, только констатирующей определенную
преемственность человека в ряду животного мира, но не объясняющей
необходимости появления человека и социума, в глобальном эволюционизме
утверждение закономерности появления человека – принципиальная исходная
позиция, определяющая программу поиска механизмов согласования разных
типов эволюции: от космической – до социальной.

В зависимости от схемы анализа единого эволюционного процесса: по
«восходящей» линии (от элемента – к сложно организованным системам) или
по «нисходящей» линии (от единой гармонии Вселенной или от самой сложной
формы материальной самоорганизации – к элементарным структурам), -
различают две позиции.
В первом случае глобальный характер эволюции
прослеживается от уровня элементарных структур (например, вихревые
образования) до сложных иерархически организованных систем в природе и
обществе. Утверждается, что генетическое и структурное единство
эволюционного процесса определяется низшими уровнями самоорганизации
материи. Этой точки зрения на эволюционный процесс придерживался
В.И.Вернадский в учении о биосфере и ноосфере, Э.Янч – в концепции
самоорганизующейся Вселенной, И.Пригожин – в неравновесной
термодинамике.

Другая линия анализа связана с именем Пьера Тейяра де Шардена,
который развивает аристотелевскую традицию христианской философии,
полагая, что генетическое и структурное единство эволюционных процессов
определено высшими уровнями самоорганизации материи.

В глобальном эволюционизме термин «эволюция» содержательно
отличается от сходных понятий изменения и развития. Эволюция связывается с
появлением принципиально новых, ранее не имевшихся параметров или
систем. Развитие - с появлением новых признаков системы, которые, однако,
не являются принципиально новыми для мирового единства. Появление клетки
как основы живой природы, например, - эволюционное явление, но обменные
процессы, а также процессы, происходящие при рождении каждой отдельной
клетки, изменения в результате ее деления, описываются термином «развитие».
Также как процессы, происходящие в современных астрономических объектах,
представляются в терминах изменения и развития (движение планет Солнечной
системы, циклы Солнечной активности и т.д.).

Различные системы можно рассматривать как эволюционные лишь на
этапе становления принципиально новых для системы и мира качеств и
структур. Эволюционное формирование наблюдаемых космических тел и
образований произошло на определенном этапе развития Вселенной. Сейчас
мы наблюдаем лишь изменение их параметров. То же можно сказать о
геологических системах.

В качестве эволюционирующих систем выделяются только две: весь Мир и
авангардная форма движения. Глобальная эволюция Мира отличается от
эволюции отдельных систем своей непрерывностью и переносом процесса
эволюционных изменений с одного вида движения на другой. Эволюционный
процесс в отдельной системе необходимо заканчивается при достижении
некоторого равновесного состояния, а эволюция продолжается в последующем
виде движения. В авангардной форме движения всегда можно выделить
эволюционный параметр, который непрерывно изменяется и связан с
появлением новых характеристик и определений данного типа движения. Этот
параметр относится к эволюционирующей системе в целом. Например, на
уровне социальной эволюции, он относится к единому социуму, а не к расцвету
и упадку отдельных государств.

Позиция глобального эволюционизма регламентирует преемственность
типов эволюции на основании временности эволюционного развития той или
иной системы. Геологическая система была авангардом эволюции на
определенном этапе эволюции Мира и завершилась образованием
геологических структур и физического мира Земли. На предыдущем этапе, в
результате космической эволюции возникла структурная Вселенная.
Возникновение биологических систем также было возможно на конкретном
этапе, при конкретных физических параметрах, которые невозможно
восстановить в данный момент.

Постоянно эволюционирующей системой выступает только мир в целом.
Отдельные эволюционные процессы: на космическом, уровне, геологическом,
химическом, биологическом, социальном – представляются собой частные
реализации глобальной эволюции мира на разных временных этапах истории
Вселенной.

Познавательная стратегия глобального эволюционизма подчеркивает, что в
рамках каждой научной системы, объясняющей и изучающей ту или иную
форму движения, должен присутствовать механизм развития, приводящий к
внутренним противоречиям, которые разрешаются при переходе к следующему
этапу или следующей системе.

Теоретические посылки глобального эволюционизма можно свести к
следующим положениям.

1) Эволюция предстает как процесс движения Мира через
самоопределение нового порядка, как поэтапное возникновение новых
равновесных состояний.

2) Научные теории, относящиеся к отдельным видам движения,
принципиально несводимы. Появление основных видов взаимодействий
происходит в эволюционной (временной) последовательности.

3) Адекватное принципиальное описание мировых взаимодействий и
форм движения, может дать не единая система уравнений, а математический
аппарат, содержащий элемент развития. Если некая система уравнений
описывает определенные процессы, то в ней должен быть параметр, при
изменении которого, система становится неоднозначной – появляются
противоречивые решения. Введение нового параметра, компенсирующего
противоречивые решения, приводит уже к другой системе уравнений, которая
не сводится математическими преобразованиями к предыдущей и описывает
уже другой тип процессов.

4) Антропный принцип, который формулируется как:

Слабый антропный принцип: разум – один из видов мирового движения.
Его носителем выступает социальная система.

Сильный антропный принцип: разум – обязательный этап эволюции Мира.

Финалистский антропный принцип: разумная форма движения Мира –
неотъемлемый этап, определяющий его дальнейшее развитие. Во Вселенной
должна возникнуть разумная обработка информации и, раз возникнув, она
никогда не прекратится.



\end{document}
