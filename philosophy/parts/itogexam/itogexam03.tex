\documentclass[exam_answers.tex]{subfiles}

\fontsize{14pt}{14pt}\selectfont

\begin{document}

\renewcommand{\baselinestretch}{\blch}
\sublinksectionold{\normalsize (3) Теоретико-познавательные и методологические аспекты естествознания.}

Анализ познавательных установок и стратегий естественнонаучного
исследования, выявление возможностей последовательного объяснения
наблюдаемых явлений и построения логически стройных теорий лежит в
области теории познания или гносеологии.
Главная философская проблема в
этой области - несовпадение теории и наблюдаемых, регистрируемых
феноменов. Центральный вопрос гносеологической проблематики - истинность
и объективность естественнонаучного знания.

Традиции в построении причинных (каузальных) моделей объяснения
явлений, позволяющих устанавливать закономерность поведения реальных
объектов, составляют специфику естествознания.


Принцип причинной связи, заявленный в античной натурфилософии,
составляет главную мировоззренческую и познавательную стратегию
естествознания в его истории. Вместе с принципом единства мира он образует
общий контекст развития естественнонаучного знания.

Гносеологическая проблематика современного естествознания определятся
вопросами осмысления обобщенных базовых моделей причинного объяснения,
в основе которых лежит та или иная форма детерминизма.

Детерминизм – мировоззренческая позиция, в которой постулируется
причинно-следственная связь природных явлений, не всегда явно
представленная в наблюдаемых событиях. Принцип всеобщей причинной связи
был четко сформулирован в атомистическом учении Демокритом в жесткой
форме, поскольку отрицал случайность в реальной онтологии мира, утверждая
однозначную связь причины и следствия. Элемент случайности был внесен в
концепцию атомизма позже Эпикуром.

В новоевропейской философии и науке эта установка была обобщена
Лапласом. «Демон Лапласа» - символ механистического детерминизма,
выделившего универсальность силового (динамического) принципа причинноследственной связи,
который позволяет точно рассчитать все состояния
объекта. Для «Демона Лапласа» мир прозрачен, предсказуем, в нем нет
случайностей. В философии науки жесткий детерминизм, механистический,
лапласовский, динамический представляют собой тождественные понятия.

Вторая форма детерминизма – вероятно-статистическая, допускающая
случайность в систему причинения, появляется с развитием термодинамики,
статистической физики и квантовой механики, выделившими приоритет
статистического закона в объяснении причинно-следственный связей. Символ
этой формы детерминизма – «Демон Максвелла», разделяющий горячие и
холодные молекулы в сосуде, что позволяет ему нагреть правую часть сосуда и
охладить левую без дополнительного подвода энергии к системе

Статистически-вероятностный детерминизм сочетает динамический и
статистический принцип в объяснении причинения, благодаря разведению
макро- и микро-характеристик термодинамической системы и введению
принципа дополнительности в описание ее поведения. Что позволяет
рассчитывать и предсказывать главную тенденцию поведения системы, которая
понимается как массовый объект. В этой форме детерминизма случайность,
которая характеризует термодинамическую систему, относится на счет
инструментария субъекта, который не может точно рассчитать скорости
микрообъектов. Например: скорости всех молекул идеального газа.

Третья форма детерминизма оформляется в конце 20в. как вероятностный
детерминизм. В этой позиции утверждается фундаментальность вероятностных
характеристик объекта, подчеркивается, что жесткость и нежесткость
причинения зависят от условий и в этом смысле относительны. Получает новое
толкование сам закон природы, который рассматривается уже не как
объективный динамический закон, инвариантный и обратимый во времени, а
как вероятный и необратимый. «Стрела времени» Пригожина указывает на то,
что в эволюции Вселенной не всегда существовали те взаимодействия и
структуры, которые классическая и неклассическая физика считает
объективными и описывает соответствующими законами.

Круг методологических проблем естествознания определен вопросами:
какие средства, установки и методы адекватны современному уровню развития
научного знания. От метода часто зависит судьба исследования в науке,
поскольку к одним и тем же фактам можно подойти по-разному и
сформулировать на этом основании неоднозначные или совершенно
противоположные выводы. Верная картина может быть получена при
адекватном подходе к изучаемому явлению. Поиск такого подхода и составляет
главную цель методологии. В этом она опирается на общую
мировоззренческую картину, чтобы выявить условия закономерного развития
действительности и обусловленные этим формы практического и
теоретического действия.

Методологические принципы задают идеалы и нормы исследования в
соответствии с представлениями о мире и конкретной дисциплинарной
теорией. Предписательный характер методологического принципа (например,
системность) вытекает из непрерывности и семантической связности общего
массива знания в науке.

Методологические принципы формируются на трех уровнях:
мировоззренческом (теоретико-познавательном), теоретическом (в конкретной
области) и эмпирическом.

На мировоззренческом уровне формируется гносеологический
(познавательный) принцип, фиксирующий характер причинной связи и
указывающий на общий подход к исследованию явлений. В рамках
мировоззренческой установки (проясняющей, как устроен мир),
гносеологический принцип указывает, как сформировать первичное
представление об объекте исследования. Например, различие в
мировоззренческих установках Средневековья и Нового времени определяли и
различие познавательных принципов. Если мир создан Богом, един и один, то
он должен подчиняться провидению. Отсюда следует телеологический принцип
в объяснении явлений: конечная причина - в Божьей воле. В Новое время
признаются два основания мира – природное и божественное. С одной стороны,
в мире действуют законы природы, познаваемые людьми. С другой стороны,
первоначальная данность пространства и времени представляет реальность
иного рода. Абсолютность пространства признается как некое Богом данное
вместилище, в котором разворачиваются природные процессы, сводимые к
принципу внешнего (механического) взаимодействия. Дуализм и
механистический детерминизм определяли познавательную стратегию в
естествознании вплоть до 20в.
Отношение предметной теории к философии и методологии неоднозначно.
Конкретные науки вырабатывают свою методологию, связь которой с
мировоззренческим, философско-методологическим уровнем неочевидна и
присутствует обычно неявно - как само собой разумеющееся основание,
которое принимается без обсуждения. Для естествознания таким основанием,
например, является материальность мира, объективность пространства и
времени. Иногда мировоззренческий контекст вовсе игнорируется (как не
имеющий ценности для позитивной науки). Однако интуитивное ощущение,
что некоторые идеи носятся в воздухе, сопоставление познавательного
(гносеологического) принципа культурной эпохи с развитием конкретных
теорий в специальных областях знания, свидетельствуют о наличии такой
связи.

В процессе интеграции современного естественнонаучного и технического
знания, в решении глобальных проблем, оценке масштабных социотехнических
инновационных проектов междисциплинарные методологические установки и
принципы играют ключевую роль. Более того, характерные для нынешнего
века проблемно ориентированные науки строятся не на базовой теории, а на
основе общей методологии.



\end{document}
