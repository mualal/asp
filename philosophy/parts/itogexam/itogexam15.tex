\documentclass[exam_answers.tex]{subfiles}

\fontsize{14pt}{14pt}\selectfont

\begin{document}

\renewcommand{\baselinestretch}{\blch}
\sublinksectionold{\normalsize (15) Междисциплинарные стратегии в естествознании XXв. Функциональный, системный, информационный подходы.}

Объектами современных междисциплинарных исследований все чаще
становятся уникальные системы, характеризующиеся открытостью и
саморазвитием. Такого типа объекты начинают определять характер
предметных областей основных фундаментальных наук, детерминируя
облик современной, постнеклассической науки.

Главная характеристика постнеклассической науки – отказ от
универсальности физических понятий и физической картины мира. Проблемы,
выявленные в дискуссиях вокруг квантовой теории и природы квантовых
явлений, а также в новой космологии, ставшей уже в первой половине XXв.
физической дисциплиной – астрофизикой, ввели в круг фундаментальных
проблем, связанных с объяснением явлений микро- и мега мира строго говоря
нефизические понятия, фиксирующие не характерные для классической и
неклассической физики принципы целостности и эволюции. Принципы
построения новой физики в этом контексте намечены в разработке трех
главных проблем философии физики конца XXв.: онтологической проблемы
сингулярности (как особого состояния Вселенной, для которого нет
адекватного физического объяснения и описание которого неизбежно включает
временной фактор эволюции), гносеологической проблемы дополнительности
теоретического описания явлений, особенно при переходе от макро- к микро
явлениям (что поставило под сомнение возможность создания единой
унифицированной физической теории), проблемы самоорганизации.

Другим стимулом становления междисциплинарной области в системе
научного знания было появление новых наук о сложных системах, предметом
которых стали процессы управления и организации, рассматриваемые в
абстракции от физической природы самих систем. На этой почве оформился
новый общенаучный понятийный аппарат. Исторически первую роль в этом
движении концептуальной интеграции естественных и социальных наук
сыграла кибернетика, методологическим следствием которой стали
междисциплинарные познавательные стратегии системного,
функционального, информационного подхода. А в системе современных наук
появились такие дисциплины как информатика, системный анализ,
искусственный интеллект, когнитивная наука и др.

Кибернетический способ исследования сложных систем и явлений
получил название функционального подхода. Главным в этом подходе является
установка на изучение реакций системы в ответ на внешнее воздействие, 
которое имеет сигнальный характер. Система предстает в качестве «черного
ящика», имеющего вход (на который поступает некоторый сигнал) и выход
(действие, реакция, программа поведения). Внутренняя структура сложной
системы не конкретизируется и вообще не рассматривается, анализируются
только ее наблюдаемые ответные действия и необходимые для их реализации
функции.

До кибернетики подобный поведенческий подход разрабатывался в
психологии. Особенно эффективно - в дрессировке животных. Поведенческий
подход стал основанием бихевиоризма (behavior - англ. поведение) –
популярной концепции в психологии, трактующей психику через отношение
«стимул-реакция». Кибернетика использовала поведенческий принцип для
разработки абстрактных принципов эффективного управления системой.
Утверждая универсальность принципа обратной связи в изучении и
конструировании сложных систем, строение которых невозможно точно
описать, кибернетика распространила функциональный подход на широкий
класс явлений неживой и живой природы.

В основании функционального подхода лежат две идеи:

1) общность закономерных процессов связи и управления для разнородных
материальных систем;

2) взаимосвязь целесообразности и управления в организации действия системы.

Закономерности, которые открыла кибернетика, позволили выделить новую
область функциональных свойств и новые объекты научного исследования -
функциональные системы.

Теория функциональной системы, предложенная академиком П.К.Анохиным,
давала описание взаимосвязи систем разного уровня организации в живой природе на
основе понятия опережающего отражения и информации, выделяла особое значение
в функционировании сложной системы систематизирующего фактора, которым
выступает результат действия. Базовым исходным принципом в теории П.К.Анохина
выступил принцип единства структуры и функций, применение которого к анализу
биологических систем привело к выводу, что фундаментальным фактором
становления и эволюции сложных организмов в живой природе является
возникновение особых функциональных органов, назначение которых – обеспечить
реализацию необходимого, жизненно важного действия. Например, к
функциональным органами относятся инстинкты. Более того, потребность в
определенных функциях в ходе адаптации и выживания вида становится
потенциальным фактором структурного изменения тех или иных систем организма.
Например, в процессе эволюции человека подобные изменения могла приобрести
гортань с тем, чтобы обеспечить возможность речевой функции, которая в
человеческом сообществе играет роль наиболее эффективного способа
коммуникации и управления поведением.

В функциональном подходе целесообразность и управление рассматриваются в
качестве фундаментальных оснований живых (в общем случае, организмических, или 
органичных) систем. Эти основания образуют два полюса существования такой
системы и оказываются так тесно переплетенными, что отдать предпочтение какомуто из них невозможно. Органичная система строится по принципу дополнительности.
Любой элементарный процесс управления предполагает цель, а целесообразное
поведение, так или иначе, управляемо. Поскольку управление всегда имеет в
основании некоторую информацию, информационные качества, связанные с
потенциальными возможностями в адаптации системы, для органичной системы
определяют ее жизненный горизонт.

Со стороны конкретных наук функциональный подход опирается на
теорию информации, оперирующую понятием абстрактного информационного
процесса и теорию управления, оперирующую понятием автономного процесса
управления, который строится на основе обратной связи. Автономный процесс
управления (самоуправление) и первичная информация в системе - две взаимно
дополнительные сущности каждого элементарного действия органичной
системы. Главное положение функционального подхода – нет информации вне
управления и наоборот. Таким образом, информацию можно считать и
предпосылкой процесса управления и его результатом.

Общие законы, сформулированные в кибернетике, относятся к надежности
управления действиями сложных систем.

1) Закон разнообразия: эффективное управление системой возможно
только в том случае, если разнообразие управляющей системы выше
разнообразия управляемой.

2) Закон сложности: чем выше сложность системы, тем менее она
управляема. Поэтому существует порог сложности системы, за которым
тотальный контроль поведения системы становится невозможным из-за
нарастания системных эффектов.

Теория систем и системный подход. Концептуальной базой кибернетики
вступает теория систем, в которой разрабатываются принципы системного
анализа явлений, объектов и событий на основе представления об абстрактной
системе (простой и сложной). Теория систем как междисциплинарная
общенаучная концепция сложилась во второй половине ХХ в. Австрийский
ученый Людвиг фон Берталанфи (1901-1972) в 30-40 гг. попытался дать
определение понятия системы в его общем (общенаучном) значении,
сформулировал общие принципы системного подхода и успешно применил
этот подход в изучении биологических процессов. После второй мировой
войны он выдвинул идею разработки общей теории систем. Его теоретическая
программа включала:

1) выявление общих принципов и законов поведения систем независимо
от их происхождения, природы составляющих элементов и отношений между
ними;

2) выявление и формулирование объективных законов для
биологических и социальных явлений;

3) синтез современного знания на основе сходства законов,
описывающих разные сферы жизни природы, человека и общества. 

Общая теория систем, по замыслу Берталанфи, должна была стать наукой
о системах любых типов. Эта программа не реализована и в начале следующего
века. Главная трудность в создании общей теории систем - различие
общетеоретического и конкретного знания. Стремление к универсальности в
описании систем приводило к абстрактности, более характерной для
философии, чем для естествознания.

Наибольшее развитие во второй половине века получили прикладные
математические теории описания систем, использующие аппарат теории
множеств. Прикладные теории составляют и в настоящее время
концептуальную основу моделирования поведения систем и процессов. Однако
усилия Берталанфи не пропали даром. Благодаря заявленной программе
возникли новая познавательная стратегия в естествознании, получившая
название системного подхода, новые междисциплинарные (общенаучные)
методы исследования, новый системный стиль мышления.

В конце века системный подход применяется практически во всех науках
(естественных и социогуманитарных), становится общенаучной методологией.
Еще одно достижение несостоявшейся теории систем связано с формированием
особого класса общенаучных понятий, которые играют коммуникативную роль
в развитии современного научного дисциплинарного и междисциплинарного
знания, образуя своеобразный концептуальный мост между науками,
использующими различные языки описания природных явлений.

Применение понятий системного подхода к анализу прикладных проблем в
самых разных сферах привело к выделению системного анализа в отдельную
концептуальную и предметную область. В предмет системного анализа входит
не только изучение объекта, явления или процесса, но главным образом
исследование ситуаций, прежде всего проблемных. Одной из главных задач
системного анализа выступает постановка цели или задачи, определяющей
процесс управления и самоуправления поведением сложной системы.

Теоретическую основу системного анализа составили: кибернетика, теория
информации, теория игр и принятия решений, анализ систем голосования.34
Проблемы развития системного анализа связаны с заимствованием конкретных
приложений и инструментария из смежных областей уже сложившихся в науке,
в частности в кибернетике и прикладной математике. Основные понятия
системного анализа совпадают с аппаратом теории систем: система,
целостность, элемент, структура, эмерджентность.

В общенаучном контексте система определяется как множество связанных
между собой элементов, которое рассматривается как целое. Первичным
выступает математическое понятие множества, которое допускает возможность
применения к изучению и описанию системы различных операций. Однако в
системном анализе подчеркивается организованность, которая позволяет
говорить и исследовать структурно-функциональную архитектуру системы,
которая не присутствует в математических множествах (чисто
количественных). «Архитектура» становится фундаментальным понятием,
раскрывающим смысл системности с точки зрения структурной и
функциональной организации. Примечательно, что греческий эквивалент
термина «система» - «композиция». В качестве примера, указывающего на
значение понятия архитектуры системы можно указать на очевидное различие
между кучей камней и зданием, между органическими молекулами и живой
клеткой. Термин «архитектура», взятый из области искусства, оказался очень
конструктивным в области системного анализа, поскольку наилучшим образом
позволил развести низкие уровни организованности систем (приближавшиеся к
простой сумме элементов) и высокоорганизованными.

Значение организации и организованности в начале XX в. выделил А.А.
Богданов (Малиновский), опубликовавший труд «Тектология», в котором
обосновывал новую науку об организации в обществе. Принцип структурной
организованности в первой половине века развивается также в
гештальтпсихологии, которую Берталанфи считал одним из предшественников
теории систем. В гештальтпсихологии разрабатывалась динамическая теория
мышления, в основании которой лежало понятие структуры мыслеобраза, или
гештальта (Gestalt – нем. образ) и процесса его переструктурирования в ходе
постановки и решения проблем.

В современной системе междисциплинарных знаний, сложившихся на базе
системного подхода, выделяют: техническую кибернетику (изучающую
созданные человеком, искусственные системы), экономическую кибернетику
(исследующую приложения общих законов об управлении системами к
экономике), биокибернетику (исследующую живые организмы, поведение и
мышление человека, его высшую нервную деятельность, субстратом которой
рассматривается мозг и его тонкие нейрофизиологические структуры),
интеллектуально-информационную технологию, когнитивистику (в основании
которой лежит теория искусственного интеллекта, обобщающая исследования
мышления и интеллектуального действия на основании выбора в пространстве
возможных решений, трактовки мышления как принятия решения в
пространстве выбора, общей компьютерной парадигмы интеллектуального
действия, информационной парадигмы в описании деятельности мозга и
мышления).

В конце века интенсивно развиваются прикладные методы системного
анализа. В 90-х гг. методология прикладного системного анализа
распространяется в сфере социальных исследований. Английские ученые
Р.Флад и М.Джексон предложили общую классификацию методологий
прикладного системного анализа, охватывающую и область проблем 
социальной сферы. Помимо деления всех систем на простые и сложные,
которые они различили по степени зависимости (независимости) от внешней
окружающей среды, а также по способности эволюционировать, они ввели
критерии участия элементов и подсистем (групп и индивидов) в организации
деятельности системы. Согласно этому критерию были выделены:

1) унитарные системы с высокой степенью согласия относительно
целей, ценностей и установок;

2) плюралистические системы, в которых интересы и ценности
различаются, но согласовываются посредством компромиссов и выработки
приемлемых решений;

3) системы с принуждением (принудительные системы), в которых
различие ценностей, целей и установок приводят к конфликтам и навязыванию
решений.

Такая классификация определяет шесть типов систем, поскольку каждый
из перечисленных типов может относиться и к простой и к сложной системе.

Методология исследования унитарных систем объединила методы,
ориентированные на исследование систем с четкой, неизменной структурой.
Применение формализованных количественных методов описания поведением
системы в этом случае наиболее эффективно. К унитарной методологии
прикладного системного анализа относят: исследование операций,
системотехнику (для простых систем), методологию жизнеспособных систем,
предложенную С. Биром (для сложных систем).

Методы исследования операций имеют четкое приложение в решении
задачи оптимальной организации производственных процессов. Нахождение
оптимальных (эффективных) решений ведется на базе математики и
компьютерной техники, поэтому исследование операций рассматривается как
раздел информатики.

Представление об унитарных системах, которое возникает в 70-х гг. и
опирается на кибернетический (функциональный, алгоритмический) подход, в
применении к анализу социальных систем получило название жесткого
системного подхода. Несколько десятилетий спустя в общую системную
методологию вносятся изменения, которые позволяют создать более
адекватные методы описания социальных систем. Этому способствует
формирование представления о «мягких» системах, основной особенностью
которых является слабая структурированность и плюрализм внутренних
установок. Принципы исследования таких систем были предложены
У.Черчменом и представляли собой коммуникативную стратегию принятия
коллективного решения в виде деловой игры, общая организация которой
определяется установками на участие в процессе решения всех
заинтересованных сторон, учет различных точек зрения, их интеграцию и
синтез на уровне общего плана решения проблемы, а также обучение.

Большое влияние на развитие прикладного системного анализа оказали
труды американского ученого Р. Акоффа, который проанализировал эволюцию
организаций в XX в. и ввел историческую координату в характеристику
социальных систем. Он пришел к выводу, что до 60-х гг. социальные системы
можно было рассматривать как унитарные, жесткие «машины», служащие
создателям и собственникам, либо как организмы, в которых цели подсистем
подчинены общей цели системы. После 60-х гг., когда персонал становится
более образованным и склонным к самостоятельному принятию решений, цели
подсистем далеко не всегда совпадают с общей целью. В этих условиях более
адекватной методологией системного анализа социальной системы выступает
интерактивный подход, в котором развитие системного подхода в
существенной мере оказываются связанными с коммуникативными моделями
поведения и стилем мышления. И в этом варианте системного подхода
информация – главный ресурс управления.

Исходный смысл термина «информация» связан со сведениями, сообщениями и
их передачей. Клод Шеннон в 1948г. предложил количественный способ
измерения потока информации, содержащегося в одном случайном объекте на
основе двоичной системы. С тех пор количество информации измеряется в
битах и байтах (байт - набор из 8 бит, т.е. количество информации в трех
двоичных разрядах).

Первое научное расширение понятия информации дают математические
«теории информации» (комбинаторная, топологическая, семантическая), в которых
информация предстает измеримой величиной. Но до сих пор в определении этого
понятия ученые не достигли согласия.

К свойствам информации относят:

- способность управлять физическими, химическими, биологическими и
социальными процессами (там, где есть информация, действует управление, а там,
где осуществляется управление, непременно наличествует и информация);

- способность передаваться на расстоянии (при перемещении носителя
информации).

- способность подвергаться переработке.

- способность сохраняться в течение любых промежутков времени и
изменяться во времени.

- способность переходить из пассивной формы в активную (например, когда
извлекается из «памяти» для построения тех или иных структур - синтез белка,
создание текста на компьютере и т.д.).

Можно выделить три основных подхода в интерпретации его содержания.

1) Физический подход представляет информацию как негэнтропию.
Понятие энтропии в физике – это мера нарастания хаоса (беспорядка),
следовательно, информация – это мера нарастания организованности
(Л.Бриллюэн);

2) Технический, собственно кибернетический подход представляет
информацию как меру разнообразия (У.Р.Эшби);

3) Философский подход представляет информацию как отраженное
разнообразие (А.Д.Урсул.) или функциональное отражение.

Таким образом, в современной системе научных знаний информация
представляет собой феномен, который не имеет четкого определения. Согласно
Н.Винеру, информация – не материя и не энергия. Общая тенденция в истолковании
этого феномена в конце века – переход от конкретных математических дефиниций
информации как неопределенности, вероятности, алгоритма к мировоззренческому
контексту, выделяющему категории: отражение, различие, отношение, взаимосвязь.

Познавательная стратегия информационного подхода.

Современная наука выделяет информационные процессы в качестве
фундаментальных процессов, наравне с физико-химическими. С этой точки зрения
информация составляет главный ресурс не только общества, но лежит в основании
всего сущего. Например, в качестве фундаментальных характеристик физического
вакуума современная наука рассматривает его информационные характеристики.

Исходные мировоззренческие положения информационного подхода в
современном естествознании:

- универсальность информационных процессов;

- фундаментальность единства материи – энергии – информации в
основании наблюдаемого мира и его эволюции.

Эти положения создают концептуальную базу в построении новой
«информационной картины мира» в конце XXв. В стремлении создать единую
теорию универсума современная наука (в частности физика) приходит к
представлению об универсальном поле сознания, к описанию характеристик
которого можно применить аппарат квантовой механики.39 Примером может
служить концепция Семантической Вселенной Л.В.Лескова, в которой за
исходное берется понятие универсального оператора смысла (т.е. аналог
сознания) и информация, содержащаяся в знаке. Антиэнтропийная
направленность универсального оператора (сознания) может проявиться только
в том случае, если существует внешний по отношению к нему источник
негэнтропии в виде информационного поля. А реально существующий в мире
референт информационного поля – это состояние физического вакуума,
названное мэоном. Формулируя мэон-био-компьютерную концепцию Вселенной
(МБК-концепцию) Л.В.Лесков и основания интеллекта сводит к
информационным свойствам Вселенной и физического вакуума.
Информационные качества системы, в частности физического вакуума,
получают базовое мировоззренческое значение. Объяснение механизма
эволюционной динамики связывается с семантическим давлением на систему,
способным вызвать ее разрушение.

В концепции «Биоэнергоинформатики» В.Н.Волченко постулируются три
проявления Вселенной: информация (сознание), энергия (материя), смысл. В
этой модели Вселенной, наряду с информационно-энергетическим
пространством, существует семантическое пространство, в котором заложены
все смыслы эволюции. Все системы несут информацию и могут
рассматриваться как живые, обладающие неким эквивалентом сознания.
Информационно-энергетическое пространство Вселенной образует Мир
Сознания, единый для вещественных и чисто информационных систем.
Потенциальный информационно-энергетический барьер, существующий между
вещественным и «тонким» миром преодолевается благодаря «туннельному
эффекту».

Информационные модели объяснения распространяют представление об
информационной причинности на все явления микро-, макро- и мега мира, а
также на все биосферные, химические, психические, сознательные, культурные
и социальные явления. На этой базе утверждается информационная парадигма,
выступающая в качестве концептуальной основы новых проблемных областей
исследования, в частности, в теоретической биологии, биохимии, биофизике.

Под информационной причинностью понимается закономерность действия
системных требований, которая имеет кодовый характер и проявляется в
запуске последовательности действий (или программы действия), приводящих
к определенному результату. Суть информативного кода нормирование
некоторого потенциального жизненного пространства системы. Такого рода
системная причинность, выраженная кодом, указывая неявные границы
действий, задает параметры самоопределения системы.

Базовые понятия информационного подхода вводят новые общенаучные
концепты, обладающие эвристическим потенциалом.

Информационные качества системы определяют потенциальные
возможности ее органической адаптации, указывают ее жизненный горизонт.
Представление об информационных качествах системы связывается с
количеством снятой неопределенности, что может быть выражено
математически. Предпосылкой такого представления служит взаимосвязь
системы со средой. Сложная динамическая системы (в частности биосистема)
всегда погружена в некую жизненную среду (не только природную, но и
информационную). Ситуативная связь с жизненной средой жизни и ее
регуляция выражается понятиями адаптации и целесообразности действия.

Информационный процесс понимается как некий обобщенный процесс,
предполагающий выбор и обеспечивающий формирование структур подобных
знанию (или когнитивных - познавательных) в качестве базы прогнозирующего
целесообразного адаптивного действия. Выбор – это не сам процесс, а его
завершение, результат действия. В естествознании процесс – это изменение
системы во времени («движение системы»). При этом информация как таковая
отсутствует. Не каждый процесс завершается выбором, поэтому
информационные процессы характерны только для определенного класса
систем и процессов.

С понятием микроинформации соотносится выбор, который не
запоминается. С понятием макроинформации – выбор, который запоминается и
становится базой для генерации новой информации, для прогноза и
саморегуляции системы.

Информационная система – система, способная воспринимать,
запоминать, генерировать макроинформацию, извлекать ценную информацию и
использовать для достижения своих целей.

Информационная среда в широком смысле соотносится с объективным
существованием пространства потенциального выбора действий
(потенциальных возможностей в прогнозировании действия). Информационные
среды могут быть внешними и внутренними. Иерархия информационных сред,
например, в социальном пространстве предполагает сложную семантику,
которая играет ключевую роль в формировании жизненного мира индивидуума.
Достаточно просто перечислить семантические (смысловые) уровни, к которым
можно отнести архетипы подсознания, культурные смыслы, социальные
нормы, языковые традиции, интеллектуальные и профессиональные среды,
чтобы убедиться в сложной онтологии жизненного мира на уровне
информационной среды.

Информационный подход определяет методологию исследования и
обоснования результатов в проблемно ориентированных дисциплинах,
соединяющих традиционно различные концептуальные области, предметом
которых выступают биологические системы. Ключевое понятие информация в
контексте теории динамических систем (биосистем) определяется как
случайный и запомненный выбор одного варианта из нескольких возможных и
равноправных. Таким образом, под информацией подразумевается только
зафиксированная выбором информация.43 Что в известной мере совпадает с
представлением о некотором подобии знания и структуре знания,
составляющей базовый концепт когнитивного подхода, заявленного уже в
проблемной области искусственного интеллекта. Общая методологическая
платформа для физиологии, нейропсихологии, лингвистики, антропологии,
информационной технологии во взгляде на когнитивный процесс –
представление о некоторой единой архитектуре поведения человека,
животного, машины, основание которой связывается с обработкой
информации.

Исторически развивающиеся системы представляют собой более
сложный тип объекта даже по сравнению с саморегулирующимися
системами. Последние выступают особым состоянием динамики
исторического объекта, своеобразным срезом, устойчивой стадией его
эволюции.

Развитие это – направленное, качественное, необратимое изменение
системы, вызванное ее внешними и внутренними противоречиями.
Необратимость изменений понимается при этом как появление у системы
новых возможностей, не существовавших ранее.

Саморазвивающиеся системы характеризуются кооперативными
эффектами, принципиальной необратимостью процессов. Взаимодействие с
ними человека протекает таким образом, что само человеческое
действие не является чем-то внешним, оно включается в систему и тем
самым видоизменяет каждый раз поле ее возможных состояний.
Включаясь во взаимодействие, человек уже имеет дело не с жесткими
предметами и свойствами, а со своеобразными комплексами
возможностей. Перед ним в процессе деятельности каждый раз возникает
проблема выбора некоторой линии развития из множества возможных
путей эволюции системы. Причем сам этот выбор необратим и чаще всего не
может быть однозначно просчитан.

В естествознании первыми фундаментальными науками,
столкнувшимися с необходимостью учитывать особенности исторически
развивающихся систем, были биология, астрономия и науки о Земле. В
последние десятилетия на этот путь вступила физика. Представление об
исторической эволюции физических объектов постепенно входит в
картину физической реальности через развитие современной космологии,
разработку идей термодинамики неравновесных процессов и синергетики.

Синергетика возникла в 1960-х гг. как физико-математическая теория так
называемых диссипативных систем, то есть систем открытых,
взаимодействующих с окружающей средой и сохраняющих свое
существование благодаря постоянному обмену с ней веществом и
энергией. Начало ей положили работы И.Пригожина (Бельгия), а
название «синергетика» дал Г.Хакен (Германия). Были обнаружены
универсальные свойства и закономерности самоорганизации, имеющие
место в самых разнообразных системах. Синергетика, по мнению
сторонников общей теории систем, превращается в междисциплинарное
научное направление, которое становится источником
философско-методологических выводов и обобщений.

Идеи эволюции и историзма становятся основой синтеза картин
реальности, вырабатываемых в фундаментальных науках. Историчность
системного комплексного объекта и вариабельность его поведения
предполагают широкое применение особых способов его описания и
предсказания его состояний – построение сценариев возможных линий
развития системы в точках бифуркации. С идеалом строения теории как
аксиоматически дедуктивной системы конкурируют теоретические
описания, основанные на примененииметода аппроксимации,
теоретические схемы, использующие компьютерные программы.

Аппроксимация (от лат. approximare – приближаться) –
приближенное выражение каких-либо величин через другие, более
известные величины. Процессы аппроксимации приобрели особо актуальное
значение в связи с ростом числа исследований сложных систем.
Аппроксимированная модель – упрощенная модель какой-либо сложной
системы. Эрнест Резерфорд любил проверять своих новых сотрудников
на способность выстраивать гипотетические модели, или, как говорят,
«прикинуть порядок цифр», задавая вопросы типа: «Если в Лондоне
живет девять миллионов человек, то сколько среди них настройщиков
роялей?», – и просил дать ответ через 10 секунд.



\end{document}
