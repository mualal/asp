\documentclass[exam_answers.tex]{subfiles}

\fontsize{14pt}{14pt}\selectfont

\begin{document}

\renewcommand{\baselinestretch}{\blch}
\sublinksectionold{\normalsize (10) Мировоззренческое значение общей теории относительности.}

В общей теории относительности Эйнштейн формулирует принципы
теории гравитации как теории поля, которая становится основанием новой
физической (не механической) картины мира. Выражая кратко
мировоззренческое значение общей теории относительности, Эйнштейн
заметил, что «раньше полагали, что если бы из Вселенной исчезла вся материя,
то пространство и время сохранились бы, теория относительности утверждает,
что вместе с материей исчезли бы также пространство и время».

Основание общей теории относительности (ОТО) составляют:

1) Принцип постоянства скорости света (максимальная скорость
распространения волн любой природы равна скорости света в вакууме);

2) Принцип эквивалентности (инертная и гравитационная массы
идентичны друг другу);

3) Принцип геометризации физических взаимодействий: действие
гравитации имеет характер искривления пространственно-временного
континуума в зависимости от распределения материи. Универсальной
константой этой связи пространства, времени и материи является
гравитационная постоянная $\gamma$ = 6,672*10\^{-11} м3/(кг·с2);

4) Принцип локальности метрики пространства-времени.

Первым доказательством общей теории относительности было
обнаружение отклонения света вблизи края Солнца, которое наблюдалось во
время полного затмения в 1919г. Целью проведенного в Африке эксперимента
было точное измерение положения звезд до и после затмения. Проверялись три
версии: расчетные отклонения измерений по Ньютону, по Эйнштейну или
отсутствие отклонений. Подтвердились расчеты Эйнштейна, согласно которым
должно было наблюдаться большее искривление светового луча.

Второе подтверждение ОТО - совпадение наблюдаемого и расчетного
смещения перигелия планеты Меркурий. Эллиптическая орбита Меркурия
медленно поворачивается. На основании закона тяготения Ньютона этот факт
объясняется действием других планет. Уравнения Эйнштейна предсказывают
вращение эллиптической орбиты даже в отсутствии других планет. В
отношении Меркурия расчетная орбита по Эйнштейну ближе к наблюдаемой.

Третье доказательство ОТО давало изменение длины волны света в
гравитационном поле. В сильном поле тяготения ритмические процессы
(колебания атомов) должны идти с меньшей скоростью, чем на Земле, что
должно привести к более длинным волнам (покраснению) в спектре излучения
Солнца. Этот факт наблюдался, но не принимался в качестве подтверждения
ОТО, пока в 60-х гг. не измерили красное смещение одной из линий
поглощения стронция в спектре Солнца. К этому времени имелись также
наблюдения за спутником Сириуса, создающим красное смешение в тридцать
раз большее, чем Солнце.

Самое эффектное подтверждение ОТО было получено с использованием
эффекта Мёссбауэра в лабораторных условиях. Английские физики
обнаружили, что ядерные часы, помещенные на краю быстро вращающегося
диска диаметром 15см, замедляют свой ход.

Мировоззренческие следствия общей теории относительности
представлены следующим утверждениями.

1. Гравитация не сидит ни в одном из тел, ни между ними как
необъяснимое нечто, действующее мгновенно. Она сводится к геометрическим
свойствам пространства.

2. Свет и все тела движутся по геодезическим (мировым) линиям, вид
которых зависит от внутреннего строения пространственно-временного
континуума.

3. В общем случае четырехмерный континуум искривлен, и его метрика
зависит от распределения масс. Большие массы сильнее искривляют
пространство, малые массы лишь слабо деформируют его.

Общая теория относительности, согласно которой структура пространства
и времени целиком определяется распределением и эволюцией материи, легла в
основание астрофизики и эволюционной космологии, начало которой положила
теоретическая модель нестационарной Вселенной Александра Фридмана.
Идея эволюции Вселенной вытекала из решения уравнений ОТО, однако
противоречила космологической концепции самого Эйнштейна.



\end{document}
