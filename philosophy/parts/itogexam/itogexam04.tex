\documentclass[exam_answers.tex]{subfiles}

\fontsize{14pt}{14pt}\selectfont

\begin{document}

\renewcommand{\baselinestretch}{\blch}
\sublinksectionold{\normalsize (4) Базовые модели естественнонаучного объяснения.}

Базовые модели объяснения в естествознании опираются на принцип детерминизма.\\

Детерминизм -- мировоззренческая позиция, в которой постулируется причинно-следственная связь природных явлений, не всегда явно представленная в наблюдаемых событиях.

Принцип всеобщей причинной связи был сформулирован в атомистическом учении Демокритом в жесткой форме,
поскольку он утверждал однозначную связь причины и следствия, отрицал случайность в мире.

Элемент случайности был внесен в концепцию атомизма позже Эпикуром.

В новоевропейской классической науке, эта установка получила подкрепление и была обобщена Лапласом.

"<Демон Лапласа"> -- символ и метафора механистического детерминизма, выделившего универсальность силового (динамического) принципа причинно-следственной связи, который позволяет точно рассчитать все состояния объекта.

Для "<Демона Лапласа"> мир прозрачен, предсказуем, в нем нет случайностей.

Первая форма детерминизма в философии, жёсткий детерминизм, механистический, лапласовский, динамический представляют собой тождественные понятия.

Вторая форма детерминизма -- вероятно-статистическая, допускающая случайность в систему причинения.
Появляется с развитием термодинамики, статистической физики и квантовой механики в начале XX в.,
выделившими приоритет статистического закона в объяснении причинно-следственных связей.
Неопределённость соответствует самой объективной физической реальности.

Метафора этой формы детерминизма – "<Демон Максвелла">, разделяющий горячие и холодные молекулы в сосуде,
что позволяет ему нагреть правую часть сосуда и охладить левую без дополнительного подвода энергии к системе.

Третья форма детерминизма -- вероятностный детерминизм (конец XX века).
Холистский детерминизм.
Фундаментальность вероятностных характеристик любого объекта.
Роль причины может играть случай (случайная флуктуация может изменить ход).
Опирается физику высоких энергий и элементарных частиц.
Метафора Пригожина "<Стрела времени">.

В современной системе естествознания методологические аспекты
научного исследования наиболее явно выражены проблемами адекватности
базовой модели причинного объяснения. Выделяют линейную, статистическивероятностную и нелинейную модели объяснения, которые, отличаясь формой
детерминизма и приоритетного закона, соотносятся с тремя историческими
типами научной рациональности: классической (механизм), неклассической
(релятивизм) и постнеклассической (холизм).

Постнеклассическую методологию естествознания характеризует
расширение понятия причинности.
Современное естествознание оперирует уже,
по крайней мере, пятью моделями объяснения причинных связей:
динамической ("<детерминистской"> -- на основании действующей силы),
статистической ("<индетерминистской"> -- включающей случайность в цепь
причин и следствий), телеономической (рассматривающей движение системы
к конечному результату), телеологической (целевой), синхронической
(выделяющей фундаментальность повторяющегося совпадения событий).

Телеономическая модель объяснения предполагает разные пути развития
(движения) системы к конечному состоянию (например, скатывание шарика с
горки, которое объясняется законом сохранения энергии). Другой пример --
действие или развитие по некоторой программе (инстинкт, генетический код).
Представление о телеономических процессах распространилось в
естествознания благодаря биологу Эрнсту Майру, который выделил общность и
различие телеоматических, телеономических и телеологических процессов.
Объединяет три типа процессов направленность к некоторому конечному
состоянию.
Телеоматические процессы пассивны, автоматически регулируются
внешними силами или обстоятельствами (например, падение камня в пропасть
под действием гравитации).
В телеономических процессах достижение
конечного состояния контролируется встроенной в них программой, конечное
состояние при этом оно не является действующей причиной.
Например, развитие организма в соответствии с генетической программой.
В телеологических процессах конечное состояние является действующей
причиной. Телеологические процессы не просто направлены к конечному
состоянию, они целенаправленны.

Истоки представления о синхронизме как новом типе связей (в отличие от
необходимо причинного и напротив -- случайного) усматриваются в
аналитической психологии К.-Г. Юнга.
Исследуя психику человека, он пришел к выводу, что понятий причинности и случайности недостаточно для ее
объяснения, решающее значение имеет повторяющееся совпадение событий.
Синхроническая модель объяснения разрабатывается в современной космологии (А.Линде).
В экологии и в социобиологии эта модель представлена принципом коэволюции.



\end{document}
