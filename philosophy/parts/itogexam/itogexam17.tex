\documentclass[exam_answers.tex]{subfiles}

\fontsize{14pt}{14pt}\selectfont

\begin{document}

\renewcommand{\baselinestretch}{\blch}
\sublinksectionold{\normalsize (17) Научная картина мира и философские проблемы естествознания. Проблемы физической картины мира (механической, электродинамической, квантовой).}

Научная картина мира -- умозрительная система представлений, в которой
соединяются естественнонаучный и мировоззренческий (философский) уровни
знания.

Умозрительные картины мира, выделяющие естественные первоначала и
причины явлений, складываются в Античности. Создание и обоснование
такой картины – главная цель натурфилософии.

В XVII в. естественнонаучная и натурфилософская картины мира не
совпадают.

Научная картина мира абстрагируется от религиозных, философских,
мифологических, житейских представлений о мире, стремится представить мир
и его законы независимо от сознания людей и духовных предпочтений. Все же
она не свободна от мировоззренческих, религиозных, познавательных установок
своей эпохи.

Содержательно научная картина мира определена концепциями
естествознания, раскрывающими природу материи, пространства,
времени, движения, взаимодействия.

Научную картину мира как признанную сообществом теоретическую
модель характеризуют:

- натурализм (отрицание существования сверхъестественных сил),

- связь с физическими представлениями о природе материи и принципах
взаимодействий,

- обоснованность;

- эмпирическая проверяемость (или возможность опытного опровержения);

- историчность (содержание НКМ постоянно обновляется).

Особенность современной науки - наличие разных моделей
реальности, определенных теоретическими принципами
соответствующих областей знания (физики, химии, биологии,
кибернетики и др.).

Научная картина мира (НКМ):

- синтезирует достижения в разных предметных областях -
мировоззренческая функция,

- играет роль неэмпирического критерия обоснования научного статуса
выдвигаемых проблем и гипотез.
Теоретические построения в той или иной области всегда проходит двойную
проверку: на эмпирическую проверяемость фактами и соответствие
признанной НКМ.

- Коммуникативная функция НКМ связана с распространением новых идей
и теоретических установок в самых разных интеллектуальных слоях
общества.
Популяризация сложных построений современной науки разворачивается на
уровне общих представлений о мире. Ведущую роль играет философия.
Начиная с Галилея и Ньютона, фундаментальные основания для синтеза
знания в общей картине мира давало развитие физических теорий. Однако в
конце XX в. в интеграции знания о мире фундаментальное значение
приобрели нефизические принципы системности, самоорганизации,
эволюции.

Эволюция картины мира в европейской истории соотносится с
научными революциями, которые кардинально меняют
мировоззренческую и методологическую парадигму в развитии
знания о мире.

В развитии естествознания можно выделить три больших
исторических периода, которые различаются научной картиной мира.

Механическая картина (XVII – XIXв.), в основании которой лежит
классическая механика Ньютона, соответствует периоду классической
науки. Стиль научного мышления, определенный установками
механической картины мира, - классический идеал научной
рациональности.

Физическая картина, в которой прослеживаются два этапа:
электродинамический и квантово-механический (XXв.), соответствует
периоду неклассической науки.

Синтетическая картина (конец XX в.), в основании которой лежат
принципы системности, самоорганизации, глобального эволюционизма,
соответствует периоду постнеклассической науки.

\includegraphics[width=\textwidth, page=5]{lecture07.pdf}

\includegraphics[width=\textwidth, page=6]{lecture07.pdf}

\includegraphics[width=\textwidth, page=8]{lecture07.pdf}

Проблемы
электродинамической
картины мира были
связаны с объяснением
строения атома и его
устойчивости.

Выяснилось, что электромагнитных сил недостаточно для
соединения и удержания вместе элементов ядра. 

Проблема строения материи вылилась в исследование
элементарных частиц, которое привело к открытию микромира.

Классическая физика, включая электромагнитную теорию,
оказалась не пригодной для объяснения явлений микромира.

Электродинамическая картина мира сыграла свою конструктивную
роль в становлении научной картины мира, выявив
фундаментальность статистических закономерностей.

Принцип причинно-следственной связи в электродинамической
картине допускал случайность в ходе развития событий.

Исследование природы элементарной частицы привело к
представлению о корпускулярно-волновом дуализме микрочастицы
(Луи де Бройль), которое отразило неопределенность ее природы и
роль случайности в описании ее проявлений.



\end{document}
