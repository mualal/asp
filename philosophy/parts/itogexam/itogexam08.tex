\documentclass[exam_answers.tex]{subfiles}

\fontsize{14pt}{14pt}\selectfont

\begin{document}

\renewcommand{\baselinestretch}{\blch}
\sublinksectionold{\normalsize (8) Идеалы теоретического естествознания. Принципы построения логически строгой теории. Высшая математика и естествознание.}

Новый стиль научного мышления, который формируется в естествознании
начала XXв., характеризуется особым содержательно-формальным подходом к
описанию и обобщению экспериментальных фактов. В научном сообществе
формулируются требования к логической строгости выдвигаемых концепций и
теорий, формулировке вводимых понятий, постановке научных проблем,
способам обоснования и проверки гипотез.

Стандарт логически строгой теории оформился в начале века в
математике. Его содержание раскрывают следующие положения.

1) Любая (математическая и физическая) теория имеет дело с одним
или несколькими множествами объектов, соответствующим образом
идеализированных и формально математически представленных и связанных
между собой некоторыми отношениями, которые также представлены
формально (например, в виде функции).

2) Основные (фундаментальные) свойства объектов и принципы
отношений формулируются в виде аксиом (в математике), постулатов, законов
или принципов (в физике, например, закон сохранения энергии, принцип
относительности).

3) Теория должна быть применима к любой системе объектов, для
которых фиксируются отношения, удовлетворяющие системе аксиом или
основных принципов, положенной в ее основу.

4) Теория может считаться логически строго построенной, если при ее
развитии все новые объекты, их свойства и отношения между ними, выводятся
формально из аксиом, постулатов или принципов.

В физике начала века оформляется область чисто теоретических
исследований. Ее предметом становится обнаружение и анализ скрытых
фундаментальных свойств и отношений, которые принципиально не
наблюдаются и проявляются только опосредовано (как следствия).

Идеал теоретического естествознания в науке XX в. дает физическая
теория, для которой характерны:

- формальный математический язык описания явлений;

- аксиоматическое основание теории в виде постулатов или
фундаментальных принципов;

- выводное, гипотетико-дедуктивное построение теоретического знания;

- разработка математических моделей, выражающих концептуально
построенное знание.

Геометрические модели, характерные для физики прошлого века,
сменяются формальными, символическими построениями, в которых реальные
процессы мыслятся. Постулаты новых теорий не очевидны. Физическая
реальность предстает в таком виде, что ее понимание и наглядная
интерпретация оказывается очень сложной задачей. 

До Галилея, Декарта и Ньютона, заложивших основы математического
аппарата описания движений, связь математики и естествознания не была
очевидной. Идею о том, что математика выражает реальные отношения и
закономерности, высказывал еще Коперник, но в инструмент
естественнонаучного познания она превращается уже в статусе высшей
математики, язык которой становится языком строгой теории.
Натурфилософское обоснование нового статуса математики дает Декарт,
рассматривая материю как протяженное тело, отождествляя физическое
(реальное, материальное) пространство с протяженностью (абстрактным
пространством). С тех пор физическая реальность раскрывается
математическим описанием процессов в независимых переменных координат и
времени.

В XIXв. стимулом развития высшей математики выступают прикладные
задачи в области механики, геодезии, гидро- и аэродинамики. Оформляются
основные разделы математического анализа, векторной алгебры и
аналитической геометрии, теории сходимости рядов и функций комплексного
переменного.

Выдающуюся роль в становлении высшей математики сыграл директор
астрономической обсерватории, профессор Геттингенского университета Карл
Фридрих Гаусс (1777-1855). Создавая математический аппарат изучения формы
земной поверхности,
он разработал универсальные дифференциальногеометрические методы исследования криволинейных поверхностей.
Другая его работа «Арифметические исследования» (1801) расценивается как начало
современной теории чисел. Гаусс провел первое систематическое исследование
сходимости рядов, ввел геометрическое представление комплексных чисел,
соотнеся их с точками на плоскости. Ему принадлежит открытие
эллиптических функций, а также первые сомнения в отношении априорной
данности пространства, допускающего только одну евклидову геометрию.
Гаусс допускал, что для больших масштабов должна быть другая геометрия.

К средине XIXв. создается теория пределов, и на ее основании методы
исчисления бесконечно малых объединяются в особую теоретическую область
математического анализа. Возникнув на почве прикладных задач
естествознания и техники, дифференциальное и интегральное исчисления
становятся разделом чистой математики, замкнутой на своих собственных
проблемах, далеких от конкретных задач естествознания. В теоретическом
оформлении математического анализа большое значение имели работы
Ж.Фурье (1768-1830), О.Коши, (1789-1857), Н.Абеля (1802-1829), Б.Больцано
(1781-1912), К.Вейерштрасса (1815-1897). В это же время У.Гамильтон (1805-
1865) и Г.Грасман (1809-1877) разрабатывают теорию комплексных чисел,
возникает новая математическая дисциплина – векторное исчисление.

Выдающимся событием в развитии математики стала неевклидова
геометрия, первое публичное изложение которой принадлежит Николаю
Лобачевскому (1792-1856).
В работе «О началах геометрии» он вывел
уравнения, позволяющие представить неевклидово пространство аналитически.
Новая область геометрии впоследствии получила название гиперболической
геометрии. Идея Лобачевского о многообразии геометрических систем, а также
идея о зависимости геометрических свойств пространства от его физической
природы – были величайшим достижением мысли XIXв., которое не было
оценено. Независимо от Лобачевского, спустя два года венгерский математик
Янош Больяи (1802-1860) излагает идею неевклидовой геометрии в работе
«Абсолютная наука о пространстве», которая также не встретила понимания.

Утверждение новых идей в геометрии связано с работами Георга Римана
(1826-1866), который рассмотрел геометрию как учение о непрерывных
N-мерных многообразиях (совокупностях однородных элементов), развил идею
математического пространства, дал общее определение пространства
«многообразия», которое охватывает функциональные и топологические
пространства. В математике возникло новое понятие «риманово пространство»,
которое обобщило (на основании общего свойства – кривизны) пространства
геометрии Евклида и Лобачевского, а также пространства созданной Риманом
эллиптической геометрии. Риман выявил проблему относительности геометрии
к масштабам пространства и свойствам материи, развил учение о кривизне
пространства в отношении реального физического мира. Теория искривленного
пространства с произвольным числом измерений в начале XXв. легла в
основание новой физической теории – теории относительности.

В начале XXв. ученые стремятся овладеть методами математики и
эффективно применить ее средства для выражения физической сущности,
лежащего за пределом наблюдаемого. В связи с этим расширяется и
дисциплинарная область математики. В ее круг входят отношения между
векторами и операторами в функциональных пространствах, разнообразие
пространств любого числа измерений. Расширяется область прикладных
вычислительных методов, возникает область философских проблем, связанная
с переосмыслением исходных положения теории множеств и логических
приемов доказательства.

На рубеже XXв. область математики определялась четырьмя
направлениями: арифметика и алгебра, математический анализ, геометрия,
аналитическая механика и механическая физика. В XXв. она включает:
математическую логику, алгебру, теорию чисел, геометрию, топологию,
аналитическую геометрию, комплексный анализ, теорию вероятностей,
математическую статистику, теорию представлений, вещественный и
функциональный анализ, дискретную математику, комбинаторику,
информатику и теорию групп. В конце XIXв. в области математики работало
около тысячи человек, к последнему десятилетию XXв. - работает около 300
тысяч специалистов.


\end{document}
