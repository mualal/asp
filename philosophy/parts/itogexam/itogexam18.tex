\documentclass[exam_answers.tex]{subfiles}

\fontsize{14pt}{14pt}\selectfont

\begin{document}

\renewcommand{\baselinestretch}{\blch}
\sublinksectionold{\normalsize (18) Идея эволюции и концепция тонкой подстройки в физической картине мира.}

Идея эволюции в физической картине мира получила развитие
преимущественно на уровне космологических моделей строения и
происхождения Вселенной, в которых астрофизические
исследования и расчеты строятся в соответствии с общей теорией
относительности и квантовой теорией.

Установленные факты, подтверждающие эволюцию Вселенной:

- Расширение Вселенной, в соответствии с обнаруженным красным
смещением в спектрах удаленных космических объектов (Э. Хаббл).

- Преобладание вещества в структуре Вселенной, асимметрия между
веществом и антивеществом.

- Однородность и изотропность светящейся материи в масштабе
расстояний 100 мегапарсек.

- Существование реликтового фонового излучения, энергия которого
соответствует температуре порядка 2,7 К.

- Существование галактик и галактических скоплений, имеющих
разный возраст.

- Ячеистая структура Вселенной на метагалактическом уровне.

Попытки увязать идею эволюции и сохранение физического мира,
для которого характерны фундаментальные мировые константы,
привели к концепции «тонкой подстройки Вселенной» и
формулированию нефизического объясняющего принципа,
декларирующего наличие взаимосвязи между параметрами
Вселенной и существованием в ней разума, который получил
название антропного принципа.
\\

Теория относительности и квантовая теория не дают ответа на вопрос о
происхождении наблюдаемых структур Вселенной. Почему возникает
именно такая Вселенная, которая характеризуется именно такими
законами сохранения и ограниченным набором физических констант, -
остается открытыми в современной физике.

Термин «тонкая подстройка Вселенной» подчеркивает роль физических
констант, фундаментальных калибровочных симметрий и асимметрии
физического вакуума (в качестве исходного состояния пра-материи
Вселенной).

Содержание концепции тонкой подстройки определяется положением,
что универсальные физические константы однозначно определяют
(предопределяют) структуру нашей Вселенной.

Основанием концепции тонкой подстройки послужила численная
взаимосвязь параметров микромира (постоянной Планка, заряда электрона,
размера нуклона) и глобальных характеристик Вселенной (ее массы,
размера, времени существования).

Анализ возможных изменений основных физически параметров показал, что
даже незначительное изменения мировых физических констант, приводит к
невозможности существования нашей Вселенной в наблюдаемой форме и не
совместимо с появлением в ней жизни.

В среде физиков возникла идея о существовании некоторого
фундаментального принципа, в соответствии с которым осуществляется
тонкая подстройка Вселенной (А.Эддингтон, П.Дирак, Дж. Барроу, Р.Дикке,
Б.Картер).

Взаимосвязь между параметрами Вселенной и появлением в ней разума
была выражена в формулировании антропного принципа космологии.



\end{document}
