\documentclass[exam_answers.tex]{subfiles}

\fontsize{14pt}{14pt}\selectfont

\begin{document}

\renewcommand{\baselinestretch}{\blch}
\sublinksectionold{\normalsize (7) Философские и теоретические основания химии как предметной области естествознания.}

Кардинальная перестройка химического знания в истории науки была
связана с переходом от рецептурной схемы, определявшей действия алхимика
(ученого, исследователя, врача, металлурга) к теоретической, концептуальной
основе, в становлении которой ключевую роль сыграла натурфилософская
проблема строения вещества.

Французский мыслитель Пьер Гассенди (1592-1655), интерпретируя
забытое наследие античных материалистов, высказывает идею, что Бог создал
определенное количество атомов, отличающихся друг от друга формой,
величиной и весом, из нескольких десятков атомов природа создает великое
множество тел. Крупные соединения атомов, доступные ощущениям он назвал
молекулами. Для обнаружения таких частиц, невидимых глазом, Гассенди
использовал энгиоскоп (микроскоп).

Основанием химии как точной экспериментальной науки стало
представление о корпускулярном строении вещества (corpusculum – лат. тельце,
маленькое тело). Английский ученый Роберт Бойль (1627-1691) в книге
«Химик-скептик» (1661) изложил взгляд на химию как самостоятельную науку,
имеющую независимый от алхимии и медицины предмет, дал первое
толкование понятия «химический элемент», определил корпускулу как простое
тело, которое уже не может быть разделено на другие тела. Элементы –
вещества, которые нельзя разложить, состоят из однородных корпускул. Таково
золото, серебро, олово, свинец. Например, киноварь, которая разлагается на
ртуть и серу, представляет собой сложное вещество. Бойль одним из первых
получил и описал водород, фосфор и некоторые его соединения. Соединив
учение об элементах с атомистическими представлением о строении вещества,
Бойль полагал, что задача химии как экспериментальной науки – выделение
отдельных веществ в чистом виде (химических элементов), установление их
состава и комплекса свойств, которыми оно обладает.

Развитие корпускулярной теории в XVIIв. связано с именем Ньютона,
который немало времени посвятил химическим опытам, исследовал кислоты, 
химическое действие, распад веществ и их образование. Ньютон полагал, что
корпускулы созданы Богом, неделимы и неуничтожимы. В его теории строения
вещества соединение корпускул происходит за счет сил притяжения, которые и
определяют химическое сродство разных веществ.

Положения корпускулярной теории строения вещества, которые
полностью признаны современной наукой, сформулировал М.В.Ломоносов:

- все вещества состоят из корпускул (мельчайших частиц);

- корпускулы находятся в постоянном, беспорядочном движении;

- корпускулы взаимодействуют между собой.

Факт движения мельчайших частиц вещества был экспериментально
подтвержден английским ботаником Р.Броуном (1773-1858).

Центральная проблема химии XVIIIв. – процесс горения. Для его
объяснения выдвигается теория флогистона (невесомой субстанции,
содержащейся в каждом горючем теле, которая утрачивается при горении).
Лавуазье показал, что явления горения и обжига происходят при наличии
«чистого воздуха» и объясняются гораздо проще без флогистона. В 1769г. он
опубликовал «Начальный курс химии», где систематизировал накопленные к
тому времени химические знания, изложил кислородную теорию горения, дал
определение элемента и классификацию простых веществ.

К концу XVIII века химия из совокупности множества не связанных друг с
другом рецептов, превратилась в последовательную систему знания о строении
и свойствах веществ (простых и сложных). Был сформулирован закон
сохранения массы вещества при химических реакциях (М.В.Ломоносов –
1756г., А.Л.Лавуазье – 1789г.): масса веществ, вступающих в химическую
реакцию, равна массе веществ, образующихся в результате реакции.

Из закона сохранения вещества вытекало, что вещество нельзя создать из
ничего, и нельзя уничтожить совсем. Закон сохранения вещества Ломоносов
связывал с законом сохранения энергии. Количественным выражением закона
сохранения энергии при химических реакциях стал тепловой баланс.

Оформление химии в классическую естественнонаучную дисциплину,
предметом которой является исследование природных элементов и их
соединений, связано с развитием атомно-молекулярного учения (химической
атомистики). Первый шаг к созданию этого учения сделал учитель из
Манчестера Джон Дальтон. Впервые положения атомистической теории
Дальтона были заявлены в лекции «Об абсорбции газов водой и другими
жидкостями», которую он прочитал 20 октября 1803г. в литературно-философском
обществе Манчестера. Дальтон строго разграничил понятия атом
и молекула, которую называл сложным, или составным атомом, подчеркивая,
что эта сложная частица является пределом химического деления
соответствующих веществ. Состав вещества однороден в отношении молекул,
свойства веществ определяются свойствами молекул. Корпускулярное учение о
строении вещества приобрело современный понятийный аппарат. Появилось
понятие атомного веса химического элемента. Было проведено разграничение
между строением химического элемента (зависящим от атомного веса) и
молекулярным строением вещества, между свойствами атомов и молекул.

В 1804г. английский химик Т.Томсон изложил атомистическую теорию
Дальтона в третьем издании своей книги «Новая система химии». Тем не менее,
понадобилось еще почти полвека для окончательного утверждения атомно-молекулярного учения.
Этому способствовало развитие способов определения
атомных и молекулярных весов, открытие ряда количественных законов: закона
постоянства состава (Ж. Пруст - 1808), закона простых объемных отношений
для газов (Ж. Гей-Люссак - 1808), закона Авогадро (1811), - которые хорошо
объяснялись с позиции атомно-молекулярного учения. Экспериментальное
обоснование оно получило в работах Й.Б.Берцелиуса. Официально атомно-молекулярное
учение было признано на I Международном конгрессе химиков (1860).

В середине XIXв. атомно-молекулярное учение дополняется
представлениями о валентности и химической связи, которые легли в
основание теории химического строения органического вещества
(А.М.Бутлеров – 1861). На этой базе оформилась стереохимия (Дж.Г.ВантГофф - 1874),
исследующая пространственное строение органических соединений.

Общий теоретический подход в химии, сложившийся в XIXв., ставил
определение свойств химических веществ в зависимость не только от
элементного состава, но и от структуры элементов и их соединений. Развитие
атомно-молекулярного учения привело к идее о сложном строении не только
молекулы, но и атома. Первым эту мысль в XIX в. высказал английский ученый
У.Праут, обобщив результаты измерений, показавших, что атомные веса
элементов кратны атомному весу водорода. Праут выдвинул гипотезу, согласно
которой атомы всех элементов состоят из атомов водорода.

Идею о сложном строении атомов развивал Д.И.Менделеев (1834-1907),
который предположил, что между химическими элементами природы
существует закономерная связь, определяемая возрастанием атомного веса
элемента. В созданной им Периодической Системе Химических Элементов
(1871) известные природные элементы расположены последовательно в
соответствии со своим атомным весом, выделено их химическое родство,
которое выражается в сходстве реакций. Система Менделеева сделала
наглядным единство природных химических элементов, их связи и возможные
превращения. 

Периодический закон сыграл решающую роль в развитии ряда смежных с
химией наук, а также химической технологии и промышленности,
стимулировал развитие учения о строении атома, открытие новых элементов.

В XXв. формируется электронная теория строения вещества,
складываются новые направления химических исследований: физическая
химия, химическая кинетика (учение о скоростях химических реакций), теория
электролитической диссоциации, химическая термодинамика.



\end{document}
