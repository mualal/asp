\documentclass[main.tex]{subfiles}

\begin{document}

\linksection{Лекция 01.02.2024 (Шипунова О.Д.)}

\insertlectureslide{1}{08}

\sublinksection{Методология развития научного знания}

\insertlectureslide{2}{08}

\sublinksection{Интеллектуальные ресурсы динамики науки}

\insertlectureslide{3}{08}

\sublinksection{Формы развития знания}

\insertlectureslide{4}{08}

\sublinksection{Характеристика проблемы как формы развития знания}

\insertlectureslide{5}{08}

\sublinksection{Требования к постановке проблемы}

\insertlectureslide{6}{08}

\sublinksection{Роль мнимых проблем в динамике науки}

\insertlectureslide{7}{08}

\sublinksection{Общие критерии обоснованности гипотезы}

\insertlectureslide{8}{08}

\sublinksection{Критерий истины в практике обоснования гипотез}

\insertlectureslide{9}{08}

\sublinksection{Методы фактического обоснования гипотез}

\insertlectureslide{10}{08}

\sublinksection{Моделирование как метод фактического обоснования гипотез}

\insertlectureslide{11}{08}

\sublinksection{Функции модели в проверке гипотез}

\insertlectureslide{12}{08}

\sublinksection{Практика логического обоснования гипотез. Структура аргументации}

\insertlectureslide{13}{08}

\sublinksection{Формы аргументации}

\insertlectureslide{14}{08}

\sublinksection{Аргументация и доказательство}

\insertlectureslide{15}{08}

\sublinksection{Умозаключения по аналогии на основании сходства признаков}

\insertlectureslide{16}{08}

\sublinksection{Эвристические функции аналогии}

\insertlectureslide{17}{08}

\sublinksection{Примеры рассуждений по аналогии}

\insertlectureslide{18}{08}

\sublinksection{Индуктивные умозаключения}

\insertlectureslide{19}{08}

\sublinksection{Научная индукция как форма аргументации}

\insertlectureslide{20}{08}

\insertlectureslide{21}{08}

\sublinksection{Примеры рассуждений по индукции}

\insertlectureslide{22}{08}

\insertlectureslide{23}{08}

\sublinksection{Виды косвенного обоснования}

\insertlectureslide{24}{08}

\sublinksection{Цель аргументации}

\insertlectureslide{25}{08}

\sublinksection{Логическое обоснование гипотез}

\insertlectureslide{26}{08}

\sublinksection{Гипотетико-дедуктивный метод в развитии научного знания}

\insertlectureslide{27}{08}

\sublinksection{Мысленный эксперимент как способ проверки гипотез}

\insertlectureslide{28}{08}

\sublinksection{Практика конструктивного обоснования}

\insertlectureslide{29}{08}

\sublinksection{Законы логики}

\insertlectureslide{30}{08}

\sublinksection{Закон тождества}

\insertlectureslide{31}{08}

\sublinksection{Закон противоречия}

\insertlectureslide{32}{08}

\sublinksection{Закон исключенного третьего}

\insertlectureslide{33}{08}

\sublinksection{Закон достаточного основания}

\insertlectureslide{34}{08}

\insertlectureslide{35}{08}

\sublinksection{Примеры нарушения законов тождества и противоречия}

\insertlectureslide{36}{08}

\sublinksection{Примеры нарушения закона достаточного основания}

\insertlectureslide{37}{08}

\sublinksection{Критика и опровержение}

\insertlectureslide{38}{08}

\sublinksection{Критика тезиса}

\insertlectureslide{39}{08}

\sublinksection{Критика аргументов}

\insertlectureslide{40}{08}

\sublinksection{Правила и логические ошибки в процедурах обоснования и опровержения}

\insertlectureslide{41}{08}



\end{document}
