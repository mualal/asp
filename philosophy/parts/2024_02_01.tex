\documentclass[main.tex]{subfiles}

\begin{document}

\section{Лекция 01.02.2024 (Шипунова О.Д.)}

\insertlectureslide{1}{08}

\subsection{Методология развития научного знания}

\insertlectureslide{2}{08}

\subsection{Интеллектуальные ресурсы динамики науки}

\insertlectureslide{3}{08}

\subsection{Формы развития знания}

\insertlectureslide{4}{08}

\subsection{Характеристика проблемы как формы развития знания}

\insertlectureslide{5}{08}

\subsection{Требования к постановке проблемы}

\insertlectureslide{6}{08}

\subsection{Роль мнимых проблем в динамике науки}

\insertlectureslide{7}{08}

\subsection{Общие критерии обоснованности гипотезы}

\insertlectureslide{8}{08}

\subsection{Критерий истины в практике обоснования гипотез}

\insertlectureslide{9}{08}

\subsection{Методы фактического обоснования гипотез}

\insertlectureslide{10}{08}

\subsection{Моделирование как метод фактического обоснования гипотез}

\insertlectureslide{11}{08}

\subsection{Функции модели в проверке гипотез}

\insertlectureslide{12}{08}

\subsection{Практика логического обоснования гипотез. Структура аргументации}

\insertlectureslide{13}{08}

\subsection{Формы аргументации}

\insertlectureslide{14}{08}

\subsection{Аргументация и доказательство}

\insertlectureslide{15}{08}

\subsection{Умозаключения по аналогии на основании сходства признаков}

\insertlectureslide{16}{08}

\subsection{Эвристические функции аналогии}

\insertlectureslide{17}{08}

\subsection{Примеры рассуждений по аналогии}

\insertlectureslide{18}{08}

\subsection{Индуктивные умозаключения}

\insertlectureslide{19}{08}

\subsection{Научная индукция как форма аргументации}

\insertlectureslide{20}{08}

\insertlectureslide{21}{08}

\subsection{Примеры рассуждений по индукции}

\insertlectureslide{22}{08}

\insertlectureslide{23}{08}

\subsection{Виды косвенного обоснования}

\insertlectureslide{24}{08}

\subsection{Цель аргументации}

\insertlectureslide{25}{08}

\subsection{Логическое обоснование гипотез}

\insertlectureslide{26}{08}

\subsection{Гипотетико-дедуктивный метод в развитии научного знания}

\insertlectureslide{27}{08}

\subsection{Мысленный эксперимент как способ проверки гипотез}

\insertlectureslide{28}{08}

\subsection{Практика конструктивного обоснования}

\insertlectureslide{29}{08}

\subsection{Законы логики}

\insertlectureslide{30}{08}

\subsection{Закон тождества}

\insertlectureslide{31}{08}

\subsection{Закон противоречия}

\insertlectureslide{32}{08}

\subsection{Закон исключенного третьего}

\insertlectureslide{33}{08}

\subsection{Закон достаточного основания}

\insertlectureslide{34}{08}

\insertlectureslide{35}{08}

\subsection{Примеры нарушения законов тождества и противоречия}

\insertlectureslide{36}{08}

\subsection{Примеры нарушения закона достаточного основания}

\insertlectureslide{37}{08}

\subsection{Критика и опровержение}

\insertlectureslide{38}{08}

\subsection{Критика тезиса}

\insertlectureslide{39}{08}

\subsection{Критика аргументов}

\insertlectureslide{40}{08}

\subsection{Правила и логические ошибки в процедурах обоснования и опровержения}

\insertlectureslide{41}{08}



\end{document}
