\documentclass[main.tex]{subfiles}

\begin{document}

\setcounter{section}{102}

\linksectionold{Вопросы к зачёту и экзамену}

\vspace{3mm}
{\parindent0pt\textbf{(I) Общие проблемы философии науки.}}

\begin{enumerate}[nosep,leftmargin=*]
	\item Предмет философии науки (определение содержания термина "<наука">, различие научного и вненаучного знания, критерии научного знания, специфика науки как сферы деятельности).
	\item Философские основания науки
	\item Соотношение позитивного научного и философского знания
	\item Основные стадии эволюции науки как системы познавательной деятельности: преднаука и развитая наука; классическая, неклассическая, постнеклассическая наука.
	\item Предпосылки философии науки в Античную эпоху и Новое время (умозрительные методы познания и классификации наук по Аристотелю, натурфилософские концепции о строении мира, истинный метод науки в эмпиризме и рационализме, Ф. Бэкон, Р. Декарт).
	\item Позитивистские концепции в философии науки. Демаркация науки и философии (О. Конт, Г. Спенсер, Дж. Милль). Эмпириокритицизм о структуре опыта (Р. Авенариус, Э. Мах).
	\item Неопозитивистская концепция науки: принципы верификации, конвенционализма, физикализма. Логический позитивизм о структуре опыта и языке науки.
	\item Концепции постпозитивизма. Критический рационализм К. Поппера.
	\item Концепция исследовательских программ И. Лакатоса.
	\item Т. Кун об исторической динамике науки.
	\item Эпистемологический анархизм П. Фейерабенда.
	\item Социология науки в постпозитивизме. Проблема интернализма и экстернализма в понимании механизмов развития науки.
	\item Динамика науки как процесс порождения нового знания. Формы развития научного знания: проблема и гипотеза.
	\item Эмпирический уровень научного познания. Формы организации знания на эмпирическом уровне. Эмпирические методы.
	\item Теоретический уровень научного познания. Формы организации знания на теоретическом уровне. Теоретические методы.
	\item Методология обоснования. Научная форма обоснования. Законы логики и принципы аргументации.
	\item Методология развития научного знания. Требования к постановке проблем и обоснованию гипотез.
	\item Модели научного объяснения. (+ Объяснение, понимание, интерпретация как основание трансляции опыта науки, популяризации и развитии научного знания)
	\item Исторические типы научной рациональности и научные революции
	\item Особенности современного этапа развития науки. Междисциплинарные взаимодействия, общенаучные понятия, системная методология.
	\item Предпосылки формирования экофилософии (учение о биосфере и ноосфере В.И.Вернадского, русский космизм, Римский клуб о глобальных кризисах)
	\item Принципы синергетики (теории самоорганизации) и универсальный эволюционизм в формировании современной научной картины мира.
	\item Отношение общества к науке. Сциентизм и антисциентизм
	\item Этапы развития науки как социального института.
	\item Место и роль науки в культуре техногенной цивилизации. Проблема ценности научно-технического прогресса.
\end{enumerate}

\vspace{12mm}
{\parindent0pt\textbf{(II) Философские проблемы естествознания. Общие вопросы}}

\begin{enumerate}[nosep,leftmargin=*]
	\item Предмет философии естествознания.
	\item Онтологические проблемы естествознания
	\item Теоретико-познавательные и методологические аспекты естествознания
	\item Базовые модели естественнонаучного объяснения.
	\item Первая система естествознания -- натурфилософия: познавательная установка, метод, круг проблем.
	\item Мировоззренческие и методологические принципы классического естествознания. Динамический детерминизм. Выявление границ механического объяснения на рубеже 20в.
	\item Философские и теоретические основания химии как предметной области естествознания
	\item Идеалы теоретического естествознания. Принципы построения логически строгой теории. Высшая математика и естествознание.
	\item Методологические установки в создании теоретической физики. СТО и становление релятивистской физики
	\item Мировоззренческое значение общей теории относительности
	\item Философские аспекты квантовой теории. Проблема индетерминизма
	\item Философские проблемы теоретической биологии. Принципы наследственности и изменчивости в становлении генетики
	\item Проблемы концептуального синтеза генетики и теории эволюции
	\item Эволюционная биология -- проблема естественного отбора и механизмов биоэволюции.
	\item Междисциплинарные стратегии в естествознании XXв. Функциональны, системный, информационный подходы.
	\item Синергетическая парадигма: основные понятия и принципы.Теория самоорганизации.
	\item Научная картина мира и философские проблемы естествознания. Проблемы физической картины мира (механической, электродинамической, квантовой).
	\item Идея эволюции и концепция тонкой подстройки в физической картине мира.
	\item Междисциплинарные принципы в формировании естественнонаучной картины мира (системность и самоорганизация).
	\item Глобальный эволюционизм -- новая натурфилософская позиция в системе современного естествознания. Картина мира в глобальном эволюционизме
\end{enumerate}

\vspace{12mm}
{\parindent0pt\textbf{(III) Философские проблемы естествознания. Вопросы по специальным разделам естествознания и математики}}

\vspace{4mm}
\centerline{Философские проблемы физики}
\begin{enumerate}[nosep,leftmargin=*]
	\item Место физики в системе естественных наук.
	\item Философские проблемы становления концепций теоретической физики. Теория относительности. Теория строения атома и физика элементарных частиц.
	\item Онтологические проблемы физики
	\item Физический вакуум и поиск единой теории
	\item Проблема пространства и времени
	\item Проблема детерминизма. Индетерминизм в квантовой механике.
	\item Квантовая механика и объективность научного знания. Проблема природы квантовых явлений.
	\item Системные идеи в физике.
	\item Теоретические и эмпирические основания биофизики.
	\item Представление о квантовом компьютере.
	\item Основания и концептуальная структура современных астрофизических теорий.
	\item Изменение представлений о характере физических законов в связи с концепцией "<Большого взрыва"> в космологии и формированием синергетики.	
\end{enumerate}

\vspace{4mm}
\centerline{Философские проблемы химии}
\begin{enumerate}[nosep,leftmargin=*]
	\item Специфика предмета химии и его эволюция в истории науки.
	\item Становление химии как области экспериментального естествознания.
	\item Представление о концептуальных системах химии.
	\item Развитие учения об элементах.
	\item Возникновение структурных теорий в процессе развития органической и неорганической химии.
	\item Химическая кинетика и проблема поведения химических систем.
	\item Концепции о самоорганизации химических систем.
	\item Взаимодействие физики и химии. Тенденции физикализации химии.
	\item Междисциплинарные концептуальные системы в химии. Биохимия и геохимия. Биосферная концепция В.И.Вернадского.
	\item Эволюционные проблемы в химии.
\end{enumerate}

\vspace{4mm}
\centerline{Философские проблемы биологии}
\begin{enumerate}[nosep,leftmargin=*]
	\item Предмет философии биологии. Цели и объекты биологического исследования: история и современность.
	\item Проблемы построения теоретической биологии.
	\item Методологические установки и парадигмы в биологии. Перспективы информационного подхода.
	\item Проблемы эволюционной теории. Синтетическая теория эволюции. Представления о механизмах эволюции.
	\item Эволюционная биология и эпистемология К.Лоренца.
	\item Сущность живого. Проблема возникновения жизни. Соотношение биохимической и биологической эволюции.
	\item Проблема детерминизма в биологии. Место целевого подхода в биологических исследованиях. Функциональные описания.
	\item Философские проблемы биотехнологий, генной и клеточной инженерии, клонирования. Предмет биоэтики.
	\item Принцип системности в биологии. Концепция биосферного единства. Коэволюционная стратегия глобального эволюционизма.
	\item Социобиология. Концепция геннокультурной коэволюции.	
\end{enumerate}

\vspace{4mm}
\centerline{Философские проблемы математики}
\begin{enumerate}[nosep,leftmargin=*]
	\item Специфика математики. Понятие математической реальности и математического объекта. Природа математического мышления.
	\item Структура математического знания. Теоретическая математика, прикладная математика, метаматематика.
	\item Философские проблемы в истории математики.
	\item Проблема потенциальной и актуальной бесконечности.
	\item Аксиоматический метод в математике. Аксиоматизация и формализация.
	\item Проблема полноты формализованной системы. Проблема соотношения формальных и содержательных теорий.
	\item Проблема обоснования математики: программа логицизма, интуитивизма, формализма.
	\item Философско-методологические проблемы математизации знания.
	\item Математика и информатика. Понятие информации.
	\item Компьютерная революция и математика.
	\item Моделирование и вычислительный эксперимент.
\end{enumerate}

\vspace{4mm}

\textbf{Примечание:} по данному разделу \textbf{(III)} формулируются вопросы, связанные с темой диссертации.

\textbf{Дополнительный вопрос.}
Пять первоисточников по специальности по выбору аспиранта из рекомендуемой литературы в программе, базовом учебнике, из списка первоисточников.


\end{document}
