\documentclass[main.tex]{subfiles}

\begin{document}

\linksection{Вопросы к зачёту и экзамену}

\vspace{3mm}
{\parindent0pt\textbf{(I) Общие проблемы философии науки.}}

\begin{enumerate}[leftmargin=*]
	\item Предмет философии науки (определение содержания термина "<наука">, различие научного и вненаучного знания, критерии научного знания, специфика науки как сферы деятельности).
	\item Философские основания науки
	\item Соотношение позитивного научного и философского знания
	\item Основные стадии эволюции науки как системы познавательной деятельности: преднаука и развитая наука; классическая, неклассическая, постнеклассическая наука.
	\item Предпосылки философии науки в Античную эпоху и Новое время (умозрительные методы познания и классификации наук по Аристотелю, натурфилософские концепции о строении мира, истинный метод науки в эмпиризме и рационализме, Ф. Бэкон, Р. Декарт).
	\item Позитивистские концепции в философии науки. Демаркация науки и философии (О. Конт, Г. Спенсер, Дж. Милль). Эмпириокритицизм о структуре опыта (Р. Авенариус, Э. Мах).
	\item Неопозитивистская концепция науки: принципы верификации, конвенционализма, физикализма. Логический позитивизм о структуре опыта и языке науки.
	\item Концепции постпозитивизма. Критический рационализм К. Поппера.
	\item Концепция исследовательских программ И. Лакатоса.
	\item Т. Кун об исторической динамике науки.
	\item Эпистемологический анархизм П. Фейерабенда.
	\item Социология науки в постпозитивизме. Проблема интернализма и экстернализма в понимании механизмов развития науки.
	\item Динамика науки как процесс порождения нового знания. Формы развития научного знания: проблема и гипотеза.
	\item Эмпирический уровень научного познания. Формы организации знания на эмпирическом уровне. Эмпирические методы.
	\item Теоретический уровень научного познания. Формы организации знания на теоретическом уровне. Теоретические методы.
	\item Методология обоснования. Научная форма обоснования. Законы логики и принципы аргументации.
	\item Методология развития научного знания. Требования к постановке проблем и обоснованию гипотез.
	\item Модели научного объяснения. (+ Объяснение, понимание, интерпретация как основание трансляции опыта науки, популяризации и развитии научного знания)
	\item Исторические типы научной рациональности и научные революции
	\item Особенности современного этапа развития науки. Междисциплинарные взаимодействия, общенаучные понятия, системная методология.
	\item Предпосылки формирования экофилософии (учение о биосфере и ноосфере В.И.Вернадского, русский космизм, Римский клуб о глобальных кризисах)
	\item Принципы синергетики (теории самоорганизации) и универсальный эволюционизм в формировании современной научной картины мира.
	\item Отношение общества к науке. Сциентизм и антисциентизм
	\item Этапы развития науки как социального института.
	\item Место и роль науки в культуре техногенной цивилизации. Проблема ценности научно-технического прогресса.
\end{enumerate}

\vspace{7mm}
{\parindent0pt\textbf{(II) Философские проблемы естествознания.}}

\begin{enumerate}[leftmargin=*]
	\item	
\end{enumerate}


\end{document}
