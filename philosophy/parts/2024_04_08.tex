\documentclass[main.tex]{subfiles}

\begin{document}

\linksection{Лекция 08.04.2024 (Шипунова О.Д.)}

\sublinksection{\,\,Эволюция физического и органического мира Земли в естествознании}

\insertlectureslide{1}{13}

\sublinksection{\,\,Концепция о происхождении Солнечной системы шведских астрономов}

\insertlectureslide{2}{13}

\sublinksection{\,\,Физические характеристики Солнца}

\insertlectureslide{3}{13}

\sublinksection{\,\,Основной спектр солнечной радиации}

\insertlectureslide{4}{13}

\sublinksection{\,\,Солнечная система}

\insertlectureslide{5}{13}

\sublinksection{\,\,Физические характеристики Земли}

\insertlectureslide{6}{13}

\sublinksection{\,\,Геомагнитное поле}

\insertlectureslide{7}{13}

\sublinksection{\,\,Электромагнитное поле Земли}

\insertlectureslide{8}{13}

\insertlectureslide{9}{13}

\sublinksection{\,\,Строение Земли}

\insertlectureslide{10}{13}

\insertlectureslide{11}{13}

\insertlectureslide{12}{13}

\insertlectureslide{13}{13}

\sublinksection{\,\,Эволюция макромира Земли}

\insertlectureslide{14}{13}

\sublinksection{\,\,Эволюция физического мира Земли}

\insertlectureslide{15}{13}

\sublinksection{\,\,Концепции о строении и эволюции земной коры (литосферы)}

\insertlectureslide{16}{13}

\sublinksection{\,\,Концепция глобальной тектоники}

\insertlectureslide{17}{13}

\sublinksection{\,\,Геохимическая концепция эволюции Земли}

\insertlectureslide{18}{13}

\sublinksection{\,\,Биогенная миграция элементов}

\insertlectureslide{19}{13}

\sublinksection{\,\,Живое вещество в эволюции Земли. Закон Вернадского}

\insertlectureslide{20}{13}

\sublinksection{\,\,Техногенные аномалии}

\insertlectureslide{21}{13}

\sublinksection{\,\,Геохимические этапы Эволюции физического мира}

\insertlectureslide{22}{13}

\sublinksection{\,\,Органический мир Земли на основе углеводородных соединений}

\insertlectureslide{23}{13}

\sublinksection{\,\,Условия Химической эволюции}

\insertlectureslide{24}{13}

\sublinksection{\,\,Биохимическая эволюция}

\insertlectureslide{25}{13}

\insertlectureslide{26}{13}

\sublinksection{\,\,Этапы биохимической эволюции}

\insertlectureslide{27}{13}

\sublinksection{\,\,Переход к биологической эволюции органического мира Земли}

\insertlectureslide{28}{13}

\sublinksection{\,\,Переход к эволюции многоклеточных структур}

\insertlectureslide{29}{13}

\sublinksection{\,\,Общая картина биологической эволюции}

\insertlectureslide{30}{13}

\sublinksection{\,\,Эволюция животных организмов}

\insertlectureslide{31}{13}

\sublinksection{\,\,Проблема жизни в современном естествознании}

\insertlectureslide{32}{13}

\sublinksection{\,\,Определение сущности жизни}

\insertlectureslide{33}{13}

\sublinksection{\,\,Необходимые условия жизни на Земле}

\insertlectureslide{34}{13}

\insertlectureslide{35}{13}

\sublinksection{\,\,Концепции о происхождении жизни}

\insertlectureslide{36}{13}

\sublinksection{\,\,Эволюционизм}

\insertlectureslide{37}{13}

\sublinksection{\,\,Концепция биогенеза}

\insertlectureslide{38}{13}

\sublinksection{\,\,Концепция абиогенеза}

\insertlectureslide{39}{13}

\sublinksection{\,\,Коацерватная гипотеза А.И.Опарина}

\insertlectureslide{40}{13}

\sublinksection{\,\,Генетические гипотезы о переходных структурах}

\insertlectureslide{41}{13}

\sublinksection{\,\,Абиогенная концепция происхождения жизни}

\insertlectureslide{42}{13}



\end{document}
