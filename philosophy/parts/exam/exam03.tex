\documentclass[exam_answers.tex]{subfiles}

\fontsize{14pt}{14pt}\selectfont

\begin{document}

\renewcommand{\baselinestretch}{\blch}
\sublinksectionold{\normalsize (3) Соотношение позитивного научного и философского знания}

Философия оперирует очень общими категориями и не имеет прямого практического применения.
На этом построена проблема демаркации (разделения научного и философского знания).
Отрицание рациональной роли философии для науки.

В начале XIX века было заявлено, что наука и научное знание – это то, что может быть полезно человеку и которое приложимо.
Сила знания проявляется таким образом, что позитивное знание считается ценным, полезным, приложимым, а философия – это метафизика, т. е. это рассуждения, которые не имеют смысла для науки и для человека (другими словами, её полезностью можно пренебречь).

Окончательно разорвать связь с философией учёным не удалось.
\\

Формирование и трансформация философских оснований науки требует не только философской, но и специальной научной эрудиции исследователя.

В настоящее время этот особый слой исследовательской деятельности обозначается как философия и методология науки.

Обособление этой области связано с оформлением в XIX в. позитивизма, разграничившего область научного знания (практически полезного – позитивного) и область метафизических сущностей и понятий, противопоставив науку и философию.

Однако значение мировоззренческих оснований науки вновь вышло на первый план в постпозитивизме и аналитической философии в связи с проблемами научного реализма и обоснования новых концепций и теорий.

Отличительные черты научного знания:

1) Рациональность всех содержащихся в научном знании положений и выводов (в научном знании не может быть ничего не доступного человеческому пониманию)

2) Объективность, общезначимость, безличность

3) Воспроизводимость и проверяемость

4) Логическая строгость, точность и однозначность, что обеспечивается фиксацией условий получения знания; установлением точных (в пределах интервала допустимой погрешности) количественных значений изучаемых параметров

5) Логическая взаимосвязь различных элементов научного знания, в силу которой оно представляет собой не сумму разрозненных сведений, а логически упорядоченную систему.
Взаимосвязь и единство существуют не только в рамках отдельных наук, но и между ними.
\\

Указанные особенности научного знания придают ему большую достоверность.
Оно является более надёжным, чем любое другое знание.


\end{document}