\documentclass[exam_answers.tex]{subfiles}

\fontsize{14pt}{14pt}\selectfont

\begin{document}

\renewcommand{\baselinestretch}{\blch}
\sublinksectionold{\normalsize (17) Методология развития научного знания. Требования к постановке проблем и обоснованию гипотез.}

Проблемы возникают в любых сферах деятельности человека.
В области науки, где спецификой является решение познавательных проблем, существует традиционная практика обоснования проблемы, которая представлена требованиями к формулировке и постановке проблемы:

-- наличие обоснованного вывода о том, что избранная проблема не решена в мировой науке или предлагаемые решения неудовлетворительны (неполны, не аргументированы, содержат ошибки, имеют частный характер и т. д.);

-- анализ предшествующего опыта исследования по выявленной проблеме, чтобы избежать дублирования. В технике необходим анализ патентного фонда.
Это требование предполагает:
а) знание явлений, процессов, законов развития данной предметной области;
б) знание истории вопроса: возможные подходы, методы исследования, неудачные попытки решения;

-- обоснование актуальности проблемы для общества в дополнении к личной убеждённости, что её необходимо решать.
Это требование подчёркивает вопрос о реальности проблемы: насколько она назрела и возможно ли её разрешение в обозримом будущем;

-- выявление основного противоречия проблемной ситуации;

-- формулирование целей и задач исследования (что составляет стратегию конкретного исследования).
\\

Общие критерии обоснованности гипотезы:

-- гипотеза должна быть чётко сформулирована на принятом языке, в определённых терминах и иметь правдоподобный смысл;

-- содержание гипотезы должно быть связано с предшествующим знанием или хотя бы ему не противоречить в случае полной оригинальности;

-- гипотеза должна быть эмпирически проверяема
\\

Обоснованность гипотезы – необходимое условие её приемлемости в качестве имеющего смысл научного утверждения.
Отсутствие обоснования дискредитирует гипотезу настолько, что она не может быть предметом дальнейшего обсуждения в научном сообществе.


\end{document}