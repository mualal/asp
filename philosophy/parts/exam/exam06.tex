\documentclass[exam_answers.tex]{subfiles}

\fontsize{14pt}{14pt}\selectfont

\begin{document}

\renewcommand{\baselinestretch}{\blch}
\sublinksectionold{\normalsize (6) Позитивистские концепции в философии науки. Демаркация науки и философии (О. Конт, Г. Спенсер, Дж. Милль). Эмпириокритицизм о структуре опыта (Р. Авенариус, Э. Мах).}

В первом позитивизме фиксируется проблема, которая получила название демаркации (разделения) науки и философии.
Формулируется понятие позитивной науки, с которой связывается понятие рациональное знание.

Рациональное знание в духе позитивизма – это только то, что вырабатывает наука и которое приложимо в социальной жизни.

Универсальные (очень абстрактные) законы философии (которые далеки от социальной жизни) уходят в разряд метафизики.

О. Конт: принцип демаркации позитивной науки и метафизики.

Конт специализировался на классификации наук, а Милль занимался вопросами методологии науки. Он предложил методы естествознания перенести в область социологии.
Его установка (объяснить историю общества исходя из природы человека) трактуется как психологизм.
Милль поставил вопрос о необходимости разработки новых методов в исследовании общества.

Спенсер разработал «систему синтетической философии».
Он свёл все законы науки к закону эволюции, имея в виду постепенный, плавный переход из «неопределённой бессвязной однородности в определённую и связную разнородность».
В теории познания Спенсер развивал концепцию трансформированного реализма, утверждая, что ощущения не похожи на предметы, однако каждому изменению предмета соответствует определённое изменение структуры ощущений и восприятий.
Идеи Спенсера пользовались большой популярностью в конце XIX в. и оказали значительное влияние на второй и третий этапы развития позитивизма.

Второй позитивизм (эмпириокритицизм) в конечном счёте определяется неясностью самого опыта науки.

Первый позитивизм: что такое чистая/позитивная наука?

Второй позитивизм: что такое опыт?
Исходное положение эмпириокритицизма: «существует только опыт».
Цель научного познания – накопление опытных данных и наиболее экономное описание элементов опыта.
Теории – косвенные описания многообразия наблюдений для удержания в памяти (здесь это просто способ упорядочивания данных, а не отражение универсальных связей).
Эмпириокритицизм выступает против механистической картины мира, указывает на то, что она не достаточна.
Пытается создать новую модель реальности, которая опирается не на материю, пространство, время, а на некоторые функциональные отношения между элементами мира.

Познание – особый аспект жизнедеятельности.

Авенариусу принадлежит идея принципиальной координации, которая подчёркивает, что опыт сам по себе, представляя некую изначальную реальность, тем не менее существует только в координации (слово координация заменяет слово ощущение).
Другими словами, идея принципиальной координации подчёркивала, что опыт представляет собой изначальную реальность, в которой нет расщепления на субъект и объект.
Иначе говоря, не существует объекта без субъекта и не существует субъекта без объекта.
Мир дан нам только в «принципиальной координации» как опыт.
«Второй позитивизм» сделал вывод о том, что наука не даёт подлинной картины реальности, а доставляет лишь «символы, знаки, отметки для практики».
Таким образом, «второй позитивизм» пришёл к отрицанию объективной реальности, отражаемой нашим сознанием.


\end{document}