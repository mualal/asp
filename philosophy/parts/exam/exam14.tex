\documentclass[exam_answers.tex]{subfiles}

\fontsize{14pt}{14pt}\selectfont

\begin{document}

\renewcommand{\baselinestretch}{\blch}
\sublinksectionold{\normalsize (14) Эмпирический уровень научного познания. Формы организации знания на эмпирическом уровне. Эмпирические методы.}

Эмпирическое знание добывается в опыте, в непосредственном или опосредованном (через приборы) контакте исследователя с существующими вне его сознания объектами.
Познание на эмпирическом уровне идёт от конкретного реального объекта к абстрактному, затем от него – к конкретному множеству реальных объектов.

Главной задачей в эмпирическом познании является получение научных фактов.
Основными эмпирическими методами являются наблюдение и эксперимент.

Научное наблюдение – это целенаправленное и специально организованное восприятие явлений.
Главное требование к научному наблюдению – объективность, точность даваемых им сведений.
Наблюдение должно проводиться так, чтобы вмешательство наблюдателя не исказило картину изучаемых явлений.

Эксперимент – это управляемое и контролируемое воздействие на изучаемый объект в целях получения информации о нём.

Итогом наблюдений и экспериментов должно быть установление научных фактов.
Чтобы свести к минимуму влияние случайностей и возможные ошибки, наблюдения и эксперименты многократно повторяются и их результаты подвергаются математической (статистической) обработке.
Только после этого они становятся достоверными научными фактами.

В теории познания фактом называется эмпирическое высказывание, суждение о событии. 
Факт – это не само событие, а утверждение о событии, описание события.
Множество событий шире множества фактов.
Событие становится фактом, если оно вошло в сферу человеческого познания.

Накапливая факты и подвергая их систематизации, классификации, обобщению, ученые находят зависимости между ними – эмпирические законы или закономерности.
Совокупность эмпирических законов, относящихся к некоторой области явлений, иногда называют феноменологической теорией этих явлений.
Однако такая теория не выходит за рамки эмпирического описания явлений и не объясняет их сущности.
Например, эмпирические законы теплового расширения не объясняют ни механизма этого явления, ни линейного характера зависимости объёма от температуры.

Объяснение найденных эмпирических фактов и закономерностей требует перехода на более высокий, теоретический уровень научного познания.


\end{document}