\documentclass[exam_answers.tex]{subfiles}

\fontsize{14pt}{14pt}\selectfont

\begin{document}

\renewcommand{\baselinestretch}{\blch}
\sublinksectionold{\normalsize (15) Теоретический уровень научного познания. Формы организации знания на теоретическом уровне. Теоретические методы.}

Теоретик работает не с самими объектами, а с их мысленными образами.
Материальные орудия деятельности теоретика: карандаш, бумага, компьютер.
Затраты на развитие теоретических исследований на два порядка ниже, чем на развитие эмпирических.

Признаком теоретического познания является создание идеальных объектов, раскрывающих сущность эмпирически наблюдаемых явлений.

Теория – это логически упорядоченная система знаний о каких-либо явлениях, в которой строятся их мысленные модели и формулируются законы, объясняющие и предсказывающие наблюдаемые факты и закономерности.

Важную роль играют разнообразные мысленные эксперименты – умозрительное исследование теоретической модели, её «поведения» в различных мысленно представляемых условиях.
Изучение теоретических моделей в мысленных экспериментах позволяет сформулировать понятия и принципы, которые отражают свойства этих моделей.

Из основных принципов теории должны быть логически выведены возможные следствия и развёрнута система понятий, что и образует содержание теории.

Мысленные модели выступают как промежуточное звено между теорией и действительностью.
Теоретическая модель всегда основывается на упрощении, схематизации, идеализации реальности, поэтому и теория всегда отражает реальность лишь в упрощённом, схематизированном и идеализированном виде.
Теоретические законы описывают свойства идеальных объектов.
Чтобы применить теоретические законы к реальным объектам, необходимо построить для них соответствующие теоретические модели.

Аксиоматическое представление придаёт теории логическую стройность, строгость, чёткость.

Формализация – метод изложения теории особым языком со строго фиксированным синтаксисом.
Язык вводится набором исходных символов, а также правил образования из них языковых выражений (формул) и правил операций – перехода от одних формул к другим.
Теория, изложенная в формализованном языке, превращается в формализованную систему.

Аксиоматический метод находит применение не только в математике, но и в естественных науках (механика, термодинамика и др.), но возможности его применения в естествознании ограничены, так как содержание естественнонаучных теорий должно обосновываться и корректироваться опытом, а данные опыта могут не укладываться в рамки принятой заранее аксиоматики.

\end{document}