\documentclass[exam_answers.tex]{subfiles}

\fontsize{14pt}{14pt}\selectfont

\begin{document}

\renewcommand{\baselinestretch}{\blch}
\sublinksectionold{\normalsize (4) Основные стадии эволюции науки как системы познавательной деятельности: преднаука и развитая наука; классическая, неклассическая, постнеклассическая наука.}

В современной философии нет единого мнения в толковании научного знания.

Существуют две крайние точки зрения.

Согласно первой, термины «наука» и «знание» обозначают один социальный феномен.

Наука (лат. scientia — знание) – любое сохраняемое и передаваемое знание, которое возникает в глубокой предыстории вместе с культурой изготовления орудий труда и передачей опыта их использования.

Практическое, обыденное и теоретическое знание (концептуальное, обоснованное) в этом контексте не различаются.

Тогда начало науки можно отнести к периоду неолита (VII тыс. до н.э.), когда кардинально меняется образ жизни человека (кочевой – оседлый).

Неолитическая революция – меняется образ жизни (появляется деятельность, направленная на преобразование природы под необходимости выживания человека – переход от присваивающего хозяйства к производящей экономике).
Основа неолитической революции – передача знания (примером или магическими ритуалами).
Посвящение – имеешь право стать дважды рождённым.
Платон был посвящён в геометры.

Пранаука традиционных культур (древняя математика и астрономия);
протонаука, которая базируется на умозрительной практике доказательства (Античная наука; натурфилософия эпохи эллинизма;
опытная наука позднего Средневековья);
преднаука (эпоха Возрождения и Нового времени).

Другая точка зрения на проблему начала науки трактует научное познание как специально планируемую исследовательскую деятельность, которая имеет особые методы и язык описания.
В этом случае начло науки имеет исток в эпоху Возрождения.

Классическая наука в современном понимании оформляется в XVII-XVIII вв. вместе с точным экспериментальным естествознанием, утверждающим особую практику научного обоснования, которая опирается на виды умозрительного (логического, математического) доказательства и экспериментальное (эмпирическое) подтверждение.
В этом контексте предшествующее знание о природе и природных явлениях трактуется как донаучное.

Классический этап в развитии науки охватывает период с XVIIIв. (когда утверждается система точного экспериментального естествознания на базе классической механики и натурфилософии Ньютона) до первой трети XXв. (когда формулируются законы квантовой механики и представления о статистическом законе и утверждается квантовая механика).

Неклассический период в развитии науки (30-50гг. XXв.) характеризуется дополнительностью в описании причинных связей (динамические и статистические законы), проблемами исследования и описания микромира. Принцип неопределённости Гейзенберга, принцип вероятностного описания.

Постнеклассический период в развитии науки относят к концу XXв., когда фундаментальное значение в развитии научного знания получают междисциплинарные познавательные стратегии и принципы системности, эволюции, самоорганизации.

Переход к новому периоду в развитии науки в философии связывается с представлением о научной рациональности, содержанием которой выступает изменение стиля научного мышления (типа научной рациональности), базовых моделей описания и объяснения причинных связей (форм детерминизма), категориального и математического аппарата науки, универсальных принципов в научной картине мира.


\end{document}