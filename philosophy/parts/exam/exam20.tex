\documentclass[exam_answers.tex]{subfiles}

\fontsize{14pt}{14pt}\selectfont

\begin{document}

\renewcommand{\baselinestretch}{\blch}
\sublinksectionold{\normalsize (20) Особенности современного этапа развития науки. Междисциплинарные взаимодействия, общенаучные понятия, системная методология.}

Объектами современных междисциплинарных исследований всё чаще становятся уникальные системы, характеризующиеся открытостью и саморазвитием.
Такого типа объекты начинают определять характер предметных областей основных фундаментальных наук, детерминируя облик современной, постнеклассической науки.

Главная характеристика постнеклассической науки – отказ от универсальности физических понятий и физической картины мира.
Попытки объяснить явления микро- и мега мира ввели в круг фундаментальных проблем строго говоря нефизические понятия, фиксирующие не характерные для классической и неклассической физики принципы целостности и эволюции.

Стимулом становления междисциплинарной области в системе научного знания было появление новых наук о сложных системах, предметом которых стали процессы управления и организации, рассматриваемые в абстракции от физической природы самих систем.
На этой почве оформился новый общенаучный понятийный аппарат.
Исторически первую роль в этом движении концептуальной интеграции естественных и социальных наук сыграла кибернетика.
Кибернетический способ исследования сложных систем и явлений получил название функционального подхода.
Общие законы, сформулированные в кибернетике, относятся к надёжности управления действиями сложных систем:

1) закон разнообразия: эффективное управление системой возможно только в том случае, если разнообразие управляющей системы выше разнообразия управляемой;

2) закон сложности: чем выше сложность системы, тем менее она управляема.
Поэтому существует порог сложности системы, за которым тотальный контроль поведения системы становится невозможным из-за нарастания системных эффектов.

Концептуальной базой кибернетики выступает теория систем, в которой разрабатываются принципы системного анализа явлений, объектов и событий на основе представления об абстрактной системе (простой или сложной).
Людвиг фон Берталанфи выдвинул идею разработки общей теории систем.
Его теоретическая программа включала:

1) выявление общих признаков и законов поведения систем независимо от их происхождения, природы составляющих элементов и отношений между ними;

2) выявление и формулирование объективных законов для биологических и социальных явлений;

3) синтез современного знания на основе сходства законов, описывающих разные сферы жизни природы, человека и общества.

Главная трудность в создании общей теории систем – различие общетеоретического и конкретного знания.
Стремление к универсальности в описании систем приводило к абстрактности, более характерной для философии, чем для естествознания.
Но благодаря заявленной программе возникли новая познавательная стратегия в естествознании, получившая название системного подхода, новые междисциплинарные (общенаучные) методы исследования, новый системный стиль мышления.

В конце века системный подход применяется практически во всех науках (естественных и социогуманитарных), становится общенаучной методологией.

Применение понятий системного подхода к анализу прикладных проблем в самых разных сферах привело к выделению системного анализа в отдельную концептуальную и предметную область.
Теоретическую основу системного анализа составили: кибернетика, теория информации, теория игр и принятия решений, анализ систем голосования.


\end{document}