\documentclass[exam_answers.tex]{subfiles}

\fontsize{14pt}{14pt}\selectfont

\begin{document}

\renewcommand{\baselinestretch}{0.75}
\sublinksectionold{\normalsize (18) Модели научного объяснения. (+ Объяснение, понимание, интерпретация как основание трансляции опыта науки, популяризации и развитии научного знания)}

В современной системе знания выделяют линейную, статистически-вероятностную и нелинейную модели научного объяснения, которые отличаясь формой детерминизма и приоритетного закона соотносятся с тремя историческими типами научной рациональности и стилем мышления (классический, неклассический, постнеклассический).
\\

Научная рациональность – комплекс процедур и стилей мышления, который характеризует ту или иную эпоху в развитии науки. Каждую историческую эпоху горизонт знания всегда ограничен. Ньютон не смог бы сформулировать теорию Эйнштейна.
\\

Линейная модель – механистический детерминизм (однозначная связь причины и следствия) – аналитическая геометрия, дифференциально-интегральное исчисление – классическая механика, классические теории в физике.
\\

Статистически-вероятностная модель – статистический детерминизм (нежёсткая связь причины и следствия) – теория вероятностей – статистическая физика, теория относительности, квантовая механика.
\\

Нелинейная модель – вероятностный детерминизм, относительность жёсткого и нежёсткого механизмов причинения – теория катастроф, теория автоколебаний – неравновесная термодинамика, теория самоорганизации.
\\

В функции философии науки также входит коммуникативная функция, которая заключается в продвижении новых научных идей, распространении и популяризации построений науки в широких интеллектуальных культурных слоях.
\\

Объяснение – это совокупность утверждений, составленная для описания набора фактов таким образом, чтобы стали понятными их причины, контекст и последствия.

Понимание – это способность постичь смысл и значение чего-либо и достигнутый благодаря этому результат.
Для понимания характерно ощущение ясной внутренней связности, организованности рассматриваемых явлений.

Интерпретация – работа мышления, состоящая в раскрытии уровней значения и расшифровке смысла какого-либо явления, события или текста, процесс разъяснения и толкования их.


\end{document}