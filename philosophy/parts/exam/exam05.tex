\documentclass[exam_answers.tex]{subfiles}

\fontsize{14pt}{14pt}\selectfont

\begin{document}

\renewcommand{\baselinestretch}{\blch}
\sublinksectionold{\normalsize (5) Предпосылки философии науки в Античную эпоху и Новое время (умозрительные методы познания и классификации наук по Аристотелю, натурфилософские концепции о строении мира, истинный метод науки в эмпиризме и рационализме, Ф. Бэкон, Р. Декарт).}

Античная наука строит умозрительное знание о мире на основании созерцания мира и методов сомнения и рассуждения (критическое мышление).
Подчёркивает различие научного знания и мнения одного человека или группы.

Пифагорейская школа: в основании гармонии Космоса – число.

Элейская школа (Ксенофан) первоначалом мира полагает единство и незыблемость Бытия.

Платон и Аристотель – наследники Элейской школы. Аристотель полагает в основании мира форму как активное организующее материю начало.
Аристотель вводит понятия Материя (в значении потенциальной возможности вещей) и Энергия (для обозначения актов перехода потенции в её реализацию, как характеристику изменения/движения к цели).
В картине мироздания Аристотель выделил 4 причины (формальную, материальную, действующую и целевую).
Физика, согласно Аристотелю, раскрывает действующую причину.

Наука о движении тел под действием внешней причины – физика.

Наука о скрытой сущности (причине) – метафизика.

Наука о методах получения знания – аналитика.
Аристотель формулирует 3 закона логики: закон тождества, противоречия, исключённого третьего.

Наука об обществе – политика.

Наука о добродетели – этика.

Наука о душе – психология.

Натурфилософские картины мира тоже нельзя назвать научными, поскольку они содержат умозрительные, отвлечённые представления, а нередко также и религиозно-мифологические.
Тем не менее, логическая упорядоченность, обоснованность, преобладающий рациональный характер приближают её к научной картине мира.

Натурфилософия Платона частично следует атомистическому учению и частично элементаризму.

Натурфилософия Христианского запада.
Утверждение принципа двойственной истины, требующего признания прав «естественного разума» наряду с Христианской верой.

Натурфилософия и наука в Средние века.

Натурфилософия эпохи Возрождения.
Отождествление Бога и Природы.

Натурфилософия И. Ньютона.
Решает проблему физического обоснования гелиоцентрической системы Галилея-Кеплера.
\\

Последовательно изложил и обосновал идеи эмпиризма выдающийся английский философ Френсис Бэкон.
Он предложил реформу научного метода – обращение к опыту и обработка его методом индукции.
Эмпиризм связывает источник истинного знания с чувственными ощущениями.

Позиция рационализма в методологии познания подчёркивает первенство разума над чувствами в познании, независимость разума от чувственных восприятий.
Выдающийся представитель рационализма – Рене Декарт.
Метод Декарта развёртывается как логическая дедукция и включает в себя 4 правила.


\end{document}