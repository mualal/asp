\documentclass[exam_answers.tex]{subfiles}

\fontsize{14pt}{14pt}\selectfont

\begin{document}

\renewcommand{\baselinestretch}{0.75}
\sublinksectionold{\normalsize (12) Социология науки в постпозитивизме. Проблема интернализма и экстернализма в понимании механизмов развития науки.}

Социология науки изучает динамику науки в её взаимоотношении с обществом.

Ключевой вопрос социологии науки – почему развивается наука?
В решении проблемы движущих факторов развития науки сложились альтернативные концепции интернализма и экстернализма.

Согласно интернализму, развитие науки имеет внутреннюю детерминацию, то есть обусловлено внутренне присущими научному познанию закономерностями.

Согласно экстернализму, развитие науки имеет внешнюю детерминацию, то есть обусловлено действием внешних социально-исторических факторов.

Интерналисты подчёркивают, что идеи возникают только из идей.
Существует логическая последовательность, в которой они рождаются. Нарушить эту последовательность внешние воздействия не в состоянии.
Интернализм не отрицает того, что общественные условия влияют на ход развития науки, но полагает это влияние несущественным, неопределяющим.

Экстерналисты, наоборот, настаивают на том, что нельзя понять причины развития науки, абстрагируясь от социальных условий, в которых она развивается.

Интерналисты склонны поддерживать кумулятивистское понимание роста научного знания, а к экстернализму тяготеют сторонники антикумулятивных взглядов.

Интерналисты недооценивают роль социального заказа, предъявляемого обществом к науке. Интерналистский взгляд на науку не даёт возможности понять, почему рост научных знаний исторически неравномерен, почему он бурно идёт в одних странах, тогда как другие в то же время никакими научными достижениями не блещут.
Ответ на подобные вопросы интерналисты дать не могут, так как причины здесь надо искать не внутри науки, а в социальных условиях её существования.

Экстернализм одностроннне и упрощённо трактует зависимость достижений науки от вненаучных факторов.
Экстерналисты не учитывают того, что достижения науки сами влияют на формирование социальных потребностей.
Экстерналисты игнорируют логику развития научных идей и свободу научного творчества учёного, который сам выбирает круг решаемых им задач.
И социальные потребности не могут заставить науку сделать то, что она не способна сделать.

В слабом интернализме по Попперу источники развития научного знания связываются с динамикой постановки и развития проблем.

В слабом интернализме по Лакатосу источники развития научного знания связываются с развитием ядра исследовательских программ.

В слабом интернализме по Тулмину источники развития научного знания связываются с развитием концептуальных структур.

Дилемма экстернализм – интернализм представляется неразрешимой только тогда, когда позиции того и другого абсолютизируются.
Наиболее плодотворной представляется идея диалектического единства внутренней и внешней детерминации развития науки, когда движущие силы развития науки находятся в отношении дополнительности.


\end{document}