\documentclass[exam_answers.tex]{subfiles}

\fontsize{14pt}{14pt}\selectfont

\begin{document}

\renewcommand{\baselinestretch}{\blch}
\sublinksectionold{\normalsize (19) Исторические типы научной рациональности и научные революции}

Под научной рациональностью в философии науки понимают стиль познавательной деятельности, который складывается в XVII-XVIII вв. на базе точного экспериментального естествознания и который характеризуется математическим языком описания и формой обоснования знания, сочетающей логическое доказательство и фактическую (экспериментальную) проверку.

Научная рациональность – комплекс процедур и стилей мышления, который характеризует ту или иную эпоху в развитии науки.

Каждую историческую эпоху горизонт знания всегда ограничен.

Ньютон не смог бы сформулировать теорию Эйнштейна.

Исторический тип научной рациональности определяется базовой моделью объяснения причинно-следственных связей (формой детерминизма), базовой теорией и стилем мышления, типом исследуемых объектов, математическим инструментарием.

Типы научной рациональности: классический (механизм; линейная модель), неклассический (релятивизм; статистическая модель и вероятность), постнеклассический (холизм; нелинейная модель).
\\

Глобальные научные революции, изменившие тип научной рациональности, а также философские основания науки, происходили 4 раза.

Первая научная революция XVII в. привела к становлению экспериментальных и математических методов классической науки.

Вторая научная революция конца XVIII – первой половины XIX в характеризуется оформлением теоретических оснований классических дисциплин: физики, химии, биологии.

В конце XIX в. механическая картина мира берётся под сомнение, прежде всего в физике.

Третья научная революция охватывает период с конца XIX до середины XX столетия. В этот период существенно изменяются представления о физической реальности, пространстве, времени, материи.
Формируется электродинамическая картина мира в рамках классического типа научной рациональности.
Выдвигается квантово-механическое объяснение явлений микромира.
Утверждается приоритет статистического закона в описании материальных явлений.
Это подтверждается открытием специфики законов микро-, макро-, мегамира в физике и космологии, исследованием механизмов наследственности.
Утверждаются принципы релятивизма и дополнительности в причинных моделях объяснения явлений, характеризующих неклассический тип научной рациональности.

Четвёртая научная революция началась во второй половине XX в.
Новый этап в развитии науки соотносят с пост-неклассическим типом научной рациональности.
На первый план выдвигаются междисциплинарные методы познания и проблемно-ориентированные формы исследовательской деятельности.
Реализация комплексных программ порождает необходимость в единой системе теоретических и экспериментальных исследований, прикладных и фундаментальных знаний, интенсификации прямых и обратных связей между ними.
Формируется общенаучный концептуальный аппарат на базе теории систем, функционального и информационного подхода в объяснении сложных явлений.
В качестве приоритетного принципа исследования объектов выдвигается принцип системности и самоорганизации.


\end{document}