\documentclass[exam_answers.tex]{subfiles}

\fontsize{14pt}{14pt}\selectfont

\begin{document}

\renewcommand{\baselinestretch}{0.75}
\sublinksectionold{\normalsize (9) Концепция исследовательских программ И. Лакатоса.}

Имре Лакатос выступает против метода фальсификации, поскольку этот метод не позволяет обосновать истинное знание и построить фундаментальные теории, поэтому его основная идея выражена в понятии исследовательская программа (как всё-таки развивается наука и что обеспечивает устойчивость научного знания?)

Исследовательская программа предполагает комплекс взаимодействующих и развивающихся теорий. Программа для развития теорий, которая имеет определённую структуру и включает:

ядро программы – фундаментальная проблема, идея, представление (сохраняется при появлении опровергающих положений);

предохранительный пояс (исследователи, реализующие программу, выдвигают гипотезы, защищающие это ядро);

негативную эвристику и позитивную эвристику.
\\

Развитие науки – соперничество исследовательских программ, т. е. концептуальных систем, организованных вокруг некоторых фундаментальных проблем, идей, понятий и представлений, образующих концептуальное «твёрдое ядро» научно-исследовательской программы.

Исследовательская программа, которая перестаёт предсказывать факты, не справляется с появлением новых фактов, не может объяснить их, вырождается.
\\

Гипотезы предохранительного пояса могут быть взяты чисто интуитивно, но формулируются именно для того, чтобы сохранить исходную позицию «ядра».

Пример: работы Уильяма Гарвина (XVII век) по теории кровообращенияв человеческом теле. Защитная гипотеза теории кровообращения: есть тонкие сосуды (капилляры), которые соединяют два круга (артерии и вены).
\\

Негативная и позитивная эвристики предполагают некие исследования или поиск таких факторов, которые либо опровергают, либо поддерживают «ядро».

Методология исследовательской программы: серии сменяющих друг друга теорий, объединённых определённой совокупностью базисных идей и принципов; «Логика открытия» -- ряд правил (даже не особенно связанных друг с другом) для оценки готовых, хорошо сформулированных теорий, на основе которых можно сформулировать новые идеи и новые теории. Именно «логика открытия» связана с понятием эвристики.
\\

Эвристическую логику открытия Лакатос формулирует против (или в альтернативу) метода проб и ошибок.

Основная методологическая задача эвристики – построение моделей процесса поиска нового для субъекта (или общества в целом) решения проблемы. 
Задача эвристики распространяется не только на действия людей (методологию научной деятельности), но и лежит в основании разработки машинных эвристических программ, которые построены на правдоподобных рассуждениях и развиваются в области интеллектуальных технологий.

Наиболее интенсивно задача поиска общей структуры и алгоритмов эвристического поиска разрабатывается в области проблем искусственного интеллекта, где сформированы модели слепого поиска, лабиринтная и структурно-семантическая модели эвристической деятельности.


\end{document}