\documentclass[exam_answers.tex]{subfiles}

\fontsize{14pt}{14pt}\selectfont

\begin{document}

\renewcommand{\baselinestretch}{\blch}
\sublinksectionold{\normalsize (25) Место и роль науки в культуре техногенной цивилизации. Проблема ценности научно-технического прогресса.}

В рамках биотехнологии и генной инженерии особенно остро стала осознаваться необходимость развития научной и инженерной этики, непосредственно включённых в канву естественно-научного и инженерного исследования.
Экологические технологии высветили внешние границы научно-технического развития для человечества в рамках биосферы, стимулировав выработку новой философии устойчивого развития.
\\

Ценность научно-технического прогресса, критерии научно-технического прогресса и этические проблемы возникают вследствие экологических глобальных кризисов, которые фиксируются во второй половине XX века и рождают идею прогнозирования или хотя бы примерного представления о будущем цивилизации (сохранение или уничтожение).
\\

Цель научной организации и управления научно-техническим прогрессом заключается в поддержании стабильного равновесия общества и человека с природой, более осторожной, продуманной и осмотрительной деятельности, органического встраивания технического прогресса в культурные традиции человечества и естественное жизненное пространство.
\\

Новые критерии научно-технического прогресса в рамках концепции устойчивого развития:

1) Равновесие общества и природы, мира природного и мира искусственного.

2) Защита окружающей среды (биосферы) от антропогенных воздействий.

3) Диалог «человека и природы», в котором природа, окружающаая человека среда, -- самоценный компонент, обладающий правом голоса, а в ситуации экологического кризиса часто даже правом первого голоса.

4) Принцип коэволюции в биосферном единстве.

5) Социальная ответственность конкретных лиц, принимающих решения о проектах, могущих принести вред человеку и человечеству.
\\

Понятия устойчивого развития, глобализации научно-технического прогресса приобретают социально-политическое значение, которое меняется в зависимости от страны, региона, социальной группы, политического режима.
\\

Устойчивое развитие осуществимо лишь в результате формирования новой системы ценностей.
Необходима переориентация не только технического мышления, но и вообще общественного сознания и самосознания каждого индивида на совершенно новое представление о научно-техническом прогрессе, критериями которого выступают устойчивое развитие и социальная ответственность.
Это означает параллельное институциональное развитие, оценку последствий новой техники и технологии, социально-экономическую экспертизу научных, технических и хозяйственных проектов.


\end{document}