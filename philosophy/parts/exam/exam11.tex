\documentclass[exam_answers.tex]{subfiles}

\fontsize{14pt}{14pt}\selectfont

\begin{document}

\renewcommand{\baselinestretch}{\blch}
\sublinksectionold{\normalsize (11) Эпистемологический анархизм П. Фейерабенда.}

На почве момента перехода в развитии научного знания (как развивается научное знание, если есть совершенно разные несовместимые парадигмы) развивает свою идею американский философ Пол Фейерабенд.
Его позиция получила название методологический анархизм (или эпистемологический анархизм).

Фейерабенд показывает, что если рациональность состоит в следовании определённым правилам рационального действия, то в реальной науке рациональность, то есть соблюдение определённых правил, смешана с иррациональностью, то есть с их нарушением.
В противном случае наука вообще не смогла бы развиваться.
Фейерабенд выдвинул методологический принцип пролиферации (размножения) теорий: учёные должны стремиться создавать теории, несовместимые с существующими и признанными теориями, что способствует их взаимной критике и ускоряет развитие науки.

Принцип пролиферации призван обосновать плюрализм в методологии научного познания. Фейерабенд приходит к тезису о несоизмеримости конкурирующих и сменяющих друг друга альтернативных теорий.
Их нельзя сравнивать как в отношении к общему эмпирическому базису, так и с точки зрения общих логико-методологических стандартов и норм, так как каждая теория устанавливает свои собственные нормы.

В такой интерпретации наука ничем не отличается от любой другой формы духовного общения людей, теряет какие-либо определённые очертания, растворяется в духовной культуре общества и её истории.

Фейерабенд пытается противопоставить концепцию исторического релятивизма обычной концепции научной рациональности.
То есть у Фейрабенда нет концепции научной рациональности, а есть концепция исторического релятивизма, по которой стандарты рациональности меняются от эпохи к эпохе, от учёного к учёному, от одной научной школы к другой научной школе.
В этом отношении «методологический анархизм» Фейерабенда смыкается с концепцией науки Куна, где научная революция отождествляется с «религиозным переворотом» в воззрениях учёных, в ходе которого меняются не только теории, но и критерии их оценки.
Представители критического рационализма единодушно квалифицируют взгляды Фейерабенда и Куна как откровенный иррационализм, получая в ответ обвинение в скрытом иррационализме.

В основе исторического релятивизма Фейрабенда лежит характерное для позитивизма отрицание объективной истины в научном знании.

Фактически Фейерабенд очень ярко представляет ту модель развития научного знания, которая получила название антикумулятивная модель.


\end{document}