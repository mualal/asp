\documentclass[exam_answers.tex]{subfiles}

\fontsize{14pt}{14pt}\selectfont

\begin{document}

\renewcommand{\baselinestretch}{0.75}
\sublinksectionold{\normalsize (7) Неопозитивистская концепция науки: принципы верификации, конвенционализма, физикализма. Логический позитивизм о структуре опыта и языке науки.}

Неопозитивизм – следующее развитие позитивистской традиции и оно тоже опирается на проблематичность самого опыта.
В основе науки по-прежнему лежат эмпирические данные и ощущения.

Неопозитивизм выступает альтернативой эмпириокритицизму, потому что он акцентирует тот момент опыта, который связан со знанием.
Опыт знания науки и языка науки.
То есть в системе эмпирического опыта науки мы должны учитывать не только непосредственно эмпирические данные, которые получаем через ощущения (приборы и так далее), но и то знание, которое мы получаем.
И это знание передаётся через язык.

Одно из новшеств, введённых неопозитивистами, - понятие «логическая конструкция».
В учении о логических (теоретических) конструкциях проводится принципиальное отождествление объекта и теории объекта, хотя и признаётся разница между ощущениями и результатами их рациональной переработки.

Главная задача неопозитивизма в философии науки – это изучение языка науки и языковых моделей, которые отождествляются с моделями реальности.

Принцип верификации призван осуществить «демаркацию» (разграничение) между утверждениями, имеющими смысл для науки, и утверждениями, лишёнными научного смысла.

Конвенционализм постулирует существование в науке произвольных соглашений, действующих в виде исходных (аксиоматических) положений логической структуры науки.

Физикализм – требование адекватного перевода предложений всех наук, содержащих описание предметов в терминах наблюдения, на предложения, состоящие исключительно из терминов, которые употребляются в физике.
Распад физикализма привёл к обеднению неопозитивистской доктрины, чему также способствовало «ослабление» принципов верификации и конвенционализма.

Одна из главных проблем: проблема семантики, которая связана с языком как знаковой системой.
То есть та система, которая как раз имеет отношение к наращиванию опытом знанию.

Всякий знак имеет два значения – предметное и смысловое.

Предметное значение – это объект, который представлен знаком (обозначен).

Смысловое значение – это характеристика объекта, представителем которого выступает знак.

Язык – знаковая символическая система, которая выступает наиболее эффективным средством коммуникации в человеческом сообществе.

Семантический треугольник присущ любому термину.
Любое слово как термин имеет предметное значение и смысловой контекст.
И этот контекст оказывается разным.
Смысловой контекст может трактоваться по разным каналам.

Проблема понимания может возникнуть на каждом из трёх уровней действия знаковой системы: на синтаксическом (из-за незнания правил), на семантическом (омонимия, полисемия), на прагматическом (психический барьер в восприятии речи или знака).


\end{document}