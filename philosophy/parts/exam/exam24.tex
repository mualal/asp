\documentclass[exam_answers.tex]{subfiles}

\fontsize{14pt}{14pt}\selectfont

\begin{document}

\renewcommand{\baselinestretch}{0.75}
\sublinksectionold{\normalsize (24) Этапы развития науки как социального института.}

\vspace{-10pt}

Предпосылки становления науки как социального института в конце XVIII в.

1) Увеличение объёма и разнообразия научных знаний в конце XVIII – первой половине XIX в. Складывалась ситуация, при которой учёному все труднее было овладевать накопленной научной информацией, необходимой для успешных исследований.

2) Специализация знания. Нарастающая специализация способствовала оформлению предметных областей науки, приводила к дифференциации наук, каждая из которых не претендовала на исследование мира в целом и построение некой общей картины мира, а стремилась вычленить свой предмет исследования, отражающий аспект реальности.

3) Появление нового типа субъекта научной деятельности – коллективного.

Развитие средств трансляции научного знания вызвало к жизни становление форм научной коммуникации и социальных институтов науки.

1) В науке XVII столетия главной формой закрепления и трансляции знаний была книга (манускрипт, фолиант), в которой должны были излагаться основополагающие принципы и начала «природы вещей».

2) Переписка между учёными

3) В XVIII в. – особый тип сообщества – «Республика учёных» объединяло исследователей Европы. Переписка выступала не только как форма трансляции знания, но и служила основанием выработки новых средств исследования (в частности, мысленный эксперимент).

4) Во второй половине XVIII столетия в различных странах образуются сообщества исследователей-специалистов, часто поддерживаемые общественным мнением и государством (например, сообщество немецких химиков). Коммуникации осуществляются на национальном языке.

5) Новое средство научной коммуникации – статья в научном журнале. Адресована анонимному читателю и требует более тщательного выбора аргументов для обоснования выдвигаемых положений.

6) Организация и выпуск периодических научных журналов.

Становление социальных институтов науки.

1) В XVII в. возникают академические учреждения. 

2) Исследователи из разных областей знаний объединяются в научные сообщества.

3) Новый тип профессиональной деятельности – университетский профессор, и система подготовки научных кадров.

4) Целенаправленная подготовка научных кадров – повсеместно создаются и развиваются научные и учебные учреждения, в том числе университеты.

5) Складывается система дисциплинарно-организованного обучения.

6) Систематизация знаний в процессе преподавания выступила как один из факторов формирования конкретных научных дисциплин.

7) Специальная подготовка научных кадров оформляла особую профессию научного работника.

8) В XX в. возникает Большая наука. Резка возрастает число занятых в науке профессиональных исследователей. Усиливается специализация научной деятельности.

9) В Большой науке возникает разнообразие типов научных сообществ. Возникают «незримые колледжи», в которых исследователи по интересам поддерживают неформальные контакты.

10) Наука стала производительной силой общества. Наука – область специального финансирования.


\end{document}