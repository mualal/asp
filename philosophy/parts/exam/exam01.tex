\documentclass[exam_answers.tex]{subfiles}

\fontsize{14pt}{14pt}\selectfont

\begin{document}

\sublinksectionold{(1) Предмет философии науки (определение содержания термина "<наука">, различие научного и вненаучного знания, критерии научного знания, специфика науки как сферы деятельности).}

Философия работает с категориями, а категория – это предельно общее понятие. По аналогии в физике используем понятие материя и при этом не сосредотачиваемся на том, каким образом она выражается. Как только начинаем её конкретизировать, то начинаем исследовать структуру материи и так далее.

Категории составляют основной предмет философии.

Когда мы говорим о философии науки (т.е. как философия совмещается с научным познанием), то сталкиваемся с достаточно общим понятием – самим термином "<наука">.

Термин "<наука"> достаточно универсальный, так как мы конкретно не говорим ни о физических, ни о математических науках. И от того, как мы сформулируем термин «наука», к ней будет привязана и история самой науки.

Современное понимание:
Наука -- динамическая система объективных истинных знаний о существующих связях действительности, получаемых в результате специфической общественной деятельности и превращаемых в непосредственную практическую силу общества (наукоёмкие технологии, научно-производственные объединения).

В русском языке термин "<наука"> не совпадает с термином знание, а в греческом и латинском -- совпадает.
В греческом языке гносис – это знание с точки зрения его движения; а эпистема – это знание с точки зрения его структуры. В советское время был распространён термин гносис и гносеология как теория познания. А сейчас очень модный термин эпистема или эпистемология – учение о структурах знания. Это характерно для современной технологической культуры, особенно если мы рассматриваем кибернетику, искусственный интеллект, то для них как раз актуально учение о структурах знания.

Научная рациональность подчёркивает особый язык, причинные модели объяснения явлений, строгую форму логического и фактического обоснования утверждений и концепций.

Научное знание – специфически организованная система объективного знания, которая отвечает определённым критериям (предметность, воспроизводимость, объективность, обоснованность, полезность) и является фундаментальной базой технологических инноваций. Научное знание не зависит от конкретного учёного, а существует как объективная истина. Научное знание обладает собственным концептуальным языком построения. Научный дискурс (язык описания явлений) представлен чётко определёнными понятиями, построен в соответствии с признанными принципами объяснения (канонами научной рациональности), абстрагирован от эмоциональных и субъективных оценок.

Критерии научного знания:

1) рациональность всех содержащихся в научном знании положений и выводов (в научном знании не может быть ничего не доступного человеческому пониманию);

2) объективность, общезначимость, безличность;

3) воспроизводимость и проверяемость;

4) логическая строгость, точность и однозначность, что обеспечивается фиксацией условий получения знания; установлением точных (в пределах интервала допустимой погрешности) количественных значений изучаемых параметров;

5) Логическая взаимосвязь различных элементов научного знания, в силу которой оно представляет собой не сумму разрозненных сведений, а логически упорядоченную систему


\end{document}
