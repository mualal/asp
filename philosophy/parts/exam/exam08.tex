\documentclass[exam_answers.tex]{subfiles}

\fontsize{14pt}{14pt}\selectfont

\begin{document}

\renewcommand{\baselinestretch}{\blch}
\sublinksectionold{\normalsize (8) Концепции постпозитивизма. Критический рационализм К. Поппера.}

Поппер критикует два главных устоя логического позитивизма – принципов верификации и конвенционализма с точки зрения односторонности индуктивизма и психологизма в теории познания.

По Попперу проверка научной осмысленности и истинности научных теорий должна осуществляться не через их подтверждение, а преимущественно (или даже исключительно) лишь через их опровержение.

Неодинаковую роль подтверждающих и опровергающих фактов Поппер назвал познавательной «ассиметричностью».
На этом основании Поппер требует заменить принцип верификации принципом фальсификации.

Собственно научных утверждений (и теорий) не существует; имеют место лишь гипотезы, которые никогда в статус истинных научных теорий перейти не смогут.
Они используются лишь временно.
Другими словами, любые относительные истины – лишь принятые на время заблуждения.
\\

Следующее нововведение, которое Карл Поппер вводит в систему философии науки – это идея третьего мира знания.
Он, строго говоря, рассматривает структуру реальности через 3 уровня (через 3 мира), взаимодействие которых определяет развитие науки.
Первый мир – это мир физических сущностей.
Второй мир – духовные состояния человека, включающее его сознательное и бессознательное.
Третий мир – это мир «продуктов человеческого духа», который включает в себя средства познания, научные теории, научные проблемы, предания, объяснительные мифы, произведения искусства и т. п. 

Важно, что этот мир знания существует отдельно и независимо от каждого индивидуального субъекта познания.
И этот мир знания – то, с чем работает наука.

Объективированные идеи третьего мира живут благодаря их материализации в книгах, скульптурах, различных языках.
\\

Поппер строит определённую эволюцию научного знания.
В дальнейшем появляется термин эволюционная эпистемология (или эволюционная теория познания), где он предлагает модель смены научных теорий.

Если рассматривать модель развития научного знания Поппера (абстрактную, которая не завязана на сознании человека, а завязана только на росте научного знания, т. е. как растёт знание в третьем мире), то она представлена следующим образом (используется метод проб и ошибок): сначала есть некая исходная проблема, дальше её предположительное решение или «пробная теория», далее эту гипотезу подтверждают или опровергают экспериментальные результаты и затем формулируется новая проблема.
Метод фальсификации Поппера работает в этом случае, когда проводится решающий эксперимент и отбрасываются гипотезы (или переформулируются в новую проблему).

Важно то, что Поппер подчёркивает, что в процессе выдвижения гипотез участвуют не только собственно научные представления, но и другие идеи, поскольку его исходная позиция – это взаимосвязь трёх миров (мира природы, мира сознания человека и мира знаний), то соответственно мировоззренческие идеи тоже участвуют в процессе выдвижения гипотез.


\end{document}