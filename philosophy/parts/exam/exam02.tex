\documentclass[exam_answers.tex]{subfiles}

\fontsize{14pt}{14pt}\selectfont

\begin{document}

\sublinksection{Философские основания науки}

Философские основания науки – это некоторые позиции, которые относятся именно к познавательным процедурам. В системе науки большую интегрирующую роль играют только 2 принципа: единство мира (что именно изучает данная конкретная наука в общей единой системе) и принцип детерминизма (причинно-следственные связи; пример: в мире есть только атомы и пустота и всё, что мы наблюдаем – это движение атомов в пустоте). Причинно-следственная связь является ключевой связью в формировании закономерности.

Философские принципы в основании науки:

- онтологический принцип единства мира (как основание интеграции знания в научной картине мира);

- гносеологический принцип детерминизма (как основание познавательных стратегий науки в описании закономерности явлений, установка исследования на поиск причинно-следственной связи).

"<Онтос"> -- бытие, "<гносис"> -- знание, "<логос"> -- учение.

Онтология – учение о бытии. Гносеология – учение о познании.

В механическом взгляде причины связаны чисто с механическими действиями сил. В статистике причина – это случайность. В теории самоорганизации роль причины играет случайная флуктуация. Причинно-следственная связь всегда должна быть, но по-разному её трактуют.

Философские основания науки включают две взаимосвязанные подсистемы категорий (предельно общих понятий):

- онтологические категории (используются для описания объективной и субъективной реальности): "<пространство">, "<время">, "<материя">, "<состояние">, "<причинность">, "<необходимость">, "<случайность">, "<вещь">, "<свойство">, "<отношение">, "<сознание">, "<процесс">.

- гносеологические категории (используются в познавательных процедурах): "<истина">, "<метод">, "<знание">, "<описание">, "<объяснение">, "<доказательство">, "<теория">, "<факт">.

Уровни реальности в философской онтологии:

- объективная реальность существует независимо от наблюдателя, регистрируемого явления, регистрирующего прибора, мышления;

- феноменологическая реальность – существование наблюдаемого или регистрируемого явления, причины которого могут быть скрытыми и неизвестными, а физический смысл не ясен.

Идеальный уровень реальности – существование мысленных конструкций как самодостаточных и самостоятельных объектов в виде абстракций.

Сейчас система науки настолько разнообразна, что может изучать не только конкретные вещи, ни ещё и эфемерные вещи (функции), которые исследуются отдельно.
Основанием функциональных систем является не стабильная структура, а разнообразные структуры, которые соединяются в некий функциональный устойчивый цикл (например, круговорот живого вещества, круговорот воды).
Сначала появились в биологии, теперь есть и в кибернетике.



\end{document}
