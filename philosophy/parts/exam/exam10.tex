\documentclass[exam_answers.tex]{subfiles}

\fontsize{14pt}{14pt}\selectfont

\begin{document}

\renewcommand{\baselinestretch}{\blch}
\sublinksectionold{\normalsize (10) Т. Кун об исторической динамике науки.}

Следующая позиция в постпозитивизме получила название «историческая школа». Её основатель – Тома Кун, который написал всем известную работу «Структура научных революций».

Томас Кун знаменит тем, что он вводит в оборот конкретно-исторический субъект познания, который мы сейчас знаем под термином «научное сообщество».
Это достаточно пионерское введение, поскольку до этого момента субъект познания ассоциировался с индивидом (одним человеком и его сознанием).
А здесь получается, что субъект познания уже абстрактный субъект, называемый «научным сообществом».
Из этого следует, что каждое «научное сообщество» принимает собственные стандарты рациональности.

Кун вводит понятия парадигмы и доктрины.
Парадигма как то основание, которое даёт стандарт научной рациональности в том или ином научном сообществе.
Не случайно Томас Кун относится к тому движению, которое критикует преемственность научного знания (согласно Куну история науки предстаёт как совокупность разобщённых и не понимающих друг друга научных сообществ).

Ещё одна характеристика рациональности в концепции Куна связана с понятием нормальная наука, то есть отличительным признаком науки в данном случае является не сама по себе какая-то рациональность, а некие признаки или совокупность признаков, которыми характеризуется нормальная наука.

Нормальная наука последовательно развивает интерпретации и методы исследования мира из какой-то одной признанной парадигмы.

Когда начинаем рассуждать о научном знании, то если мы берём систему Куна, то у него всё привязано к нормальной науке, которая фиксирует через парадигму некие знания в научном сообществе.
А всё остальное попадает во вненаучную рациональность.

 Проводится граница между наукой и здравым смыслом (или обыденным знанием; повседневным рациональным действиям).
 
Переход от одной парадигмы к другой невозможен как последовательная плавная смена и наращивание знания.
А переход осуществляется через скачок (или как гештальтпереключение).
Поэтому периоды нормальной науки сменяются научными революциями, то есть процесс развития научного знания дискретный.
\\

Возникает проблема: как же всё-таки формируется новая парадигма?
На каком основании?
Этот вопрос остаётся открытым.
Хотя Томас Кун пишет, что в этой смене (переходе) участвуют не только чисто внутринаучные факторы, но и вненаучные (философские, эстетические, религиозные и вообще любые).
\\

Характерную особенность философии Кун усматривает в том, что в ней никогда не существовало единой общепризнанной концепции – парадигмы.
Каждый крупный философ создаёт свою собственную философскую систему, и философия в целом всегда представляет собой поле битвы различных точек зрения.

В науке же плюрализм теорий и их взаимная критика чрезвычайно редки, обычное состояние науки характеризуется объединением всех исследований в рамках одной господствующей концепции.

Учёные ведут себя подобно философам только тогда, когда должны выбирать между конкурирующими теориями.

В периоды кризисов наука перестаёт быть наукой и уподобляется философии.


\end{document}