\documentclass[exam_answers.tex]{subfiles}

\fontsize{14pt}{14pt}\selectfont

\begin{document}

\renewcommand{\baselinestretch}{\blch}
\sublinksectionold{\normalsize (16) Методология обоснования. Научная форма обоснования. Законы логики и принципы аргументации.}

Логическое обоснование гипотез в зависимости от характера исходных положений можно разделить:

-- на рассуждения, которые опираются на гипотезы или эмпирические обобщения, истинность которых ещё надо установить;

-- на рассуждения, которые опираются на посылки заведомо ложные или ложность которых может быть установлена.
В этом случае выведение следствия, противоречащего хорошо известным фактам или истинным утверждениям, позволяет скорректировать исходные позиции исследования.
Сведение к абсурду – наиболее распространённый способ опровержения, который дополняется проверкой следствий опытным путём.
\\

Законы логики представляют собой общие нормы рассуждения, регулирующие процессы речевого общения на уровне трансляции смысла (мыслекоммуникации).

Общие принципы логики, сформулированные ещё в Античные времена, направляют интеллектуальную деятельность человека посредством интерсубъективных критериев, выступающих ориентирами внутреннего (личностного) осмысления языковых выражений.
Нарушение законов логики, которое квалифицируется как логическая ошибка и парадокс, демонстрирует, прежде всего, ситуацию скрытого или явного непонимания.
\\

Закон тождества: в процессе доказательной аргументации нельзя подменять данную мысль другой.
Каждая мысль должна быть тождественна самой себе.
Нарушение закона тождества ведёт к ошибке, которая называется подменой понятия или тезиса.
Неявная подмена понятия совершается в шутках, обыгрывающих многозначность словесных выражений.
\\

Закон противоречия: об одном и том же, в одно и то же время, в одном и том же отношении нельзя утверждать и отрицать.
\\

Закон исключённого третьего: если одна мысль представляет собой простое отрицание другой мысли, то они не могут быть вместе ни истинными, ни ложными.
\\

Закон достаточного основания: мысль может быть признана истинной лишь в том случае, если она достаточно обоснована.
В вопросительных и оценочных ситуациях закон достаточного основания играет роль смысловой границы, ориентирует на контроль истинности высказываемых утверждений и формулирование более точного ответа, отсекают нехарактерные оценки.
\\

Гипотетико-дедуктивный метод в логической проверке предположений позволяет выбрать одну из конкурирующих гипотез посредством опровержения другой.


\end{document}