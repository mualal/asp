\documentclass[exam_answers.tex]{subfiles}

\fontsize{14pt}{14pt}\selectfont

\begin{document}

\renewcommand{\baselinestretch}{0.75}
\sublinksectionold{\normalsize (23) Отношение общества к науке. Сциентизм и антисциентизм}

Эпоха Просвещения в истории европейской культуры выдвинула идею прогресса, отождествив науку с разумом, а прогресс науки как коллективного разума – с общественным прогрессом.
С тех пор наука выступала в качестве неоспоримой культурной ценности, которая играла ключевую роль критерия социального прогресса.

Во второй половине XX в. отношение общества к науке становится неоднозначным в связи с глобальными проблемами экологического плана.
Складываются две альтернативные позиции: сциентизм и антисциентизм.

Сциентизм трактует науку как величайшую ценность.
Сторонники этого взгляда убеждены в необходимости и благотворности научного подхода к решению всех проблем жизни людей.
В противоположность им антисциентисты говорят об антигуманности науки и необходимости ограничить её развитие, дают негативную оценку достижениям науки, акцентируют их разрушительные последствия.

Сциентизм настаивает на том, что только дальнейшее развитие науки может спасти человечество от бед, порождённых научно-техническим прогрессом.
В антисциентизме выражается разочарование как в научно-техническом прогрессе, так и в науке.

Современная наука пугает многих своей заинтересованностью милитаристскими проектами и недоступностью (для тех, кому не хватает знания и таланта).
Распространяются слухи об ужасных открытиях и изобретениях, которые грозят человечеству поголовным зомбированием, гибелью генофонда, рабством под властью машинного интеллекта, созданными в научных лабораториях вирусами.
В то же время в глазах общества наука продолжает оставаться важнейшей силой, с помощью которой решаются разнообразные социальные задачи.
Более того, в современной практике наукоёмкие технологии задают темпы экономического развития, становятся критерием государственной образовательной и технической политики.


\end{document}