\documentclass[exam_answers.tex]{subfiles}

\fontsize{14pt}{14pt}\selectfont

\begin{document}

\renewcommand{\baselinestretch}{0.75}
\sublinksectionold{\normalsize (22) Принципы синергетики (теории самоорганизации) и универсальный эволюционизм в формировании современной научной картины мира.}

Принципы синергетики:

-- признание универсальности согласованных процессов в природе;

-- признание универсального характера адаптации как закономерного поведения сложной системы любой природы;

-- признание закономерности критического состояния в эволюции сложной системы любой природы (эволюция системы анализируется в терминах порядка и хаоса);

-- утверждается относительность простоты и сложности системы (всякую систему одновременно можно рассмотреть на макроуровне как целостность, описываемую немногими параметрами порядка, и на микроуровне как сложное взаимодействие множества элементов);

-- общая картина эволюционного процесса предстаёт как смена условных состояний порядка и хаоса, которые соединены фазами перехода к хаосу (гибель структуры) и выхода из хаоса (самоорганизация);

-- вероятностный детерминизм как основание прогноза состояний сложных систем.
\\

Универсальный эволюционизм в построении научной картины мира опирается на междисциплинарные принципы системности, самоорганизации, эволюции.
\\

Элементарный объект в синергетике – колеблющийся элемент (или циклический процесс) – осциллятор.

Картина эволюции системы представляется графически – как непрерывное изменение координаты и скорости.

Точка, изображающая состояние системы, движется по фазовой траектории, которая для линейного осциллятора представляет собой эллипс.

В случае затухания колебаний фазовые траектории при любых начальных условиях заканчиваются в точке, которая соответствует состоянию покоя в положении равновесия.

Эта особая точка в фазовом пространстве как бы притягивает к себе со временем все фазовые траектории, поэтому получила название аттрактора.

Другой вид аттракторов (помимо особой точки) представлен предельными циклами, которые указывают на некоторый установившийся ритмический режим, например, биение сердца.

Аттрактор выступает обобщением понятия равновесия в эволюции системы.


\end{document}