\documentclass[exam_answers.tex]{subfiles}

\fontsize{14pt}{14pt}\selectfont

\begin{document}

\renewcommand{\baselinestretch}{0.75}
\sublinksectionold{\normalsize (21) Предпосылки формирования экофилософии (учение о биосфере и ноосфере В.И.Вернадского, русский космизм, Римский клуб о глобальных кризисах)}

Ещё одна область, которая получила развитие в философии науки, получила название экофилософии.
У неё, конечно, есть более глубокие философские научные корни (философские учения русского космизма).
Это работы Вернадского и Циолковского.

В становлении экофилософии решающее значение сыграло учение В.И.Вернадского (1863-1945) о биосфере, в котором ключевое положение занимает трактовка живого вещества как единой системы всех растительных и животных организмов планеты, естественного компонента земной коры, наряду с минералами и горными породами.

Согласно системному биокосмическому принципу Вернадского необходимо рассматривать живую природу Земли как целостную систему, взаимодействующую с вещественно-энергетическими процессами, протекающими в земных, околоземных и отдалённых пространствах Космоса.

Такое обобщение вводит новые функциональные системы в виде обменных циклов (биогеоценозов), позволяет рассматривать биосферное единство в его внутренних и внешних взаимосвязях:

-- изменение системных макроусловий оказывается эволюционным фактором, меняющим потенциальную норму жизни ситемы, что вызывает её кардинальную перестройку;

-- новая структура и её новые свойства вроде бы не имеют видимых оснований.
Такой характер возникновения специфических для новой целостности свойств получил название эмерджентной эволюции (наглядный пример – принцип действия калейдоскопа).
В этом же ключе развиваются представления о системной детерминации в современной биологии.

-- Жизненное пространство, образующее макроуровень жизни органической системы, очерчено единством системных условий, которые с точки зрения элементов самой системы (микроуровня) воспринимаются как априорные ограничения
\\

Биогеоценозы позволяют исследовать взаимосвязь не только человека и природы, но прежде всего взаимосвязь климатических зон экосистем в едином пространстве биосферы, которая живёт как некое единое жизненное пространство.


\end{document}