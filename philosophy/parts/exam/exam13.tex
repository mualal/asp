\documentclass[exam_answers.tex]{subfiles}

\fontsize{14pt}{14pt}\selectfont

\begin{document}

\renewcommand{\baselinestretch}{0.75}
\sublinksectionold{\normalsize (13) Динамика науки как процесс порождения нового знания. Формы развития научного знания: проблема и гипотеза.}

Специфика творчества в науке определяется взаимосвязью трёх компонентов:

1) проблемы (задачи);

2) эвристических методов;

3) интеллекта и психологических особенностей человека.

Научное творчество – это всегда интенсивная интеллектуальная работа.
\\

Проблема. Проблемная познавательная ситуация характеризуется скрытым вопросом. Проблема выражается неразрешимым противоречием – антиномией.

В общем смысле под проблемой понимается отражаемая системой вопросов и высказываний ситуация, для которой характерно наличие цели и отсутствие знания о путях её достижения.

Проблема – нечёткая смысловая структура, имеющая некую информационную «среду обитания», контекст.
Упорядоченный контекст в виде связной системы понятий и представлений образует фрейм проблемы – семантическое пространство, в котором осуществляется поиск путей её решения.
Проблема – это достаточно фундаментальная в практическом и теоретическом отношении познавательная ситуация, способы решения которой неизвестны или известны не полностью.

Различают неразвитую и развитую проблемы.

Неразвитая проблема – это нестандартная задача, не имеющая алгоритма решения, которая возникла на базе определённого знания и направлена на устранение противоречия между смысловой и фактической стороной познавательной ситуации.

Развитая проблема – это «знание о некотором незнании», дополненное указанием путей устранения очерченного круга незнания.
Другими словами, это некоторая ограниченная область поиска, в которой просматривается возможный результат и хотя бы общая стратегия исследования.
\\

Формулировка проблемы – сложная интеллектуальная операция, которая включает в себя, как правило, три части:
1) систему исходных утверждений или описание фактических данных;
2) постановку вопроса – что нужно найти;
3) методологический принцип – систему указаний на возможные пути решения, другими словами, стратегию поиска или эвристику. Для неразвитой проблемы невозможно или трудно выполнить третийпункт.
\\

Гипотеза.
Познавательная ситуация характеризуется ориентацией на ответ.
Гипотеза выражается вероятным знанием – идеей, моделью, версией, разрешением проблемной ситуации.

Гипотеза должна удовлетворять ряду требований, соблюдение которых хотя и не обеспечивает их истинность, но даёт им право на существование в науке.
Важнейшие требования: логическая непротиворечивость; принципиальная проверяемость; фальсифицируемость; предсказательная сила; максимальная простота; преемственность.


\end{document}