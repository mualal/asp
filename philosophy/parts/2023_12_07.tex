\documentclass[main.tex]{subfiles}

\begin{document}

\section{Лекция 07.12.2023 (Шипунова О.Д.)}

\insertlectureslide[0]{1}{05}

Сегодня тема очень обширная.
Социология науки.
Она будет предполагать с одной стороны некое продолжение обзора концепций постпозитивизма, а с другой стороны и историю социального института, наука как социальный институт и характеристику уже современных социальных аспектов развития науки, которые связаны с понятием инновационных процессов.
Социология уже более современная.

\subsection{Проблема механизма развития научной деятельности в философии науки}

\insertlectureslide[0]{2}{05}

Первое с чего мы начинаем -- это с конца прошлой лекции, где говорили, что одна из проблем, которая возникает в рамках течений постпозитивизма связана с тем, что исследуется абстрактный субъект научного познания "<научное сообщество">.
Это во-первых.
А во-вторых, возникает проблема механизмов развития самой науки как таковой, научного знания, которое включается в состав (понятие) структуры опыта науки, и соответсвенно механизмов развития собственно научной деятельности в таком интегральном виде (не как индивидуальной деятельности, а как деятельности "<научного сообщества">).
С этим как-раз и связаны те принципы, которые формулируются в рамках постпозитивистских учений, которые формируются как 2 альтернативных принципа, но смысл этих принципов связан с поиском стимулов развития научной деятельности.
Каковы стимулы и причины, движущие развитием (ростом) научного знания?

В рамках течений позитивизма выдвигаются два таких принципа: принцип интернализма и принцип экстернализма.

В слабом интернализме по Попперу источники развития научного знания связываются с динамикой постановки и развития проблем.

В слабом интернализме по Лакатосу источники развития научного знания связываются с развитием ядра исследовательских программ.

В слабом интернализме по Тулмину источники развития научного знания связываются с развитием концептуальных структур.

Сильный интернализм усиливается тем, что вообще отрицается социокультурное влияние на процесс научного познания.
А в слабом варианте интернализма это влияние учитывается.
Именно Карл Поппер вводит идею о том, что социокультурный контекст влияет на рост научного знания.

Противоположный принцип экстернализма фиксирует внешние причины, социально-исторические традиции и ценностные факторы, которые выступают причиной развития науки.
Это более свежая концепция.
Она формируется в XX веке, поэтому если мы посмотрим на авторов, которые разрабатывают эту идею, то все они живут в XX веке и в XXI веке.

Опыт науки показывает, что принцип интернализма и принцип экстернализма находятся в отношении дополнительности.
Внутренняя детерминация определяет логику развития научных идей, а внешняя определяет ту доминирующую тенденцию в развитии науки, которая связана с социальными условиями, социальными потребностями и социальным заказом.

\subsection{Положения интернализма}

\insertlectureslide[0]{3}{05}

Более подробно концепция интернализма связана с общей идеей, что идеи возникают только из идей.
С одной стороны это развитие принципа Декарта о некоторых порождённых идеях, интеллектуальной интуиции.
Но уже на уровне постпозитивизма предполагается, что эти идеи -- это не просто идеи индивидуальной головы (одного единственного субъекта познания), а это идеи, которые в науке доказаны.

Поэтому внутренняя детерминация развития науки определяется необходимым уровнем начальных знаний (интеллектуальным потенциалом) и необходимым техническим аппаратом (или техническим потенциалом).

Общественные условия влияют на ход развития науки, но не являются приоритетными.
С точки зрения интернализма идеи всё-таки возникают из идей.
Рост научного знания абстрагируется от конкретных исторических условий.

\subsection{Положения экстернализма}

\insertlectureslide[0]{4}{05}

Положения экстернализма опираются на то, что нельзя понять причины развития науки вне тех социальных условий, в которых она развивается.
Существующее социально-культурное пространство жизни влияет на развитие науки, потому что наука (как и любое знание) есть порождение общества и является одной из отраслей общественного труда.

Социальные потребности выливаются в такое понятие как социальный заказ.

Понимание роста научного знания следует здесь антикумулятивной модели.
Здесь тоже выдвигается сильный принцип экстернализма, согласно которому любое знание социально детерминировано.

Как развивается и функционирует наука как социальный институт?
Не как индивидуальная деятельность, а как социальный институт.
И не просто как "<научное сообщество"> по Томасу Куну, а уже более сложный институт, который имеет разные формы.

Наиболее известный представитель слабой позиции экстернализма -- это Майкл Малкей.
Он видит цель социологии в выявлении социальных условий и мотивов исследовательской деятельности.
В этом плане как раз таки социология науки формируется как отдельный предмет в отличие от философии и собственно науки.

\subsection{Концепции социологии науки}

\insertlectureslide[0]{5}{05}

В постпозитивизме наиболее классический вариант концепции социологии науки выдвигает Роберт Мертон (этот вариант называют "<стандартная концепция"> в социологии знания).
Он формирует такие принципы, которые получили название структурно-функциональная онтология.

Мертон вводит понятие научный этос.
Можно говорить о девиантном (отклоняющемся) поведении учёного, если этот учёный не следует ценностно-нормативному набору. 

\subsection{Программа "<дискурс-анализа"> Майкл Малкей}

\insertlectureslide[0]{6}{05}

Следующая концепция получила название "<дискурс-анализа">.

Само научное знание здесь трактуется в духе релятивизма.
Какого-то выделенного статуса в системе знания научное знание не имеет.

"<Дискурс-анализ"> пытается исследовать научное знание через смысловые уровни или смысловые планы собственно языка.

"<Дискурс-анализ"> показывает разные уровни смысла даже в научном знании.
Выделяет 3 взаимосвязанных уровня смысла.

\subsection{Концепция идеальных типов Макса Вебера}

\insertlectureslide[0]{7}{05}

Следующая концепция, которая тоже имеет значение к социологии науки -- это концепция идеальных типов Вебера.

Макс Вебер интересен тем, что он вводит такие понятия как целерациональность, формальная рациональность, познавательный интерес, идеальный тип.

Именно целерациональность предполагает расчёт (исчисление, вычисление) полезного результата действия (который выступает основой формальной рациональности).
Именно эта идея проявляется во второй половине XX века в компьютерных вычислениях.
Это базовая идея для вычислительной техники, которая сводит мышление и познание к исчислению полезного результата действия. 

\subsection{Принцип отнесения к ценности в концепции М. Вебера}

\insertlectureslide[0]{8}{05}

Ещё одно введение Макса Вебера связано с понятием отнесения к ценности, где он проводит различие между ценностью как интересом эпохи и ценностью как просто частный интерес.
А отнесение к ценности -- некая внутренняя установка к познавательной деятельности, которая предполагает некое индивидуальное движение и объясняет те познавательные тенденции, которые в науке существуют.

Макс Вебер считается одним из основателей философии ценностей.
Он пытается сформулировать понятие ценности и принципа отнесения к ценности в отличие от, так скажем, философии Канта.

Факт отнесения к ценности определяет мотивы и направляет индивидуальные впечатления к некоторой субъективной оценке.
С точки зрения Вебера, чтобы человек сформулировал оценочное суждение, он сначала интуитивно должен этот выбор совершить.

\subsection{Концепция коммуникативной рациональности Юрген Хабермас}

\insertlectureslide[0]{9}{05}

Ещё один представитель, которого можно отнести к социологии науки -- это Юрген Хабермас, который формулирует концепцию коммуникативной рациональности.
Важно то, что он вообще формулирует теорию коммуникативного поведения и при этом он пытается рационально представить эту систему взаимодействия через противопоставление двух миров.
Двухступенчатое строение общества как системы объективных обстоятельств, объективных структур и жизненного мира, в котором эти все структуры и системы находятся во взаимодействии с личностью.
Этот момент, который характеризует взаимосвязь объективных структур (технические, экспериментальные манипуляции; практические вопросы) и систем жизненного мира (общества и человека), где как-раз коммуникация является базовым принципом взаимосвязи.

Наибольший вклад Хабермаса в систему связан с тем, что он формулирует с одной стороны различие общества как системы, а с другой стороны связь личности и других личностей во внутрисубъективном пространстве через концепт жизненного мира, который в себя включает и объективные системы и смыслы, и деятельность, её мотивы и так далее.

Это обсуждались некоторые такие идеи по социологии науки (Мертон, Малкей, Вебер, Хабермас) для постпозитивизма. 

\subsection{Наука как социальный институт}

\insertlectureslide[0]{10}{05}

Далее характеризуем науку как социальный институт и её историю становления.

\subsection{История становления форм научной коммуникации}

\insertlectureslide[0]{11}{05}

История становления форм научной коммуникации -- это одна из новых идей в социологии науки, потому раньше на это как-то меньше обращали внимание.
Напомню, что вплоть до середины XX века в системе науки считалось, что субъект познания -- индивидуум.

Понимание о коллективном (групповом) субъекте (о "<научном сообществе">) -- это уже идея XX века.

\insertlectureslide[0]{12}{05}

Функции научных журналов и их статей, которые сложились в середине XIX века, практически не изменились.

Мы не стали вести историю от Средневековья, потому что до эпохи Возрождения вся система развития знания и образования была под эгидой Духовенства.

Мы обсуждаем уже Новое время.

\subsection{Становление социальных институтов науки}

\insertlectureslide[0]{13}{05}

Следующий этап развития социальных движений в науке -- это становление социальных институтов.

Наиболее показательный социальный институт в науке -- это Академическое учреждение.

В Академических учреждениях работали учёные, которые занимались только наукой.
Это не Декарт, не Алессандро Вольта, не Бенджамин Франклин, которые просто занимались наукой, потому что было интересно.
Франклин при этом был чуть-ли не президентом США.

Государство содержит Академические учреждения и обеспечивает деятельность учёного.

Также возникают свободные ассоциации учёных и дисциплинарные научные сообщества по интересам, которые складываются как раз вокруг научных журналов.

Новый тип профессиональной деятельности (университетский профессор) и подготовка научных кадров как раз и связаны с развитием университетов.

\subsection{Развитие университетов}

\insertlectureslide[0]{14}{05}

Первые университеты (в XII-XIII веках) создаются на базе духовенства.

На базе университетов складывается система дисциплинарно-организованного обучения.

Это расширение деятельности в системе науки и дисциплин приводит к новой системе организации науки, которая получила название Большая Наука.

\subsection{Большая наука}

\insertlectureslide[0]{15}{05}

Эффективность научных исследований связана с симбиозом формальных коллективов (научных лабораторий) с неформальными формами общения.

\subsection{Наука -- производительная сила общества}

\insertlectureslide[0]{16}{05}

Ещё один момент, характеризующий социальный статус науки -- это так называемое понимание науки как производительной силы общества.

\subsection{Инновационная деятельность}

\insertlectureslide[0]{17}{05}

И следующая характеристика, которая соответствует современному состоянию общества -- это понятие инновационной деятельности, которая тоже связывается с развитием науки как производительной силы (здесь конечно больше акцент на создание новых продуктов и наукоёмких технологий).

Социальная структура науки в этом плане вписывается  в контекст инновационной деятельности.

Есть понятие новация -- это нечто новое.

А инновация -- это не просто новая идея, а именно внедрение, которое приводит к активному или пассивному изменению системы в отношении к внешней среде.

\subsection{Инновационная политика}

\insertlectureslide[0]{18}{05}

Поскольку инновационная деятельность не определяется одним институтом/университетом, то она требует интегрального подхода, который получил название инновационная политика.

Политика, которую конечно может иметь частная фирма.
Но прежде всего это, конечно, политика, которая связана с государственными установками и прежде всего такая политика должна учитывать некоторые объективные факторы инновационного цикла.

Объективные факторы связаны с тем, что этот цикл требует больших вложений и имеет много ступеней (3 основных -- исходный этап, второй этап и третий этап).

\insertlectureslide[0]{19}{05}

Здесь приведены примеры, которые подтверждают объективные факторы инновационного цикла, который связан именно с экономическим аспектом.

\subsection{Модели инновационного развития}

\insertlectureslide[0]{20}{05}

Здесь представлено небольшое обобщение так называемых моделей инновационной политики, которые сложились к началу XXI века.

Научно-технический прогресс регулируется (управляется) определённой моделью инновационной политики.
На слайде представлены модели, которые можно выделить.

ТНК = транснациональные компании.
Внегосударственная модель, которая может прийти к противоречию с интересами государства.

\subsection{Модели научно-технического развития в государственной политике}

\insertlectureslide[0]{21}{05}

Если рассматриваем модели технического развития в рамках государственной политики, то здесь можно выделить 3 модели.

\subsection{Проблема ценности научно-технического прогресса}

\insertlectureslide[0]{22}{05}

Следующий момент, который связан с развитием социологии науки -- это конечно те проблемы, которых раньше в науке не было, но они появились и связаны с такими понятиями как ценность научно-технического прогресса, критерии научно-технического прогресса и так называемые этические проблемы.

Почему они возникают?
Потому что кризисы экологические, глобальные, которые фиксируются во второй половине XX века, рождают идею прогнозирования или хотя бы примерного представления о будущем цивилизации (сохранение или уничтожение).

Если мы говорим о критериях природопользования, то здесь выдвигается 2 принципа: комплексное использование ресурсов природы (безопасность биосферного пространства жизни, в котором осуществляется деятельность) и критерий ресурсосберегающих технологий.

\insertlectureslide[0]{23}{05}

Как в условиях интенсивного влияния на природу технологических процессов (и создания так называемой техносферы, которая давит на природные циклы) обеспечить устойчивое развитие и человеческого общества, и природного пространства жизни?

Это входит в понятие ценностных ориентиров развития цивилизации, то есть органичное встраивание технического прогресса в естественное жизненное пространство и культурные традиции человечества.

Ориентиром является равновесие общества и природы, мира природного и мира искусственного и так далее по слайду.

Одна из задач, которая из этого (этих критериев) вытекает -- это формирование нового мировоззрения в эпоху глобальных кризисов.
Направления этого нового мировоззрения, которые характеризует современную философию науки: философия устойчивого научно-технического и хозяйственного развития, биоэтика, принцип коэволюции в биосферном единстве, экофилософия.

\subsection{Биоэтика}

\insertlectureslide[0]{24}{05}

Интересный термин.
На самом деле этика (наука о нравственности) -- это система, которая обеспечивает гармонию общественных отношений.

А биоэтика касается не только человека, но и любых живых организмов.

Сейчас более развита отдельная область научной специализации, которая получила название биоэтика.
В узком смысле это специализация или специальность научная, которая обращена к кругу этических проблем во взаимодействии в медицине (врача и пациента).

\subsection{Экофилософия}

\insertlectureslide[0]{25}{05}

Ещё одна область, которая получила развитие в философии науки, получила название экофилософия.
У неё конечно есть более глубокие философские научные корни (философские учения русского космизма).
Это, конечно, работы Вернадского и Циолковского.

Наибольший приоритет в этой области, конечно, у Вернадского.

Вообще надо сказать, что первая идея Вернадского, которая получила название геохимической концепцией в эволюции Земли -- это идея о том, что объяснить эволюцию Земли и её расширение (есть такая проблема) можно идеей о миграции атомов химических элементов.
То есть его геохимия построена на понятии миграции химических элементов в оболочках Земли.
А дальше Вернадский переходит уже к биосферной концепции, где формулирует концепцию о круговороте живого вещества.

Вообще идея Вернадского связана с так называемым биокосмическим принципом, согласно которой живая природа Земли (биосфера) выступает как целостная система, взаимодействующая с вещественно-энергетическими процессами, которые протекают в земных, околоземных и отдалённых пространствах Космоса.

Обсудили момент, связанный с экофилософией, его концептуальную основу, а собственно говоря дальше на этой базе начинают разворачиваться и новые идеи.
Сейчас в науке мы видим дисциплины экологического цикла, которые строятся на концепции биосферы и охватывают совершенно новые единицы исследования природы.

Биогеоценозы позволяют исследовать взаимосвязь не только человека и природы, но прежде всего взаимосвязь климатических зон экосистем в едином пространстве биосферы, которая живёт как некое единое жизненное пространство.

\insertlectureslide[0]{26}{05}

Просмотр видеоролика о 20 открытиях XXI века.

\end{document}
