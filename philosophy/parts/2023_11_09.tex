\documentclass[main.tex]{subfiles}

\begin{document}

\linksection{Лекция 09.11.2023 (Шипунова О.Д.)}

\insertlectureslide{1}{01}

\insertlectureslide{2}{01}

\insertlectureslide[0]{3}{01}

Чем занимается философия науки?
И чем она полезна для других наук?
Почему возникает коллизия между позитивным научным знанием (позитивизмом) и философией?

\insertlectureslide{4}{01}

\sublinksection{Предмет философии науки. Исходные понятия}

\insertlectureslide[0]{5}{01}

Философия работает с категориями, а категория -- это предельно общее понятие.
По аналогии в физике используем понятие материя и при этом не сосредотачиваемся на том, что это такое (каким образом она выражается).
Как только вы начинаете её конкретизировать, то вы начинаете исследовать структуру материи и так далее.

Категории составляют основной предмет философии.

Когда мы говорим о философии науки (у нас такая достаточно прикладная дисциплина с точки зрения философии), т.е. как философия совмещается с научным познанием.
И мы сталкиваемся с достаточно общим понятием -- самим термином наука.

И термин наука оказывается достаточно универсальным, так как мы конкретно не говорим ни о физических, ни о математических науках, а говорим в общем (не конкретизируем).
И от того, как мы сформулируем термин наука, к ней будет привязана и история самой науки.

Есть много определений термина наука.
Здесь представлено развёрнутое определение, которое характеризует современную науку и как систему знаний, и как специфически организованную познавательную деятельность, и как непосредственную производительную силу общества (наукоёмкие технологии, научно-производственные объединения).

К началу XX века наука о познании природы создала такой мощный потенциал знания, что весь XX век этот потенциал воплощается в железо.

Ранее тоже были примеры воплощения накопленного потенциала в железо: XVIII век -- век часов -- на основе теории математического маятника; XIX век -- век паровых машин и потом уже электрических.

В русском языке термин наука не совпадает с термином знание, а в греческом и латинском совпадают.

В греческом языке гнозис -- это знание с точки зрения его движения, а эпистема -- это знание с точки зрения его структуры.

В советское время был распространён термин гнозис и гносеология как теория познания.
А сейчас очень модный термин эпистема или эпистемология -- учение о структурах знания.
Это характерно для современной технологической культуры, особенно если мы берём кибернетику, искусственный интеллект, то для них как раз актуально учение о структурах знания -- как они сохраняются, понимаются, накапливаются и как знание генерируется.
Эти вопросы являются и философскими вопросами.

Научное знание -- объективное знание, т.е. не зависит от конкретного учёного, а существует как объективная истина.

Вопрос: когда вы читаете учебник, то вы верите ему или считаете, что это объективная истина?
Что больше?
Верим науке?
Или понимаем?
Это ещё 1 аспект философский дискуссий -- философская вера.
Можно верить в знание.

Научное знание обладает своим собственным концептуальным языком построения.

В научном дискурсе даже больше чем в философии требуется точность определений.
Давайте договоримся о словах.
Но дать точное определение далеко не всегда возможно.
Поэтому на защите важно, чтобы оппоненты вас поняли.
Одни и те же формулировки можно понимать по разному.

До сих пор никто не дал точного определения электричеству.
Но можно заменить определение демонстрацией явления.
Электричество и магнетизм сначала фиксировали чисто опытным путём.

\sublinksection{Задачи философии науки}

\insertlectureslide[0]{6}{01}

Наука как социокультурный феномен.
Научное знание существует в социуме, только в социуме и не даётся с рождением человека.
Поэтому в древней Индии люди, которые имели отношение к ведам, ведали (т.е. знали), считались дважды рождёнными.

Наука представляет собой форму социальной памяти.
Т.е. не генетической памяти, а именно социальной памяти.

Как согласовать научные методы и научные описания разных предметных областей?
Как связать языки научных исследований с теми областями жизни, которые называются гуманитарными?

Задачи философии: как развивается конкретное научное знание (ситуация кризиса всегда разрешается выходом в некую более широкую область).
Революция в физике и в химии: появление кислородной теории горения.

Когда будете читать первоисточники, то увидите, что философские проблемы науки задают не философы, а сами учёные.

Идеал научной рациональности как некая норма познавательной деятельности, в которую входят объяснение, обоснование, язык, изложение и так далее.

\sublinksection{Научная картина мира -- Предмет философии науки}

\insertlectureslide[0]{7}{01}

Интеграция знания оказывается достаточно непростым, потому что знание постоянно растёт.
Интеграция в научную картину мира (которая противопоставляется мифам, религии и так далее).
Натурфилософия (или философия природы) формулирует 2 принципа интеграции научного знания.
Как познать единство мира в его многообразии без помощи Бога, а только с помощью умозрения?
Логическая культура обоснования -- натурфилософия.
Математическая натурфилософия -- физика.
Механика Ньютона (пространство и время).

Сейчас речь идёт о более глобальных категориях, например, самоорганизация.

\insertlectureslide{8}{01}

\sublinksection{Философские основания науки}

\insertlectureslide[0]{9}{01}

Философские основания науки -- это некоторые позиции, которые относятся именно к познавательным процедурам.
В системе науки большую интегрирующую роль играют только 2 принципа: единство мира (что именно изучает данная конкретная наука в общей единой системе) и принцип детерминизма (причинно-следственные связи; в мире есть только атомы и пустота и всё, что мы наблюдаем -- это движение атомов в пустоте).
Причинно-следственная связь является ключевой связью в формировании закономерности.

Механический взгляд: причины связаны чисто с механическими действиями сил.

Статистика: по другому трактуется причина -- случайность.

В теории самоорганизации роль причины играет случайная флуктуация.

Причинно-следственная связь всегда должна быть, просто по-разному её трактуют.

\insertlectureslide{10}{01}

\sublinksection{Уровни реальности в философской онтологии}

\insertlectureslide[0]{11}{01}

Объективная реальность (т.е. не зависящая от человеческого сознания группового или индивидуального) может включать и материальный уровень, и идеальный уровень.

А феноменологическая реальность фиксирует существование некоторого явления, закономерности которого не установлены, но зато ему можно дать описание (дискрипцию, как сейчас модно говорить).

\sublinksection{Онтологический статус исследуемого объекта}

\insertlectureslide[0]{12}{01}

Сейчас система науки настолько разнообразна, что может изучать не только конкретные вещи, но ещё и эфемерные вещи (функции), которые исследуются отдельно.
Основанием функциональных систем является не стабильная структура, а разнообразные структуры, которые соединяются в некий функциональный устойчивый цикл (например, круговорот живого вещества, круговорот воды).
Сначала появились в биологии, теперь есть и в кибернетике.

\sublinksection{Функциональные системы}

\insertlectureslide[0]{13}{01}

Учёные практически всегда фиксируют проблемы через философскую позицию: фиксируются формы бытия (где, как и что есть).

\insertlectureslide{14}{01}

\sublinksection{Соотношение позитивного научного и философского знания}

\insertlectureslide[0]{15}{01}

Почему различаются системы позитивного научного и философского знания?

Философия оперирует очень общими категориями и не имеет прямого практического применения.
На чём и построена проблема демаркации (разделения научного и философского знания).
Отрицание рациональной роли философии для науки.

В начале XIX века было заявлено, что наука и научное знание -- это то, что может быть полезно человеку и которое приложимо.
Сила знания проявляется таким образом, что позитивное знание считается ценным, полезным, приложимым, а философия -- это метафизика, т.е. это рассуждения, которые не имеют смысла для науки и для человека (другими словами, её полезностью можно пренебречь).

К сожалению для первых позитивистов, окончательно разорвать связь с философией самим учёным тоже не удалось.

\sublinksection{Возникновение науки и основные стадии её исторической эволюции}

\insertlectureslide{16}{01}

\sublinksection{Проблема начала науки}

\insertlectureslide[0]{17}{01}

Неолитическая революция -- меняется образ жизни (появляется деятельность, направленная на преобразование природы под необходимости выживания человека -- переход от присваивающего хозяйства к производящей экономике).

Основа неолитической революции -- передача знания (примером или магическими ритуалами).

Посвящение -- имеешь право стать дважды рождённым.
Платон был посвящён в геометры.

\insertlectureslide[0]{18}{01}

Знание о природе, получаемое в процессе исследовательской деятельности.
Имеет свои методы и языки описания.
Структурировано.

\sublinksection{Проблема периодизации предыстории современной науки}

\insertlectureslide[0]{19}{01}

Преднаука эпохи возрождения.
Если мы говорим о революции в науке, то это говорит о том, что наука до этого (до преднауки) была, но в других формах (пранаука, протонаука).

В Милетской школе было сформулировано понятие атом.

В Пифагорейской школе начало мира -- число, гармония.
Космос, гармоничный мир.
А раз он гармоничный, то значит он познаваемый.
А как он познаётся?
С помощью математики, геометрии.
Хаос -- нечто непознаваемое.

\sublinksection{Эволюция современной науки}

\insertlectureslide[0]{20}{01}

Система современной науки формируется всё-таки в XVIII веке, благодаря распространению и развитию классической рациональной механики Ньютона.

Классический этап развития науки -- расчёт и эксперимент.
Идея математической физики у Декарта -- тождество физической и математической реальности.

Неклассический период -- атом не является неделимым.

На микромасштабе невозможно описать явление с классической точки зрения.
Электрон иногда удобно описывать как частицу, а иногда как волну.

\insertlectureslide[0]{21}{01}

На первый план выходит проблема интеграции научного знания.

Биохимия (теория биосферы) Вернадского.

Проблема генетики в биологии.

В 50-х годах формулируется новый научный категориальный аппарат научного познания на базе теории систем.

Сейчас говорим о системах адаптивных, саморазвивающихся, самоорганизующихся.

Информациология.

\sublinksection{Исторический тип научной рациональности}

\insertlectureslide[0]{22}{01}

Научная рациональность -- комплекс процедур и стилей мышления, который характеризует ту или иную эпоху в развитии науки.

Каждую историческую эпоху горизонт знания всегда ограничен.

Ньютон не смог бы сформулировать теорию Эйнштейна.

\insertlectureslide{23}{01}

\sublinksection{Глобальные научные революции}

\insertlectureslide{24}{01}

\insertlectureslide[0]{25}{01}

Вторая научная революция называется дисциплинарной, потому что именно в середине XIX века оформляются те дисциплины, которые нам все знакомы (физика, химия, биология).

\insertlectureslide{26}{01}



\end{document}
