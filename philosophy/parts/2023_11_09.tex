\documentclass[main.tex]{subfiles}

\begin{document}

\section{Лекция 09.11.2023 (Шипунова О.Д.)}

\insertlectureslide{1}{01}

\insertlectureslide{2}{01}

\insertlectureslide{3}{01}

\insertlectureslide{4}{01}

\subsection{Предмет философии науки. Исходные понятия}

\insertlectureslide{5}{01}

\subsection{Задачи философии науки}

\insertlectureslide{6}{01}

\subsection{Научная картина мира -- Предмет философии науки}

\insertlectureslide{7}{01}

\insertlectureslide{8}{01}

\subsection{Философские основания науки}

\insertlectureslide{9}{01}

\insertlectureslide{10}{01}

\subsection{Уровни реальности в философской онтологии}

\insertlectureslide{11}{01}

\subsection{Онтологический статус исследуемого объекта}

\insertlectureslide{12}{01}

\subsection{Функциональные системы}

\insertlectureslide{13}{01}

\insertlectureslide{14}{01}

\subsection{Соотношение позитивного научного и философского знания}

\insertlectureslide{15}{01}

\subsection{Возникновение науки и основные стадии её исторической эволюции}

\insertlectureslide{16}{01}

\subsection{Проблема начала науки}

\insertlectureslide{17}{01}

\insertlectureslide{18}{01}

\subsection{Проблема периодизации предыстории современной науки}

\insertlectureslide{19}{01}

\subsection{Эволюция современной науки}

\insertlectureslide{20}{01}

\insertlectureslide{21}{01}

\subsection{Исторический тип научной рациональности}

\insertlectureslide{22}{01}

\insertlectureslide{23}{01}

\subsection{Глобальные научные революции}

\insertlectureslide{24}{01}

\insertlectureslide{25}{01}

\insertlectureslide{26}{01}



\end{document}
