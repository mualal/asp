\documentclass[main.tex]{subfiles}

\begin{document}

\section{Лекция 04.03.2024 (Шипунова О.Д.)}

\subsection{Философия естествознания}

\insertlectureslide{1}{09}

\subsection{Сферы научного познания}

\insertlectureslide{2}{09}

\subsection{Предмет философии естествознания}

\insertlectureslide{3}{09}

\insertlectureslide{4}{09}

\subsection{Базовые модели естественнонаучного объяснения}

\insertlectureslide{5}{09}

\insertlectureslide{6}{09}

\subsection{Каузальные модели в постнеклассической науке}

\insertlectureslide{7}{09}

\insertlectureslide{8}{09}

\subsection{Характеристика исторических типов научной рациональности}

\insertlectureslide{9}{09}

\subsection{Интеллектуальные практики объяснения и обоснования}

\insertlectureslide{10}{09}

\subsection{Форма научного обоснования}

\insertlectureslide{11}{09}

\subsection{Логика объяснения}

\insertlectureslide{12}{09}

\insertlectureslide{13}{09}

\subsection{Логика понимания}

\insertlectureslide{14}{09}

\subsection{Методология развития естественнонаучного знания}

\insertlectureslide{15}{09}

\subsection{Представление о продуктивном мышлении}

\insertlectureslide{16}{09}

\subsection{Методы активизации поиска решения}

\insertlectureslide{17}{09}

\insertlectureslide{18}{09}

\insertlectureslide{19}{09}

\subsection{Матрица идей Г.Буша}

\insertlectureslide{20}{09}

\subsection{Интенсивные эвристические методы ТРИЗ}

\insertlectureslide{21}{09}

\subsection{Психологические методы активизации творчества}

\insertlectureslide{22}{09}

\insertlectureslide{23}{09}

\insertlectureslide{24}{09}

\subsection{Проблема самоорганизации в продуктивной деятельности}

\insertlectureslide{25}{09}


\end{document}
