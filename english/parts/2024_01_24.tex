\documentclass[main.tex]{subfiles}

\begin{document}

\linksection{Лекция 24.01.2024 (Смольская Н.Б.)}

So let us begin.
Well, I hope that you're ready with those points that I asked you to think about.
I mean the discussion topic or the broadening of the discussion topic that we covered last class.
I mean "<Ethical problems of modern science">.
This is the case that I would like us to continue discussing.
And so then it would become a solid discussion topic and a solid monolog presentation of yours.

But I think we will begin with grammar.
With those sentences that are uploaded into the course.
Are you ready with them?
Yes or no?
Not all.
Так. Тогда подождите, а я сейчас принесу распечатки.
Те, кто сделали, молодцы.

Раздаю распечатки, которые были вашим домашним заданием.
Так, ну что, тогда придется, учитывая, сколько не делали...
Для того, чтобы сократить время подготовки, сейчас я подумаю, что бы дать тем, кто готов...
Тогда сконцентрируйтесь, пожалуйста, на определенных предложениях, которые, собственно говоря, я и хотела с них начать обсуждать, но раз немножко будет время на подготовку...
Значит так.
Выделяйте те, кто не делал, предложения a, c, f, g, i, k, m, ну и последнее самое t.
Делайте эти 8 пока, для начала.
Это не значит, что мы не обсудим (но возможно сделаем это в следующий раз) оставшиеся 12.
Просто 12 оставшихся, они посложнее.
Так, so, those who are ready.
Чтобы вам не терять время, или вы что-то делаете?
Те, кто готов?
Нет, я записываю то, что вы сказали.
А, ну то есть те предложения, которые...
Что я сказала?
Вы сказали предложения, на которые нужно особое внимание обратить.
Да, это для тех, кто не готов, чтобы мы не тратили время на 20 предложений.

Тогда те, кто готовы поднимите руки, чтобы я знала сколько вас.
Кто дома делал?
Не скромничайте.
Раз, два, три, четыре, пять, шесть, семь.
Вас не так много, но ничего.
За то я буду знать.
Тогда с вас, пожалуйста, по три вопроса по теме.
Четвертая тема, которую мы возьмем, это "<Ведущие научные школы в моей области знаний">.
Под научными школами понимается то, где концентрируются те центры бифуркации, которые продуцируют основные идеи в вашей области исследований.
Вот.
Ну хорошо давайте не в вашей области, а возьмём вашу научную специальность, то есть у нас математика и механика.
И что ещё?
Физика тут есть?

Хорошо, как бы мне...
Я понимаю, что вы немножко разные.
Хорошо, тогда нет, там есть вторая часть: "<Университеты как научные центры">.
Давайте по два вопроса, которые вы будете адресовать своим коллегам.
Подумайте, какие бы вопросы вы задали, если бы вы были экзаменатором, задали бы своим коллегам, если бы у них была вот эта тема на монолог.
Университеты как научные центры.
Только такие вопросы какие-то посерьезнее.
То есть здесь без привязки к научной области, а именно, что университеты как места, где проводятся научные исследования, а не только образовательная деятельность.
Какие-нибудь такие немного tricky questions.
Okay.

Постарайтесь ничем не пользоваться, только своей головой.

\newpage
\sublinksection{Grammar exercise (tenses and aspects)}

--- Выполнение задания ---
\\

\textbfind{Выполненное задание}

\begin{enumerate}[nosep,leftmargin=8mm,label=\alph*.]
	\itemsep\eitsp
	\item[a.] The famous book by Frederick W. Taylor Scientific Management (had been written; has been written; \textbf{was written}; is written) \underline{in 1947}.
	\item[c.] Jack Richards left the company he (has worked; \textbf{had worked}; has been working; have worked) with for ten years in order to set up his own business.
	\item[f.] Hurry up. The train (leaves; left; will leave; \textbf{will be leaving}) \underline{in a minute}.
	\item[g.] (\textbf{Have you seen}; Did you see; Do you see; Will you see) any interesting films \underline{lately}?
	\item[i.] I \underline{usually} read the newspaper \underline{while} I (wait; \textbf{am waiting}; have waited; have been waiting) for the bus.
	\item[k.] The team (will carry out; will be carried out; \textbf{will have carried out}; will be carrying out) the experiment \underline{by September}.
	\item[m.] \underline{Since 1965} many measurements of the microwave background (\textbf{have been made}; were made; had been made; are being made).
	\item[t.] He has been to London (formerly; \underline{\textbf{lately}}; not long ago; recent).
\end{enumerate}
\ 

So, ready? Those who are working with the home task at the moment.
\\

Okay.
Let's begin discussing.
So these eight sentences are, well, they're not easy at all, but they're easy from the point of view of their understanding.
I have chosen them because they clearly present the instances of using particular, specific verb tense forms.
And according to these sentences we usually have something that helps us to understand what tense to use.
Because as for the other sentences (for the sentences left) sometimes you should just know the rule and I will give you the some more time because that's not so easy just to choose the correct form on the spot.
So I hope that for the next class you will have more time to look through the 12 sentences left and we will also discuss those rules that help us to understand the grammar phenomenon.

So let's begin with these eight sentences.
So what about the sentence (I wanted to say number) but sentence under the item (a)?
So what is the correct answer?
What is the letter of the correct answer that we should choose here?
So that is (C) definitely.
So what form is it?
What tense form is it?
That's Past Simple.
So why have we chosen Past Simple for this sentence?
So there is the marker that helps us to understand it.
So we have not the date but the year in this case because when we name the date, it means that we should name the month and the day.
So the date contains what?
The number and the month.
So in this case, this solid item is called date.
So in this case, we have a year, which is also enough to nominate the Past Simple tense form here.
Absolutely correct.
And it is one of the easiest examples so that is why we begin with it.

So let's proceed and sentence (c) is a little bit more difficult, but still it is easy.
Especially taking into account that maybe if there are no answers, you could have some difficulties, but you are restricted by the answer choices.
То есть, если бы не было этих вариантов ответов, то может быть у кого-то бы закрались какие-то сомнения и вы бы задумались, но здесь мы к счастью на тестовых заданиях, когда выберем правильный ответ из данных, в общем-то ну тут сначала 50 на 50 можно звонок другу, но чаще всего все-таки после 50 на 50 уже можно догадаться, что же будет однозначно правильно.
Итак, в этом случае.
So in this case.
Who would like to be the next?
So here we choose...
C?
No.
So for sentence (c) the correct answer is (B).
The correct answer is (B).
So who would like to explain why we should choose (B) here? You can use the Russian language.
So it's just for me because I usually enjoy talking to you in English.
That's a nice chance for me to practise my English, because that's not so often.
So that's why I speak English.
But we can switch into Russian.
So who would like to explain?
Кто даст нам объяснение?
I will try.
Maybe.
Because it's highlighted that he was working in this company and then he left it.
So, here we have left the company, but he was working there before.
Right, so he was working for the company for some period of time, then the next action happened, so he was working and working and working, then he left.
In this case we have what?
The sequence of facts or the sequence of actions.
Здесь у нас есть некая последовательность.
Когда что-то было до, а что-то после.
И оба этих действия произошли в прошлом.
И в этом случае мы с вами знаем, что английский язык в таких фактуальных случаях для того, чтобы показать предшествование, использует Past Perfect.

В случае, если мы с вами это обсуждали, если мы хотим подчеркнуть еще, что действие длилось, как в нашем своем случае, да?
А, нет, здесь у нас нет Past Perfect Continuous, здесь нет Continuous, поэтому в данном случае.
Это наше желание: мы могли бы с вами сказать had been working, но поскольку нам здесь не дано выбора иного в Past, кроме как Past Perfect...
Потому что все остальные, в чем положительный трюк такого рода тестов, тестовых заданий?
И мы их успешно используем и во время экзаменационных тестов для того, чтобы максимально позволить не ошибиться в выборе правильного ответа.
Потому что мы могли бы здесь ничего не давать, а поставить глагол to work в скобочках, и ваша задача была бы правильно раскрыть скобки.
Было бы это больше в сторону правильных ответов или наоборот я не знаю, но тем не менее мы с вами понимаем именно анализируя эту ситуацию.
Если вы помните, когда мы с вами обсуждали вот, что у Future мы можем подчеркивать, это же как бы желание говорящего, подчеркиваем мы длительность действия, да, вот того действия, которое производилось, или нам это не принципиально, мы просто подчеркиваем, что это было действие как факт.
В данном случае это выбор говорящего.
Конечно, если бы вам здесь дали две формы на выбор, то вы имели бы право в тесте, хотя просто в тестах таких методологически не должно такого быть, выбрать два этих ответа.
Один и второй, had worked и had been working, они оба здесь будут правильные, но ни в коем случае не вот A, C и D, которые у нас здесь даны, потому что они, все оставшиеся три, стоят в Present.
Разные формы там, да, Present Perfect, Present Perfect Continuous.
А Present Perfect в данном случае (в этом предложении) использоваться не может по требованию грамматического элемента.

Так, немножко живее.
Сейчас у нас атмосферное давление скачет.
И зима закончилась на два дня, наверное, потом опять придет.
Давайте будем ещё живее, все участвуют, и у нас всё будет хорошо.

Итак, предложение (f).
Who would like to work with this sentence?
You are welcome!
Hurry up. The train will leave in a minute.
Так, есть какие-то другие варианты?
Leaves?
Нет.
Действие в будущем?
Да, действие в будущем, но как раз вот это тот вариант.
Я решила просто его оставить.
В этом варианте можно использовать или will leave, или will be leaving.
То есть в любом случае мы имеем дело с тем действием, которое происходит в будущем.
Hurry up! Left нам уже тут не поможет, это уже не hurry up будет призыв, да?
Stop and rest, да?
Да.
Calm down в этом случае будет.
The train left a minute ago.
То есть здесь мы с вами сразу же откидываем left.
B -- это неправильный ответ однозначно.
А дальше вы должны размышлять, потому что три оставшихся...
Эй, вы, как это, трое, подойдите оба ко мне, как-то так, ...
Итак, три оставшихся претендуют на то, чтобы быть выбранными сюда, но нужно уже смотреть на контекст и на маркеры, которые у нас есть.
Если мы используем leaves для обозначения?
Это был вопрос.
Когда?
Когда в русском языке мы используем Present Simple для обозначения действия в будущем?
Это речь идет о расписании, по-моему.
О расписании, абсолютно верно.
То есть, в данном случае in a minute нам не требуется.
Здесь, в этом предложении мы ...
Вот hurry up создаёт для нас иной контекст.
Мы можем говорить the train leaves at и указывая в 7:45 там предположим, то есть расписание будь оно в прошлом, настоящем или в будущем всегда требует указания конкретного времени.
Здесь у нас это время не указано, и поэтому вариант leaves в этом предложении у нас не имеет права на свое существование.
Хотя вы абсолютно правы, Present Simple может использоваться для обозначения действия в будущем, если это происходит по расписанию, но не в этом случае.
А вот теперь разберёмся, если мы выберем (c) will leave.
Что это за значение будущего времени?
Потому что они отличаются.
Will leave -- это Future Simple; will be leaving -- это Future Continuous.
В чем будет отличаться их значение?
Нужно или не нужно длительность процесса указывать?
Ну, абсолютно верно, да, конечно, но здесь не только еще длительность.
Will leave -- это просто описание, констатация факта, который произойдет в будущем.
Нейтральная, без каких-либо ...
Сама форма глагола не содержит никакого дополнительного значения -- просто вот указание на то, что действие в будущем произойдет.
А вот will be leaving ...

У Future Continuous не так много значений.
И will be leaving это, как сказать, некое недлительное действие.
Мы понимаем, что он отправится, но он же не будет отправляться в течение часа.
Это будет некое краткосрочное недлительное действие, в совершении которого мы уверены.
То есть мы абсолютно уверены, что через минуту он трогается.
И на короткий момент времени это действие будет происходить, а дальше он уже будет ехать.
Не трогаться, а уже будет ехать.
Вот это значение will be leaving.
Will live -- это просто будущее.
Понятно, что если вы скажете will live ...
I will read this book, I will meet my friend, это всегда будет правильно, это констатация факта в будущем.
А вот если вы используете, а у нас в этом случае есть in a minute -- вы уверены, что он тронется через минуту.
У вас есть уверенность, и само действие предполагает собой не некую длительность в течение большого промежутка времени, а только небольшая часть временного следующего континуума, то можно смело в этом случае вот, когда вы хотите сказать, что я буду бежать I will be running to catch my dog and then I will definitely fall down.
Предположим.
То есть ненадолго вас хватит, но вы уверены, что побежите свою собачку догонять.
Вот таково одно из значений Future Continuous.
Здесь мы с вами с ним встретились.

Второе значение, это простое, когда мы хотим просто подчеркнуть, что в будущем будет происходить длительное действие в течение какого-то времени.
Когда вы скажете during three hours we'll be watching an interesting comedy, и вы указываете в течение какого времени.
Future Continuous.
Но это просто.
А вот в рассматриваемом случае (предложение (f)) это такой вот специфический, такой tricky instance.
Так, двигаемся дальше.

Предложение (g).
Давайте пойдем не от правильного ответа, поскольку тут вообще почти слов нет в данном предложении, ну и в ответе тоже нет слов почти, но тем не менее, сразу четко мы с вами определяем, где у нас индикатор, который сразу же вынуждает нас ...
Lately.
Lately -- это тот индикатор, не будь которого мы могли бы засомневаться, но раз у нас есть lately.
Lately -- это один из указателей совершения действия, который требует использования Present Perfect.
В любом случае Perfect.
Не Simple, не Past Simple, хотя вроде как перевод на русский язык, а мы не всегда должны...
Перевод на русский язык в большинстве случаев нам помогает понять смысл, но не грамматически корректно оформить.
Поэтому в данном случае мы отсекаем значение, да, мы поняли хорошо, а вот lately нас сразу же restricts, и мы знаем, что нужно использовать Perfect, поскольку здесь нет никакого предшествования, то однозначно тогда Present Perfect.
Lately!
Какое еще слово знаете?
Recently!
Да, то есть мы понимаем, что если мы видим эти слова в переводе или в вашей голове они как переводятся?
Недавно.
В недалёком прошлом.
Они требуют использования Present Perfect.
Ну и теперь мы смотрим и отсекаем сразу же все то, что здесь нам не нужно и правильный ответ здесь (A) Have you seen any interesting films lately?
Если мы посмотрим на этот очень простой вопрос, то есть проще некуда вопрос составить с точки зрения грамматики, я не про ответ говорю.

Что здесь ещё, какой есть важный грамматический момент, не имеющий отношения к глаголу, но тем не менее имеющий отношение к этому типу предложения, к вопросу, который вы тоже должны знать?
Как изучающие грамматику, как вспоминающие грамматику английского языка.
Any?
Да, any.
Что такое any?
Какой-либо.
Перевод я уверена, что вы знаете.
Как часть речи?
Итак, это местоимение.
Какие еще коллеги у него в ряду выстраиваются?
Чаще всего это some, с которым мы можем его перепутать.
Ну, и еще no.
Some, any, no в зависимости от типа предложения.
Ну и теперь объясните мне, вы, наверное, понимаете, к чему я веду.
Почему any, а почему не some?
Вопросительное предложение.
Потому что в вопросительных предложениях и в отрицательных, когда отрицание стоит при глаголе, то есть когда мы говорим, как сказать, anybody didn't know the answer.
Хотя это не совсем красиво звучит, но тем не менее.
В отрицательных и вопросительных предложениях никогда не используется местоимение или производные от него с some.
Ни местоимение some, ни производные от него в отрицательных и вопросительных предложениях не используются.
Вместо some, если some у нас использовалось в утвердительном предложении, а вам нужно задать какой-то вопрос, где этот some будет дальше фигурировать на уровне смысла, то вы здесь чисто формально в английском языке производите замену на any.
Да, формальные замены.

Теперь по поводу no.
То же самое слово nobody.
Что мы с вами ...
Я не помню, говорили ли мы, кажется, мы как-то с вами все-таки вспоминали об этих особенностях английского языка.
Итак, no, nobody, местоимение.
В каком случае предложение в английском языке, я, наверное, так пойду издалека, будет считаться отрицательным?
Мы, по-моему, с вами об этом говорили.
Какое предложение в английском языке считается отрицательным?
Есть отрицание.
Логично.
Тогда мы пойдем еще выше, еще более аналитически подойдем к этой проблеме.
И в русском языке я могу сказать, никто никогда мне об этом не рассказывал.
Никто никогда не рассказывал -- 3 отрицания.
В русском языке это отрицательное предложение.
Я не слышала, как кто-то зашел в аудиторию -- одно отрицание, но всё равно предложение отрицательное.
И русский язык -- это язык полинегативный, когда мы можем выстроить...
Ну, вот сколько позволит разум поставить нам отрицаний в предложении, и столько мы этих отрицаний поставим.
А английский язык -- это язык мононегативный.
Мы в предложении имеем право поставить только одно отрицание, но одно отрицание поставить-то мы должны, то есть я имею в виду не больше одного, но какое предложение будет отрицательным?
Отрицание в предложении.
Чтобы предложение стало отрицательным, нам нужно отрицать саму ситуацию, а ситуация это либо деятель, либо действие.
Все остальное, если эти две основных составляющих ситуации положительны, то тогда предложение в английском языке будет считаться утвердительным.
Да?
I saw nobody in the room.
Это какое предложение?
Отрицательное?
Нет.
Правильный ответ -- утвердительное.
Я же действие произвела.
Ну, просто там никого не было.
А вот nobody was present at the class.
Вот это уже отрицательное, потому что мы отрицаем деятеля.
I didn't participate in the discussion last seminar.
Я отрицаю действие.
Ситуации нет, потому что этого действия не было.
То есть с этой точки зрения, это уже более глубинные такие слои организации наших двух языков, нашего родного языка и иностранного, который мы с вами изучаем.
Это уже, так сказать, более другие подходы.
Это теоретическая грамматика языка.
То есть это не практика, которую мы изучаем.
Но тем не менее, просто для того, чтобы вы даже задумались, насколько философия языка отдельно взятого влияет на той презентации материала, которая вот формально используется, эти формальные, а даже неформальные конструкции, а разговорные, фактические.

А есть же тоже в английском двойное отрицание?
Пример: I'll never do you no harm.
Это вы откуда это взяли?
Пол Маккартни, Битлз.
Ну мы цитат сейчас можем...
По сути дела, отрицание на отрицание даёт утверждение.
Здесь как бы усиление отрицания в данном случае.
Ну, в данном случае это будет утвердительное предложение.
Когда отрицание на отрицание -- это не отрицательное предложение.
Хотя это отклонение от правил, то есть скажем так...
Ну может быть это можно назвать термином окказионализм.
Я по-моему вам говорила.
Есть неологизм -- то есть новые слова, или в данном случае какие-то новые конструкции, которые язык принимает, и они фиксируются в учебниках, словарях и так далее, тогда это слово или феномен становится неологизмом -- новым для языка, но тем не менее принятым нормой.
А если это, выражаясь, так сказать, просто изобретение отдельно взятого индивидуума для того, чтобы рифму выдержать, размер стиха выдержать, чтобы он ложился, тогда начинаются вот эти отклонения от правил.
И хотя здесь, вероятно, какой-то подтекст, то, что вы сказали, минус на минус даёт плюс, вполне дополняю, что я тебе сейчас всё скажу.
I'll never do you no harm.
Ну, я думаю, что здесь все-таки отрицание, и тут надо весь контекст, я честно говоря.
Это какая песня? Послушаю, почитаю.
Oh! Darling.
Надо будет послушать, я уже так давно не слушала, а жаль.
Обращу внимание.
Вот.
То есть, это может быть оказионализм, который просто вот в отдельном, конкретно взятом случае появился, но это не значит, что он в норме языка.
В данном случае он противостоит норме языка, но тем не менее вот какие-то причины для образования этой отходящий от правила конструкции они были.
Но носители языка понимают же?
Для этого нужно слушать весь контекст и какой здесь смысл, какой смысл будет конкретно в этой конструкции.
Вот почему она была использована.
Так any разобрались и вспомнили вот это some, any, no.

Предложение (i).
Давайте тоже пойдем от анализа предложения в целом.
И здесь тоже сразу же есть маркер.
В данном случае это conjunction.
Conjunction это что?
Conjunction.
Это какая часть речи, подсказываю?
While.
Это хорошо, что вы уже прямо само слово назвали, но я просто...
Да, это союз.
Да, да, да.
Вот оно, пожалуйста, оно и есть.
Поэтому здесь (в этом предложении) единственный союз, потому что если мы с вами проанализируем целиком предложение, то мы видим здесь, что есть две пары главных членов предложения.
I read и дальше I и пробел.
Да?
То есть мы с вами считаем количество предложений в сложном предложении по количеству пар подлежащее-сказуемое.
Может быть несколько сказуемых, но одно подлежащее, или несколько подлежащих и одно сказуемое, но в данном случае мы группируем их, абстрагируя как элемент и элемент.
В нашем случае два, две пары, значит союз точно будет один.
Ну и тут вы уже его правильно назвали -- это союз while.
И я тоже думаю, что со школы (или бакалавриата) вы помните, что после союза while используются, ну преимущественно, если только автор не забыл, не знает или не захотел.
Преимущественно какие времена или группы времен?
Continuous.
While (пока) подчеркивает, что дальше мы хотим сказать, что что-то длилось.
Не when, по сути дела, when и while -- это два близкостоящих.
То есть после них однозначно идёт описание какого-то действия, приведения какого-то действия, но when после себя обычно использует точечные события, и это будут события либо Simple, либо Perfect, а while будет после себя использовать Continuous.
Это может быть просто Continuous или Continuous Perfect, но тем не менее Continuous, то есть форма -ing у нас однозначно в ответе быть должна.
Ну и теперь соответственно точно кто хочет стать миллионером -- мы сразу же от отбрасываем с вами... 
То есть каким образом, да?
Не мне вам объяснять.
Вы здесь сидите все те, кто точно аналитикой пользуются всё время.
Мы сразу же отбрасываем с вами какие ответы?
(A) и (C) мы сразу отбрасываем, потому что они не содержат какой формы обязательной для continuous?
Как называется -ing, который используется при образовании Continuous?
Что это?
Что за форма?

Participle I.
Просто, чтобы вы вспоминали, что есть Participle I, который от глагола образуется и используется в сказуемом, а есть -ing форма, которая что?
Называется как?
Герундий.
Это, по сути дела, отглагольное существительное, то есть существительное, и тогда оно в форме сказуемого никогда не будет использоваться.
Оно будет стоять либо на месте подлежащего reading is important или в форме дополнения I like reading, как I like coffee, объект.
Точно так же I like reading.
Только в этих двух местах используется -ing.
Но при этом I am reading this interesting book now.
Вроде как они похожи, но по месту их расположения в предложении мы понимаем, что это разные формы.
Итак, значит, у нас остались (B) и (D).
Что нас теперь ещё в предложении заставляет выбрать единственно правильный ответ?
А здесь он единственно правильный.
Usually, конечно.
Вот после того, как мы с вами while, помогло нам выбрать Continuous, дальше мы уже более глобально рассматриваем предложение, и знаем, что usually -- это наречие, которое нам в целом про ситуацию говорит, что ситуация у нас регулярно повторяется, значит, соответственно, если она регулярно повторяется, то это просто настоящее, без всякого Present Perfect, потому что Present Perfect это что по сути своей, это что за форма?
Это форма, которая указывает, что действие либо уже произошло в прошедшем, либо началось в прошедшем и длится до настоящего времени.
А здесь мы просто должны показать регулярность.
И поэтому мы выбираем себе единственную возможную форму (B) Present Continuous.
Так, ну что, больше мне не за что зацепиться в этом предложении.
Двигаемся дальше.

Предложение (k).
Итак, здесь тоже, мне тоже даже хочется, чтобы, ну, дальше пойти этим путем, если есть такая возможность.
Где в этом предложении, вот вы дошли до этого предложения, смотрите, ответов опять четыре, вам нужно какие-то выбрать.
Где подсказки?
И здесь у нас подсказка by.
Абсолютно верно.
Подсказка by.
Дальше мы смотрим с вами на возможные ответы.
Все у нас will, все содержат will, поэтому здесь мы ничего не фильтруем, но если у нас есть by, то в этом случае мы должны выбрать только единственно возможную группу времен и это Perfect.
By требует использования Perfect.
Это либо Past perfect мог бы быть, либо Future Perfect.
Либо мы к прошлому говорим, что до какого-то момента в прошлом действие было совершено, либо в будущем оно будет совершено до какого-то момента.
By -- это предел.
Это не at nine o'clock, не in May, когда мы этот, вот этот момент включаются в период.
А by -- это стенка.
Май, а вот до мая должно быть закончено, а дальше начинается май.
И поскольку у нас всё с will, а мы знаем, что перфектная группа времён должна быть, то мы с вами ищем Future Perfect.
Ищем, ищем и находим (C).
Я всё время когда вот ищем, вспоминаю, когда уральские племени только начинались, там я шла, шла, нашла, мы нашли с вами Future Perfect.
Единственное.
Все остальные нам не подходят, потому что либо Future Simple, Future Continuous, Future Passive, по-моему, да?
У нас единственный правильный вариант -- это вариант (C).
The team will have carried out the experiment by September.

Здесь есть тоже интересный момент, на который можно обратить внимание.
To carry out.
Как переводится?
Проводить.
В данном случае эксперимент.
И поскольку мы с вами уже говорили про то, что наш всё-таки интерес за небольшое количество встреч, это все-таки Academic English.
То есть это ваше частично письменное общение, устное, но все равно в рамках наших научных интересов и той академической коммуникации, которая предполагается.
В прошлый раз мы с вами уже вспоминали специфику академического английского, да, там no contracted forms, что у нас там еще было, что мы с вами вспомнили, что типа будем коллекционировать.
Ну, тем не менее, вот to carry out, что это за глагол, скажите мне, пожалуйста.
Это phrasal verb, потому что глагол to carry сам по себе без out, это что?
Это нести.
А to carry out, как вы правильно перевели, это проводить, в данном случае, эксперимент, что там еще можно, что-то в общем, какое-то мероприятие организовывать.
Фразовый глагол.
Мы с вами говорили, что это специфика английского языка.
Фразеологизмы есть и в русском, и в английском.
А фразеологизм -- это сочетание минимум двух знаменательных частей речи, а вот английский язык у него есть еще минимальный уровень, когда мы наблюдаем с вами сочетание знаменательной части речи, то есть глагола, который значение имеет, и служебной части речи, потому что out -- это preposition, после или предлог, но тем не менее.
В русском языке у нас этого феномена нет.
Но мы наблюдаем ту же самую трансформацию, как и при фразеологизмах, когда сочетание двух слов имеет абсолютно, ну, не абсолютно, но тем не менее, в общем имеет значение, не имеющие отношения к значению каждого из слов в него входящих.
То же самое и здесь.
И to carry out -- это фразовый глагол.
Академический английский не допускает использования бытовых фразовых глаголов.
Как, например, что там, я не вспомню, простых.
Ну, то есть, не то, что не допускает, он не...
Господи, как-то по-русски-то.
Он не приветствует использование фразовых глаголов в том случае, если у фразового глагола есть эквивалент (синоним) полнозначного, простого, правильного глагола.
Правильного с точки зрения не правильный/неправильный, а нормального, назовём его так, глагола.
Синоним, пожалуйста, я к чему веду-то?
К самому простому.

Дайте мне синоним фразового глагола to carry out.
To perform.
Можно.
To perform.
Хорошо, но чаще всего используется другой глагол.
Это я просто к чему?
Бывает, что вы просто будете устно рассказывать где-то своим коллегам на английском языке, как вы проводили эксперимент, и вам нужно несколько раз вот это вот сказать.
Он был организован тогда-то, мы его проводили потому-то, чтобы ваша речь была более вариативной, в этом случае единственный выход -- это использовать (а) синонимы, (б) описательные фразы.
Ну, легче все-таки сначала себе синонимы назвать.
Итак.
Set?
Нет, to set это некое точечное.
Вот установить, поставить цель и так далее.
Proceed?
Тоже можно, но реже, что?
Execute?
Реже используется...
To apply?
Нет, to apply к эксперименту нет.
Do?
Это просто, да.
Чаще всего в академическом английском к фразовым глаголам английского языка синонимами являются, по сути дела, заимствованные из латинского языка глаголы.
Хотя to perform and to execute -- это тоже всё правильно, но я просто...
Вот чаще всего используется глагол to conduct.
To conduct an experiment.
Это прямо устойчивая фраза.
Эти глаголы тоже, то есть можете себе, так сказать, записать, потому что во втором семестре они уже нам будут пригождаться.
Вот, но тем не менее.
Ну, просто мы здесь с вами столкнулись с этим глаголом, это для меня такая лишняя возможность его вспомнить.
Так ну совсем чуть-чуть осталось, и мы с вами дальше будем двигаться.

Предложение (m).
И опять же прежде чем мы перейдем к выбору правильного варианта тоже давайте сразу же укажем на маркер, который нас сразу же определяет, указывает вектор, что нам нужно выбрать.
Since?
Да, абсолютно верно.
Несмотря на то, что у нас здесь есть конкретный год 1965.
Тем не менее наличие since говорит нам о том, что мы будем использовать что?
Perfect, абсолютно верно, а если мы дальше проанализируем всё предложение, то ещё не просто Perfect, а Present Perfect, абсолютно верно, то есть если вы видите since, и само предложение не состоит из двух частей.
Since к Past Perfect не имеет отношения.
Если только это не будет, когда предложение действительно содержит предшествование.
Потому что если бы это было, там, предположим, before 1965, вот тогда был бы Past Perfect.
А если у нас есть в простом предложении since, то тогда мы используем только Present Perfect.
И в нашем случае единственный возможный вариант -- это (A).
То есть мы Past Perfect тоже отмели, хотя это тоже перфект.
И у нас остаётся правильный вариант.
То есть в таких тестовых заданиях нужно не лихорадочно искать вариант правильный, а сначала в самом предложении разобраться.

Тесты -- это условные предложения, они всегда составляются так, чтобы они содержали какой-то указатель.
И наша с вами задача, когда вы будете делать письменный перевод, и мы будем с вами практиковаться, именно в том, чтобы чётко понимать соответствие того, что мы имеем материала в исходном варианте (в данном случае в англоязычном).
Ну и соответственно здесь мы ищем конечно здесь немножко не в ту сторону, что я про перевод говорю,  но тем не менее соответствие мы должны искать.
Увидели since соответствие Present Perfect.
Увидели, как в первом, конкретную дату, тем более, относящуюся к прошлому, значит, мы точно знаем, что будет Past Simple.
При переводе при письменном, как я вам и говорила, ваша задача точно знать, что вот здесь вот такой грамматический феномен, значит, в русском языке он будет отражаться таким-то порядком русских слов.
Не точно так же, не дословный перевод, а определенные трансформации.
А такое очень часто существует.
То есть вы должны показывать соответствие.

Так, ну и последнее предложение (t).
В данном случае я решила просто лишний раз.
На самом деле ничего сложного нет.
Просто, чтобы лишний раз уже подтвердить для себя то, с чего мы там где-то начинали.
В (t) мы выбираем (B) lately.
И просто это лишний раз вам напомнить, что вот каково значение вот этого has been.
И не в этом предложении, а если я скажу He never was in London, то что значит это предложение?
На самом деле так не говорят не потому, что это некрасиво звучит, а потому, что у этого предложения есть конкретное значение, то есть оно скрыто под текстом.
He has never been in London.
Или.
He never was in London.
Я могу предположить, что человек умер, и он уже никогда не сможет побывать в Лондоне.
Абсолютно верно.
То есть если вы про себя говорите, да, вот люди, которые начинают изучать английский, и говорят я никогда не был.
Всё, was, never, всё построили, правильное предложение, а вот это значение предполагает именно то, вот этот скрытый смысл, что всё, человека больше нет, у него никогда больше такого шанса не будет.
I have never been in London, это значит, ну вот пока я еще не был, вот на настоящий момент пока ситуация предшествующая такова, но перспективы не отменяем.
Просто для того, чтобы вот у себя, чтобы вы в голове вот это вот значение держали.

Так, хорошо.
So now those who are ready for today with these sentences (so the hometask for the rest) so you will now take this exercise with you and we will again start with these sentences at our next class.

\newpage
\sublinksection{О следующем занятии и февральском зачёте}

Кстати, учитывая то, что вы помните, я в самом начале легко перенесла (по касательной) грипп.
У нас пропало с вами одно занятие.
Я это помню, не знаю как вы.
Я это держу в голове.
Поэтому у меня предложение, просто ваши коллеги, которые во второй группе, они это предложили, мы можем в следующую среду позаниматься два занятия.
У кого сколько хватит сил.
Честно скажу.
Если вам там нужно (семья, дети или что-то ещё), то вы можете уйти.
Начинаем в 18:00?
Да, мы начинаем в шесть, вот как у вас.
И туда две пары.
Нет, просто я понимаю, что это тяжело.
Перерыв сделаем?
Сделаем, конечно.

Да, и причем, мне бы хотелось попробовать, чтобы мы попрактиковали перевод и мы бы посмотрели, что получится, но не сложные не ваши специфические.
Да ну я просто ...
Есть маленькие кусочки (подборки), которые как раз содержат вот те самые моменты, которые вот посмотреть, знаете ли вы как это на русский язык переводится или же вы будете мне дословно переводить.
Вот такого плана.
Ну и чтобы раздать вам всё, что у меня накопилось и собрать то, что вы мне принесете еще, и дать вам установки ...
Я значит планировала ...
Можете просто, чтобы все остальные знали, что у нас зачёт будет 14 и 21.
У вас будет 14.
У второй группы 21.
Если вы вдруг по какой-то причине 14-го февраля не можете прийти, то это не страшно, потому что ведомости мы будем закрывать уже, так сказать, ближе к финалу.
Ну, мало ли, кто-то не придет 14-го (заболел, в командировках), то можно прийти 21.
По времени во второй половине дня это точно, не могу сказать, будет зависеть от аудитории, потому что уже начнётся учебный процесс, поэтому сложно.
В 18:00 конечно свободно, но может быть и раньше получится немножко.
Не раньше 16:00 и не позже 18:00.
А раз мы заговорили о зачете, можете тогда повторить, пожалуйста, что нужно на зачет?
Да, это я сейчас повторю.
Но только зачёт предполагает, что вы все придёте к началу зачета.
Потому что я вам говорила, что будет небольшой тест вот такого плана.
Ну, мне просто вот хочется, чтобы вы показали, и себе тоже показали, что вы то что-то знаете, что-то не знаете.
Максимально решается все в пользу студента, поэтому не пугайтесь, если вы мне там ничего не сделаете.
Я просто устно могу там какие-то вопросы задать для того, чтобы поднять самооценку.

Итак, вы все приходите вместе, садитесь, я вам раздаю эти тесты.
Ну, может быть, не 20 предложений, но хотя бы там 10-12.
Чтобы вы быстро выбрали мне правильные ответы.
Дальше я смотрю, сколько там ошибок получилось.
И в следующий раз я раздам, что-то я проверила, смотрю, где-то проверено, а где-то нет, но я в следующий раз всё раздам.
И вы принесёте ...
Если где-то у меня есть замечания по вашим темам, вы просто принесете мне все ваши темы, которые вы мне сдавали, и я вам дала проверенными (отдала с замечаниями или с плюсиками), чтобы я видела, сколько тем вы мне сдавали, сколько вы написали, и сколько их у вас на руках.
Просто я могу на зачёте, конечно же, раздать вам и прокомментировать на зачете, но вдруг вы что-то там измените.
Потому что есть, сразу скажу, моё основное замечание, уж сейчас мы будем дальше двигаться.
Замечание по вашим темам.

Это не у подавляющего большинства, но тем не менее присутствует то, что все равно есть темы, которые написаны как замечательные письменные эссе.
Прям читаешь и радуешься, но это не устное высказывание.
Потому что вы должны понимать, что это разные функциональные стили.
Есть разговорный стиль, а есть научный стиль (конкретно жанр эссе).
Вот встречаются такие замечательно написанные эссе.

Или же встречаются устные; у меня вроде как нет претензий, но при этом linking words это хорошо и хорошо когда вы их используете это типа however, therefore, там что-то еще, но вот когда на 12 предложениях эти therefore идут через предложение, то начинает раздражать.
Поэтому какую-то вариативность тоже показывать или же устная речь не предполагает ...
Если это просто монолог, а не ваш доклад по теме вашей вашего исследования, а просто рассказ о чем-то.
А в нашем случае это именно ваш рассказ, my scientific interest, это не report перед научной аудиторией.
Здесь просто, чтобы вы легко могли бы об этом рассказать.
Поэтому therefore, however, nevertheless, они немножко странновато в этом жанре звучат, поэтому можно там одно вставить, но сильно этим не жонглировать.
Вот есть вот такое второе формальное замечание.

Ну а третье это то, что иногда меня радует чётко 12 предложениями по 4 слова.
Я радуюсь, но пишу мало.
Нужно мне ещё.

Всё, так что вы приходите, я вам раздаю, вы пишете тест быстро, дальше, может быть, пока вы пишете, я пройду раздам ваши работы, или там, какие остались, и дальше мы с каждым быстро общаемся, я вам ставлю замечательную оценку, и всё.
У вас где-нибудь в зачётках написано, кто там, бюджет, контракт?
У студентов всегда пишут, чтобы преподаватель знал.
Где-то сейчас вот это было, во ВКонтакте-то, цитата преподавателя.
Где-то какой-то там на электромехе или где-то ставит четверку и смотрит зачетку, а там одни отлично.
И говорит: "<А мог бы стать отличником!">.
У нас, кстати, все на бюджете.
Да, на ваших направлениях (научных специальностях) бюджетных мест много было.

\newpage
\sublinksection{Universities as a Science Centres}

Okay.
Questions.
Questions for persons who are no so diligent as whose who are ready with all the twenty sentences.
So I would like now ask your questions.
So that will be your responsibility to address your questions to your colleagues.

So you choose, gentlemen, whom to ask.
So you are welcome!
Who would like to be the first?
Okay.
Choose the victim and ask the question.

What kind of laboratories does your institution have?
And what types of research you can perform there?

I work in the department of applied mechanics.
So we have some laboratories, but unfortunately, I have never fulfilled any tasks in these laboratories.
Oh, I lied.
There was one laboratory where I did a project.
It was related with frequencies.

Окей, кстати, хороший глагол, достойный того, чтобы его вспомнить.
Глагол to lie.
Во-первых, просто тоже, если мы вспоминаем с вами специфику английского языка.
Во-первых, lie переводится ...
Это может быть и глагол, и существительное.
Да, или лгать, или ложь.

Что нужно знать по поводу существительного lie?
Нельзя во множественном?
Нет, почему?
Можно.
А? Вот устойчивое выражение, да? Честно говоря или не честно говоря, да?
Как будет по-английски?
Честно говоря.
To tell the truth.
Да, а как будет покривить душой, или он покривил душой?
Ну, неправду сказал, соврал.
Да, to tell a lie или to tell lies.
Почему так?
To tell the truth, но to tell a lie или to tell lies.
Что правда она всегда одна.
Абсолютно верно!
А вот её антагонистов можно придумать огромное количество.
Это первое.

А дальше глагол.
Глагол to lie.
Это какой глагол?
Мы его правильно использовали.
С точки зрения уже грамматических характеристик.
Это правильный глагол.
И поэтому мы имеем с вами lie -- lied -- lied.
А Continuous будет lying.
Но в английском языке есть омоним.
Что такое омоним?
Какое слово называется омонимом?
Пишется и произносится одинаково, но обозначает разные вещи.
В русском языке коса-коса.
В общем, что-то еще.
В английском языке то же самое.
Есть глагол, который пишется точно так же, но обозначает он совсем другое.
Это что за глагол?
Который пишется точно так же, но обозначает совсем другое.
Лежать.
Глагол лежать.
Это какой глагол?
Неправильный.
Если это неправильный, то вы должны знать все три формы, то есть Past Simple и Past Participle от этого глагола.
Как они звучат?
To lie (лежать) -- lay -- lain.

В английском языке есть так называемый каузативный глагол, который показывает, что это действие было, вернее, действие было произведено над объектом, и он в результате вот приобрел некие характеристики.
Если to lie -- это лежать, активное действие, то каузативный глагол -- это глагол положить.
Камень лежит, потому что его положили -- над ним произвели действие, кладя на что-то.
И каузативные глаголы в этом случае, здесь связь очевидна, да, лежать или положить, то есть результат.
Они в английском языке (выстраивается прямо целая, так сказать, группа таких глаголов) произошли от прошедшего времени, от прошедшего времени неправильных глаголов.
И поэтому в английском языке вы сталкиваетесь вот с этим глаголом.
Положить -- это to lay.
И глагол to lay -- это какой глагол?
Неправильный.
Какие у него формы?
To lay -- laid -- laid.
To pay -- paid -- paid.
To say -- said -- said.
Это та группа неправильных глаголов, которые у нас тоже объединяются в одну группу неправильных глаголов.
Неправильные глаголы распадаются на шесть групп.
Их, конечно, большое множество, но, тем не менее, можно вывести некие определенные закономерности.
Ну это просто вот мы вспомнили, вы случайно использовали слово, зато есть возможность вспомнить и поговорить, и что-то ещё новое узнать.
А разве у этого глагола нет связи с горизонтальным положением или что-то такое?
Да, ну, как.
Положить.
Но это не всегда горизонтальное, это результат.
Результат.
Окей.
Так, окей, ну давайте продолжим.

So would you please to ask your question to any of your colleagues?
Okay.
Do you think that universities must be funded only by government or also by organisations interested in their research?
Зря вы этот тип вопроса используете.
Что это за тип вопроса у нас?
Do you think?
Yes or no question.
А я могу просто добавить and why.
And why?
What do you think on the problems of funding?
Если честно, я не понял вопроса, даже не расслышал.
Можно тогда повторить?
О, по-английски только.
Would you please repeat the question?
What do you think about the funding of universities?
Do you think that they must be funded only by government or also by organisations interested in their research?
А если по-русски?
Университеты должны спонсироваться только государством  или всё-таки в том числе организациями, которые в этих исследованиях заинтересованы?
In my opinion, there is no any bad issues in funding by organisations which are interested in scientific researches because government funding is cool, but it isn't enough.
That's all.
Okay, thank you.
At least some answers for a while.
Good.

Your question, and we continue the sequence of answers.
Could you name any last important conferences that were held at your university?
Maybe it's better to say something about conferences related to your major.
It's a difficult question for me, but I'll try to answer.
I know that in August the conference of Applied Mechanics was held in the Polytechnic University.
In August.
I know that was a very interesting conference.
Okay, thank you.
Кто следующий?

What university made the biggest impact in your scientific sphere?
And why?
Владислав, я задаю этот вопрос вам.
А вы в ответ, давайте.
Can you repeat your question?
Это был вопрос, да?
What university made the biggest impact in your scientific sphere?
And why?
I think the Polytechnic university.
This university give me much information, so I ...
This question is not about you.
Who would like to answer the question?
So, who would like to say a few words on this problem?
Who is ready?
Don't be afraid because that is the chance for you to speak.
What university?
Any world university is one of the most significant in terms of any impact.
So who would like to name?
Yes, please.
I think a lot of impact is applied or involved in my scientific area.
It is computational fluid dynamics.
It's not a particular university, but some great people or professors.
Individuals.
Not even the teams, but individuals.
Maybe the team, but I know only a particular person or some leaders of this team.
One of these persons abroad, it's Lilux, it's the stuff related to turbulence.
In Russia, I think it's Лойцянский, and some period of time is Эйлер.
Are these people related to any institutions or they just conduct their research individually without any link?
They of course are related to universities.
The last two are related to the Polytechnic University, of course.
Эйлер?
В Политехе?
Тогда же Политеха ещё не было?
Эйлер -- это же СПбГУ.
Yes, in Saint-Petersburg, but not in the Polytechnic University.

Are they supported by their institutions?
Or they just...
Эйлер же в академми наук?
No, I mean supported from the point of view of presenting, of giving the opportunity to use the equipment, the laboratories, to develop the schools, the followers.
Or they exist on their own and it doesn't matter where they are.
So they can move to any other institution and they will also conduct the same research.
Or they are somehow related and they are dependent, if I can say so.
These two scientists are not related, they depend on each other, I think.
But he was in charge of the department at this university, so I think, of course, he had some support from the ministry and the civil service.
So we can also maybe say that there is some team of all responsible.
Euler was the leader of Russian scientific department.
The department of Russian Academy of Sciences.
Okay, thank you.

Gentlemen, who else is ready with the questions?
Would you please ask any questions to your colleagues?
Слушаем вопрос.
My question is very easy.
What scientific events are held at your university as a research centre?
Let it be anyone of you.
Practise, practise speaking, but it's not so difficult.
I do not give any bad marks for any mistakes.
So, would you please repeat the answer?
Because it is very simple.
What scientific events are held at your university as a research center?
What scientific events that you would like to mention or that you know?
Conferences, I don't know, discussions, workshops?
Давайте, кто из вас?
Да, давайте.
Maybe very important scientific event is Advanced problems in Mechanics.
Because in this conference there are a lot of different topics which discussed during this conference.
And this conference bring people from different regions of Russia and they can discuss their field of science with colleagues in the Polytechnic University and receive some helpful pieces of advice to continue their research.
Good, thank you.

Advice.
What do you know about this word?
Совет.
No, I'm always interested in the words when I ask you from the point of view of grammar.
What is interesting about this word?
Существительное и глагол одинаковые.
Нет, не одинаковые.
To advise или advice.
To advise глагол оканчивается на se.
А совет как существительное оканчивается на ce.
Но с точки зрения грамматики то же самое, что и со словом research.
Advice -- неисчисляемое.
То есть to give advice.
А если вы хотите сказать, что он мне дал несколько важных советов, то тогда вы можете сказать ...
Many?
Many -- это если мы можем посчитать.
Much не звучит.
Some pieces of advice.
A piece of advice и в этом случае можно уже тогда и говорить.
Ну, two pieces of advice не совсем, а some pieces of advice вполне нормально.
Это очень важно.
Это то же самое, что с research.
Ни в коем случае не говорим researches.
Как исследования множественного числа нет.

Так, ну и давайте.
Фёдор, у вас есть вопросик?
Давайте, мы найдем жертву.
What scientific research are held at your university in your area of science?
Gentlemen, what scientific research?
Can you repeat, please?
What scientific research are held at your university in your area of science?
I think, I don't know, but I know that there are scientific laboratories in my high school.
These laboratories have a lot of good specialists, who can resolve a lot of problems in the sphere of Oil Engineering.
Okay.

\newpage
\sublinksection{Ethical problems of modern science. Three G-s problem}

Так ладно.
И буквально быстро, потому что время есть.
Three G-s problem.
Have you found any information about that?
Yes.
Okay.
So now let's begin with, I don't know, gift authors.
Gift authors.
What does it mean?
So what kind of, let's say, member of a team can be qualified as a gift author?
So who would like to say a few words about a gift author?
Well, if I understood correctly, it refers to a term when we speak about somebody who did not conduct the research, actually, but he was really gifted this right to be written among the authors because of his, for example, previous achievements or maybe some personal relationships with those who...
Or the present official status.
Yes, yes.
Like an academic advisor.
Because in some cases even his name can attract some attention to this article or piece of work.
Good.
Do you agree or would you like to add something?
Perhaps you can give your authorship to some junior specialists to promote this person in the academic environment.
Good.
It was said that this gift authorship could attend some attention to the research.
I think this is connected with guest authorship, not gift authorship.
It is different.
So that is why I wanted to ask.
So okay, so what was the information about gift authorship?
Oh, sorry, guest authorship.
So let's now try and separate these two, because I think for a ghost authorship, it is quite clear if you have found the information.
So it is definitely different.
So let's separate gift and guest.
So, and from your opinion, or from the information that you have found, so what is a gift author?
So then a gift...
I agree with the last speaker, but this point that gift authorship could attract some attention to the article, this point is connected with guest author.
Gift author is just a goodwill of someone to add the rating or citation index of certain person (to make a gift to this person).
The person who is added it is...
Я не знаю как это сказать, в общем, он в плюсе от этой ситуации, что его добавили as a gift author.
Guest authorship is another situation.
It is, according to information that I have found, the aim of this is to pay extra attention to the article by the name of the cool senior specialist.

Свадебный генерал это кто?
Это gift или guest?
Who is it?
Он привлекает внимание?
Guest?
Да, потому что он привлекает внимание.
Я понимаю, что это не к научной точки зрения, но иногда бывает и на конференции приглашается человек, который за счёт своего статуса придаёт значимость событию.
В данном случае с вами для публикации статьи.
И это guest.

А gift -- это когда вам помогли, ну чтобы вот где-то нужно там третью ВАК-овскую статью, вот и тогда вы будете gift.
Если вас приписывают.
То есть gift -- это когда приписывают для того, чтобы помочь человеку.
Guest -- это когда приписывают для того, чтобы оказать помощь статье и публикации.

Окей, what is a ghost author?
То есть gift и guest, вот они, вот их не путать, ну просто эти термины сейчас в научном сообществе начинают использоваться как раз вот в отношении плагиата, да, чтобы правильно...
И плагиат, и ...
Тут даже нет, не плагиат.
Тут как это назвать -- нет это вот этические нормы по сути дела, такие скрытые некие этические аспекты, которые начинают постепенно обсуждаться.

Okay, so the ghost author.
I think this is the worst case.
Ghost authorship is a practice when the professional writer or a well-known scientist contributes to the research, but the company that paid him not shown him in the authors list.
The company or the team of authors does not show this person in the list of the authors.
So in this case that's a ghost author.
His name is not present in the list of the authors but the result of the work and the maybe the text (some pieces of the text) of the article are written by another person who is not included into the list.

This case and the opposite case when some companies sponsored research but not shown.
And all risks of this research lay on scientists that was included in the list of the authors.
Okay.
What could be the Russian equivalent in this case?
Литературный негр?
Может быть, просто вы слышали, я, например, про русские термины пока не слышала, называем ли мы как-то это, да?
Хотя у меня в голове было slavery, конечно, ну, английское слово, slavery, да, в данном случае, somehow.

Хорошо.
Ладно.
So are there any additions to these problems?
Well, so actually these are three different problems that do exist in modern science.
And they are now being discussed in the scientific society.
So all these attitudes and relationships between or among the team of the authors.
So I think that somehow you can start, if you want, you can speak not only about some specific problems, ethical problems, when you speak during your examination on the ethical problems of modern science, you can also start with these problems that are not directly related to your field of research but still are related to the problem of conducting modern scientific research.

\newpage
\sublinksection{Your home task}

Okay, so, your home task.
Итак, в следующий раз мы закончим с вами обсуждение тех предложений, которые остались.
Там тоже очень много таких интересных моментов, на которые стоит обратить внимание.
Я так думаю, что те, кто их сделали уже, вы, наверное, просто, так сказать, использовали свои знания, да?
Ну, вот я знаю, думаю, что вот это правильный ответ.

И те, кто не делал, и те, кто сделал.
Вот у меня сейчас установка постараться в предложении, где есть вот эти индикаторы, постараться их отметить.
То есть не просто так давать мне правильный ответ, а объяснить почему он правильный.
Есть ли в предложении что-то, а для нас именно это важно, увидеть в предложении те направляющие, те рельсы, которые нас выведут к правильному ответу.
Потому что, как правило, даже в простых...
И если вы научитесь эти индикаторы вставлять, зная, что если я вот этот индикатор поставил, я вот это, должен вот такую форму глагола использовать.
Это очень важно.
И тогда вы как бы тоже перестанете избегать, вернее, перестанете допускать какие-то, пусть и небольшое количество, но тем не менее ошибок.
Будете знать, что если recently, то это Perfect.
Всё сказал recently, значит скажу в Perfect.

Дальше, поскольку у нас вот здесь вот вопросы были, я не просто так хотела, но на следующий раз, мне бы хотелось тогда в следующий раз, пожалуйста, принесите мне написанную тему "<Ethical problems of modern science">.
Я соберу и как раз ethical problems отдам вам уже на зачёте.
Ethical problems of modern science: вот three G-s, plagiarism, animal testing, то есть тема, которую мы в прошлый раз обсуждали, и чуть-чуть сегодня.
Вот это вот three G-s, это вот как бы вот эти вот три вида авторства, ghost, guest и gift.
Это тоже один из моментов, которые мы тоже обсуждаем.

Итак, значит, вот "<Ethical problems of modern science"> постарайтесь написать, потому что это сложно.

И было бы здорово, если бы каждый из вас, потому что всё-таки вы в данном случае имеете отношение к разным лабораториям, к разным институтам, к разным высшим школам.
Как раз был вопрос: какие научные исследования проводятся непосредственно в вашей высшей школе?
Но только я бы сказала исследования, которые вышли за рамки университета.

Вот Кама, например.
Или машина на солнечных батареях.

Вот расскажите, есть ли что-то, пусть даже это не широкому кругу людей, но тем не менее, когда исследования и результаты исследования, проведенные в ваших высших школах, вышли за границы нашего уважаемого университета?
Где наш имидж университета действительно поддерживается и вовне?

Выбрать исследования кафедры, на которой учимся, или кафедры, которую заканчивал?

Мне все равно, просто, чтобы вы о чём рассказали действительно о значимом.
Что значит всё-таки вышла за пределы?
Ну то что не только в узком кругу ограниченных людей, как говорил великий Егор Лигачёв, знают об этом.
Но, вот я говорю, либо это стало достоянием и используется в отрасли, в промышленной отрасли, да.
Либо это просто вот уже даже в быт куда-то вышло.
Либо это какие-то вот математические изыскания, которые привели...
Как Боровков да с его лабораторией там Covid рассчитал, когда там будут вспышки, когда там что, например.

Просто я вот говорю, когда каким-то образом то, что было исследовано и предложено в Политехе, получило какое-то развитие за рамками университета.
А про явления?
Пожалуйста.
Мне на самом деле важно, чтобы вы что-то сказали.
И вот тут хоть вас и много, но мне бы хотелось, чтобы давайте каждый, поскольку у нас будет две пары, будет возможность высказаться.
Буквально подготовьте по три минутки, не больше, совсем чуть-чуть, но тем не менее, вот озвучить, сказать, что это и куда это вылилось.

Но это не из списка ...
Это получается будет как четвёртый монолог или отдельно?
Да, это четвёртый монолог -- это "<Universities as a science centres"> ("<Университеты как научные центры">) и "<Ведущие научные школы в моей области знаний">.
Университет как научный центр по сути дела.
То есть вы об этом можете смело говорить.
Университет это не только образовательное учреждение, которое обучает бакалавров, магистров, специалистов, но это еще и научно-исследовательский центр, который ведёт исследования, которые выходят ещё и на определённый уровень (имеют определённый implementation; определённую реализацию).
То есть вот в этом смысл.
Университет -- это не только школа.

Тогда в следующий раз я вам ещё раз скажу про зачёт.

\end{document}
