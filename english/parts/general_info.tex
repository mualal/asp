\documentclass[main.tex]{subfiles}

\begin{document}

{\parindent0pt

\textbf{Курс лекций по иностранному языку. Общая информация.}

\vspace{3pt}

Курс читает Смольская Наталия Борисовна.

Материалы к лекциям (аудиозаписи, дополнительная литература) \href{https://drive.google.com/drive/folders/1iSx2EjJJ98rhPpOsM4xGcJA7-7qitQHM?usp=sharing}{доступны по ссылке}.

Все упражнения с занятий: \hyperref[sec:all-tasks-lk]{\color{blue}{УПРАЖНЕНИЯ}}.

Учебное пособие по английскому языку для аспирантов: \href{https://elib.spbstu.ru/dl/2/s19-119.pdf/info}{ИБК СПбПУ}.

Портал аспирантуры: \href{https://portasp.spbstu.ru/login/index.php}{PORTASP SPBSTU}.

\vspace{3pt}

Аспектно-временные (аналитические) формы глагола (16 форм в активном залоге и \newline 10 форм в пассивном залоге): \href{https://mualal.github.io/asp/english/EnglishTensesAspectsVoicesPoster.pdf}{Tenses and Aspects}.
Случаи употребления: \hyperref[subsec:tenses-usage]{\color{blue}{Tenses Usage}}.

\vspace{3pt}

Неличные формы глагола (инфинитив, причастие, герундий): \hyperref[subsec:impersonal-lk]{\color{blue}{IMPERSONAL FORMS}}.

\vspace{3pt}

Косвенная речь: \hyperref[subsec:reported-speech]{\color{blue}{REPORTED SPEECH}} -- не относится к сослагательному наклонению!

\vspace{3pt}

Типы условных преложений: \hyperref[subsec:conditionals-types]{\color{blue}{CONDITIONALS}}. Условные предложения (Conditional Sentences) являются частным случаем сослагательного наклонения (Subjunctive Mood).

\vspace{3pt}

Сослагательное наклонение в сложных предложениях: \hyperref[subsec:subjunctive-mood-lk]{\color{blue}{SUBJUNCTIVE MOOD}}.

Формы сослагательного наклонения: \hyperlink{subjunctive-mood-forms-table}{\color{blue}{Subjunctive Mood Forms}}.

\vspace{3pt}

How to Summarize a Text: \hyperref[subsec:tips-on-summarizing]{\color{blue}{TIPS ON SUMMARIZING}}.

List of Useful Phrases for Making Summaries: \hyperref[subsec:list-of-useful-phrases-for-making-summaries]{\color{blue}{LIST OF PHRASES}}.

}

\end{document}
