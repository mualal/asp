\documentclass[main.tex]{subfiles}

\begin{document}

\setcounter{secnumdepth}{0}

\setcounter{section}{103}
\linksectionold{Упражнения из учебного пособия}

Далее представлены выполненные упражнения из учебного пособия для аспирантов: \href{https://elib.spbstu.ru/dl/2/s19-119.pdf/info}{ИБК СПбПУ}.

\setcounter{subsection}{1}
\sublinksectionold{Система времён английского глагола (активный залог)}

\textbfind{Number 1 (Present Simple/Present Continuous OR Past Simple/Past Continuous).}

\begin{enumerate}[nosep,leftmargin=*]
	\itemsep\eitsp
	\item Inflation \textbf{erodes} our real wages.
	\item The full-year figures \textbf{showed} a downturn.
	\item About one American worker in four \textbf{depends} for his job on the activities of government directly or indirectly.
	\item Some enterprises are going bankrupt, and others \textbf{are becoming} profitable.
	\item The lossed in this year's conflict (the Chechnya campaign) \textbf{are happening} at a rate equal to those in Afghanistan.
	\item The year \textbf{is drawing} to a close.
	\item It is more than 25 years since she \textbf{left} England to begin a new life in the Middle East.
	\item The group \textbf{was having} talks about launching a subscription television channel.
	\item But what we \textbf{are seeing} in practice is the inadequate bankruptcy law.
	\item This group \textbf{is considering} whether or not to invest in a business.
	\item I thought I \textbf{was being} very businesslike when I asked to see the (accounting) books.
	\item He called me at home while I \textbf{was having} lunch.
	\item Italy's workers \textbf{are now seeing} cuts in their real wages.
	\item The fact the car companies \textbf{are having} to use other routes \textbf{is having} a major affect on us.
	\item The \textbf{are just being stupid} by not keeping the market informed.
	\item We \textbf{are still feeling} a little nervous after last year's stock market performance.
	\item The federal bureaucracy \textbf{shapes} and \textbf{implements} federal policy.
	\item The Defense Department's share of the total Federal budget usually \textbf{ranges} between 25 an 30 percent.
	\item The IMF \textbf{is having} its annual meeting this month.
	\item But anyone who expects immediate changed \textbf{is not being} objective.
\end{enumerate}
\ 

\textbfind{Number 2 (Present Simple/Present Continuous OR Past Simple/Past Continuous).}

\begin{enumerate}[nosep,leftmargin=*]
	\itemsep\eitsp
	\item They \textbf{are having} (испытывают трудности) a rough time adjusting to this life.
	\item The firm \textbf{is deciding} (решает) whether to lower its price to increase sales.
	\item Without doubt the car industry \textbf{is experiencing} (испытывает) a new phase of innovation.
	\item Many British an American companies \textbf{are beginning} (начинает) to realize that languages are becoming increasingly useful because they have a bigger range of clients.
	\item When F.D.Roosevelt took office in 1933 banks in many parts of the country \textbf{were failing} (становились банкротами).
	\item While other European countries \textbf{were progressing} (успешно развивались) Britain \textbf{was moving} backwards.
	\item Last Monday he \textbf{briefed} (кратко информировал) me on the full details of the agreement.
	\item Profitability \textbf{is} (маячит впереди) on the horizon.
	\item We \textbf{are} (находимся) in the forefront of areas that will provide growth for the whole continent (Latin America).
	\item An economic transformation \textbf{is gathering momentum} (набирает обороты; разворачивается) in Saudi Arabia.
\end{enumerate}
\ 

\textbfind{Number 3 (почти везде Present Perfect).}

\begin{enumerate}[nosep,leftmargin=*]
	\itemsep\eitsp
	\item He has been out of work \textbf{for a long time} (в течение длительного времени).
	\item \textbf{Over the last 10 years} (За последние 10 лет) the Japanese market place has grown from 8\% to 41\%.
	\item Tremendous changed have taken place in the world \textbf{during the past 30 years} (за прошедшие 30 лет).
	\item He has been in electronics \textbf{for most of his life} (в течение большей части своей жизни).
	\item \textbf{Once} (as soon as; как только) an employee has learned a second language it is easier for him to learn a third.
	\item \textbf{Historically} (Исторически/Традиционно), Europe has depended on US grain and minerals but Europe has now become a major grain exporter.
	\item \textbf{In the last five to ten years} (За последние пять-десять лет) more and more women have entered the ranks of management.
	\item He has visited this country \textbf{for the twentieth time} (в двадцатый раз) to bring 15 tonnes of humanitarian cargo.
	\item There is a saying that no man has tasted the full flavour of life \textbf{until} (до тех пор, пока) he has known poverty, love and war.
	\item But government efforts have \textbf{so far} (до сих пор) had little success.
	\item \textbf{Since the 30s} (С 30-х годов) the United States has had unbalanced federal budget most of the time.
	\item \textbf{In the past} (В прошлом/Раньше), women have tended to have fewer years of education.
	\item The company has built nearly 100m bicycles \textbf{since} (начиная с) 1949.
	\item \textbf{Until now} (Вплоть до настоящего времени), fees have been tied to inflation.
	\item The dangers of a worldwide recession have eased \textbf{in recent weeks} (в течение последних недель/за последние недели).
	\item This city has \textbf{historically} (исторически/традиционно) been the most attractive place for the majority of investment.
	\item \textbf{In the last two years} (В течение последних двух лет), the emphasis has not been on productivity but on competitiveness.
	\item The State Statistics Committee has \textbf{in the past} (в прошлом/раньше) taken into account the shadow economy, but the new method for calculating growth has multiplied its share in industrial production.
	\item The nations of Europe have fought many wars \textbf{over the centuries} (на протяжении веков).
	\item Learning has been a virtue \textbf{throughout the history} (на протяжении всей истории) of Japan.
\end{enumerate}
\ 

\textbfind{Number 4 (tenses: Present, Past, Future; aspect: Perfect).}

\begin{enumerate}[nosep,leftmargin=*]
	\itemsep\eitsp
	\item By the post-World War II era, the economy \textbf{had become} (стала) more sophisticated, and market-driven forces began to dominate.
	\item The authorities were unaware that he \textbf{had visited} (посетил) the country.
	\item 50 years ago there were 95 private banks in Switzerland. By the end of the 1980s, the total \textbf{had shrunk} (сократилось) to 22. Since then, numbers \textbf{have fallen} (уменьшилось) still further.
	\item The conference \textbf{had not come to an end} (ещё не закончилась) yet when the announcement was made.
	\item Today, a new era \textbf{has blossomed} (полностью вступила в свои права).
	\item I \textbf{had not done} (не сделал) half my work by the time they left.
	\item This is not the first time the two men \textbf{have met} (встретились).
	\item Yet after only two years in office things were not going the way the President \textbf{had planned} (планировал).
	\item But a lot \textbf{has happened} (произошло) over the past six years.
	\item By the time he gets to North Korea, he \textbf{will have visited} (уже посетит/побывает в) about a dozen countries in two months.
	\item Within ten years these changes \textbf{will either have arrived} (уже произойдут) or be coming very soon.
	\item He \textbf{will have had his dinner} (уже пообедает) by the time you come back.
	\item I'll be back at five. I hope you \textbf{will have had a good rest} (уже отдохнёшь) by that time.
	\item If a book costs 20 DM to produce, said Mr.X., its price \textbf{will have doubled} (уже удвоится) by the time it makes its way to the local bookstore.
	\item Soon, three months \textbf{will have passed} (уже будет/исполнится) since they approved Ms reappointment as prime minister by an overwhelming majority.
\end{enumerate}
\ 

\textbfind{Number 5 (tenses: Present, Past, Future; aspect: Perfect Continuous).}

\begin{enumerate}[nosep,leftmargin=*]
	\itemsep\eitsp
	\item Since the Great Depression of the thirties, the American economy \textbf{has been improving} (продолжает/всё время/развивается).
	\item Man and machine \textbf{have been competing} (всё время/постоянно/конкурируют) for the same jobs since the industrial revolution.
	\item Foreign investors \textbf{have been briskly pulling money out of} (продолжают стремительно свёртывать финансовую деятельности) Russia's bond market in recent weeks.
	\item About 800 families \textbf{had been living and waiting} (проживали и ждали) in emergency shelters since the mines closed a couple of years ago.
	\item By the 1st of September this year he \textbf{will has(?) been working} (уже будет работать) at this laboratory for 3 years.
	\item At that time farmers \textbf{had been salivating} (предвкушали) at the prospect of juicy profits.
	\item For several weeks the media \textbf{has been reporting on} (сообщают о) the decline of the dollar.
	\item I \textbf{had been persistently ringing you up} (непрерывно звонил) from 4 o'clock but your number was engaged.
	\item I \textbf{have been managing} (уже руковожу) groups of people since I was 25 years old.
	\item For nearly 10 years forecasters \textbf{have been saying} (говорят) that computer hardware is getting cheaper at the rate of 25\% to 30\% per year.
\end{enumerate}

\newpage
\setcounter{subsection}{2}
\sublinksectionold{Система времён английского глагола (пассивный залог)}

\textbfind{Number 1 (Passive Voice; Present, Past or Future SIMPLE).}

\begin{enumerate}[nosep,leftmargin=*]
	\itemsep\eitsp
	\item The reduction in the official rate of exchange of one currency for another \textbf{is called} (называется) devaluation.
	\item His discovery \textbf{is much spoken of} (много говорят о).
	\item Some people may talk a lot, but \textbf{are generally not listened to} ((их) чаще всего не слушают) by other members.
	\item These conclusions \textbf{were arrived at} (получили/к ним пришли) by by many experimenters.
	\item All obstacles \textbf{were made away with} (устранили/были устранены).
	\item A host of good stories \textbf{are told of} (рассказывают о) him.
	\item This theory \textbf{is not questioned} (не подвергается сомнению) by most people.
	\item This method \textbf{was done away with} (отказались от/отвергли) many years ago.
	\item These studies \textbf{were begun} (были начаты) in the late 1960s.
	\item Further growth \textbf{will be fuelled} (будет подстёгиваться/стимулироваться) by rising demand.
	\item No one \textbf{is denied} (никому не отказывают в праве) entry to higher education because of their financial circumstances.
	\item British retailers \textbf{are highly sought after} (очень востребованы) because of the training they get here.
	\item My future \textbf{was taken care of} (позаботились о).
	\item The trip \textbf{was paid for} (была оплачена) in part by his supporters in UK and in part by the Taiwanese side.
	\item They say they \textbf{were not warned about} (их не предупредили о) the changes.
	\item He \textbf{is refused} (ему отказывают) assistance.
	\item The land here \textbf{is worked} (обрабатывается) by machinery.
	\item The commands \textbf{were generally obeyed} (им обычно подчинялись) to avoid punishments.
	\item He \textbf{is often consulted} (у него часто консультируются) by courts for independent advice.
	\item He \textbf{was sworn in recently} (его привели к присяге) as the new mayor of Las Vegas.
\end{enumerate}
\ 

\textbfind{Number 2 (Passive Voice; Present or Past SIMPLE).}

\begin{enumerate}[nosep,leftmargin=*]
	\itemsep\eitsp
	\item Many European economies \textbf{were run} on the basis of central plans.\\
	Экономики многих стран Европы управлялись на основе централизованных планов.
	\item He \textbf{was very well thought of} by his supervisor.\\
	О нём был очень хорошего мнения/хорошо отзывался его начальник.
	\item Unrest \textbf{was reported} some other parts of the country.\\
	О волнениях поступили сообщения из некоторых других частей страны.
	\item This leader \textbf{is well liked} by subordinates.\\
	К этому руководителю хорошо относятся подчинённые.
	\item Measures to boost the economy with global cooperation \textbf{were given highest priority}.\\
	Самое большое значение придававлось мерам, стимулирующим развитие экономики с использованием международного сотрудничества.
	\item But even in a democracy there are a lot of things a person \textbf{is denied} the right to do.\\
	Но даже в демократическом обществе есть много такого, чего человек не имеет права делать.
	\item These programs \textbf{are viewed} in more than a hundred countries.\\
	Эти программы смотрят более чем в ста странах.
	\item If additional economic projects \textbf{are started} an additional demand \textbf{is created} for materials, plant and labour.\\
	Если приступают к реализации/берутся за выполнение дополнительных экономических проектов, то создаётся дополнительный спрос на материалы, оборудование и рабочую силу.
	\item These employees \textbf{were rated} more effective by their supervisors.\\
	Эти сотрудники оценивались/считались/рассматривались руководителями как более эффективные работники.
	\item They \textbf{were not warned} about the changes.\\
	Их не предупредили об изменениях.
	\item They \textbf{were not consulted}.\\
	С ними не посоветовались.
	\item They \textbf{were denied} pay increases.\\
	Им отказали в повышении зарплаты.
	\item Productivity in the building industry rises as \textbf{more use is made of} large prefabricated components.\\
	Производительность труда в строительной промышленности растёт, так как всё больше используются крупные сборные компоненты.
	\item The construction industry \textbf{is influenced} by these trends.\\
	На строительную отрасль оказывают влияния эти тенденции.
	\item Regulation forcing them to pay higher wages \textbf{is demanded}.\\
	Требуется принять постановление, заставляющее их платить более высокие зарплаты.
	\item Gold \textbf{is traditionally thought of} as the best investment in a crisis.\\
	Золото традиционно считается/рассматривается как самым лучшим капиталовложением в случае кризиса.
	\item She \textbf{was showed with} awards.\\
	Её засыпали наградами.
	\item His desk \textbf{is piled high with} papers.\\
	Его стол завален бумагами.
	\item This type of analysis \textbf{is the least paid attention to}.\\
	Этому типу анализа уделяется наименьшее внимание.
	\item She \textbf{was not denied} a promotion or a raise.\\
	Ей не отказывали в повышении по должности или зарплаты.
\end{enumerate}
\ 

\textbfind{Number 3 (Passive Voice; tenses: Present, Past, Future; aspects: Simple, Continuous, Perfect).}

\begin{enumerate}[nosep,leftmargin=*]
	\itemsep\eitsp
	\item Much use \textbf{is being made} these days of various kinds of testing devices.
	\item The side effects of this move \textbf{are being seen} now.
	\item At that time he \textbf{was being considered} for a high-level post in the president administration.
	\item You \textbf{are always being confronted} with different problems.
	\item Little progress \textbf{has so far been made} in reducing military tension between them.
	\item In many industrial countries, the essential demand for all types of construction \textbf{has now been made}.
	\item The estimations \textbf{had not been known about} at the time.
	\item They learned late last week that he \textbf{had been refused} an exit visa.
	\item Serious consideration \textbf{is being given} to growth of the union.
	\item Reuters lost less business than \textbf{had been feared}.
	\item Necessary reforms \textbf{are at last being attempted}.
	\item Half-year results from Racal Electronics \textbf{will be eagerly awaited}.
	\item A new wave of NATO expansion \textbf{will be decided} later.
	\item Traditionally in Rome the new \textbf{has been seen} as the destruction of the old -- and hence resisted.
	\item Nearly everything we do in the modern world \textbf{is helped} by computers.
	\item The opinions of these employees \textbf{were sought}.
	\item The position of Sales Department Manager \textbf{will be filled} by Mr.X.
\end{enumerate}

\newpage
\setcounter{subsection}{3}
\sublinksectionold{Модальные глаголы и их эквиваленты}

\textbfind{Number 1 (can, may, must, should, ought to).}

\begin{enumerate}[nosep,leftmargin=*]
	\itemsep\eitsp
	\item A manager nowadays \textbf{must} gather considerable evidence in support of firing someone.
	\item The content of education \textbf{should} correspond to the nation's actual needs.
	\item In all that people can individually do for themselves, government \textbf{ought not to} interfere. (A.Lincoln).
	\item You(?) \textbf{should} only borrow what you can comfortably afford to repay.
	\item Some of those views \textbf{may} be of some value to others.
	\item If the market becomes too saturated demand \textbf{can} drop off.
	\item You \textbf{can not} have improvement without invention.
	\item Humanized surroundings for employees \textbf{can} produce better workers.
	\item The customer \textbf{may} not always be right.
	\item The good old days of his leadership \textbf{may} never return.
\end{enumerate}
\ 

\textbfind{Number 3 (active or passive form of infinitive).}

\begin{enumerate}[nosep,leftmargin=*]
	\itemsep\eitsp
	\item All federal taxes, however, still \textbf{have to be paid}.
	\item The British government \textbf{ought to be worried} about the declining quality of Britain's world-class universities.
	\item You \textbf{have to get} someone who can make the tough, mean, unpopular decisions.
	\item What \textbf{has to be learned} is not only the firm's budgeting system but also performance standards.
	\item The staff \textbf{will have to be trained} in the use of equipment.
	\item Consideration \textbf{must be given} to these changes.
	\item These measures \textbf{should be given} highest priority.
	\item This Zone \textbf{may not be entered} without a police permit.
	\item Preliminary decisions \textbf{can be taken} at an early stage.
	\item These key issues \textbf{must be addressed} at each stage of the process.
	\item Every country that wants to industrialize \textbf{has to have} steel.
	\item These recommendations \textbf{should be acted upon} with the minimum of delay.
	\item The instructors themselves \textbf{may need} to be retrained.
	\item The number of illegal immigrants \textbf{can only be guessed at}.
	\item They know what \textbf{is to come}.
\end{enumerate}
\

\textbfind{Number 4 (modal verbs).}

\begin{enumerate}[nosep,leftmargin=*]
	\itemsep\eitsp
	\item The manager \textbf{will need to be able to state} (суметь) his problem objectively. (Future Simple)
	\item We \textbf{have now been able to develop} (уже сумели) a highly cost effective product. (Present Perfect)
	\item They \textbf{have had to delay} (уже были вынуждены) these schemes. (Present Perfect)
	\item Considerable care \textbf{must be taken} (необходимо/нужно) by the senior company management in the selection of highly skilled employees. (Present Simple)
	\item Top management \textbf{will now have to spend} (теперь будут вынуждены/придётся) an increasingly greater proportion of its time on people rather than machines, methods or money. (Future Simple)
	\item The proper balance \textbf{must be struck} (нужно/необходимо) between long and short-term objectives (strike the balance -- сохранять равновесие). (Present Simple)
	\item Multinationals everywhere \textbf{are having to work} (сейчас вынуждены) hard on their image. (Present Continuous)
	\item Even less sophisticated investors \textbf{may be beginning to see} (возможно/может быть) the benefits of independent advice. (Present Continuous(?))
	\item A organization \textbf{cannot survive} (не может) long without an appropriate degree of loyalty or commitment from its employees. (Present Simple)
	\item Scientists \textbf{have not been able to make a detailed examination} (так и не сумели) of the environmental damage. (Present Perfect)
\end{enumerate}
\

\textbfind{Number 5 (modal verbs; translate to English).}

\begin{enumerate}[nosep,leftmargin=*]
	\itemsep\eitsp
	\item An economic programme \textbf{must} be well prepared.\\
	Экономическая программа должна быть хорошо подготовлена.
	\item She \textbf{will have to} change her plans.\\
	Ей придётся изменить свои планы.
	\item They \textbf{will be able to} double their production.\\
	Они сумеют удвоить выпуск продукции.
	\item Global cooperation \textbf{should} be given highest priority.\\
	Международному сотрудничеству следует придавать первостепенное значение.
	\item He \textbf{could} be relied on to do independent creative work.\\
	Ему можно было поручить выполнять самостоятельную, творческую работу.
\end{enumerate}

\newpage
\setcounter{subsection}{4}
\sublinksectionold{Модальные глаголы + Perfect Infinitive}

\textbfind{Number 2 (Modal Verbs + Perfect Infinitive).}

\begin{enumerate}[nosep,leftmargin=*]
	\itemsep\eitsp
	\item This report \textbf{could have been written} (мог бы быть написанным) by Lewis Carroll, so nonsensical (лишённый смысла) was it in parts. (Write)
	\item He \textbf{should have} (следовало бы) immediately \textbf{cut short} (прервать) his vacation. (Cut short)
	\item The missing Tory voters \textbf{might have stayed} (возможно остались) at home because the weather was bad. (Stay)
	\item The crisis \textbf{could not have come} (не мог бы произойти) at a worse time. (Come)
	\item He was denied a job in government. The principal reason \textbf{may have been} (вероятно была) his age. (Be)
	\item This sharp young man \textbf{could have} (мог бы) easily \textbf{been a great success} (достичь успеха) on his own. (Be a success)
	\item He took brave decisions that \textbf{should have been taken} (следовало бы принять) years ago. (Take)
	\item They \textbf{could have} (едва могли бы) hardly \textbf{avoided} (избежать) it. (Avoid)
	\item They \textbf{may have lost} (возможно потеряли) everything they cherished. (Lose)
	\item The Internet \textbf{may have cost} (вероятно стоил) inventors a fortune. (Cost)
	\item The Fed (USA) has a reputation for keeping alive firms that \textbf{should have been allowed} (надо было бы позволить) to die. (Allow)
	\item We have left undone those things which we \textbf{ought to have done} (следовало бы сделать). (Do)
	\item It costs a little more than one \textbf{might have thought} (можно было бы представить себе). (Think)
	\item Our people \textbf{might not have made} (навряд ли совершали) those trips in the past. (Make)
	\item There \textbf{may have been} (возможно было) more than altruism behind his actions. (Be)
\end{enumerate}


\newpage
\setcounter{subsection}{5}
\sublinksectionold{Различные функции глаголов "<to be"> и "<to have">}


\newpage
\setcounter{subsection}{6}
\sublinksectionold{Инфинитив}

\textbfind{Number 6 (correct forms of Infinitive).}

\begin{enumerate}[nosep,leftmargin=*]
	\itemsep\eitsp
	\item I have been fortunate \textbf{to have been involved} (\sout{to have involved}) in the real estate industry for over 25 years in the United States.
	\item We ought \textbf{to be using} (\sout{to be used}) our competitiveness in foreign markets.
	\item The only developing countries not \textbf{to have suffered} (\sout{to have been suffered}) from the Asian and Russian collapses were China and India.
	\item We really need \textbf{to be seeing} (\sout{to be seen}) increases of 2\% to 3\% a year.
	\item US President F.D.Roosevelt did not want \textbf{to be seen} (\sout{to see}) as a disabled person.
	\item At long last, the customer is beginning \textbf{to be heard} (\sout{to have heard}).
	\item He has an ambition \textbf{to succeed} (\sout{to be succeed}) J.Ch. as prime minister.
	\item There is plenty \textbf{to repair} (\sout{to have repaired}).
	\item This aid for development has often been too uncoordinated and too thinly dispersed \textbf{to have had} (\sout{to have}) any real impact on poverty.
	\item Good companied take a long time \textbf{to build} (\sout{to have built}) (создавать).
	\item Flats, food and fuel are among the main items \textbf{to be hit} (\sout{to hit}) by increases averaging 400 to 500 per cent.
	\item This company is one of the few producers in the world \textbf{to have come} (\sout{to come}) through the recent recession without making an annual loss.
	\item In order to be seen \textbf{to be doing} (\sout{to be done}) something, the government prods (подгонять, побуждать) the railways into extra safety spending.
	\item This is not the time \textbf{to be buying} (\sout{to have bought}).
	\item As the protests grew, their (students' and workers') demands expanded \textbf{to include} (\sout{to be included}) the ouster (увольнение в отставку) of corrupt leaders and democratic reforms.
	\item The power \textbf{to make} (\sout{to be made}) cuts from the budget remains with the Congress.
	\item The only executive with sufficient authority \textbf{to exercise} (\sout{to be exercised}) such control is the chief executive.
	\item Now is the best time \textbf{to be making} (\sout{to be made}) new investments in venture capital.
	\item The only country \textbf{to have tried} (\sout{to try}) this system so far has been Argentina.
	\item \textbf{To be paying} (\sout{to be paid}) a high rate of interest, this fund has to be taking a fair amount of risk.
\end{enumerate}

\newpage
\setcounter{subsection}{7}
\sublinksectionold{Сложное дополнение (Complex Object)}

\textbfind{Number 9 (translate to English).}

\begin{enumerate}[nosep,leftmargin=*]
	\itemsep\eitsp
	\item Деятки компаний увидели (стали свидетелями того), как их акции резко пошли вниз (to tumble). -- Scores of companies have seen their shares tumble.
	\item Компании увидели (столкнулись с тем), что их текущие прибыли сократились почти в два раза (to halve) с 2002 года. -- The company has seen operating profits almost halve since 2002.
	\item Некоторые люди хотят, чтобы общественное (public-sector) телевидение было освобождено (to free) от политического контроля. -- Some want public-sector television freed from political control.
	\item Большинство экономистов считает, что Америка избежит экономического спада (recession). -- Most economists expect America to escape a recession.
	\item Новые копировальные аппараты (copiers) требуют от обслуживающего из персонала умения пользоваться компьютерной техникой и Интернетом. -- New digital copiers require the service personnel to have an understanding of computers and Internet.
	\item Они читают, что она сделала головокружительную карьеру (to built a distinguished career). -- They consider her to have built a distinguished career.
	\item Придя домой она обнаружила, что пол вымыт и все вещи очищены от пыли (to dust). -- Coming home she found the floor washed and all the things dusted.
	\item Эта угроза побудила его принять решение перевести (to transfer) свои акции. -- This threat led him to decide to transfer his shares.
	\item Он пытался добиться того (to get), чтобы чиновники (the bureaucrats) приняли решение. -- He was trying to get the bureaucrats to make a decision.
	\item Россия должна иметь политическую стабильность, чтобы убедить (to get) ведущих российских бизнесменов вернуть капитал в экономику страны. -- Russia must have political stability to get Russian business leaders to return capital into the economy.
\end{enumerate}


\newpage
\setcounter{subsection}{8}
\sublinksectionold{Сложное подлежащее (Complex Subject)}

\textbfind{Number 1 (translate sentences with Complex Subject to Russian; the predicate verb in the Passive Voice is underlined).}

\begin{enumerate}[nosep,leftmargin=*]
	\itemsep\eitsp
	\item The delegates \uline{are not expected} to participate in today's talk.
	\item The Premier's speech \uline{is believed} not to have aroused wide response.
	\item They \uline{have been heard} to speak about this problem.
	\item These companies \uline{are believed} to have invested a total of \$25 billion into Russia.
	\item The president \uline{is supposed} to step down in 2000.
	\item This type of company \uline{isn't supposed} to exist in Russia.
	\item Sales of personal computers \uline{are forecast} to rise 16\% in 2000.
	\item This book \uline{is not meant} to be read over a cup of coffee in a smoky cafe.
	\item Investment incentives \uline{were announced} to attract the private sector with free land and tax exemptions (налоговые льготы).
	\item These policy measures \uline{are judged} likely to be implemented.
	\item Measures that have already been announced \uline{are assumed} to be implemented over the medium term.
	\item Capital-intensive (капиталоёмкие) technologies \uline{were seen} to be more advanced.
	\item He \uline{is known} to be very rich.
	\item He \uline{is widely believed} to be seeking the removal of several lawmakers.
	\item The Central Bank \uline{was reported} to have said that the ruble could fall below 30 to the dollar by the end of the year.
	\item The emissions of the industrialized world \uline{are thought} to be raising temperatures and may eventually melt polar ice caps.
	\item A decision \uline{was expected} to be made by early summer.
	\item He until recently \uline{was considered by US administration officials} to be one of their most important Russian allies on economic reform.
	\item Leaders of the two companies \uline{are believed} to favor a fall merger.
	\item Department of Trade and Industry officials \uline{are understood} to have traveled to Brussels last week to discuss the proposals.
	\item The editor \uline{was said by his secretary} to be unavailable for comment on Thursday.
	\item The debt burden of Mexican corporations \uline{is widely held} to be the biggest obstacle in the path of a sustained economic recovery.
	\item His company \uline{was reported} to have sold demonstrator cars as new ones.
	\item He \uline{is popularly held} to have deceived the population with his market reform programme.
	\item A senior official of one of Britain's biggest unions \uline{is said} to have ordered Mr.X. and his wife to cancel an expenses-paid trip to Russia and to repay (вернуть) \$500 they were given in advance expenses.
	\item The introduction of this system \uline{is estimated} to have saved up to \$ 5 m for the manufacturers.
	\item He \uline{was alleged} to have received the loan. (To allege -- утверждать (особ.без основания)).
	\item He \uline{was believed} to have been driving nearly 120 mph.
	\item Practical measures \uline{are being considered} to tackle the problem.
	\item He \uline{is alleged} to have been responsible for the illegal disposal of toxic waste.
	\item These demands \uline{are assumed} to be met with imported oil.
	\item The nationalized sector in Great Britain \uline{is known} to consist mainly of service industries. (Service industries $\sim$ вспомогательные отрасли; отрасли инфраструктуры; (power, fuel, transport)).
	\item The industrial revolution in Great Britain \uline{is often said} to have been based on coal an Iron.
	\item This city \uline{is supposed} to be the pacesetter (лидер; задающий тон, темп) of the country's economic reforms.
	\item His talents \uline{were held} to be necessary.
	\item The value of the old machine \uline{can be said} to be one-fifth of the cost of the new one.
	\item He \uline{is said} to be once again attempting to open a business under Ms own name.
	\item At the same time these conditions \uline{must be seen} to be fair to all market players.
	\item They \uline{should no longer be expected} to use these methods.
	\item Such problems \uline{are stated} to arise within the corporation from the technological complexity ... of production process.
\end{enumerate}
\ 

\textbfind{Number 2 (paraphrase these sentences using Complex Subject).}

\begin{enumerate}[nosep,leftmargin=*]
	\itemsep\eitsp
	\item[E.g.] It is believed that he is a good specialist in the field of economics.\newline \textbf{He} \uline{is believed} \textbf{to be a good specialist} in the field of economics.
	\item It was said that he had finished his experiments by the I-st of May.\newline
		\textbf{He} \uline{was said} \textbf{to have finished his experiments} by the I-st of May.
	\item It is expected that this complicated problem will be solved in the future.\newline
		\textbf{This complicated problem} \uline{is expected} \textbf{to be solved} in the future.
	\item It was supposed that he would take part in the discussion.\newline
		\textbf{He} \uline{was supposed} \textbf{to take part} in the discussion.
	\item It was announced that the meeting would take place in Glasgow.\newline
		\textbf{The meeting} \uline{was announced} \textbf{to take place} in Glasgow.
	\item They think that the conference is of great importance.\newline
		\textbf{The conference} \uline{is thought} by them \textbf{to be of great importance}.
	\item It is supposed that the statement will arouse wide response (вызовет широкий отклик).\newline
		\textbf{The statement} \uline{is supposed} \textbf{to arouse wide response}.
	\item The president stated that this visit was of great importance.\newline
		\textbf{This visit} \uline{was stated} by president \textbf{to be of great importance}.
	\item It is assumed that labour is economically motivated.\newline
		\textbf{Labour} \uline{is assumed} \textbf{to be economically motivated}.
	\item They said that he had lost patience.\newline
		\textbf{He} \uline{is said} \textbf{to have lost patience}.
	\item It is assumed that student have some familiarity with the basic patterns of economic systems.\newline
		\textbf{Student} \uline{is assumed} \textbf{to have some familiarity} with the basic patterns of economic systems.
\end{enumerate}
\ 

\textbfind{Number 3 (translate sentences to English using Complex Subject)}

\begin{enumerate}[nosep,leftmargin=*]
	\itemsep\eitsp
	\item Широко известно, что существующая система является устаревшей. \textbf{(To consider)}
	\item Полагают, что они постараются выполнить решения президента. \textbf{(To expect)}
	\item Говорят, он потратил миллионы, скупая землю в целом ряде восточных штатов. \textbf{(To beleive)}
	\item Роботов считают идеальными машинами для выполнения (to perform) простых монотонных задач. \textbf{(To consider)}
\end{enumerate}
\ 

\textbfind{Number 4 (translate sentences with Complex Subject to Russian; the predicate verb in the Active Voice is underlined).}

\begin{enumerate}[nosep,leftmargin=*]
	\itemsep\eitsp
	\item People \uline{seem} to be talking about me behind my back.
	\item Wednesday \uline{happened} to be his birthday.
	\item The fact, that the company has bought a computer \uline{appears} to indicate some considerable faith in the future.
	\item Conflict \uline{seems} to be an inescapable part (неотъемлемая часть) of group life.
	\item These problems \uline{seem} to have been less important.
	\item These workers \uline{seem} to have been relatively successful.
	\item The composition of the lunar surface \uline{proved} to be similar to basalts on Earth.
	\item Nothing \uline{seemed} to have been changed since her childhood.
	\item She \uline{seemed} to have been waiting for us for an hour.
	\item We \uline{happened} to be in London on that day.
	\item I \uline{chanced} to meet him in New York.
	\item He \uline{didn't seem} to have much of a sense of humour.
	\item Many forms \uline{have proved} ill-prepared for the unique problems in this marketplace.
	\item American's workers \uline{appear} to have neared their most cherished goal.
	\item These forecasts \uline{turned out} to be inaccurate (faulty).
\end{enumerate}
\ 

\textbfind{Number 5 (paraphrase these sentences using Complex Subject).}

\begin{enumerate}[nosep,leftmargin=*]
	\itemsep\eitsp
	\item[E.g.] It seems he opposed this proposal. (Кажется он был против этого предложения).\newline
		\textbf{He} \uline{seems} \textbf{to have opposed} this proposal.
		\newline
		It seemed that finding these books was not too difficult a task.
		\newline
		\textbf{Finding these books} \uline{didn't seem} \textbf{too difficult a task}.
		\newline
		Обратите внимание на перемещение отрицания во втором примере (при перефразировании).
		Обратите внимание на отсутствие глагола "<to be"> во втором примере.
	\item It appeared that they were away.
	\item It seems he doesn't mind what the firm pay him.
	\item It seems they are permanently dependent on government aid.
	\item It appears that historians know very little about his life.
	\item It appears that this statement does not have any particular significance.
	\item It turned out that these two control mechanisms were independent of each other.
	\item It seems that people are not in a hurry.
	\item It turned out that millions of people cheered him in New York City.
\end{enumerate}
\ 

\textbfind{Number 6 (translate sentences to English using Complex Subject)}

\begin{enumerate}[nosep,leftmargin=*]
	\itemsep\eitsp
	\item Оказалось, что он прав. (\textbf{Prove})
	\item Получилось так, что эти делегаты отсутствовали на прошлом заседании. (\textbf{Happen в отрицательной форме}; to attend -- присутствовать)
	\item Кажется, индийские менеджеры не поняли этого. (\textbf{Seem в отрицательной форме})
	\item Этот год оказался ещё одним годом разочарований. (\textbf{Prove}; another disappointing year)
	\item Кажется, эта система работает в настоящее время. (\textbf{Seem})
	\item Этот анализ оказался очень ценным. (\textbf{Turn out})
	\item Кажется (по-видимому), социальный статус квалифицированных инженеров в этой стране ниже, чем в США. (\textbf{Appear})
	\item Сейчас, кажется (по-видимому), опасность уже миновала. (\textbf{Appear}; to pass)
\end{enumerate}
\ 

\textbfind{Number 7 (translate sentences to English using Complex Subject)}

\begin{enumerate}[nosep,leftmargin=*]
	\itemsep\eitsp
	\item Consequently (следовательно), environmental requirements \uline{are more likely} to increase than abate (уменьшаться).
	\item The revolution in new materials \uline{is likely} to progress dramatically.
	\item Someday engineers \uline{are certain} to come up with a television set that can be worn on the wrist (запястье).
	\item Yesterday's manufacturing methods \uline{are not likely} to satisfy tomorrow's manufacturing needs.
	\item Everyone \uline{is certain} to welcome the new initiatives.
	\item Scientific cooperation \uline{is sure} to benefit both of the countries.
	\item He \uline{is likely} to sell some of his oil to energy-hungry nations.
\end{enumerate}
\ 

\textbfind{Number 8 (translate sentences to English using Complex Subject)}

\begin{enumerate}[nosep,leftmargin=*]
	\itemsep\eitsp
	\item Маловероятно, что эту проблему решат.
	\item Эти данные наверняка подтвердят (confirm) его гипотезу.
	\item Временные (temporary) трудности обязательно будут преодолены.
	\item Такие исследования, вероятно, приведут (lead) к увеличению производительности труда.
	\item Похоже, что цены на землю снизились (fall) с ноября.
\end{enumerate}
\ 

\textbfind{Number 9 (translate sentences with Complex Subject to Russian)}

\begin{enumerate}[nosep,leftmargin=*]
	\itemsep\eitsp
	\item \uline{The oil} was said \uline{to have been} refined in Durban (South Africa).
	\item By 2000 \uline{this problem} is likely \uline{to have reached} crisis proportions.
	\item \uline{These firms} are estimated \uline{to have created} around 20,000 jobs.
	\item \uline{He} was said \uline{to have been dealing with} American customers of late (недавно, за последнее время).
	\item \uline{The gulf} between rich and poor is likely \uline{to go on widening}.
	\item \uline{We} are unlikely \uline{to understand} what they are doing.
	\item \uline{The future} is likely \uline{to bring} an increased international friction over trade.
	\item \uline{Advancements} in artificial intelligence area are said \uline{to come} slowly with many failures.
	\item \uline{This programmer} has been known \uline{to pride himself} in writing code so good that no one else can understand it.
	\item \uline{This plant} is claimed to be the most advanced petrochemical \uline{facility}.
	\item \uline{The salaried worker} is assumed \uline{to do} his work without a direct incentive relationship between output and payment.
	\item \uline{Their income} is assumed \uline{to have declined} somewhat.
	\item Without substantial structural reforms \uline{Russia} is not likely \uline{to return} to economic growth.
	\item \uline{They} are unlikely \uline{to provide} any help.
	\item \uline{Negotiations} are likely \uline{to prove tough}.
	\item \uline{This plan} is considered \uline{unlikely} to go ahead.
	\item Too many times \uline{young recruits} (новичок; $\sim$ молодой специалист) are heard \uline{to say} after a year on the job, "<I could have done this work without going to college.">
	\item \uline{The beneficiaries} of all these changes are meant \uline{to be consumers}. (Beneficiary -- тот, кто получает экономическую выгоду от внедрения новшества).
	\item However, \uline{not many of us} can be said \uline{to be inspirational speakers} (вдохновенный оратор).
	\item When \uline{the source of this piece of information} is seen \uline{to come from} no less an august body (авторитетный источник) than the Royal Society, few will suspect its vaildity.
	\item \uline{Output} per man-hour has long been known \uline{to set} a definite limit to the ability of a society to support with goods and services a given standard of living.
	\item The hooped-for US investment \uline{boom} is unlikely \uline{to materialize}.
	\item \uline{The subject} seems \uline{not to have been researched}.
	\item \uline{I} happen (-кстати; как раз) \uline{to have enjoyed} this film.
	\item This year \uline{she} seemed \uline{to be having} more problems.
	\item The early \uline{history} of insurance doesn't appear yet \uline{to have been thoroughly explored}.
	\item Private \uline{investors} appeared \uline{willing} to pay \$15,000 for the computer programs.
	\item These \uline{shares} seemed \uline{more robust} than usual.
	\item Up to now Switzerland's private \uline{bankers} seem \uline{to have been able to hold on to} (сохранять, удерживать) their market leadership surprisingly well.
	\item The new \uline{system} is proving \uline{to be very fast and stable}.
	\item Many \uline{developers and investors} of residential buildings seemed \uline{to have emerged} from the crisis unscathed  [(прил.) невредимый].
	\item Russia's new privatization \uline{minister} seems \uline{minded} [(прил.) готовый ч.-л. делать] to bring everything he can to market.
	\item \uline{The benefits} could eventually turn out \uline{to be even greater} than those estimated.
	\item His \uline{remarks} were thought \uline{significant}.
	\item He explained that \uline{the meeting} seemed \uline{unnecessary}.
\end{enumerate}
\ 

\textbfind{Number 10 (translate sentences to English using Complex Subject)}

\begin{enumerate}[nosep,leftmargin=*]
	\itemsep\eitsp
	\item Так уж случилось, что эта (политическая) партия преобладает (to hold a majority). \textbf{(Happen)}
	\item Экономический рост, по-видимому, снизился (to slow down) до годового показателя (an annual rate of) примерно 1.5 процента. \textbf{(Appear)}
	\item Создавалось впечатление, что это мало его волнует (to care little). \textbf{(Appear)}
	\item Кажется, он никогда не рассматривал такую возможность (to consider). \textbf{(Seem)}
	\item Инфляция может оказаться более высокой, чем ожидалось. \textbf{(May turn out)}
	\item Люди, по-видимому, обращают мало внимания на эту проблему. \textbf{(Appear)}
	\item Переговоры, вероятно, продляться (to last) несколько дней. \textbf{(Be likely)}
	\item Маловероятно, что их рекомендации понравятся (to please) членам парламента. \textbf{(Be unlikely)}
	\item Против их предложений, вероятно, выступят потребители. \textbf{(Be likely)}
	\item Их терпение (patience) может оказаться небезграничным (infinite -- безграничный). \textbf{(Might not prove)}
\end{enumerate}

\newpage
\setcounter{subsection}{9}
\sublinksectionold{Предложный инфинитивный оборот (Prepositional Infinitive Complex)}

\textbfind{Number 2 (translate parts in brackets to English).}

\begin{enumerate}[nosep,leftmargin=*]
	\itemsep\eitsp
	\item The Emergency Situations Ministry set Dec.15 as deadline \textbf{for the heat to be turned on} (для того, чтобы отопление было подключено; to turn on the heat).
	\item A huge step would be \textbf{for Russia to join the WTO} (чтобы Россия вступила в ВТО; to join the World Trade organization).
	\item There are two basic principles \textbf{for the sales manager to remember} (которые должен помнить менеджер по продажам; to remember).
	\item There will be a need \textbf{for them to have initial working capital} (чтобы они имели первоначальный капитал; to have initial working capital).
	\item The next moves -- developing good ideas and carrying them out -- will be very much harder \textbf{for him to make moves} (ему сделать/предпринять; to make moves).
	\item Norway says that it is quite happy \textbf{for foreigners to take part in its privatization programme} (что иностранные граждане принимают участие в её приватизационной программе; to take part).
	\item Russia always needs to be ready \textbf{for the oil price to end} (к тому, что везенье, связанное с ценами на нефть, закончится).
\end{enumerate}


\newpage
\setcounter{subsection}{10}
\sublinksectionold{Причастие I (Participle I)}

\textbfind{Number 2 (use Present Participle (active or passive) or Perfect Participle (active or passive)).}

\begin{enumerate}[nosep,leftmargin=*]
	\itemsep\eitsp
	\item \textbf{Having lived} (прожив; to live) in London long he speaks English well.
	\item \textbf{Having signed} (подписав; to sign) the treaty the heads of Government left the city.
	\item \textbf{Arriving} (прибывая; to arrive) at the airport the heads of Government usually make a short statement.
	\item The company has announced plans to shut down one of their divisions \textbf{throwing} (превращая в безработных; to throw) 300 workers onto the employment lines.
	\item Many of the buildings \textbf{being built} (строящихся; to be built) do not function as well as they should.
	\item \textbf{Having been discussed} (после того как обсуждали; to be discussed) for 7 hours the declaration was approved.
	\item \textbf{Having been defeated} (после того как потерпел поражение на выборах; to be defeated -- потерпеть поражение на выборах) the Premier was obliged to resign.
	\item Projects now \textbf{being considered} (рассматриваемые/обсуждаемые; to be considered) are an aqua park and cinema center.
	\item Investments decisions are not things \textbf{being made} (обсуждаемые/принимаемые; to be made) daily; it is a matter of carefully judging.
	\item He spent the last forty years of his life \textbf{traveling} (путешествуя; to travel) around the country.
	\item Einstein was a late starter, \textbf{being barely able} (который плохо умел/испытывал затруднения; to be barely able) to read at age eight.
	\item There are so many other problems \textbf{crying out for} (требующих; to cry out -- взывать; требовать) active measures.
	\item \textbf{Being chemically different} (отличаясь от; to be chemically different) from wool and cotton, new synthetic fibres require special detergents (моющие средства) to wash them properly.
	\item \textbf{Having defined an objective} (определив цель; to define an objective), it is essential to assess the size of market, buying patterns, output required and pricing structure.
	\item Work on standards \textbf{being done} (проводимая/осуществляемая; to be done) in the UK is not separated from that being done elsewhere.
\end{enumerate}


\newpage
\setcounter{subsection}{11}
\sublinksectionold{Независимый причастный оборот}


\newpage
\setcounter{subsection}{12}
\sublinksectionold{Причастие II (Participle II)}

\newpage
\setcounter{subsection}{13}
\sublinksectionold{Герундий (Gerund)}


\newpage
\setcounter{subsection}{14}
\sublinksectionold{Герундиальные обороты}


\newpage
\setcounter{subsection}{15}
\sublinksectionold{Глаголы, вызывающие особые затруднения (Confusing Verbs)}


\newpage
\setcounter{subsection}{16}
\sublinksectionold{Степени сравнения прилагательных и наречий}


\newpage
\setcounter{subsection}{17}
\sublinksectionold{Предлоги}


\end{document}
