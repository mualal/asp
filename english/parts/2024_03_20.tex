\documentclass[main.tex]{subfiles}

\begin{document}

\linksection{Лекция 20.03.2024 (Смольская Н.Б.)}

So, let us begin.
Well, taking into account that there are not so many of you at the moment, I think that we will...
I was planning to start with the discussion points.
But I think we will go the other way around.
So, I will give you the...
So, it's like a sort of a test.
Well, that's not the test, but that's a sort of a test paper.
And you will work with it, because this is the collection of different cases of using subjunctive mood in the sentences.
So your task will be to work through, so then we will get together and discuss those areas or those aspects of using subjunctive mood in the sentences and finish with the sort of the discussion and your (what is it) reminiscences (воспоминания) about participating in different grant projects and scholarships, scholarship projects that you had the chance to participate in during your previous life.
Maybe you are taking all these competitive processes at the moment, I don't know, because this is the period when, as far as I know, the process of applying for different scholarships and grants is actually taking place, so maybe that would be the point for you to speak about.
So that will be the second part of our class.

Okay, because your colleagues are coming, so that is why we will, we won't interrupt your monologues, but it will be like a sort of a passive participation with the activity of your knowledge and your brains and your mind of thinking about the subjunctive mood cases.
Well, I hope that it won't take much time for you to...
Actually, there is the opposite side of paper (задания есть и на обратной строне листа).

It won't take too much time because, yes, there are some points that can be difficult for you, but we'll discuss it after that.
But at the moment, I think you will try and work with it quite quickly.

So today subjunctive mood, because after that we will switch to, as I have said, as I have told you before, to gerunds, infinitives, and participles, which is also quite a problematic aspect of the grammar of the English language.
So take your time with the subjunctive mood and then we will discuss it.
На листах можно ручкой отмечать?
Yes, these are the copies for you, so you can use them and then you can add your notes if it is necessary (на листах с заданиями можно делать пометки).
These are your copies.
It is like a self-test, because after that you will be checking and evaluating yourself actually.

And if I'm not mistaken, we were going to discuss again to refresh the procedure of the examination, right?
Just for those who have forgotten the most important aspects of the exam.
So for me not to forget.

\newpage
\sublinksection{Таблица Subjunctive Mood (в сложных предложениях)}
\label{subsec:subjunctive-mood-lk}
{\parindent-10pt\includegraphics[width=1.04\textwidth, page=1,trim={0.7in 0.72in 1.15in 1.2in},clip=true]{SubjunctiveMood.pdf}}

\vspace{-5.7pt}
{\parindent-10pt\includegraphics[width=1.04\textwidth, page=2,trim={0.7in 1.15in 1.15in 1.2in},clip=true]{SubjunctiveMood.pdf}}

\placetextbox{0.127}{0.777}{\scriptsize(условные предложения;}

\placetextbox{0.127}{0.767}{\scriptsize типы условных}

\placetextbox{0.127}{0.757}{\scriptsize предложений разбирали}

\placetextbox{0.127}{0.747}{\scriptsize на прошлом занятии)}

\newpage
\sublinksection{Грамматика сослагательного наклонения (примеры + теория)}

\hypertarget{ltask:2024-03-20}{--- Выполнение задания ---} (\hyperref[task:2024-03-20]{\color{blue}{перейти к тексту задания}})
\\

\textbfind{Выполненное задание (первые 9 предложений теста на разные случаи употребления сослагательного наклонения, которые успели разобрать на паре)}

\begin{enumerate}[nosep,leftmargin=*]
	\itemsep\eitsp
	\item The general ordered that the enemy \uline{\textbf{be}} stopped.
	
		((A) is; (B) was; \textbf{(C) be}; (D) were)
	\item The judge insisted that the witness \uline{\textbf{tell}} the truth.
	
		((A) tells; (B) told; \textbf{(C) tell}; (D) will tell)
	\item ``Do you like to eat steak?''\\``Certainly. And I wish \uline{\textbf{I got}} it more often.''
	
		(\textbf{(A) I got}; (B) I am getting; (C) my getting; (D) I have got)
	\item ``I'am hungry.''\\``I \uline{\textbf{would have fixed}} breakfast for you if I had known.''
	
		((A) fixed; (B) have fixed; \textbf{(C) would have fixed}; (D) will fix)
	\item I wish that my sick friend \uline{\textbf{were}} out of danger.
	
		((A) is; \textbf{(B) were}; (C) was; (D) would)
	\item Would you take a trip round the world if you \uline{\textbf{had}} more money?
	
		((A) have; \textbf{(B) had}; (C) would have; (D) had had)
	\item I wish I \uline{\textbf{had gone}} with you last night.
	
		((A) went; (B) have gone; (C) was going; \textbf{(D) had gone})
	\item You should wear thick clothes in winter lest you \uline{\textbf{should catch cold}}.
	
		((A) may catch cold; (B) might catch cold; \textbf{(C) should catch cold}; (D) can catch cold)
	\item I \uline{\textbf{would have been}} glad to show you around the city, had you come here.
	
		(\textbf{(A) would have been}; (B) could be; (C) would be; (D) were)
\end{enumerate}
\ 

Okay, so let's begin.
Well, first of all, I would like you again refresh your terminology knowledge.
So we are discussing Subjunctive Mood, which is Сослагательное Наклонение.
You remember that we have (or we remembered) two more moods in the English language, which are (or not only in English, but the same as with the Russian language) which are повелительное и изъявительное.
So what about the English terms that we use?
So what about повелительное? You all know.
Imperative.
Imperative mood, that's absolutely correct.
So what about изъявительное?
You remember that when discussing these moods, we have pointed out that изъявительное наклонение is the mood that speaks about the real facts or the facts in reality.
So, we actually objectively treat these facts.
So, what would be, what is the meaning?
Because actually the meaning of изъявительное наклонение explains and gives us the term.
So when we think about the fact, we actually, what do we do when we point out the fact?
What is the verb?
We indicate.
Indicative.
So this is the third term.
So we have indicative mood, imperative mood, and subjunctive mood (изъявительное, повелительное, сослагательное).

And we have discussed with you that subjunctive mood includes conditional sentences but it is not the same and we should keep it in mind.
Yes, conditional sentences are actually the most widely used examples or cases of the subjunctive mood, but there is quite a number of cases where forms of subjunctive mood in the English language are used, and the table that was uploaded into the course and that some of you actually have, gives us the understanding that conditional sentences are not the only ones that when we should use forms of the subjunctive mood.
One more point for us to understand.

So what about the meaning of subjunctive mood?
Итак, что обозначает сослагательное наклонение?
Какую идею оно передает о действиях?
О каких действиях мы говорим или, так сказать, держим их в голове, когда мы должны использовать сослагательное наклонение?
Какие действия?
Если при изъявительном наклонении мы говорим об объективно происходящих, произошедших или будущих действиях, которые произойдут, то при сослагательном наклонении мы говорим о предполагаемых, о возможных, о желаемых действиях (потенциальных действиях).
То есть мы передаем нечто субъективно на это воспринимаемое.
Не факт, что это будет стопроцентная реальность.
То есть это некие наши предположения, мечты, приказы в определённых предложениях.

Не всегда только imperative mood в английском языке будет передавать приказы.
Мы можем это и передавать через определенные конструкции в предложениях, которые вы тоже тут должны были увидеть (в задании, которое вы выполняли), как бы снимая напряжение, которое передавало бы imperative mood. Потому что когда мы используем повелительное наклонение, то это приказ прямо вербализованный, но при этом приказ можно подавать как-то косвенно, за счет построения другого типа предложения.
И вот здесь как раз нам на пользу приходит конструкция сослагательного наклонения.

В итоге нам нужно развести условное, просто у вас в голове должен произойти вот этот сплит, что условные предложения –- это одно, а сослагательное наклонение -- это специфические конструкции (в английском языке всё состоит из конструкций), которые должны использоваться для выражения определенных понятий, в данном случае предполагаемых или возможных действий.
Это просто набор определенных кубиков, которые должны сложиться, и в этом случае они будут передавать именно ту идею, которая нам нужна.

В русском языке в данном случае все проще, потому что русский язык, пользуясь своими синтетическими характеристиками для сослагательного наклонения будет использовать только что?
Сослагательное наклонение в русском языке, по сути дела, приравнивается, к условному наклонению в русском языке (поэтому очень часто мы используем другой термин, который и ввёл вас в заблуждение).
В русском языке это практически синонимичные, если можно так сказать, взаимозаменяемые термины.
И условное/сослагательное наклонение в русском языке образуется с помощью частицы БЫ.
Больше у нас никаких вариантов для условного наклонения, для сослагательного/условного наклонения нет в русском языке.

Английский язык сослагательное наклонение (то есть особые формы) использует для выражения вот тех вот понятий, так скажем, или оттенков значения глагола, которые в таблице у вас тоже отражены.

Ну давайте пойдем от теста и будем с вами либо группировать, ну, не от теста, а от упражнения, точно, bless you.
Кстати, bless you -- это...
Что?
Повелительное наклонение?
Нет, это как раз сослагательное наклонение.
Я же...
Повелительное наклонение -- это либо приказ, либо просьба.
А bless you...
Пожелание?
Да.
Да благословит тебя Бог.
Господь?
Да, Господь.
Так что вот это вот предложение на самом деле...
Несмотря на то, что форма вроде похожа, но при этом предложение это выражает пожелание в вашу сторону.
Поэтому вот оно сослагательное наклонение.
Как раз вы в тему нам сейчас тут невербально выразились сослагательно.

Так, давайте пойдем по предложениям и будем с вами соотносить вот с тем, что мы наблюдаем в каждом предложении, чтобы понять, что же такое сослагательное наклонение.
Я говорю что, в данном случае область сослагательного наклонения в английском языке шире, чем в русском языке, поэтому нам нужно немножко отступить от наших знаний русского и позволить всем тем возможностям и просторам грамматики английского языка записаться где-то с той стороны лобной кости.
Простите за метафору.

Итак, ну что, давайте, наверное, пойдем один за другим.
Я думаю, что 23 вас точно тут есть, поэтому на каждого хватит возможности озвучить свою точку зрения.
Начинаем с вас.
Пожалуйста, самое первое.
Поскольку это тест и необходим выбор правильного ответа.
Смысл, конечно, в том, чтобы вы мне прочитали потом целое предложение, но прежде чем вы будете его читать начинаем всё-таки с выбора ответа чтобы не допускать ошибок, то есть сначала мы называем буквочку.

Итак, какой вы считаете будет правильный выбор? A, B, C or D?
D?
Ещё у кого какие варианты?
Видите, всё-таки вам все подсказывают C.
C -- это правильный ответ.
Пожалуйста, прочитайте нам всё предложение.
The general ordered that the enemy \textbf{be} stopped.
Итак, что это за случай у нас?
У кого есть табличка перед глазами или кто помнит?
Что это такое за предложение?
Это дополнительное предложение, потому что чисто формально это предложение присоединяется союзом that.
Союз that присоединяет формально, на уровне формальности.
Да, мы с вами говорили, да, что английский язык высокоформальный и формульный.
Даже здесь, наверное, так нужно выбрать правильное слово.
Поэтому союз that присоединяет с точки зрения грамматики или синтаксиса только дополнительные типы предложений.
Но другое дело то, что предшествует этому союзу that в главном предложении, вынуждает нас использовать в придаточном предложении форму сослагательного наклонения.
Для того, чтобы определиться, нам нужно посмотреть какой глагол.
Они здесь у вас (в табличке) в общем-то самые широко распространённые здесь перечислены.
Suggested, ordered, proposed, demanded.
Сюда может глагол wanted подойти.
Required, если продолжать список.
Вот если мы видим какие-то из этих глаголов, которые говорят о неких действиях.
Предположил, пожелал, захотел.
То есть некие пожелания, возможности, предположения, то после этих глаголов в дополнительных предложениях будет использоваться форма сослагательного наклонения.
И просто, чтобы понимать, почему здесь глагол to be, нужно, на самом деле, отталкиваться от универсальной одной из форм.
Или если мы немножко с вами отойдем в сторону.

Итак, кто-нибудь может мне рассказать с точки зрения форм сослагательного наклонения.
Никто не поинтересовался или где-то в каких-то теоретических работах не почитал.
Существуют две формы сослагательного наклонения.
Subjunctive I и Subjunctive II, так называемые.
Если выражаться русским языком, то Subjunctive I -- это так называемая синтетическая форма, а Subjunctive II -- аналитическая.

\hypertarget{subjunctive-mood-forms-table}{}
{\parindent0pt\includegraphics[width=\textwidth,page=1,trim={0.95in 5in 0.95in 0.9in},clip=true]{SubjunctiveMoodForms.pdf}}

\textbfind{Важное замечание!} В таблице (и далее по тексту) для Subjunctive I указаны Present Simple, Past Simple и Past Perfect.
Но так говорить не совсем правильно, так как для сослагательного наклонения эти формы соответственно называются Present Subjunctive I (простой инфинитив без частицы "<to">), Past Subjunctive I (форма глагола Past Simple) и Past Perfect Subjunctive I (had + Participle II).\newline
Аналогично для Subjunctive II есть Present Subjunctive II (should / would / might / could + простой инфинитив без частицы "<to">) и Perfect Subjunctive II (should / would / might / could + перфектный инфинитив без частицы "<to">).

Слово "<аналититическая"> -- поскольку вы уже, наверное, поняли за те наши полгода (хоть и размазанные), но тем не менее полгода мы с вами общаемся -- я уже вам не раз упоминала этот термин.
Когда упоминаю аналитические формы, то я всегда говорю о чём?
Что значит аналитическая форма?
То есть она как формула.
Да.
А если попроще?
Если вдруг кто-то не понимает что значит аналитическая форма?
Что входит в аналитическую форму?
Опишите, пожалуйста.
Итак, это набор, состоящий из двух и более слов.
Аналитическая форма.
То есть это либо два каких-то слова будут включаться сюда, либо может быть и больше.

А когда мы говорим про синтетическую форму, то здесь мы будем с вами говорить о том, что эта синтетическая форма выражается одним словом.
Синтетическая форма -- это форма, которая не будет выходить за рамки слова.
Это то, что мы в русском языке в большинстве грамматических форм имеем -- когда всё, что нам нужно выразить, мы выражаем в рамках одного слова.
Шёл, пришёл, перешёл, не дошёл.
Все равно все эти значения прошедшего времени, отрицательные формы, они как бы будут в рамках одного слова.

Таким образом, если мы говорим с вами про форму, которую можно называть Subjunctive I, то исходя из тех примеров, которые у вас были в табличке, из тех примеров, которые мы с вами уже обсуждали на прошлом занятии, какие формы для образования сослагательного наклонения из ваших предложений будут относиться к Subjunctive I?
Когда мы имеем дело только с одним словом.
Одно слово имеется в виду при переводе на русский или что?
Нет, одно слово в предложении в английском языке.
Что за формы это будут?
Это либо \textbf{форма Present Simple}.
Используется в нулевом и первом типах условного наклонения (эти типы разбирали на прошлой паре).
Например, для первого типа условного наклонения:
If I \textbf{call} you we will go.
\textbf{Сall} -- это Subjunctive -- это специфическая форма, ведь на уровне содержания у нас будущее действие -- если я тебе позвоню (в будущем).
Тут нам русский язык в помощь для понимания.
Но в придаточном предложении условия, которые относятся к будущему времени, мы, тем не менее, используем форму Present Simple.
Это форма Subjunctive.
Английский язык (то есть абстрагируйтесь от русского) в условных предложениях, которые обозначают реальные условия, требует использовать форму, которая похожа на Present Simple.
И эта форма называется Subjunctive I.
И ещё, когда мы говорим о том, что выражается одним словом, это какая форма?
\textbf{Форма Past Simple}.
Точно так же.
Используется, например, во втором типе условных предложений (разбирали на прошлой паре).
If we \textbf{went}, we would come.
Если бы мы пошли, мы бы пришли.
Ну уж простите за простоту смысла предложения.
То есть либо Present Simple, либо Past Simple.

Далее давайте сначала перейдём к Subjunctive II, а потом уже допишем сюда еще одну форму к Subjunctive I.

А когда мы говорим про Subjunctive II, то это значит, что в английской грамматике Subjunctive II образуется из смыслового глагола и некоего глагола, который рассматривается именно как некое дополнение для выражения Subjunctive.
Это что за формы?
Это формы с глаголами would и should.
А дальше мы сюда уже с вами подставляем то, что нам нужно в зависимости от типа предложения, например, условного.
Либо, как это сказать, V1, то есть Bare Infinitive (инфинитив без to), либо это будет Perfect Infinitive (например, have done).
Вот это называется аналитическая форма, когда глаголы would и should, как некие сторонние глаголы, начинают использоваться именно в нужных нам предложениях.
В данном случае мы с вами не берем Future In The Past, это другой случай в английском языке, а вот в тех требованиях, когда по смыслу мы передаем желательные действия, would и should будут использоваться как показатели сослагательного наклонения и использоваться в так называемых Subjunctive II формах.

Ну и сюда к Subjunctive I мы должны приписать еще, хотя с точки зрения грамматики это аналитическая форма глагола, но это не аналитическая форма Subjunctive.
Это форма Past Perfect.
Потому что Past Perfect, если бы у нас было занятие по грамматике английского языка, как вы, так сказать, занимались уроками русского языка в школе, и вы бы выделяли, подчеркивали бы сказуемые (помните, в школе наверняка на уроках русского подчеркивали, искали, и потом определяли типы сказуемых).
Когда вы говорили о типах сказуемых, вы прежде всего делили сказуемые на две основные группы.
Какие?
Простые и составные.
Так вот, в английском языке то же самое, но только подход наш к определению классификации, вот что это будет, квалификации, вернее, простое ли это сказуемое, будет зависеть от того, будет ли это видовременная форма глагола или же сказуемое, образованное с помощью различных дополнительных компонентов.
И вот если бы мы с вами встретили предложение.
Yesterday he \textbf{went} to the cinema.
He \textbf{had gone} to the cinema before he \textbf{went} to school.
Had gone и went в английском языке будут простыми сказуемыми.
Простыми, потому что они образуются с помощью таблички вот времён и видов английского глагола.
Вот если вы форму глагола возьмёте из этой таблички, то тогда это простое сказуемое.
Несмотря на то, что из двух слов состоит.
А вот если мы с вами построим, например, вот Subjunctive возьмем, I \textbf{would go} if I had, или предположим, что там можно еще построить, чтобы было would или should.
Даже если это будет модальный глагол, то это уже будет составное сказуемое, потому что для того, чтобы построить вот эту форму с помощью глаголов would или взять модальный, то вы уже в табличке в этой (в таблице аспектно-временных форм глагола) уже этой формы не найдёте.
То есть для английского языка градация между простым и составным сказуемым является, вернее не градация, а лакмусовая бумажка.
Это найдем ли мы эту форму, сможем ли мы её образовать с помощью таблицы аспектно-временных форм глагола или же уже нет.
И тогда мы строим разницу между аналитическими формами или синтетическими.
Понятно, да?
Ну так более-менее.

Теперь мы возвращаемся к тому предложению, которое у вас самое первое.
И несмотря на то, что здесь у нас из того, что выбор нам дан, и вы правильно здесь его нашли, это вариант из типов придаточных -- это дополнительное, мы здесь выбираем вторую по аналогии из таблицы: He suggested/ordered/proposed/demanded \uline{that the work (should) be done} by tomorrow.
Should у нас здесь стоит в скобочках.
Мы сейчас где?
Я потерялся.
Мы сейчас на первом предложении из теста: The general order that the enemy \textbf{be} stopped.
Для того, чтобы правильно выбрать вариант (C), те, у кого есть таблица, вот такая, как у меня, вот я вижу, что у некоторых она есть.
Вы пальчиком здесь смотрели и нашли, что это третья строчка.
И я просто отсюда пример прочитала.
Да, но не все сразу поняли...
Извините...
Я быстро переключаюсь.

Итак, мы здесь с вами посмотрели.
В таблице should взято в скобочки.
Это нам говорит о том, что should может здесь использоваться, а может не использоваться.
Но тем не менее, в предложениях дополнительных, которые стоят в сложном предложении после глаголов типа suggest, order, propose, suggested, demanded и присоединяются с помощью союза that будет использоваться так называемая форма Subjunctive II, аналитическая, которая в идеале состоит из двух слов should плюс так называемый bare infinitive (инфинитив без to; или мы поставим с вами V1), но в нашем предложении, как вы увидели, should у нас нет и в современном английском вот эта вот тенденция опускать should, но оставлять при этом V1.
Ещё раз -- что такое V1?
Это инфинитив без to.
Никакого окончания -s к нему не добавляется.
Вы просто опускаете should.
Но вот эта часть (bare infinitive) остается так, как она должна быть.
Может дёрнуться рука, и вот специально вас в этом тесте мучают, путают и предлагают возможный вариант (A) is (это третье лицо глагола be).
Вот эта форма здесь недопустима.
После глаголов ORDERED, SUGGESTED, PROPOSED в придаточном предложении Present Simple третьего лица единственного числа использоваться не может.
Если бы у нас было: He says that he goes to the cinema every day.
Он говорит, что ходит.
Тогда будет, потому что глагол say не выражает пожеланий, приказов, мечт и так далее.
Глагол says -- это просто простой indicative глагол (в изъявительном наклонении).
Поэтому после него это правило не работает.
А вот если вот эти глаголы -- вот просто чтобы понять, что английский язык формальный, вот он (английский язык) принял для себя решение, что после вот этих глаголов, в придаточных мы должны использовать формулу should + V1 и только так мы должны это делать.

Можно вопрос?
Should be же звучит как-то более привычно?
А если убираем should, то, на мой взгляд какая-то ерунда получается.
Почему?

Это вы мыслите с точки зрения русского языка, русскоговорящего человека.
Потому что это уже правило, которое зафиксировано в грамматике.
И вот такую форму, вот эту форму без should фиксируют как более предпочтительную.
Мы говорим о том, что есть некая формула (should + V1), которая позволяет усеченный вариант ее использовать.
Либо, первое, в американском варианте, в American English будет использоваться вариант без should.
American English.
Вообще, мы с вами говорили и в самом начале, что American English нацелен очень жёстко на сокращение формальных моментов английского языка.
Потому что я уверена, что вы...
Вы же все интернет перепиской и в английском языке, наверное, пользовались, и пользуетесь, и встречались вот с такими вот сокращениями типа 4U (for you)?
Да, это воспринимается, как интернет-язык, да?
Но, тем не менее, это не британский вариант, это оттуда пошло (из American English).
Да?
Но это простой пример.
А вот такое написание thru (вместо through) в американской yellow press очень активно встречается, в официальных ещё пока нет.
Но при этом мы с вами прочитали и мы поймем, о чём это.
В британском варианте английского языка вы никогда это не встретите.
В американском встретите.
В контексте угадывается.
Да, но в американском варианте это, хоть пока ещё и не вошедший в норму, но тем не менее в определенных слоях этот вариант будет использоваться.
Ну, вы понимаете, что это совсем просторечие, а вот на уровне грамматики, should в американском варианте возможно опускать.
Почему?
Именно потому, что мы знаем, что должно быть ещё два показателя обязательно: союз that и наличие глаголов suggest и так далее.
Вот если вот эти условия выполняются, то тогда третье условие (should) можно опустить.
Это правило английского языка.
Мы с вами говорим о том, что английский язык живёт по правилам очень жёстко.
В данном случае понятно, что у нас здесь 66.6\% (это большая часть), т.е. $2/3$ правила выполняются, поэтому перевешивают и поэтому should можно опустить.

Но для нас с вами конечно American English это важно, но на самом деле во всех грамматических источниках, если вы откроете использование сослагательного наклонения в английском языке, то будет написано что первым случаем является научный или публицистический стиль.
А мы с вами как раз про Academic English и говорим.
Так вот, в Academic English, в учебниках, в научных статьях, написанных носителями английского языка, в этом случае вы увидите именно вот такой вариант, без should.
Если это будут британский английский разговорный, то should британцы (носители английского языка) не опустят, потому что для британской нормы (для British English) should здесь будет обязательным.
Мы с вами это должны понять.
Русский язык более креативен, более метафоричен, нежели английский.
Английский действительно сугубо аналитический язык.
Очень аналитический язык, поэтому здесь нужно откинуть вот эти вот \textit{так не выглядит}, и так далее (откинуть все другие субъективные взгляды).
Мы мыслим формулами.
Вот нужно просто заставить...
Вот для вас, извините, опять же, в хорошем смысле технарей, у которых ум аналитический, для вас это должно быть понятно.
Нужно просто отключить влияние русского языка в данном случае.
И тогда все встаёт на свои места.

Возвращаемся обратно.
Итак, первое предложение.
Из того, что у нас даны варианты, мы можем выбрать сюда только вариант (C) \textbf{be}.
И это тип придаточного -- дополнительный.
И предложение со структурой, когда в главном предложении используются определенные глаголы (suggested, proposed, insisted и так далее).
Двигаемся дальше.

Пожалуйста, кто второй?
(C)
Да, абсолютно верно.
Здесь правильный вариант (C).
Пожалуйста, всё предложение.
The judge insisted that the witness \textbf{tell} the truth.
Поскольку эти два предложения (первое и второе) похожи, insisted тоже относятся сюда же к этим глаголам.
То вот теперь мне очень важно, чтобы вы сами как бы про себя это проговорили.
Ну, раз вы второй, то вам это почётное право достаётся.
Переведите ваше предложение, пожалуйста.
\textit{Судья настоял на том, чтобы свидетель сказал правду.}
Я специально попросила вас теперь перевести, потому что мы с вами говорили, помните, когда на предыдущем занятии мы с вами чертили табличку, и вот эти второе и третье в условном наклонении, в условных предложениях, второй и третий тип.
Я вам обозначила, и вы это осознали, да, что показателем в русском языке для использования второго и третьего типа условных предложений будет что?
Наличие частицы БЫ.
И вот теперь еще раз переведите, пожалуйста, ваше предложение.
Громко, чтобы все слышали.
\textit{Судья настоял на том, чтобы свидетель сказал правду.}
Вы слышали, что прозвучало?
Чтобы.
Несмотря на то, что "<чтобы"> относится к союзу that, но тем не менее для нас, как для русскоговорящих, если в вашей голове союз that переводится как "<чтобы">, то тогда после него будет стоять аналитическая форма сослагательного наклонения Subjunctive II.
"<Чтобы"> присоединяет по сути дела по смыслу какие предложения?
На самом деле, мы должны даже их не как дополнительные рассматривать.
С точки зрения английского языка, они дополнительные, потому что с помощью союза that.
То, что союз that -- это союз что (это никто не отменял), но при определенных глаголах союз that, когда на русский переводится, то становится союзом "<чтобы">.
Это что-такое "<чтобы">?
Это цель.
Союз "<чтобы"> выражает цель.
Настаивал с какой целью?
Чтобы он сказал.
То есть у судьи была какая-то цель.

А если мы первое предложение теперь с вами переведем, то что у нас получается?
\textit{Генерал приказал, чтобы враг был остановлен.}
То есть, опять же, он отдал приказ с целью остановить врага.
Вот если в вашей голове при переводе, когда вы осознаёте, что вы предложение строите и вдруг у вас звучит союз "<чтобы">, то в английском языке вы должны использовать формы сослагательного наклонения.
Мы здесь пользуемся какими-то намёками какими-то подсказками русского языка, но в данном случае я для вас вот обращаю это внимание, здесь параллель нельзя проводить.
Потому что с точки зрения структуры предложения в английском языке -- это дополнительное предложение придаточное.
Но с точки зрения содержания или идеи, которое это предложение передает -- это цель.
То есть это некие мои пожелания на будущее.
Понимаете это?
То есть английский язык нам определенные формальные требования создает, по которым мы дальше должны выбирать ту или иную грамматическую форму.
Не просто бездумно это делать, но осознавать, в связи с чем это связано.

Двигаемся дальше.
Дальше будет немного другое.
Пожалуйста, следующее предложение.
Правильный вариант (A).
Прочитайте, пожалуйста, и мы с вами найдём и обсудим, что это такое.
Do you like to eat steak?
Certainly.
And I wish \textbf{I got} it more often.
Would you please translate it?
Ты любишь есть стейки?
Конечно! Я желаю их есть как можно чаще.
Если мы дословно будем переводить, значит, что мы можем с вами...
Давайте найдём, где у нас есть что-то подобное в таблице.
У кого таблица перед глазами?
Итак, это что?
Это у нас опять тот же пункт -- это дополнительное предложение, сказуемое выражено глаголом I wish.
И важно то, что в этих предложениях в английском языке союз that опускается.
То есть по логике он там есть, но его не пишут и не говорят.
Поэтому перевод: я хочу, чтобы я ел (чтобы была возможность есть).
Опять союз "<чтобы">.

И с глаголом I wish строится предложение, которое либо выражает размышления о настоящих событиях, которые вы хотели бы изменить, но не можете этого сделать, либо сожаление о прошлых действиях.
Если мы сожалеем о том, что мы сейчас не можем (по каким-то причинам не можем это делать), то тогда мы строим предложение I wish + Past Simple (или would).
А если мы сожалеем о том, что сделали, то тогда мы строим предложение с I wish + Past Perfect.
В нашем случае мы сожалеем...
Грубо говоря, в школе вас учили эти предложения переводить, как вот вы начали мне варианты давать, теперь мы приходим к тому, как мы по-русски передадим эту мысль, естественно, но нам нужно понять, какую мысль именно нужно передать.
То есть мы сожалеем о том, что мы не можем делать сейчас.
Жаль, что я не могу их есть всегда.
Жаль, что я их не ем.
То есть I wish -- это жаль.
На русский языке переводится.
Для чего мне важно, чтобы мы это перевели?
Для того, чтобы если в русском языке вы когда-нибудь построите предложение с жаль, и вам нужно будет своему собеседнику на английском языке его перевести, вы тут же вспомните, что если вам жаль, что вы что-то не делаете, то это I wish и дальше Past Simple.
Если же вам жаль, что я вчера (т.е. в прошлом) с тобой не пошёл, то тогда будет I wish I had gone with you yesterday.
То есть в этом случае I wish и Past Perfect.
То есть сожаление о том, что вы не сделали вчера.
В какой-то степени это сожаление примерно равноценно третьему типу условных предложений (разбирали их на прошлой паре).
Если бы у меня была возможность, я бы это сделал.
Но только там не звучит это жаль прямо очевидно.
А вот I wish -- это чёткое выражение сожаления о чём-то: или о настоящем, или о прошедшем.
Поэтому в этом случае, из того, что нам здесь дано, мы можем выбрать только форму I got, потому что после I wish можно либо Past Simple использовать, либо Past Perfect.
Больше ничего.
Have got -- нас не устраивает, как бы мы это have got не treat (не интерпретировали / не трактовали).
Потому что I've got a brother -- учили в школе, да, что у меня есть брат.
Хотя меня так не учили, я так не могу говорить.
Я все равно говорю I have a brother с глаголом to have без всяких I've got.
Это в любом случае не Past Simple и не Past Perfect.
Поэтому из всех четырех опций мы можем выбрать только одну.
I wish с Past Perfect -- это сожаление о том, что произошло.
Жаль, что я не пошёл.
Хотел бы я пойти, но не пошёл -- вот так можно ещё эти предложения строить/переводить.
Хотел бы я есть стейки каждый день, но не ем.

То есть откуда вот этот Subjunctive получается?
Именно поскольку мы выражаем свои желания и мысли, а не отношение.
А что у нас выражает отношение к действию?
Отношение к действию выражает что?
Я же столько раз вам это говорила -- модальные глаголы.
Модальные глаголы выражают отношение к действию.
Я хочу есть.
Я должен играть.
Модальные глаголы выражают отношение.
А Subjunctive Mood выражает сами желаемые действия.
Хотел бы я поесть.
Ну то есть вот формула -- само действие -- я бы их ел всю жизнь.

Так, четвертое, пожалуйста, какой у вас вариант?
Вариант (C).
I am hungry.
I \textbf{would have fixed} breakfast for you if I had known.
Абсолютно верно.
Да, это третий тип Conditional.
Как вы переведёте?
\textit{Я голоден. Я бы приготовил тебе завтрак, если бы знал.}
А почему здесь fix?
Просто выражение такое to fix breakfast -- ну это такая разговорная фраза приготовить.
Потому что to prepare breakfast нельзя сказать.
To cook breakfast тоже не по-английски.
А вот это разговорное выражение как раз приготовить завтрак.
Да, ну, собрать все вместе.
А если не cook, то как?
To make, например.
А как будет завтракать?
To have breakfast.
Обратите внимание breakfast здесь без неопределённого артикля "<a">.
Также обратите внимание, что когда мы говорим про эту фразу, то здесь у нас глагол to have, и это специфическое использование глагола to have.
Вы наверняка это знаете, что в таких случаях мы про глагол to have знаем.
To have breakfast, to have dinner, to have break, for example.
Что с глаголом to have в этом случае происходит?
В отличие I have a brother, I have a car.
С точки зрения грамматики в случаях to have breakfast, to have dinner, to have break и так далее глагол have может в таких случаях использоваться в аспекте Continuous, потому что в этом случае глагол to have не имеет своего основного значения иметь (possession).
To have breakfast -- это завтракать.
Это не to have \textbf{a} breakfast in my bag (a doggy bag).
В этом случае (когда есть неопределённый артикль "<a">) я действительно свой завтрак в сумочке принесла.
А to have breakfast (без неопределённого артикля) -- это устойчивая такая фраза.

Можно вопрос?
С простудой тоже так?
С простудой нет устойчивого выражения и поэтому будет с артиклем I have a cold или I have a flu.
I'm having a flu at the moment, sorry, I'm absent.

Можно сказать I have a breakfast, но в этом случае англичанин поймёт, что вы с собой принесли завтрак в контейнере, и он у вас где-то там в сумке лежит.
I have my breakfast with me.
Значит, что сейчас достану и буду завтракать.
А если I have breakfast every day at 7 o'clock, то это значит, что это действие завтракания, а ничего не имеющего отношения к possession (владению, обладанию, имуществу).
Понятно?
И в этом случае как раз вот этот глагол to have точно так же, как и глагол to be.
He is being polite now, because he had a very serious conversation with his boss.
He is being polite -- это не постоянная характеристика, а это его состояние в настоящий момент.
И глагол to be здесь имеет немного другое значение (не совсем параллельно -- тут не происходит такого прямо сильного изменения значения, как с глаголом to have), но тем не менее разница очевидна.
Или это сейчас ваша, ну, как это сказать, непостоянная характеристика, тогда глагол to be может использоваться в аспекте Continuous.
Если это быть/являться постоянно, I am a student, то to be a student нельзя поставить в Continuous, потому что это некая перманентная характеристика на какой-то определенный период времени.
Это такие, в общем-то, не тонкости.
Это обсуждается и в бакалавриате обычно поднимается, но я так думаю, что вы просто это забыли.
Но это очень важные моменты, которые тоже нужно...
Ну вот интересные мне кажется...

Давайте тогда дальше пройдём и будем останавливаться только на интересных моментах.
Пятое предложение, пожалуйста, что-то мы уже обсуждали, поэтому можем просто констатировать.
Правильный ответ (B).
И это вы должны со школы помнить.
I wish I were you.
Но это подсказка со школы.
Это стандартная фраза.
После глагола I wish глагол to be будет использоваться только в форме were.
Почему?
Даже если he?
Да, даже если he, то всё равно форма were.
Это будет относиться ко всем формам.
Поэтому читайте предложение!
I wish that my sick friend \textbf{were} out of danger.
Я бы хотел, чтобы мой больной друг был вне опасности.
Или же можно перевести так: Жаль, что мой друг в опасности.
That можно опустить -- жаль, что мой друг в опасности, а я бы хотела, чтобы он был вне опасности.
Ну, то есть "<не"> появится тогда дополнительное, которого в английском предложении нет.
Нужно понимать смысл, а основной смысл предложения с I wish -- это жаль.
Я желаю, чтобы он был не в опасности, то есть это значит, что я сожалению, что он сейчас в опасности.
Да!
Значит сейчас жаль, что он в опасности и вы это осознаёте.

Пожалуйста, шестой.
Вариант (B).
Пожалуйста, читайте всё предложение.
Would you take a trip round the world if you \textbf{had} more money?
Что это за предложение?
Это Conditional.
Какой?
Второй тип Conditional (обсуждали на прошлой паре).
Просто здесь у нас вопрос, да?
Чтобы вас не сбил с толку вопрос.
По сути дела, если мы уберём вопросительный знак, то у нас получится You would take a trip if you had more money.
Просто здесь вопрос.

Так, пожалуйста, седьмое.
Вариант (D).
Да, читайте.
I wish I \textbf{had gone} with you last night.
Почему здесь будет have gone?
Потому у вас там есть что?
Потому что о прошлом предложение.
Английский -- это формульный язык.
Last night нас сразу же относит к прошлому.
Из контекста можно тоже понять было бы, если у вас нет в самом предложении отсылки ко времени.
Но здесь, поскольку это тестовые задания, всегда есть какая-то лакмусовая бумажка, которая вас сразу же должна навести на мысль.
Увидели last night?
Всё.
Поэтому после I wish выбираете форму Past Perfect.

Можно вопрос?
Мне просто казалось, что сейчас в современном языке have gone / had gone говорят, когда человек умер.
Это не так?

Во-первых, когда вы хотите сказать, что человек сыграл в ящик, то вы будете использовать Past Simple.
He passed away чаще всего.
Present или Past Perfect в этом случае не будет.
Что означает passed away дословно?
Это сыграл в ящик.
Как называются такие выражения, когда мы используем другие слова вместо понятий, которые каким-то образом неприятны нам или вызывают отрицательные эмоции?
Эвфемизмы.
Да, и в рассматриваемом случае используется passed away вместо глагола died (в прошедшем времени).
А здесь в рассматриваемом примере (в седьмом предложении теста) просто Past Perfect, потому что last night.

Пожалуйста, дальше.
Восьмое предложение.
Вариант (C).
Да.
Замечательно!
Я вас попрошу перевести, потому что с lest мы ещё пока не встречались.
Я просто на этот союз обращаю ваше внимание.
Переведите, пожалуйста.
Тебе следует носить плотную одежду зимой, чтобы не простудиться.
Да, чтобы не простудиться или чтобы ты не простудился.
Потому что после lest, где у нас там этот lest в таблице?
Обстоятельственные цели требуют после себя (вот союз lest в данном случае) использование should.
И это should не переводится как "<должен"> (или как "<следует">), а это Subjunctive II, где используется should.
Здесь в этом предложении вообще нет никакой модальности.
Это просто показатель того, что здесь мы используем сослагательное отклонение, как, грубо говоря, пожелание некоторое на будущее, чтобы не простудился.
У нас опять появилось чтобы.
Но что важно по поводу союза lest?
Вы перевели его правильно.
Если вы используете союз lest, то в своем значении и при переводе русский язык нам сразу даёт это понятие.
Он несёт в себе то, чтобы что-то не произошло.
Чтобы не -- так будет переводиться этот союз.
Lest -- это чтобы не.
Отрицания вы не ставите.
Никакого not в придаточном у вас не появится, но в голове вы должны понимать, что вы как бы отрицаете то, что вы выражаете глаголом.
Если вы вдруг туда поставите not (какой-нибудь lest you should not come), то тогда вы отрицаете, чтобы он не пришёл (т.е. чтобы ты пришёл это получится -- минус на минус даёт плюс).
Я там что-то сделаю, чтобы ты пришел.
Да?
То есть сразу иметь в голове вот эту вот мысль, что отрицание заложено в союз lest, и, соответственно, отрицание следующего после него события.

Так, двигаемся дальше.
Девятое предложение, пожалуйста.
Вариант (A).
Прочитайте, пожалуйста, и мы переведём и разберём.
I \textbf{would have been} glad to show you around the city, had you come here.
Переведите, пожалуйста.
Я буду рад показать тебе окрестности города, если ты приедешь.
Смотрите, у вас есть would have been и had come.
Это что такое?
Что это, во-первых, за предложение?
Это третий тип conditional, где в главном предложении мы используем would и have, а в придаточном у нас что используется?
Past Perfect, потому что had come, но сразу же забегая вперед, потому что мы все равно с вами немножко поговорим об этом.
Найдите мне, пожалуйста, аналогичные предложения.
Какие еще предложения будут вот этого же образца?
Вот у нас had you come.
Еще в каких предложениях вы то же самое имеете?
Назовите по номерам.
В тринадцатом.
В тринадцатом абсолютно верно.
Там вы должны были выбрать вариант (D).
В тринадцатом правильный ответ.
В четырнадцатом у нас сразу же это дано.
Абсолютно верно.
Had we known the outcome of the experiment
И в семнадцатом тоже.
Что это такое?
Кто нашёл, или кто знает?
В нашей таблице нет этого, вот конкретно просто вот этого случая.
Это частный случай чего?
Вот то, что представлено в предложении 9.

Had you come here. Had it not been. Had we known the outcome. Had the company's clients paid in due time.
Что здесь такое происходит в этих предложениях?
То есть, что это такое, если нет if?
Как называются такие предложения?
Что такое if?
Это союз.
А если if нет, то это бессоюзное предложение.
В английском языке, как вы видите, такое может быть, бессоюзное предложение.
Но если союз выпадает, то что происходит?
Появляется частичная инверсия, когда вспомогательный глагол had занимает место перед подлежащим.
Чаще всего вы с такими предложениями столкнетесь при формальном стиле общения.
То есть informal style редко использует это, несмотря на то, что у нас ваше предложение это (в общем-то) informal style.
А вот если дальше мы с вами посмотрим 13, 14, 17, то вы видите, что эти все предложения взяты однозначно из научных источников.
И вот для научных источников это является нормальным случаем использования.
Опускание союза if, но в английском языке пустоты быть не может, поэтому эта пустота тут же занимается вспомогательным глаголом.
Если, по сути дела, мы можем восстановить это предложение, у нас получится что?
I would have been glad to show you around the city if you had come here.
В тринадцатом предложении: If it had not been for the invention of new types of chips, the market for the computers could have been smaller.
В четырнадцатом точно так же.

Можно вопрос?
А для чего это делается?
С какой целью?
Это, ну вот, в английском языке это как некие, на самом деле, показатели стиля.
Я ещё раз подчёркиваю: вот девятое предложение -- это редкое исключение, когда в informal English (в неформальном английском) происходит эта инверсия, а для научного стиля это просто требование (это черта характера, если хотите сказать; мы назовём это так).
Можно сказать, что это экономия печатных знаков, хотя там, конечно, два всего, но тем не менее.
Но вот это именно такая черта.
Я в начале подумал, что может быть это сделано для того, чтобы акцентировать внимание (подобно другим случаям инверсий).
Ну можно сказать, но тем не менее, видите, вот в informal style это так.
Можно сказать, что это как эмфатическая такая конструкция, да, можно таким образом объяснить, но важный момент заключается в том, что в подавляющем большинстве случаев мы такие предложения будем встречать именно в научном, в академическом стиле.
В публицистике мы этого не встретим, а вот в академических текстах, в статьях, монографиях вот это будет там присутствовать.
Соответственно, если вам когда-то нужно будет в своей статье использовать третий тип условного предложения, то не забудьте вот об этом, покажите своё знание вот этого стиля общения.
Потому что мы с вами говорили на предыдущем занятии о том, когда, где и какие типы сложных предложений мы можем использовать.
Вот это вот один из признаков академического стиля.
А это именно для третьего типа, да?
Да, это для третьего.
Может быть и во втором, но просто мечты в Academic English редко.
Ну, гипотезы могут быть, но всё-таки тем не менее.

Можно ещё вопрос?
Да.
Вопрос по второй части. Как это перевести?
Если ты придешь?
Да.
И помните, что перевод и то, что в английском языке формально мы имеем, никак не связаны.
Условное предложение оно всё равно условное, а условные предложения обозначаются союзом if.
Просто здесь союз if убран (опущен), и инверсия выполняет его функцию.
Инверсия.

Так, сейчас я посмотрю, что мне еще интересно было бы.
Поскольку у вас эти бумажки останутся, то частично мы еще с вами посмотрим на следующих занятиях.
Значит, сейчас тогда вам домашнее задание по вот этому тесту.
Найдите, пожалуйста, каким образом вы мне объясните 18 предложение?
Какой там правильный ответ?
Правильный ответ (B).
Так, ну с 16 все понятно собственно говоря, но остальные мы с вами обсудим.
Но для предложения 18 найдите мне, пожалуйста, почему здесь будет использоваться форма B?

\newpage
\sublinksection{International and Russian programs of support of young scientists}

So what I would like us to do now is to switch with grammar because we will finish it (that will be somehow your home task).
Okay, so, and now I would like you to share your experience and of course the results of the procedure of your participation in different procedures (for applying for different grants and scholarships).
But my first question is, so if you're asked to speak about this topic, to speak on this topic during your examination, what will be the entrance sentence?
What will be the introductory sentence that you would begin this topic with?
Give your variants.
Because it is easy, actually, to start speaking about your grants but how would you generalise and open this topic.
That would be interesting.

General opinion about grants and scholarships.
If they are important or not.

I would like you to present me the sentence.
Вот как?
Чем вы начнёте впечатлять экзаменатора?
Вот с чего вы начнёте?
Потому что я ведь это вам не задала, вы готовили конкретные какие-то свои участия в программах поддержки учёных.
Какое бы первое предложение для вас было в данном случае важно?
Кто бы чем начал эту тему?

Ещё раз повторите, пожалуйста, какая тема?
Grants and scholarships, or the programs of support of young scientists.
Программа поддержки молодых ученых, это седьмая тема в программе экзамена, да?
Это я просила подумать.
Вот с чего вы начнёте?
Как бы вы начали?
Наливайте воду.
Grants and scholarships are very important for development of modern science, because with grants and scholarships young scientists can use more time to conduct new research.
Хорошо.
Ещё какие варианты?
Как бы вы начали?
I would like to tell you about scholarships.
А вот вы про scholarships.
Вот потому что здесь в прошлом ответ была вода в целом.
А вот вы, например, можете сказать.
I would like to tell you about my participation and applying for the scholarships and grants.
Ещё какие-то варианты?
Что ещё?
Как бы ещё начали?
Потому что начать ведь тоже нужно правильно.
Какие ещё?
Yes, no? Ну, видите, два аспекта мы уже поняли.
Либо вообще, так сказать, абстрагироваться, либо начать прямо с конкретного.
Что-нибудь еще?

Grant programs are the most common type of support for young scientists.
It is so important and it could give some basement for кто-там и так далее.
Я так и буду говорить.
Так не надо говорить.

Okay, who would like to give some information about your personal participation in different programs?
Let's first listen to those who were active enough and prepared the information.
So who would like to begin?
Who is ready?
Timur, are you ready?
I didn't have such experience, so I prepared some information about my prospects.
You remember that I asked you to think about two vectors.
About your past experience and about your future desires.
Now you emphasised about the past, that's why I thought that...
It's okay.
So let's listen to each other.
Maybe you will fix some information for your topic.
So, you are welcome.
Okay.
Scholarships and grants programs are very important part of young scientists' life.
And unfortunately, it's really pity that I hadn't take part in any of them previously.
But I hope that I have some chance.
So I found some information about the original contest funded by the Russian Scientific Foundation.
It's called Fundamental Scientific Research, conducted by small groups.
I should emphasise that it's not for single individuals, but for small scientific groups of two or even four scientists, young scientists, because one of the main conditions to participate, to try participating there is that not less than half of this group, scientific group, should be younger than 39 years old.
There is a certain list of...
Requirements?
Yes, requirements, but I wanted to tell about...
Oh, okay.
There is a certain list of fields of knowledge, including mathematics, physics, engineering, chemistry, material science, and so on.
And the price, the sum, the value of the grant is about 1.5 million rubles.
Okay, thank you.

Is there anyone else who has also thought about this program of support or contest?
Yes, no?
I just wanted someone to continue because it is really one of the most popular and actually this very period is the period of applying for this grant program because that would be nice when postgraduate students do participate in it because it's really what raises the value of your research.
Okay, so what about the others?
What about the prospective grants and programs that you would like to speak about?
Is there anything else that you would like to participate?
Have you thought about it?
No?
Okay.
So, Timur, thank you very much.
So maybe your colleagues will take it into account when preparing the written form of this topic for the next class.

So it means that you have prepared your past experience.
So who would like to say a few words about your experience that you actually have from your master's or bachelor's past when you have applied for any contests?
Что, никто не сделал?
Опять что ли сюда мне обращаться?
Что с остальными?
Have you participated in any of them?
Okay, so then I would like you just at the moment to name those contests that you know and just to give the main idea of it.
So, what do you know?
What contests do exist?

I haven't any experience of winning grants but I have participated in contest for young scientists of Committee of Science and Higher Education of Saint-Petersburg.
And I failed it, of course.
Never mind, because...
So, okay.
That's all.
So, and what were the requirements?
This contest is for young scientists or postgraduate students.
The value of this grant is one hundred thousand rubles (если говорим единицу, в которой измеряется, то окончание -s к цифре не добавляется) for scientists and fifty thousand rubles for postgraduate students.
There are some areas of research and your research should be in.
And also as it was already told young scientists are under 39 or 45 years old and there were some documents that should be prepared.
Good, thank you.

Yes, well actually, you shouldn't stop, all of you, even if you fail, because it is like a sort of (what is it) a skill that should be developed.
And once you win, that will be easy and you will be taken into the list of those who have the chance to win and after that, that will play for you, not against.
It means that you should participate.

Any other contest?
So, the Committee of Science and Higher Education.
Anything else?
Maybe some specialised contest, especially and specially for your field of science or field of research, is there anything?
Okay, any foreign funds and scholarships?
Have you participated in them maybe in your past?
No?
Okay, so that it means that actually being post-graduate students, you should think, well, if you didn't participate in any contest in your past, so then speak about your future and your ideas, maybe some abstract ideas, as I have said, so we should participate, we should try, and et cetera, et cetera.
Because in any case you should have a number of sentences to say during your examination.

Если такая тема попалась на экзамене, то я обязан говорить только о ИНОСТРАННЫХ программах поддержки молодых учёных?
Нет, НЕ ОБЯЗАН.
Я вам еще раз говорю, что самое главное требование на экзамене (на кандидатском), чтобы вы говорили.
Молчать нельзя.
Поэтому я и говорю воду лить.
Вы просто можете сказать, что unfortunately nowadays we do not have any opportunity to participate in any international events.
То есть ваша задача -- говорить.
Поэтому попробуйте написать.
То есть у вас нет информации, но все равно что-то сказать надо.
Поэтому подумайте, что, еще раз подчеркиваю, никто не будет проверять корректность, правильность, 100\% верифицировать, что вы говорите.
Вы можете нафантазировать, что вы в "<Умнике"> участвовали, когда вы в магистратуре учились.
Пожалуйста.

Итак.
Значит, по грамматике 18-е предложение -- это такой challenge.
Объясните, почему это правильное.
Значит дальше в курсе я открою pdf-файл с этим тестом с правильными ответами.
Он у меня там есть, поэтому вы сможете зайти и просто себя проверить.
Мы дальше всё равно некоторые предложения отсюда с вами обсудим, чтобы обратить на них внимание.
Вот это предложение 18 специфическое.
У меня вопрос почему там Subjunctive Mood используется?

Значит, на следующий раз письменная тема International and Russian Programs of Support of Young Scientists.
Напишите, пожалуйста.

И далее на следующей паре мы возьмем тему, которая у нас там, по-моему, про будущее, да?
Что у нас там?
Мы уже разобрали с вами мои научные интересы, перспективы научной карьеры, университеты как научные центры, этические проблемы современной науки.
Поэтому наука в исторической перспективе и современное состояние науки в моей области знаний/исследований (это четвертая и пятая темы программы кандидатского экзамена соответственно) -- и я бы хотела, чтобы вы подумали об этом в комплексе.
И составить один топик сразу по обеим темам.
Потому что точно так же, как и grant programs -- это и прошлое, и будущее.
В данном случае эти две темы тоже между собой связаны.
Поэтому я, чтобы вас сильно и не напрягать, сразу могу сказать, что по сути дела это одна тема, которую просто можно составить и просто поменять потом местами в зависимости от того, что вас спросят на экзамене.
Поэтому вы в любом случае можете сказать, что это древняя область исследований, ей занимались Аристотель и не знаю кто ещё, и дальше перейти...
Или же наоборот в современное время, подумайте.

Пока что не письменно, а просто основных исследователей.
Так как историческая перспектива, то значит кто там были какие-то законотворцы, нормотворцы, ...
А если современное состояние, то по сути дела это как Universities as a Science Centres просто ещё раз вспомнить, какие ведущие направления и ведущие исследования.
То есть точно так же послушаем друг друга.
Это в следующий раз.
Спасибо большое!


\end{document}
