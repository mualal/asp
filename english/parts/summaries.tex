\documentclass[main.tex]{subfiles}

\begin{document}

\setcounter{secnumdepth}{0}


\setcounter{section}{103}
\linksectionold{Summaries}


\setcounter{subsection}{1}
\sublinksectionold{Multiscale Molecular Dynamics Model for Heterogeneous Charged Systems}
As the title implies the article describes multiscale molecular dynamics model for heterogeneous charged systems.
The author proposed a multiscale approach which enables simulations on the order of micron length scales and 10's of picosecond timescales.
This approach exceeds other OFDFT-MD simulations by many orders of magnitude.\par
This article highlights that modelling matter across large length scales and timescales poses significant challenges.
The author points out that it is essential to match microscale, mesoscale, and macroscale results.
The superposition of these results should be used for further analysis.\par
The article concludes with usage of proposed method to study the heterogeneous, nonequlibrium dynamics of an inertial-confinement-fusion capsule containing a plastic ablator and deuterium-tritium ice.
The authors shows a newly founded features such as strong hydrogen jetting from the plastic into the fuel region.\par
This article is significant for inertial-confinement-fusion professionals and numerical methods specialists.
\ \\\\
\noindent{\textit{(Multiscale Molecular Dynamics Model for Heterogeneous Charged Systems, 2018)}}
\newpage


\setcounter{subsection}{2}
\sublinksectionold{Thermal transport in fullerene-based molecular junctions}
As the title implies the article describes thermal transport in fullerene-based molecular junctions.
The author concentrates on phononic thermal transport in single-molecule gold-fullerene-gold nanojunctions.
The thermal conductance of monomer, dimer and trimer fullerene molecules sandwiched between metals was computed.
In this paper three distinct MD methods for the simulation of thermal transport were used: the approach to equilibrium MD, the Langevin nonequilibrium MD, and the reversed nonequilibrium MD.\par
This article highlights that the thermal conductance decreased monotonically with the number of fullerene units.
The author draws our attention to the fact that the thermal conductance of trimer junctions was five times smaller than that of monomer junctions, and four times smaller than dimer junctions.\par
Also in this paper the author has succeeded in showing the consistency of different classical MD tools in studying thermal transport properties of nanojunctions.
The article concludes with offering perspective for future work.
For example, the results should depend on the type of the contact.
It is interesting what would be if we changed gold to graphene electrodes.\par
This article is significant for designers of high-performing thermoelectric junctions, where low thermal conductance is desired.
\ \\\\
\noindent{\textit{(Thermal transport in fullerene-based molecular junctions: molecular dynamics simulations, 2024)}}
\newpage


\setcounter{subsection}{3}
\sublinksectionold{Thermal wave propagation in graphene}
As the title implies the article describes wave propagation in graphene.
The author concentrates on the transient heat conduction process in armchair graphene ribbons pulsed by local heating with a duration of 1 ps.
In this paper nonequilibrium molecular dynamics simulations are used.\par
This article highlights two types of heat transfer: a sound wave (also called first sound), which has macroscopic momentum and propagates at the speed of sound, and a thermal wave (also called second sound) whose propagation speed is less than sound speed.
The author points out that the speed of the second wave does not change with the duration of the initial heat pulse.\par
The article concludes with the comparison of armchair graphene and zigzag graphene. The author shows that the higher proportion of ballistic transport in zigzag graphene will lead to stronger heat waves.\par
This article is significant for microelectronics professionals and researches, because graphene has extraordinary electrical and thermal properties.
\ \\\\
\noindent{\textit{(Thermal wave propagation in graphene studied by molecular dynamics simulations, 2014)}}
\newpage


\setcounter{subsection}{4}
\sublinksectionold{Front roughening of flames in discrete media}
As the title implies the article describes front roughening of flames in discrete media.
The author concentrates on the morphology of flame fronts propagating in systems composed of pointlike sources.
In this paper the superposition of the Green's function for the diffusion equation is used in order to eliminate the need to use finite-difference approximations.\par
This article highlights that propagation of fronts in a number of seemingly disparate systems (for example, bacterial colony spreading across a petri dish or polymerization fronts) are described by a single class of equations.
And the author points out that even diversified stock portfolio optimization is described by these equations, where stock capital is treated like front roughness.\par
The article concludes with a map which summarises simulation results categorized according to the flame front roughening behaviour.
The author shows a transition to classical front propagation behaviour.\par
Results proposed in this article can be used as a guide in future experimental studies.
\ \\\\
\noindent{\textit{(Front roughening of flames in discrete media, 2017)}}
\newpage


\setcounter{subsection}{5}
\sublinksectionold{A framework for solving atomistic phonon-structure scattering problems}
As the title implies the article describes a numerical approach to the solution of atomistic phonon-structure scattering problems.
In this paper a frequency-domain decomposition of the atomistic equations of motion and perfectly matched layer boundaries are used.
The accuracy of the method is demonstrated on connected monoatomic chains, for which an analyical solution is known.\par
The article highlights that the parameters defining the perfectly matched layer affect the performance of the numerical method.
And the author gives guidelines for selecting optimal parameters.\par
The extension of the method to multiple dimensions also is demonstrated by the example of 2-dimensional nanocylinder.
I conclusion the author shows that the calculations match continuum theory for long-wavelength phonons and large cylinder radii, but otherwise demonstrate complex physics associated with discreteness of the lattice.\par
This article is significant for microelectronics professionals, researches, and numerical methods specialists.
And the proposed numerical method is the object of active discussions between specialists.
\ \\\\
\noindent{\textit{(A framework for solving atomistic phonon-structure scattering problems in the frequency domain using perfectly matched layer boundaries, 2015)}}
\newpage


\setcounter{subsection}{6}
\sublinksectionold{Dramatic enhancement of interfacial thermal transport}
As the title implies the article describes a method for dramatic enhancement of interfacial thermal transport.
The author concentrates on the phonon transport across an interface between two semi-infinite leads.
The interface is modelled as masses connected by harmonic springs.
In this paper the non-equilibrium Green's function method is used.\par
The article highlights that the couplers with both geometric graded mass and geometric graded coupling can significantly improve the interfacial thermal conductance.
And the author points out that the acoustic impedance mismatch and the cutoff frequency mismatch of lead materials dramatically affect the choice of the optimized interfacial couplers.\par
Also in this paper the author has succeeded in showing the dependence of interfacial thermal conductance on type of interfacial coupler.
Six different kinds of interfacial couplers are studied.\par
This article is significant for microelectronics professionals and researches.
And author's findings may offer guidance for advanced thermal interface materials design.
\ \\\\
\noindent{\textit{(Dramatic enhancement of interfacial thermal transport by mass-graded and coupling-graded materials, 2020)}}
\newpage


\setcounter{subsection}{7}
\sublinksectionold{Ballistic and diffusive thermal conductivity of graphene}
As the title implies the article describes ballistic and diffusive thermal conductivity of graphene.
The author concentrates on thermal conductivity for diffusive and ballistic thermal transport of phonons, because thermal conductivity of graphene occurs mainly by phonons (not by free electrons).
In this paper analytical expressions are used.\par
This article highlights that the diffusive thermal conductivity increases when the mean free path of the phonon (MFP) increases.
And the author points out that the ballistic thermal conductivity monotonically decreases to zero with decreasing temperature.\par
The article concludes with figures of thermal conductivity as a function of temperature for several carbon isotope concentrations and several sample sizes.
The author shows that the sample without isotope impurity has much larger maximum thermal conductivity.
And at 300K sample length doesn't influence on thermal conductivity if sample length is greater than 2 microns.\par
This article is significant for microelectronics professionals and researches, because graphene has extraordinary electrical and thermal properties and could be useful for applying thermal conducting devices.
\ \\\\
\noindent{\textit{(Ballistic and diffusive thermal conductivity of graphene, 2018)}}
\newpage


\end{document}
