\documentclass[main.tex]{subfiles}

\begin{document}

\setcounter{secnumdepth}{0}


\setcounter{section}{103}
\linksectionold{Summaries}


\setcounter{subsection}{1}
\sublinksectionold{Multiscale Molecular Dynamics Model for Heterogeneous Charged Systems}

\ \\\\
\noindent{\textit{(Multiscale Molecular Dynamics Model for Heterogeneous Charged Systems, 2018)}}
\newpage


\setcounter{subsection}{2}
\sublinksectionold{Thermal transport in fullerene-based molecular junctions}
As the title implies the article describes thermal transport in fullerene-based molecular junctions.
The author concentrates on phononic thermal transport in single-molecule gold-fullerene-gold nanojunctions.
The thermal conductance of monomer, dimer and trimer fullerene molecules sandwiched between metals was computed.
In this paper three distinct MD methods for the simulation of thermal transport were used: the approach to equilibrium MD, the Langevin nonequilibrium MD, and the reversed nonequilibrium MD.\par
This article highlights that the thermal conductance decreased monotonically with the number of fullerene units.
The author draws our attention to the fact that the thermal conductance of trimer junctions was five times smaller than that of monomer junctions, and four times smaller than dimer junctions.\par
Also in this paper the author has succeeded in showing the consistency of different classical MD tools in studying thermal transport properties of nanojunctions.
The article concludes with offering perspective for future work.
For example, the results should depend on the type of the contact.
It is interesting what would be if we changed gold to graphene electrodes.\par
This article is significant for designers of high-performing thermoelectric junctions, where low thermal conductance is desired.
\ \\\\
\noindent{\textit{(Thermal transport in fullerene-based molecular junctions: molecular dynamics simulations, 2024)}}
\newpage


\setcounter{subsection}{3}
\sublinksectionold{Thermal wave propagation in graphene}
As the title implies the article describes wave propagation in graphene.
The author concentrates on the transient heat conduction process in armchair graphene ribbons pulsed by local heating with a duration of 1 ps.
In this paper nonequilibrium molecular dynamics simulations were used.\par
This article highlights two types of heat transfer: a sound wave (also called first sound), which has macroscopic momentum and propagates at the speed of sound, and a thermal wave (also called second sound) whose propagation speed is less than sound speed.
The author points out that the speed of the second wave does not change with the duration of the initial heat pulse.\par
The article concludes with the comparison of armchair graphene and zigzag graphene. The author shows that the higher proportion of ballistic transport in zigzag graphene will lead to stronger heat waves.\par
This article is significant for microelectronics professionals and researches, because graphene has extraordinary electrical and thermal properties.
\ \\\\
\noindent{\textit{(Thermal wave propagation in graphene studied by molecular dynamics simulations, 2014)}}
\newpage


\setcounter{subsection}{4}
\sublinksectionold{Front roughening of flames in discrete media}

\ \\\\
\noindent{\textit{(Front roughening of flames in discrete media, 2017)}}
\newpage


\setcounter{subsection}{5}
\sublinksectionold{A framework for solving atomistic phonon-structure scattering problems}

\ \\\\
\noindent{\textit{(A framework for solving atomistic phonon-structure scattering problems in the frequency domain using perfectly matched layer boundaries, 2015)}}
\newpage


\setcounter{subsection}{6}
\sublinksectionold{Dramatic enhancement of interfacial thermal transport}
As the title implies the article describes a method for dramatic enhancement of interfacial thermal transport.
The author concentrates on the phonon transport across an interface between two semi-infinite leads.
The interface is modelled as masses connected by harmonic springs.
In this paper the non-equilibrium Green's function method was used.\par
The article highlights the superiority of interface material combining with graded spring stiffnesses and graded masses in the ITC enhancement.
\ \\\\
\noindent{\textit{(Dramatic enhancement of interfacial thermal transport by mass-graded and coupling-graded materials, 2020)}}
\newpage


\setcounter{subsection}{7}
\sublinksectionold{Ballistic and diffusive thermal conductivity of graphene}
As the title implies the article describes ballistic and diffusive thermal conductivity of graphene.
The author concentrates on thermal conductivity for diffusive and ballistic thermal transport of phonons, because thermal conductivity of graphene occurs mainly by phonons (not by free electrons).
In this paper analytical expressions were used.\par
This article highlights that the diffusive thermal conductivity increases when the mean free path of the phonon (MFP) increases.
And the author points out that the ballistic thermal conductivity monotonically decreases to zero with decreasing temperature.\par
The article concludes with figures of thermal conductivity as a function of temperature for several carbon isotope concentrations and several sample sizes.
The author shows that the sample without isotope impurity has much larger maximum thermal conductivity.
And at 300K sample length doesn't influence on thermal conductivity if sample length is greater than 2 microns.\par
This article is significant of microelectronics professionals and researches, because graphene has extraordinary electrical and thermal properties and could be useful for applying thermal conducting devices.
\ \\\\
\noindent{\textit{(Ballistic and diffusive thermal conductivity of graphene, 2018)}}
\newpage


\end{document}
