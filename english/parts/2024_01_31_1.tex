\documentclass[main.tex]{subfiles}

\begin{document}

\section{Лекция 31.01.2024 часть 1 (Смольская Н.Б.)}

\subsection{Греческие и латинские заимствования. Задания}

--- Выполнение заданий на греческие и латинские заимствования ---
\\

\textbf{Выполненное первое задание (соотнесите слова, содержащие префиксы латинского или греческого происхождения с их дефинициями)}

\begin{enumerate}[nosep]
	\item Retrospect -- a review of things that have already occurred
	\item Extraterrestrial -- from beyond the limits of the earth
	\item Hyperactive -- unable to remain stationary
	\item Irrational -- senseless or absurd
	\item Polyglot -- a multilingual person
	\item Dialogue -- a conversation between two or more parties
	\item Illustrative -- serving to explain or clarify
	\item Periscope -- a long tube with mirrors used to look over the top of \underline{smth}.
\end{enumerate}
\ 

\textbf{Выполненное второе задание (соотнесите слова греческого происхождения с их дефинициями)}

\begin{enumerate}[nosep]
	\item Chronograph -- a digital clock or watch
	\item Prognosis -- a forecast or prediction
	\item Cosmopolitan -- someone who has travelled a lot and feels at home in any part of the world
	\item Misanthrope -- an individual who hates or mistrusts everyone
	\item Anachronism -- something that seems to belong to the past, not the present
	\item Autonomous -- functioning independently of outside interference or control
	\item Sophisticated -- having a lot of experience of life, good judgement; not naive
	\item Technology -- the application of pure science to the handling of specific engineering problem
\end{enumerate}
\ 

\textbf{Выполненное третье задание (соотнесите корень латинского происхождения с его значением и его производным)}

\begin{enumerate}[nosep]
	\item Alter -- other -- alternative
	\item Annu, enni -- year -- millennium
	\item Manu -- hand -- manufacture
	\item Vers, vert -- turn -- convert
	\item Vid -- see -- evident
\end{enumerate}
\ 

\subsection{Словообразование. Задание}

--- Выполнение задания на словообразование ---
\\

\textbf{Выполненное задание (заполните пробелы в тексте словами в подходящей по контексту форме, используя слова, предложенные справа для каждой строки соответственно. Вариант (0) указан как пример)}
\\

\textbf{Home-workers.}

According to government research, more people are working from home than ever before.
(0) \textbf{Consequently}, there has been an increase in (1) \textbf{loneliness} among those people who no longer have to travel to their place of (2) \textbf{employment}.
Office workers spend their day (3) \textbf{being surrounded} by friends and colleagues, while home-workers (4) \textbf{rarely} meet anyone face to face.
The most direct means of (5) \textbf{communication} a home-worker has with the world (6) \textbf{outside} is the telephone.
The fax and the internet are two more (7) \textbf{technological} links that can be used, although they still rely on the written, rather than the (8) \textbf{spoken} word.
What a home-workers really wants is the (9) \textbf{warmth} of a human voice, not the (10) \textbf{digital} bleeps of a computer.
\\

\subsection{Грамматика. Пассивный залог. Задание}

--- Выполнение задания на грамматику (пассивный залог) ---
\\

\textbf{Выполненное задание (заполните пропуски соответствующей формой глагола в страдательном залоге)}

\begin{enumerate}[nosep]
	\item You \underline{will may be offered} (may, offer) a bonus in December if sales are high.
	\item All employees \underline{were forbidden} (forbid) to talk to the press, but many did so.
	\item These samples will have \underline{been tested} (test) before a certificate is issued.
	\item The goods will \underline{be sent} (send) by rail.
	\item The missing file \underline{has} just \underline{been found} (find).
	\item Plastic bottles \underline{should be taken} (should, take) to the local recycling centre.
	\item A new item \underline{has} recently \underline{been added} (add) to the product range.
	\item Green-marketing strategies \underline{are being developed} (develop) by many companies now.
	\item The portrait is known \underline{to be painted} (to paint) by an Italian.
	\item It \underline{is believed} (believe) that a horse shoe brings luck.
	\item Some people \underline{are} easily \underline{influenced} (influence) by other people's opinions.
	\item Electrons, \underline{are} negatively \underline{charged} (charge), usually orbit as close as possible to the positively charged nucleus.
	\item The proposed legislation \underline{is thought} (think) to be unworkable at present.
	\item The research \underline{was being carried out} (carry out) over a period of six months.
\end{enumerate}

\end{document}
