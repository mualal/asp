\documentclass[main.tex]{subfiles}

\begin{document}

\linksection{Лекция 10.04.2024 (Смольская Н.Б.)}


\sublinksection{Поиск ошибки в предложениях}

\hypertarget{ltask:2024-04-10-1}{--- Выполнение задания ---} (\hyperref[task:2024-04-03-2]{\color{blue}{перейти к тексту задания}})


\newpage
\sublinksection{Перевод с английского на русский}

\hypertarget{ltask:2024-04-10-2}{--- Выполнение задания ---} (\hyperref[task:2024-04-10-2]{\color{blue}{перейти к тексту задания}})
\\

\textbfind{Выполненное задание (перевод предложений на русский язык)}
\vspace{5pt}
\begin{itemize}[nosep, leftmargin=*]
	\itemsep10pt
	\item Science is always on the move. Its preference is to find a question that nobody knew needed answering, answer it and move on, leaving technologists to turn the answer into a machine, a drug or a computer program.
	\item Decisions taken in the 1950s led to the bulk of science funding going into medicine, physical sciences such as nuclear power, and space research.
	\item Science requires the formulation of laws based upon experimental observations made under different conditions using deductive logic. There is no archaeological law comparable with Newton's laws of motion and repeated definitions are often not possible. Under this definition, archaeology is not a science.
	\item Hardware follows the law postulated by Intel co-founder Gordon Moore, which states that the computing power available for a given price doubles every year.
	\item For some years it has been common practice for everyone working on an experiment to be named on all the papers published. As a result, the vast majority of authors on a given paper do not write a single word of it, make no direct contribution to the published results and most likely do not even read it until after submission.
\end{itemize}



\end{document}
