\documentclass[main.tex]{subfiles}

\setcounter{secnumdepth}{2}

\begin{document}

\setcounter{section}{8}

\linksection{Лекция 13.03.2024 (Смольская Н.Б.)}

So, I'm also very glad to see you back here.
So, we continue dealing with those problematic questions in terms of grammar and, well actually, in general, your examination materials.
And so, our today's class will be, I would say, more grammar-oriented rather than some sort of discussion-oriented.
And I will begin with a very short task, and that will be like a sort of a warm-up and also a sort of a starting point for us to discuss a specific grammar aspect of the English language.
I will prepare the file and then I will share the screen with you.

\sublinksection{Сослагательное наклонение (Subjunctive Mood). Введение}

Well, I hope that looking at this task, you will understand what grammar material, because I used to mention it during the previous semester.
I think that it is really important in terms of our everyday communication and in terms of your academic communication, and we will discuss in what way it is important for us.
But in any case, this task is one of my favourite ones.
This is my favourite, as you remember.
To choose the correct answer of four given ones.
So your task is to pay attention to the sentence, to choose that variant that you think contains the mistake.
So choose it, mark it in your notebooks.
But of course, as you remember, I'm always interested not only in pointing out what is in this case wrong.
So what is the right answer, but what is wrong in this sentence.
But we will also discuss with you how it should sound in order to be correct.

So, take your time, ten sentences, and of course we will also practice translation, just for you to refresh your translation skills, because that will be somehow what you will have during your examination, so take your time.
Ten sentences, they are not really difficult.
I will then uprise the lower sentences for you to feel comfortable but a little bit later.
So, your task is just to find what point is wrong.
There is only one mistake in each sentence?
Yes, only one mistake.

Are you ready with, say, three first sentences?
Should I uprise the lower sentences?
Like this, just for you to feel comfortable with the lower ones.
Only 10 sentences.
So, ready? Yes, no? No.
Okay.
\\

\hypertarget{ltask:2024-03-13}{--- Выполнение задания ---} (\hyperref[task:2024-03-13]{\color{blue}{перейти к тексту задания}})
\\

\textbfind{Выполненное задание (из подчёркнутых частей каждого предложения найдена часть с ошибкой, которую исправил жирным шрифтом, а первоначальный неверный вариант занёс в скобки и зачеркнул)}

\begin{enumerate}[nosep,leftmargin=*]
	\itemsep\eitsp
	\item If the ozone gases of the atmosphere \uline{\textbf{hadn't filtered out} (\sout{did not filter out})}(A) the ultraviolet rays of the sun, life, \uline{as}(B) we know \uline{it}(C), would not have evolved \uline{on earth}(D).
	\item I wish that my university \uline{\textbf{were} (\sout{is})}(A) \uline{as large as}(B) state University because our facilities are more limited \uline{than}(C) \uline{theirs}(D).
	\item Neither the mathematics department \uline{nor}(A) the biology department at the University requires that the student \uline{\textbf{should write} (\sout{must write})}(B) a thesis in order \uline{to graduate}(C) with \uline{a master's degree}(D).
	\item When a patient's blood pressure is \uline{much}(A) higher \uline{than}(B) it \uline{should be}(C), a doctor usually insists that he \uline{\textbf{should not} (\sout{will not})}(D) smoke.
	\item \uline{If}(A) you \uline{\textbf{bought} (\sout{will buy})}(B) one box at the regular price, you would receive \uline{another one}(C) at \uline{no}(D) extra cost.
	\item If Robert Kennedy \uline{\textbf{had lived} (\sout{would have lived})}(A) \uline{a little longer}(B), he probably \uline{would have won}(C) the \uline{election}(D).
	\item If it \uline{were}(A) \uline{too far}(B) from the Sun, the Earth \uline{would be}(C) \uline{\textbf{too} (\sout{too much})}(D) cold to support any living things.
	\item If the world \uline{was}(A) considered one big country, \uline{its}(B) income inequality \uline{\textbf{far would surpass} (\sout{far surpass})}(C) \uline{that of}(D) any actual country in the world today.
	\item Some executives insist \uline{that}(A) the secretary \uline{\textbf{should be} (\sout{is})}(B) responsible for \uline{writing}(C) all reports \uline{as well as}(D) for balancing the books.
	\item The states require that \uline{every}(A) citizen \uline{\textbf{should register} (\sout{registers})}(B) before \uline{voting}(C) in \uline{an}(D) election.
\end{enumerate}
\ 

So, ready?
Should we begin discussing?
So, for our today's meeting, I have decided, well, I have chosen, I would say, the lightest return to our academic communication.
So in order not to suddenly put you into the intense discussions, I have chosen the lightest approach.
Okay, so let's begin.
Because I hope that, well, at least you have looked through all the sentences, and it's time actually for us to start discussing.

So, my first question, as it is the starting point for our discussion, is, so what is the grammar phenomenon that we are going to discuss today and probably during our next meeting.
And you will have some sort of theoretical material in our course on learning management system, in the course of the foreign language at PORTASP, in our system of distance learning.
But this is the starting point.
So what is this phenomenon that is in the center of our attention for today?
So what is it?
I don't remember how it's in English, but in Russian it's called условные предложения.
No, not exactly, because, well, I'm not surprised, because usually when we start discussing this grammar material, postgraduate students and students think always think only about условные предложения (conditional sentences) but conditional sentences are just one part of a bigger grammar phenomenon which is what?
It is сослагательное наклонение.
What is the English word for сослагательное наклонение?
Subjunctive Mood.
Наклонение -- это mood.

So probably we shall discuss it in Russian just in order for you to clearly refresh your knowledge of сослагательное наклонение (of subjunctive mood) just in general because the main idea of subjunctive mood in the English language and in the Russian language is similar but as you can see the structures of the sentences on the English language of course differ greatly in comparison to what we have in the Russian language.

Итак, сослагательное наклонение -- это одно из трёх наклонений, то есть это категория наклонения, присущая глаголу.
И в русском, и в английском языке категория наклонения существует.
Перечень наклонений одинаков.
Это изъявительное наклонение, сослагательное наклонение и повелительное наклонение.
Три наклонения, которые соответственно передают различные отношения субъекта (говорящего) к действительности.

Если мы говорим об изъявительном наклонении, то это действительный залог, можно так ещё его назвать, то есть это объективно констатация реальности, то есть тех фактов и действий, которые происходят в реальности, и это просто объективная их констатация.

Вот то, что касается сослагательного наклонения, то это, как правило, форма глагола, которая выражает определенное, как сказать, ну это не отношение, потому что это не модальный глагол, но тем не менее, как бы так это суммировать...
Если мы будем непосредственно определять, что выражает сослагательное наклонение, то это неуверенность, предположение, сомнение или же это требование, чувство радости какой-то, или чувство сожаления.
В качестве примеров, чтобы вы поняли, конечно, из английского языка мне бы хотелось, постараюсь привести, ну, например, как вы переведёте предложение God forbid.
God forbid.
Бог простит.
Да, это наше пожелание, некое отношение такое вот, ну, в данном случае, может быть, сожаление или чего-то.
В английском языке это предложение содержит глагол в сослагательном наклонении.
Просто вы обычно, глядя на эти предложения (в задании), вы их назвали условными, и я это прекрасно понимаю, потому что вы увидели знакомый союз if в начале предложения, и он сразу же у вас ассоциативный ряд выстроил.
Это союз if, который для всех сразу же вызывает в памяти конкретные предложения.
И это абсолютно правильно.
Но, как вы видите, не во всех предложениях в задании есть союз if.
А, тем не менее, все эти предложения содержат \textbf{сослагательное наклонение}.
Но мы чаще всего думаем о том, что сослагательное наклонение в английском языке, если мы сейчас будем дальше, так сказать, ну, сегодня и на следующем занятии с вами это обсуждать, что сослагательное наклонение используется в английском языке только в сложных предложениях, но это не так!

Вот не зря я вот это предложение вам назвала, God forbid, или, предположим, Long live St.Petersburg.
Это тоже предложение, которое содержит глагол live в форме сослагательного наклонения в английском языке.
То есть, да здравствует Санкт-Петербург!
В английском языке это предложение с сослагательным наклонением.
Поэтому сослагательное наклонение может использоваться и при образовании простых предложений тоже.
Это вот такое начало нашего обсуждения.
То есть, таким образом, мы с вами констатируем, что, обсуждая сослагательное наклонение, мы будем с вами иметь в виду и простые предложения, вы помните, что это предложения, которые имеют одну пару главных членов предложения, то есть подлежащее и сказуемое, или же, если больше, чем одна пара, то предложение становится сложным.

Но для начала просто, так сказать, проверим, насколько вы знаете или помните правила, которые нужно, или вернее, скажем так, даже не помните и знаете, а насколько у вас ваша лингвистическая интуиция на примере этих предложений развита.
Чувствуете ли вы пока что, даже без объяснения, где здесь ошибка, обращая внимание на предложение, или нет?
То, что касается этих предложений, я пока не буду теорию обсуждать с вами, пока что просто проверим, насколько мы близки к истине, вернее вы, 10 предложений, поэтому можете себя оценить (такая самооценка), грубо говоря, перевести потом эти 10 баллов в некую 5-балльную шкалу.
Итак, ну, вас немножко больше, чем 10, но тем не менее, всё равно есть возможность попрактиковаться в этих предложениях, потом в части перефразирования, и, в общем-то, перевести тоже, я думаю, нам тут это не повредит, чтобы правильно это сделать.

Ну что, пожалуйста, первое предложение.
Когда мы работаем с такими предложениями, вы помните, да, что сначала мы называем, где ошибка.
Ошибка в (A).
И практически во всех предложениях, кроме одного (одно здесь просто закралось специально, чтобы вашу бдительность немножко, так сказать, поймать) касаются именно глагольной формы.
Потому что я повторяю: сослагательное наклонение -- это категория глагола.
Поэтому в первом предложении ошибку нужно искать в форме глагола.
Абсолютно верно, ошибка в (A)
И здесь, конечно, if, поэтому вы как раз и назвали условные предложения.
Какова правильная форма, которая должна здесь стоять? 
Пока мы только называем правильную форму.
Остальные, если согласны с (A), или же не согласны, но тем не менее, можете где-то себе записывать то, как эта форма должна звучать.
А дальше мы с вами на основе вот этих вот примеров, которые у нас с вами есть, начнем вспоминать, что такое сослагательное наклонение и отдельные аспекты.
Итак, форма какая должна быть?
Hadn't?
А дальше?
Потому что hadn't -- это вспомогательный глагол, а обязательно должен быть ещё и смысловой.
Итак, hadn't filtered out.
Абсолютно верно.
Правильная форма, которая будет использоваться в пункте (A) в этом глаголе, это hadn't filtered out.
Пока двигаемся дальше.

Второе предложение, пожалуйста.
Где ошибка? 
Итак, ошибка в (A), абсолютно верно.
Все остальные тоже себе помечают.
И если это ошибочная форма, то правильная форма будет...
Was?
Was или were!
Форма were является классической в английской грамматике.
Для was сейчас тенденция, так сказать, проводить эту дифференциацию, но чуть позже, когда мы начнём с вами более подробно это обсуждать, я прокомментирую.
Может быть, кто-то тоже вспомнит, вдруг когда-то вы с этим встречались.
Итак, ошибка в (A).
Правильная форма здесь должна быть -- прошедшее время, но это не просто прошедшее время, это не Past Simple, а это Past Subjunctive.
И вот в этом весь и смысл.
Это прошедшая форма сослагательного наклонения глагола to be.

Так, пожалуйста, третье.
Ошибка в (B).
Правильно либо should write, либо просто write.
Обратите внимание, the student should write или the student write.
Не добавляем окончание -s к write, а просто оставляем write.
Мы чуть позже с вами пройдёмся по этим отдельным случаям, а дальше как раз, когда я в домашней работе открою файл с теорией, то вы уже сможете найти эти пункты в этой сводной табличке, которую я туда подгрузила.
Если мне память не изменяет, то она там есть, и я её для вас открою.
Своё домашнее задание вы ещё выполните, и после этого на втором занятии мы с вами уже попробуем систематизировать, осознать уже (как PhD students) и провести аналитическую, не только аналитическую, но еще и синтетическую работу, синтетический подход, синтезировать всё, что вы получили.
Итак, два варианта может быть.
Если вот идти по тому, что от нас хотели бы здесь организаторы, то здесь глагол must не может использоваться.
Вместо него будет использоваться глагол should.
А если двигаться дальше, то should можно убрать и останется только write.
Итак, сейчас пока не объясняю, почему это и что это такое, а просто мы с вами пометили для себя, какой правильный вариант.

Пожалуйста, четвёртое предложение.
Ошибка в (D).
Правильно будет так: should not smoke.
Абсолютно верно.
Фиксируем для себя ту правильную форму, которая должна быть в этом предложении.
Should not smoke.

Пожалуйста, пятое предложение.
Ошибка в (B), абсолютно верно.
И что вы сюда поставите как правильный вариант?
Может быть, bought.
Да, абсолютно верно.
Past simple от глагола buy.
Абсолютно верно.
Пока помечаем, дальше через несколько минут вернёмся уже целиком к предложениям.
Будем читать, переводить и обсуждать каждое предложение, и будем стараться понять, что это такое.

Так, пожалуйста, шестое предложение.
Ошибка в (A).
Но хотелось бы теперь, чтобы мне предложили форму, которая должна быть правильной, если ошибка в (A).
Условное наклонение -- это действительно такие лингвистические задачки в английском языке.
Очень такие интересные.
Это реально задачки, которые логически надо решать.
Had lived?
Да, абсолютно верно.
В этом случае будет had lived.

Так, в седьмом предложении где ошибка?
Ошибка в (D).
Это как раз вот вашу бдительность и внимательность проверяем в данном предложении.
Предложение, опять же, для нас очень знакомое.
Оно начинается с if.
Но если вы сравните глагольные формы, которые здесь стоят, то здесь с глагольными формами всё хорошо.
А ошибка на самом деле действительно в (D).
Что здесь лишнее?
Лишнее слово much.
У нас есть too far (см. в начале предложения) и будет too cold (см. в конце предложения).
А само грамматическое оформление предложения -- это проверить вашу внимательность.
Само грамматическое оформление с точки зрения subjunctive mood здесь абсолютно правильное.
В этом предложении, в том, что касается форм глагола, ошибки нет.
Но это будет для нас с вами как бы всё равно основание для того, чтобы его обсудить и вспомнить, что, как бы, проговорить ещё раз, что здесь используется в этом предложении.

Восьмое предложение, пожалуйста, кто предложит?
Ошибка в (C).
Какая форма должна быть?
Должно быть would surpass.
Здесь не хватает вспомогательного глагола would, который в этом случае должен использоваться в предложении.

Так, девятое предложение, где ошибка?
У нас было похожее.
Здесь ошибка в (B).
И обратите внимание, что если после глагола to insist используется придаточное предложение, то тогда предлог "<on"> не требуется.
И что должно быть вместо формы is?
Должно быть should be, абсолютно верно.
Или же, если вот учитывать, что это глагол is, то тогда просто be.
Или should be, или be.

Так, и десятое предложение.
Ошибка в (B).
Здесь у нас форма registers, но окончания -s в данном случае быть не должно.
Просто следует использовать register.
А по аналогии можно здесь should поставить?
Да, можно, и сейчас как раз мы с вами постараемся систематизировать.

Итак, у нас 10 предложений; они не показывают всю вариативность использования сослагательного наклонения.
Всю вариативность будем разбирать на следующей паре.
А сейчас это только, так сказать, для начала.
Но даже глядя на вот примеры, которые представлены здесь, мы можем уже чётко с вами выделить, по крайней мере три основных варианта использования сослагательного наклонения в английском языке.
Или не варианта, а случая.
Три случая использования.

Самый очевидный, который вы все как раз знаете (и с чего мы и начали) -- это условные предложения.
Вот сегодня мы с вами сейчас начнём вспоминать условные предложения на примере этих предложений, и посмотрим, как дальше пойдёт.

\newpage
\sublinksection{Условные предложения (частный случай Subjunctive Mood). Теория}

Итак, условные предложения.
Conditionals.
Почему вы так хорошо их знаете?
Потому что в школе этому уделяют очень много времени.
Итак, что такое условные предложения?
Когда мы в английской грамматике с вами называем: о, это условное предложение!
Что это такое за предложение?
Сначала мы начинаем...
Если мы его называем условное предложение, то мы под этим подразумеваем какое предложение?
Сложное (сложноподчинённое) предложение (это в обязательном порядке), которое содержит придаточное предложение условия.
То есть предложение придаточное, которое чаще всего присоединяется союзом if.
Это самый распространенный случай, хотя есть и другие варианты, которые используются тоже, но пока давайте на союзе if сконцентрируемся, чтобы пока, так сказать, в голове новой информации излишне не было.
Итак, условное предложение -- это придаточное предложение, которое присоединяется к главному с помощью союза, чаще всего союза if.
Но при этом мы с вами говорим условные предложения, и это нас приводит к чему?
К тому, что мы должны в вашей памяти поднять целый пласт, который говорит о том, что условные предложения могут быть достаточно разнообразны.
В чём их разнообразие?
В том что они могут быть разных типов, то есть 4 типа.
Я думаю, что вы все эти типы сейчас, наверное, вспомните.
Что это за типы?
Если их 4, то значит какие это типы?

Итак, conditionals могут быть of zero type, first type, second type and third type.
Условные предложения бывают четырёх типов: нулевой тип, первый тип, второй тип и третий тип.
Если мы их попытаемся объединить между собой, делятся ли они на какие-то категории?
Да, нулевой и первый -- это одна категория.
А второй и третий -- это другая категория.
Они вот так вот распадаются, симметрия во всем, я тоже очень люблю везде симметрию, чтобы везде было все логично.
Если они распадаются на эти две части, значит эти две части чем-то общим каким-то содержанием объединяются.
Чем?
По содержанию?
Как они называются?
Они называются real conditionals (нулевой тип и первый тип) и unreal conditionals (второй тип и третий тип).
Наш русский язык здесь нам поможет особенно осознать, что такое Unreal Conditionals.
Чисто формально -- это те предложения, в которых при переводе на русский язык вы будете использовать частицу "<бы">.
Вот если у вас есть частица "<бы">, то это Unreal Conditionals.
То есть это то, о чём мы думаем как о чём-то возможном (так сказать, желательном), но не описывающем реальным факт.

Итак, а дальше, естественно, говоря ещё и об этом делении на типы, не зря их всё-таки четыре типа, мы должны дальше уже их описывать с точки зрения тех форм глагола, которые мы используем.
Итак, у нас есть zero conditional, first conditional, second conditional и third conditional; между ними мы проведем borderline и определим real conditionals (включают в себя нулевой и первый типы) и unreal conditionals (включают в себя второй и третий типы).
И для себя просто в качестве подсказки припишем, что для нас важно, что в русском языке при переводе здесь в unreal conditionals будет частица "<бы">.

Ну, давайте пойдем сверху вниз.
Сначала поговорим про real conditionals.
Несмотря на то, что и нулевой, и первые типы объединены у нас в одну категорию (real conditionals), но тем не менее между ними очень четко прослеживается понятийная разница.

Давайте начнем с первого.
Он самый такой простой и в школе его всегда учат (первое с чего начинают).
Что это такое?
Что обозначают предложения первого типа условных предложений.
Это наши некие планы на будущее при наличии каких-то условий.
Осуществление планов на будущее при каких-то условиях.
Если у нас будут билеты, мы пойдём в театр.
How would you translate this sentence?
If we buy tickets, we will go to the theatre.
Это такое самое простое предложение, потому что нам достаточно этого предложения, чтобы проанализировать конструкцию этого предложения и те формы глагола, которые в нем используются.
Итак, раз это сложное предложение, то в нём мы сразу же определяем главное предложение и придаточное.
Мы знаем, что придаточное предложение -- это предложение условия и присоединяется оно самым очевидным союзом if (условным союзом; союзом "<если">).
Важный момент с точки зрения пунктуации в английском языке...
Мы с вами частично уже пунктуацию затрагивали, когда косвенную речь обсуждали, здесь ситуация та же самая.
Посмотрите, пожалуйста, что здесь интересного.
Вернее, для нас, для русскоязычных, тут, собственно говоря, ничего интересного нет.
Запятая есть, стоит на месте, которое для нас очевидно, как для русскоговорящих.
Но если мы поменяем местами главное предложение и придаточное, то в английском языке в этом случае запятая ставится не будет (союз в этом случае уже и так показывает, где заканчивается главное предложение и начинается придаточное, поэтому дополнительный маркер в виде запятой не нужен).
То есть запятая ставится только в том случае, если придаточное предложение предшествует главному, тогда придаточное отделяется от главного запятой.
Да, получается, что нам, чтобы показать, что это начало уже главного предложения, мы поставим здесь графический показатель в виде comma.
Ну и теперь посмотрим, потому что в английском языке очень важно правильно соотносить грамматические формы глаголов.
Вот в русском языке у нас нет такой серьёзной зависимости, как в английском языке.
Давайте посмотрим на это предложение.
И назовите мне, пожалуйста, форму глагола, который у нас используется в придаточной части и в главной.
Итак, в придаточном предложении где форма глагола?
Buy.
А в главном предложении?
Will go, абсолютно верно.

Мы с вами не раз уже говорили о том, что разница между английским и русским языками заключается в том, что в русском языке (как в языке синтетическом) все категории присущие различным частям речи демонстрируются за счет элементов внутри слова, то есть слово будет присоединять к себе все те грамматические показатели, которые ему нужны, но не будет выходить за рамки самого слова.
Если мы по-русски это скажем, то будет глагол ПОЙДЁМ.
С помощью приставки ПО от глагола ИДТИ мы образуем будущее время.
В английском языке ситуация другая.
Английский язык -- это язык аналитический, поэтому различные значения передаются дополнительными словами, которые передают ту или иную функцию.
И в этом случае мы с вами говорили, что это форма Future Simple, где у нас есть глагол will, который, по сути дела, является модальным глаголом, но он является показателем будущего времени.
Мы с вами говорили о том, что просто как бы вот это никогда особо не акцентировалось, наверное, до нашей с вами встречи -- о том, что всё-таки глагол will (как и глагол shall) всё равно частично сохраняют присущие им вот эти значения долженствования и волеизъявления.
Поэтому, используя или shall, или will при образовании Future Simple вы определённый акцент смысловой (в зависимости от ваших намерений) всё равно можете сделать (we will go или we shall go).
If my grandma calls me, I shall go and visit her.
То есть если не позвонит, то не поеду, а если позвонит, то я должен буду всё-таки поехать.

Ну и теперь мы приходим с вами к требованиям глагольных форм при построении первого типа Conditionals.
И вы мне все чётко можете здесь теперь сказать, что жёсткая зависимость временных форм заключается в чём?
В главном предложении мы будем использовать наши планы (а планы выражаются в будущем времени, да), поэтому в главном предложении будет использоваться Future Simple или Future Continuous, может быть и Future Perfect, если мы показатель какой-то сюда добавим.
А вот в придаточном предложении будущее время, которое на уровне содержания есть (например: если купим -- то есть как бы всё равно намерение относится к будущему времени), но на уровне плана выражения (а план выражения это наши буквочки) будут использоваться только настоящие времена.
Или Present Simple, или Present Continuous.
Это очень важно!
В школе вы это правило учили как?
Что после if и when будущее не используется.
Вот это правило как раз из этого случая.
Итак, здесь у нас наши планы на будущее при выполнении определенных условий -- таково значение первого типа условных предложений.

У нас есть ещё Zero Type, про который обычно забывают, но тем не менее он имеет право на существование, и тем более для нас с вами (и для вас в частности), как для academicians, как для людей, которые Academic English будете использовать, Zero Type вполне себе в академическом английском используется.
Что он обозначает?
Well-known facts, абсолютно верно.
Если выражаться научным языком, то это аксиомы, которые приняты или уже были доказаны и больше доказательств не требуют.
Пожалуйста, по-английски какой-нибудь пример.
Ну, давайте по-русски.
Если воду нагреть до 100 градусов, она закипит.
Ну, хотя бы, да, самое такое простое.
If water is heated due one hundred degrees, it boils.
Поскольку это физические/химические/экономические/социальные законы (законы бытия), то в этом случае (и это очень важно) в Zero Type мы точно также с вами прослеживаем две части.
Главное предложение и придаточное, которое тоже присоединяется здесь союзом if, но если мы с вами обратим внимание на формы глаголов, то вы знаете, что в таких предложениях используются только формы Present.
В качестве лакмусовой бумажки для этого типа, по сути дела, можно использовать то, что союз if в этом случае равен чему?
Равен союзу "<when">.
То есть в этих предложениях if может быть заменён на союз when, и смысл от этого не поменяется.
When water is heated it boils = If water is heated it boils.
То есть в любом случае взаимоотношения между условием и следствием одинаковы.
Потому что в первом типе мы уже этого не можем сделать, потому что если мы поставим здесь when, то тогда условия уже не будет, тогда это будет просто констатация факта: когда купим билеты, тогда и пойдем в театр.
Завтра купим -- завтра пойдем, послезавтра -- послезавтра.
А здесь в Zero Type при замене if на when условие сохраняется.
Но в любом случае, как вы видите, это то, что даже формы, которые у нас используются -- это формы будущего (Future) и настоящего (Present) видовременных форм, они всё-таки дают нам возможность мечтать о том, что мы рождены, чтобы сказку сделать былью.
То есть здесь вы точно уверены, что что-то будет.
Всё-таки мы живём в её ожидании (First Type) или же уже её получили, как в Zero Type, наблюдаем её изо дня в день.

А вот то, что ниже этой ватер-линии, которую мы с вами обозначили, я такой двойной полосой отделила второй и третий типы -- это вот так называемый Unreal, потому что действительно русский язык за счет вот этого перевода, который он нам даёт, позволяет почувствовать эту Unreality.
По-русски какое-нибудь предложение мне приведите, которое подойдет под нереальные условия.
Если бы у меня было это, я был бы рад.
Да, если бы у меня был миллион, то я был бы рад.
Итак, если здесь (при выборе между First Type и Zero Type) я вам назвала лакмусовой бумажкой союз "<when">, то в этих случаях (чтобы отличить unreal conditionals от real conditionals) проверочным словом для нас будет наличие вот этой частицы "<бы">.
Условной частицы "<бы">, которая при русском переводе поможет вам сразу же определить: если вы перевели "<бы">, то точно нужно смотреть unreal conditionals.
Вы будете решать между вторым или третьим типом.

Чем отличаются второй и третий типы в грамматике?
Чем они отличаются?
Чем выше номер рассматриваемого типа, тем дальше мы идём к нереальности.

Давайте даже с третьего типа начнём.
Его легче понять.
Это что?
Что обозначают эти предложения?
Это печаль.
Да, это печаль по поводу того, что уже имело место быть.
Но само предложение, оно как бы передаёт нам, что бы вы хотели сделать (хотели, чтобы что-то произошло), но этого уже не произойдёт, потому что факт уже состоялся.
По-русски, пожалуйста, пример.
Я бы хотел на прошлой неделе прийти на английский, но я не пришёл.
Чувствуете мечты?
Да, действительно, то есть при построении этих предложений в качестве такого, ну, не 100\% обязательного элемента, но тем не менее -- это либо контекст будет на это показывать, либо вот необходимый элемент, который я добавила (указание -- именно на прошлой неделе).
То есть должен быть некий указатель (обстоятельство времени), который будет относить нас к прошлому.
Если бы я вчера купил билеты, мы бы пошли вчера в этот театр.
Как будет по-английски это звучать?
Давайте теперь запишем.
If we had bought tickets yesterday, we would have gone to this theatre.
И для нас yesterday здесь обязательно таким показателем является.
И давайте теперь здесь посмотрим на глагольные формы.
В придаточном предложении глагольная форма какая?
Had bought.
Для нас, опять же, подчёркиваю, в английском языке принципиально важно, потому что это какая форма?
Аналитическая, которая состоит из двух частей -- из смыслового глагола bought и из вспомогательного глагола had, который указывает нам на определённую видовременную форму.
Сейчас мы её чуть позже назовём.
В главном предложении какая будет форма глагольная?
Would have gone.
Мы с вами наблюдаем, что здесь даже целых три слова, которые в купе (в сумме) выражают вот это вот особое значение.
Где-то здесь в русском варианте затерялась частица "<бы"> тоже, но на уровне содержания.
Итак, теперь если просто формулировать грамматическое правило, то тогда мы с вами что должны сказать?
Что в главном предложении будет использоваться?
Future in the past?
А это НЕ Future in the past!!!
Это как раз тот самый момент английского языка, когда одно и то же слово может в качестве служебного выражать разные значения.
И вот здесь would -- это не Future in the past.
Здесь would -- это вспомогательный глагол Subjunctive Mood.
Вот как раз эти should и would в предложениях, которые требуют использования сослагательного наклонения, будут вспомогательными глаголами сослагательного наклонения.
Они просто омонимичны к Future in the past.
Омонимичны, да?
Что такое омонимичны?
Пишутся и звучат одинаково, а выражают разные вещи.
Поскольку это вспомогательные глаголы, то это реальность, которой в русском практически нет.
У нас нет вспомогательных глаголов в русском языке.
Ну, кроме глагола "<будут"> в русском языке.
Пример: Я буду читать.
Вот это может быть единственный вспомогательный глагол для образования будущего времени несовершенного вида.
А вот в английском языке вспомогательные глаголы -- это служебные части речи, которые используются для образования различных форм глагола.
Итак, здесь would -- это вспомогательный глагол плюс что?
А вот это что?
Have done?
Have gone?
Present Perfect?
Нет!!!
Present Perfect -- это тот случай, когда форма глагола является сказуемым сама по себе.
We have gone.
Точка.
Мы ушли.
Вот там have gone -- это Present Perfect.
А здесь давайте вспомним, что у нас идёт после вспомогательного глагола.
Например, при образовании...
Вот у нас там есть will go.
Will -- это вспомогательный глагол -- в данном случае для образования будущего времени с частичным сохранением значения.
А go -- это что?
Это базовая форма глагола.
Или же это называется, мы чуть позже на следующих занятиях с вами будем это обсуждать -- это Simple Infinitive (bare infinitive = инфинитив без "<to">).
У этой формы есть грамматическое обозначение (название) Simple Infinitive.
Но мы с вами договорились эту форму называть Base Form.
А вот то, что используется в нашем рассматриваемом примере...

В английском языке какие неличные формы глагола существуют?
Что значит неличные формы глагола?
Это какие формы глагола?
У которых нет чего?
Лица.
Личные формы глагола -- это формы глагола, которые используются как сказуемые.
Я читаю, ты читаешь, они читают и так далее...
Личные форма глагола.
А "<читающий"> (в русском языке причастие), "читая" (в русском языке деепричастие) -- это тоже неличные формы глагола.
Ну, чтение у нас уже не является формой глагола, а является существительным (а в английском языке это бы был герундий).
Или если мы возьмём неличные формы глагола в английском языке, то это Infinitive, Participle I, Participle II и Gerund.
То есть это те формы глаголов, которые образовались от глагола, но лицо они не выражают.
Категорию лица (I, we, he, she, it) к ним никак не присоединить.
Они не соотносятся, да, они вместе не коррелируют.

Так вот это в нашем примере, хорошо, забегаю вперед, если вы не можете мне сказать, have gone -- это Perfect Infinitive.
Это Perfect Infinitive.
Ну или, чтобы вам просто, может быть, для себя более чётко было понятно, как это образуется...
Would -- это вспомогательный глагол Subjunctive Mood;
have -- это вспомогательный глагол Perfect как целой категории (а не как Present Perfect), потому что мы глагол have в зависимости от того, какой нам нужен Perfect -- прошедший (Past Perfect) или настоящий (Present Perfect) -- мы будем ставить соответственно в то время, в котором нам нужно этот глагол поставить.
А gone -- это что?
Participle II.
То есть можно записать таким образом: вспомогательный глагол Subjunctive Mood + вспомогательный глагол Perfect + Participle II, но ни в коем случае вот это в рассматриваемом примере (if ..., we would have gone to this theatre) не Present Perfect.
Present Perfect будет только в том случае, если эта форма будет без каких-либо ещё к нему присоединенных слов использоваться напрямую с подлежащим.
А здесь у нас есть перемычка, которая сразу же не даёт нам эту форму называть Present Perfect.

Итак, что нам важно?
Если мы относим вот это вот нереальное событие к тому, что он уже состоялась и нам жаль...
Что значит If we had bought tickets yesterday, we would have gone to this theatre?
Это значит, что we didn't buy tickets yesterday and we didn't go to this theatre.
То есть событие не состоялось и мы сожалеем ...
А сожаление -- это как раз Subjunctive Mood.
Я как раз вам и говорила, что это одно из значений, мыэто сожаление передаем с помощью вот этого вот предложения в третьем типе условных предложений.

Ну и теперь осталось разобрать второй тип условных предложений.
Как вы определите общее настроение второго типа?
Это некие мечты, это просто наши мечты.
Мы не знаем, осуществимы ли они, потому что это не First Type.
В First Type тоже были, по сути дела, планы и мечты, но в First Type мы показываем, что они осуществимы, а здесь в Second Type -- это скорее наши мечты, о осуществимости которых мы не знаем.
Если бы у меня был билет на самолёт, то я бы завтра улетел, но я не знаю отпустит ли меня начальник.
В First Type мы почти уверены, что что-то осуществится.
Почти -- так как мы говорим о том, что будущего и в философии не существует и тем более в английском языке, потому что в английском языке будущее зависит от вашего настроя (волеизъявления или долженствования).
А в Second Type существенно сниженная вероятность того, что эта мечта может стать реальностью.
И вот для того, чтобы показать, что это ваши мечты (может быть, в каком-то далеком будущем и осуществимые), мы и используем второй тип условных предложений.
То есть это такие сказочные мечты.
И, пожалуйста, как по-английски будет звучать, что если бы у меня был билет на самолёт, я бы улетел.
Мечтаем!
Как вы про эти мечты свои будете говорить.
If I had a ticket, I would fly.
Посмотрим, что мы имеем здесь.
В придаточном предложении (то есть после союза if) мы имеем had, которое, по сути дела, отражает форму Past Simple, ну или Past Continuous может здесь тоже быть.
А в главном предложении мы опять имеем вот этот would, который является вспомогательным глаголом сослагательного наклонения.
И базовая форма глагола, то есть Simple Infinitive, который сюда после него (после would) присоединяется.
Simple Infinitive или base form of the verb.

\newpage
\sublinksection{Разбор примеров условных предложений (Conditional Sentences)}

Так давайте посмотрим теперь на те предложения, которые мы с вами сегодня разбираем.
И теперь мы с вами читаем предложения.
Кто у нас сегодня еще пока не активничал?
So you are welcome.
And let us translate.
Итак, мы читаем предложение уже с правильной формой, называем, что это за тип условного предложения (первый, второй, третий или нулевой), и переводим.
Здесь такие не совсем уж прямо everyday English, поэтому есть возможность попрактиковать ваш перевод.
Пожалуйста.
So who would like to read and to translate?
Okay, you are welcome.

If the ozone gases of the atmosphere \textbf{hadn't filtered out} the ultraviolet rays of the sun, life, as we know it, \textbf{would not have evolved} on earth.
So, what type is it?
That's type 3 (the third type).
Because of what?
Because we have what?
Because: \textbf{would not have evolved} and \textbf{had not filtered out}.
So, would you please translate it?
Если бы озоновый слой не фильтровал ультрафиолетовые лучи Солнца, то жизнь, как мы её знаем, не зародилась бы на земле.

Okay, so what sentence is the next that we will insert into the same idea of conditional sentences.
Меня интересуют любые Coditionals.
Но никто ничего не хочет сказать еще...
Да, четвёртое предложение -- это по сути дела Zero Type.
Я не зря вам напоминала, что when тоже может быть;
if и when являются взаимозаменяемыми.
Это очень интересное предложение (четвёртое предложение), потому что в нём два разных аспекта или два разных случая использования сослагательного наклонения.
С точки зрения Conditional, это у нас где?
So who would like to read this sentence and to translate?
Yes, please.
When a patient's blood pressure \textbf{is} much higher than it should be, a doctor usually \textbf{insists} that he \textbf{should not smoke}.
Замечательно.
Где у нас Zero Type?
Назовите формы глаголы, которые относятся к Zero Type.
В придаточном предложении это будет is.
А в главном предложении insists.
Потому что вот то, что мы с вами исправляли -- это другой случай использования Subjunctive Mood.
Следующий, который мы будем обсуждать на следующей паре (это уже будет не Conditionals).
So, would you please translate?
Когда кровяное давление пациента намного выше, чем оно должно быть, доктор обычно настаивает, чтобы он не курил.
Да, чтобы он не курил.
Вот соотношение -- когда давление кровяное, то доктор настаивает -- это и есть тот самый Zero Type, потому что мы when можем смело заменить на if и у нас по сути дела взаимозависимость между главным и придаточным всё равно останется та же самая (как условие и следствие).
ЕСЛИ кровяное давление гораздо выше или же КОГДА кровяное давление -- то есть здесь специально такое завуалированное для вас, чтобы вы обратили внимание, что это Zero Type.
То, что мы исправляли, как я сказала, это уже следующий случай, и мы пока его не обсуждаем (будем обсуждать на следующей паре).

Пожалуйста, пятое предложение.
Yes, please.
If you \textbf{bought} one box at the regular price, you \textbf{would receive} another one at no extra cost.
Итак, здесь мы исправили на bought.
Что это за тип условного предложения?
Это второй тип, абсолютно верно.
В придаточном предложении у нас bought -- это Past Simple, в главном предложении would + base form.
Это тоже у нас показатели второго типа.
Переведите, пожалуйста.
Если бы ты купил одну коробку за обычную цену, ты мог бы получить другую без дополнительных затрат (по оптовой цене, наверное, имелось в виду).
Да, и при переводе важно понимать, что это Unreal.
Это вот как бы помечтай, да?
Если надумаешь, то иди.
Но вот если бы купил, то получил бы.
Здесь нет у нас отнесения к прошлому, поэтому мы здесь однозначно не говорим, что тебе не повезло, что ты этого не сделал неделю назад.
Так, замечательно.

Следующее, пожалуйста.
Вы уже называли.
Да, давайте шестое.
If Robert Kennedy \textbf{had lived} a little longer, he probably \textbf{would have won} the election.
What type is it?
It is type 3.
And translate, please.
Если бы Роберт Кеннеди жил немного бы дольше, то, вероятно, он бы победил на выборах.
Да, замечательно.

Следующий, пожалуйста.
В этом задании (в этой, так сказать, затравочке), которое мы с вами начинаем в качестве обсуждения Subjunctive Mood обсуждать, очень много именно условных предложений.
Просто потому что я специально так выбрала, чтобы показать, что условные предложения -- это действительно то, чем мы часто пользуемся.

Пожалуйста, какое следующее предложение?
Седьмое предложение (number seven).
If it \textbf{were} too far from the Sun, the Earth \textbf{would be} too cold to support any living things.
You remember that from the point of view of grammar everything is okay here.
We do not correct anything.
But it is just for practice.
Если бы Земля была слишком далека от Солнца, то Земля была бы слишком холодная для поддержки живых организмов.
Замечательно!

Так, и у нас есть ещё одно Conditional Sentence.
Это восьмое предложение.
If the world \textbf{was considered} one big country, its income inequality far \textbf{would surpass} that of any actual country in the world today.
Good.
So what do we have here?
What type is it?
Type number two.
Its -- это притяжательное местоимение в предложении.
Если бы мир был одной большой страной, то его неравенство в доходах значительно бы превосходило любое неравенство, которое есть в любой существующей ныне стране.
Мы с вами говорили, что вот это слово actual -- это не актуальность, а это то, что существует сейчас.
То есть то, что действительно.
Лишний раз, так сказать, на это обратить внимание.

Okay, so now if we keep in mind that conditional sentences are very wildly used, what will be the instances of academic English when you can use conditionals?
What would you express through conditional sentences?
Okay, the laws and that will be what?
That will be Zero Type.
In terms of academic English, то есть с точки зрения вашего научного окружения, научного языка, где вы будете использовать (если по-английски) эти предложения?
Для выражения чего?
Итак, laws and some statements that you can use as the basis for your discussion.
And what is really important is that conditional sentences can be used as hypothesis for your research, for your article.
Вы же, как правило, всё-таки при написании статьи (и уж тем более диссертации) вы формулируете нечто, что вы хотите доказать и вот это как раз Conditional Sentences, то есть правильно соотносить.
Другое дело, что можно вашу гипотезу, которую вы точно докажете, то это будет какой Conditional?
Это будет Conditional One, потому что вы однозначно планируете это доказать.

А вот type two and three (второй и третий типы) can be used for what?
Их вы как раз при описании каких-то несостоявшихся доказательств предшественников используете.
То есть когда нужно показать, что он пытался, но не смог.
Но если бы он использовал такое-то доказательство, то тогда бы получил это.
То есть на самом деле в Academic English conditional sentences are really very widely used.
Я надеюсь всё-таки, что мы всё равно будем продолжать писать статьи однозначно по-английски, поэтому давайте это тоже принимать во внимание.

\newpage
\sublinksection{Гипотезы и планы -- это первый тип условных предложений}

And this is actually the case that I would like each of you to try (I mean in terms of the hypothesis of your research; of your prospective research)...
So would you all please write down the sentence, which is the hypothesis of your research.
Пожалуйста, попробуйте сформулировать в форме conditional sentence type 1.
Итак, что вы хотите доказать?
Ну, наверняка у вас какие-то есть разные там задачи, которые вы перед собой ставите.
Запишите, пожалуйста, и будет интересно, что вы планируете сделать.

Okay, would you please now share your sentences just in order to check your grammar and just share your plans, your hypotheses that you keep in mind and have in mind.
So, who would like to begin?
Yes, please.

If I develop a new model, it will predict durability of turbine blades more correctly than actual models.

Кто ещё готов? Кто сформулировал? Да, давайте.
If we affect the fluid flow with sonic waves, the convective heat exchange will intensify.
Okay, very good, thank you.
You definitely have Present Simple in subordinate clause and Future Simple in the main clause.

Не выдумывайте ничего такого очень сложного.
Наверняка вы имеете какие-то соотношения в виде если и то.
Попробуйте их просто сказать по-английски.
If I create a simulator, I will define the influence of each parameter on physical affects.

Ещё раз проговорим, что если я прошу вас записать, то это некое апеллирование к письменному английскому, к Academic writing.
А мы с вами говорили о том, что Academic Writing не допускает сокращенных форм.
Еще раз, как сокращенные формы?
Contracted forms.
То есть никаких апострофов в письменном английском (в академическом) и в статьях не должно использоваться.
Это обязательное правило, которому нужно следовать.
Ну, если хотите, такой научный этикет письменного английского, академического.
Поэтому I'll в нашей речи однозначно можете использовать, когда мы что-то обсуждаем.
А когда вы пишете статью, то имейте в виду, что никаких сокращений быть не должно.
Их не должно быть.
Они по правилам этикета не допускаются.
Пожалуйста, кто ещё?
You are welcome.

If I make a mathematical models of combining, I will know different type of combination materials for creating cheap and quality instruments.
Хорошо, спасибо, очень хорошо.

If I create a method for calculating residual resources, I will help my work company.
Okay, that's also like a sort of a hypothesis.

If I get analytical solution for equations, I will confirm or deny numerical results.
Good, very good, thank you.

В каком типе сейчас нужно придумать?
В первом или в нулевом?
В первом типе, то есть всё-таки не закон, а гипотезу.

If I introduce machine learning in optimisation task, I will decrease the solution time.
Good, thank you.
Кто-нибудь еще?

Итак, ваше домашнее задание.
Я на днях открою теорию на портале аспирантуры.
Там у меня сделана большая табличка по всем возможным (то, что я смогла собрать; может быть, конечно, что-то еще дополнить можно, но тем не менее) случаям использования сослагательного наклонения.
Здесь собраны все возможные случаи использования сослагательного наклонения.
Условное здесь идет первой строчкой -- то, что мы с вами сегодня обсуждали.
А дальше еще про сослагательное наклонение.
И, значит, соответственно, ознакомьтесь с этой табличкой.
Дальше я вот это упражнение, которое у нас было, я его тоже туда подгружу.
Вы его увидите и после ознакомления с этой табличкой определите, к какому типу сослагательного наклонения или к какому случаю сослагательного наклонения будут относиться оставшиеся предложения.
Изучите эту табличку, потому что на следующий раз, мы в следующий раз начнем уже работать с более сложными предложениями.
Прямо здесь на следующем занятии будем по ходу идти (без особой подготовки) и выяснять, что это, как и почему.
Это то, что касается грамматики.

Ну и вы помните, что остались у нас еще, так сказать, на обсуждение темы, поэтому к следующему разу мы возьмем такую самую, так сказать, энергозатратную, пока вы еще не сильно устали, есть тема грантовой программы.
По-моему она идёт под номером семь в программе кандидатского экзамена.
То есть программы грантовой поддержки молодых ученых.
У меня к вам просьба.
Наверняка вы участвовали в каких-то конкурсах или подавали заявки на гранты.
Поделитесь, пожалуйста, вашим прошлым опытом, если вы подавали (как это было, что нужно было сделать).
И я просто с точки зрения грамматики, это мне интересно, да, потому что если вы будете описывать свой прошлый опыт, то, соответственно, вы будете описывать это, используя Past Simple и Present Perfect.
И в какой программе вы бы хотели поучаствовать.
А это уже сослагательное наклонение (и первый тип, и второй тип), то есть для того, чтобы вы ещё грамматику свою тоже попрактиковали.
То есть существуют разные программы -- может быть, кто-то из вас (будучи бакалавром) участвовал в Умнике (есть опыт), а теперь вы хотите подаваться на премию правительства России или президента.
Опишите, как сможете.
Мне в данном случае важна грамматика.
Conditional -- это план, это в основном Conditional First.
То есть мой интерес: это и попрактиковать грамматику (те самые Conditionals или же просто времена глагола), но и ещё и такая своеобразная подготовка к тому, чтобы вы потом эту тему мне тоже написали.
Это мы пока просто готовимся к обсуждению?
Да, пока просто устно, письменно пока не надо.
Пока просто подумайте, о чём бы вы рассказывали, но я вам вот эти векторы обозначила.
Один вектор назад, а второй вектор, наоборот, вперед.
То есть в чём бы хотели поучаствовать и в чем уже участвовали?
То есть это практика вашей грамматики тоже.
Чтобы не просто так вы отвлеченно рассказывали бы о чём-то, а всё-таки чтобы вы уже осознанно/сознательно на грамматику тоже обращали внимание.

Можно вопрос?
Да.
Это тема про международные и российские программы поддержки молодых учёных?
Да, абсолютно верно.
Сюда подходят конкурсы, премии, гранты и всё, что хотите.
Что вас поддерживает в этой жизни как учёного?

Скажет кто-нибудь ещё примеры Conditional Sentences Type 1?
Да, пожалуйста.

If I find the relationship between frequency spectre of seismic signal and amplitude of oscillation, I will introduce a new method for defining seismic resistance of mechanics systems.

If I make a mathematical model for strain-stress state of intermodal connections, I will create methods of calculation and designing of them for engineers.

Хорошо, замечательно.
Тогда для первого раза в этом семестре я вас отпускаю, чтобы сильно не утомлять.
До следующей среды.


\end{document}
