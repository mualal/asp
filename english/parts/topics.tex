\documentclass[main.tex]{subfiles}

\begin{document}

\setcounter{secnumdepth}{0}


\setcounter{section}{102}
\linksectionold{Topics}


\setcounter{subsection}{1}
\sublinksectionold{The prospects of my scientific career}

My research areas are Theoretical Mechanics and Mathematical Modelling of Physics processes.
There are two widely used approaches to solve Theoretical Mechanics problems: the analytical approach and the numerical approach.
I focus on numerical (computer) simulations of propagation of waves through the interface between two nanoscale media with different properties.
Numerical simulations are very useful for complex cases when analytical solution can't be obtained.

My research areas are closely related to other fields of science, for example, computer science.
So for me it is very important to explore new methods and trends in computer science.
These new methods can help to improve and to accelerate computer simulations of mechanics and physics processes.

So the first prospect in my scientific career is to study well know and new analytical solutions of waves propagation in nanoscale medias.
The second prospect is to create new numerical methods or to use already existing methods to solve new problems in mechanics.
And the third prospect is to discuss results with scientific community.
And possibly these results will be useful not only for science, but also for high-tech industry.
\\

\noindent{\textit{(Перспективы моей научной карьеры)}}

\newpage


\setcounter{subsection}{2}
\sublinksectionold{My scientific interests and my academic activity}

My research interests include mathematical modelling of physics processes, numerical algorithms for complex partial differential equations systems, and high performance computing.
I was majoring in Mathematical Modelling for Oil and Gas Industry.
During my master's research, I worked on development and implementation of numerical algorithm for water-induced hydraulic fractures growth modelling.

But now as a postgraduate student, I have the opportunity to work with my academic adviser on modelling of propagation of waves through the interface between two nanoscale media.
And my current research focuses on the implementation of effective numerical algorithms and comparison of numerical results with analytical solutions.

More generally, I am interested in parallel algorithms for scientific computing and particle methods for Theoretical Mechanics problems.
Besides academic research, I also enjoy presenting and demoing the Python programming language, especially scientific visualization, data analysis and 3D visualization, particularly in its applications to computer-aided engineering (CAE) and Theoretical Mechanics.
\\

\noindent{\textit{(Мои научные интересы и научная деятельность. Моя кафедра)}}

\newpage


\setcounter{subsection}{3}
\sublinksectionold{Ethical problems of modern science}

The rapid development of modern science raises a lot of ethical problems, such as the usage of nuclear power, animal testing, the usage of genetic engineering, military issues, and so on.
But exponential growth of scientific publications also leads to some ethical problems, such as plagiarism, the usage of false or unproven statements, three g's (ghost, guest, gift authorship) problem.
I would like to focus on the three g's problem because this problem includes three subproblems which are now actively being discussed in the scientific society.

The first problem is the guest authorship when somebody who didn't conduct the specific research is included in the list of authors because he is a well-known successful scientist.
And only the fact that the new article is published by him attracts attention of thousands of scientists to this article.
I think that this problem can be successfully addressed by adding well-know author not to the list of authors but to the list of reviewers.

The second problem is the gift authorship when the team of researches give a gift to somebody and add him to the list of article authors.
In other words, the team of researches want to help the person to achieve some goals in the rating or citation index. I think that this problem can be somehow addressed by adding information about each author experiments and results to the article.

The third problem is the ghost authorship when somebody who took part in the research isn't included in the list of authors. His name is not present in the list but the result of the work and some piece of the text of the article are written by this author.
I think that if team don't want to include this person to list than this person won't be working with this team anymore.
But if it is an agreement between the team and the person it's strange and it is difficult to be addressed.

So three g's problem is not directly related to my field of research but still is related to the problem of conducting modern scientific research and publishing the results of research.
\\

\noindent{\textit{(Вопросы научной этики и гражданской ответственности учёных)}}

\newpage


\setcounter{subsection}{4}
\sublinksectionold{University as a science centre}

Universities are often considered to be one of the most important science centres in the world.
One of the key roles of university as a science centre is to create an environment of intellectual curiosity and collaboration.
Through their research programs, universities bring together experts from a wide range of fields to work together on interdisciplinary projects that can lead to new discoveries and breakthroughs.

As the example, the scientific group of the Higher School of Theoretical Mechanics, where I am studying now, discovered a new physical phenomenon of "<ballistic resonance">, where mechanical oscillations can be excited only due to internal thermal resources of the system.
The discovered phenomenon describes that the process of heat equilibration leads to mechanical vibrations with an amplitude that grows with time.
The effect is called ballistic resonance.
These discoveries also provide an opportunity to resolve the paradox of Fermi Pasta-Ulam-Tsingou (when oscillations in the chain of particles connected by springs first almost decayed, but then revived and reached nearly the initial level).
For mechanicians and physicists, this experiment is vital because a chain of particles connected by springs is a good model of crystal material.
According to experts, the theoretical approach proposed by scientists of SPbPU demonstrates a new approach to how we understand the heat and temperature.
It may be fundamental in the development of nano-electronic devices in the future.

Universities are an essential part of the scientific community, providing a platform for research, education, and innovations that helps to drive progress and improve our understanding of the world around us.
\\

\noindent{\textit{(Университеты как научные центры. Ведущие научные школы в моей области знаний)}}

\newpage


\setcounter{subsection}{5}
\sublinksectionold{International and Russian programs of support of young scientists}

Support programs for young scientists are very important for fostering innovation, collaboration, and research development.
Grants and scholarships give the opportunity for scientists to continue their research.
If there are not any scholarships in some countries, the scientists in this countries will spend all time working at job.
And consequently there won't be enough time to conduct any research or experiments.
So grants and scholarships play a very significant role in the modern science development.

Unfortunately, nowadays we don't have any opportunity to participate in any international events.

Russian Science Foundation provides funding for scientific research across various disciplines.
Many Russian universities have their own programs to support young scientists through grants, mentorship, and research facilities.
It is really pity that I didn't take part in any of them previously.
I wish I would participate to some grant programs, because this programs really would raise the value of my research.

I found some information about the contest funded by Russian Science Foundation but I should emphasise that this contest is not for single individuals but for small scientific groups of two or even four scientists.
And there is a certain list of requirements.
For example, this contest is carried out only for young scientists who are younger than 39 years old.

So it is very important and essential to participate in the contests while being a young scientist.
\\

\noindent{\textit{(Международные и российские программы поддержки молодых учёных)}}

\newpage

\setcounter{subsection}{6}
\sublinksectionold{The retrospective (the historical background) of my field of knowledge}

The theoretical mechanics history is very rich and spans several centuries, reflecting the evolution of scientific thought.

Early ideas about motion and forces can be traced back to ancient Greece.
Philosophers like Aristotle proposed their views of motion and the nature of objects but their views were often qualitative rather than quantitative.

The earliest quantitative formulation of classical mechanics is often referred to Newtonian mechanics.
It consists of the physical concepts based on the 17th century foundational works of Isaac Newton,
and the mathematical methods invented by Gottfried Wilhelm Leibniz and Leonhard Euler to describe the motion of bodies under the influence of forces.
And methods based on energy and momentum were developed leading to the development of analytical mechanics which includes Lagrangian mechanics and Hamiltonian mechanics.
These advances are used in all areas of modern physics.

Computational methods have transformed theoretical mechanics.
Simulations of complex systems that are analytically intractable can be made numerically.
Techniques like finite element analysis (FEA) and computational fluid dynamics (CFD) are widely used.

In conclusion I would say that classical mechanics provides accurate results when studying objects that are not extremely massive and have speeds not approaching the speed of light.
But for a vast majority of engineering problems classical mechanics methods can be used to get a solution with enough precision or accuracy.
\\

\noindent{\textit{(Наука в исторической перспективе)}}

\newpage

\setcounter{subsection}{7}
\sublinksectionold{The current state of my field of knowledge}

The current state of theoretical mechanics is a composition of classical mechanics, advanced mathematical formulations, and interdisciplinary approaches that integrate concepts from fields like quantum mechanics, theory of relativity, and computational physics.

Nowadays not only the analytical approaches but also the numerical approaches are used in theoretical mechanics in order to solve complex problems.
For example, molecular dynamics methods can be used to simulate thermal wave propagation in nanostructures like graphene.
The thermal conductivity of graphene is actively discussed in scientific community.
And both ballistic and diffusive thermal conductivity of nanostructures can be described in terms of theoretical mechanics and using theoretical mechanics methods.

Another actively evolved area of theoretical mechanics is topological mechanics which specialises on creation of metamaterials.
Metamaterials are engineered to have properties not found in nature, such as a negative optical index of refraction, one-way vibrational waves, or exotic elastic behaviour.
Structures at length scales as short as a micron can now be fabricated with advanced material processing like 3D printing.

In conclusion I would say that nowadays theoretical mechanics is a dynamic field at the intersection of various disciplines.
And it remains to be a foundation for physics and practical applications across engineering, technology, and beyond.
\\

\noindent{\textit{(Современное состояние науки в моей области знаний / исследований)}}

\newpage

\setcounter{subsection}{8}
\sublinksectionold{The role of English in the field of science}

Over the past 80 years, there has been a steady increase in the proportion of scientific publications written in English and a corresponding decrease in the use of all other languages for global scientific communication.
At first sight, this appears to offer significant gains in efficiency and universality, as scientists everywhere are required to learn, at most, one language in addition to their own.
Many scientific projects involve teams from multiple countries, and it is very convenient to use English for meetings, publications, and presentations.
Publications in English reach a broader audience, making it easier for scientists and the public to access required article.
Also the English language provides a standardised vocabulary for scientific concepts, which is essential for clear communication and understanding across disciplines.

There are, however, drawbacks to this system that may be significantly affecting both the development of global science and its social effects.
Issues of concern include the poor correlation between ability in a second language and scientific ability; the disproportionate burden of language training and technology borne by non-English countries; and a widening gap between the language of science and the languages of government policy and community life.

In conclusion I would say that the English language has become the de facto language of science. And nowadays it is essential for all scientists to speak English fluently.
\\

\noindent{\textit{(Роль иностранного языка в международном сотрудничестве и решении научных проблем)}}


\end{document}
