\documentclass[main.tex]{subfiles}

\begin{document}

\setcounter{secnumdepth}{0}


\setcounter{section}{102}
\linksectionold{Topics}


\setcounter{subsection}{1}
\sublinksectionold{The prospects of my scientific career}

My research areas are Theoretical Mechanics and Mathematical Modelling of Physics processes.
There are two widely used approaches to solve Theoretical Mechanics problems: the analytical approach and the numerical approach.
I focus on numerical (computer) simulations of propagation of waves through the interface between two nanoscale media with different properties.
Numerical simulations are very useful for complex cases when analytical solution can't be obtained.

My research areas are closely related to other fields of science, for example, computer science.
So for me it is very important to explore new methods and trends in computer science.
These new methods can help to improve and to accelerate computer simulations of mechanics and physics processes.

So the first prospect in my scientific career is to study well know and new analytical solutions of waves propagation in nanoscale medias.
The second prospect is to create new numerical methods or to use already existing methods to solve new problems in mechanics.
And the third prospect is to discuss results with scientific community.
And possibly these results will be useful not only for science, but also for high-tech industry.
\newpage


\setcounter{subsection}{2}
\sublinksectionold{My scientific interests and my academic activity}

My research interests include mathematical modelling of physics processes, numerical algorithms for complex partial differential equations systems, and high performance computing.
I was majoring in Mathematical Modelling for Oil and Gas Industry.
During my master's research, I worked on development and implementation of numerical algorithm for water-induced hydraulic fractures growth modelling.

But now as a postgraduate student, I have the opportunity to work with my academic adviser on modelling of propagation of waves through the interface between two nanoscale media.
And my current research focuses on the implementation of effective numerical algorithms and comparison of numerical results with analytical solutions.

More generally, I am interested in parallel algorithms for scientific computing and particle methods for Theoretical Mechanics problems.
Besides academic research, I also enjoy presenting and demoing the Python programming language, especially scientific visualization, data analysis and 3D visualization, particularly in its applications to computer-aided engineering (CAE) and Theoretical Mechanics.
\newpage


\setcounter{subsection}{3}
\sublinksectionold{Ethical problems of modern science}

The rapid development of modern science raises a lot of ethical problems, such as the usage of nuclear power, animal testing, the usage of genetic engineering, military issues, and so on.
But exponential growth of scientific publications also leads to some ethical problems, such as plagiarism, the usage of false or unproven statements, three g's (ghost, guest, gift authorship) problem.
I would like to focus on the three g's problem because this problem includes three subproblems which are now actively being discussed in the scientific society.

The first problem is the guest authorship when somebody who didn't conduct the specific research is included in the list of authors because he is a well-known successful scientist.
And only the fact that the new article is published by him attracts attention of thousands of scientists to this article.
I think that this problem can be successfully addressed by adding well-know author not to the list of authors but to the list of reviewers.

The second problem is the gift authorship when the team of researches give a gift to somebody and add him to the list of article authors.
In other words, the team of researches want to help the person to achieve some goals in the rating or citation index. I think that this problem can be somehow addressed by adding information about each author experiments and results to the article.

The third problem is the ghost authorship when somebody who took part in the research isn't included in the list of authors. His name is not present in the list but the result of the work and some piece of the text of the article are written by this author.
I think that if team don't want to include this person to list than this person won't be working with this team anymore.
But if it is an agreement between the team and the person it's strange and it is difficult to be addressed.

So three g's problem is not directly related to my field of research but still is related to the problem of conducting modern scientific research and publishing the results of research.
\newpage


\setcounter{subsection}{4}
\sublinksectionold{Universities as a science centres}

Universities are often considered to be one of the most important science centres in the world.
One of the key roles of universities as a science centres is to create an environment of intellectual curiosity and collaboration.
Through their research programs, universities bring together experts from a wide range of fields to work together on interdisciplinary projects that can lead to new discoveries and breakthroughs.

As the example, the scientific group of the Higher School of Theoretical Mechanics, where I am studying now, discovered a new physical phenomenon of "<ballistic resonance">, where mechanical oscillations can be excited only due to internal thermal resources of the system.
The discovered phenomenon describes that the process of heat equilibration leads to mechanical vibrations with an amplitude that grows with time.
The effect is called ballistic resonance.
These discoveries also provide an opportunity to resolve the paradox of Fermi Pasta-Ulam-Tsingou (when oscillations in the chain of particles connected by springs first almost decayed, but then revived and reached nearly the initial level).
For mechanicians and physicists, this experiment is vital because a chain of particles connected by springs is a good model of crystal material.
According to experts, the theoretical approach proposed by scientists of SPbPU demonstrates a new approach to how we understand the heat and temperature.
It may be fundamental in the development of nano-electronic devices in the future.

Universities are an essential part of the scientific community, providing a platform for research, education, and innovations that helps to drive progress and improve our understanding of the world around us.
\newpage


\setcounter{subsection}{5}
\sublinksectionold{International and Russian programs of support of young scientists}

\newpage

\setcounter{subsection}{6}
\sublinksectionold{Наука в исторической перспективе}

\newpage

\setcounter{subsection}{7}
\sublinksectionold{Современное состояние науки в моей области знаний / исследований}

\newpage

\setcounter{subsection}{8}
\sublinksectionold{Роль иностранного языка в международном сотрудничестве и решении научных проблем}



\end{document}
