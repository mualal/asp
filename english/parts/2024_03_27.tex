\documentclass[main.tex]{subfiles}

\begin{document}

\linksection{Лекция 27.03.2024 (Смольская Н.Б.)}

Что-то сегодня как-то это, не сильно активно.
Наступает активная фаза весны.
Небо начало голубеть.
Кстати, знаете, как по-английски будет (ну, такой бизнес-сленг) креативное мышление.
Не просто дословно creative thinking (как вы могли бы подумать), а \textbf{blue sky thinking}.
Это идиома.
Можно догадаться, почему так.
Это я blue sky вспомнила.
So we are in \textbf{the blue sky thinking}, all of you.


\sublinksection{Небольшое дополнение о сослагательном наклонении}

Well, probably, well, I hope.
So, the Subjective Mood is a very interesting and quite troublesome aspect.
And I hope that you have compared your answers with the correct answers that I've given.
There are some examples or some moments that can be problematic, but they are taken from the original texts.
And when comparing your answers with that letter that is correct, you just pay attention to it and just take it as it is.
I mean, these are the answers to the tasks.
\\

\hypertarget{ltask:2024-03-20-answers}{--- Предложения, о которых идёт речь ---} (\hyperref[task:2024-03-20]{\color{blue}{перейти к полному тексту задания}})
\\

\begin{enumerate}[nosep,leftmargin=*]
	\itemsep\eitsp
	\item[11.] The principle of symmetry with respect to spatial translation states that an experiment conducted in Alma-Ata \uline{\textbf{would give}} the same results when it is repeated 8,000 kilometers away in Grenoble.
		
		((A) will give; (B) gave; \textbf{(C) would give}; (D) had given)
	\item[18.] In the 1980s, countries like Germany or Switzerland were praised for keeping inflation in the 4\% to 6\% range, a range that \uline{\textbf{would have been viewed}} as high in 1950s.
		
		((A) would be viewed; \textbf{(B) would have been viewed}; (C) would view; (D) has been viewed)
\end{enumerate}
\ 

Sorry, I will point them down.
Well, it is actually 18 that I asked you to think about.
And if I'm not mistaken, it is 11, which somehow can lead to some difficulties in actually finding a correct answer.
But the answer that is given as the correct one is correct, because for the example number 11 it is (C).
It is "<would give"> (not "<will give">).
The same with 18 where the correct answer is still "<would have been viewed">.
And the thing is actually that these are the so-called, that we have discussed with you (or we discussed last week, because there should be Past Simple in this case).
This is the so-called analytical form of the Subjunctive Mood, and you remember that one of the meanings of the Subjunctive Mood in the English language is our presupposition, so our guesses.
And that is why using...
So in this case we actually go from the form to the meaning.

То есть в этом случае (особенно вот эти примеры) нам показывают, что не всегда строгие грамматические правила в английском языке играют роль, хотя чаще всего мы действительно руководствуемся теми правилами, которые в английском языке существуют, но иногда использование определенных форм для слушателя даёт значение.
И используя в данном случае...
Но если в 18 там еще можно понять почему "<would have been">, то вот вопрос например с 11 (почему там "<would give">, а не "<will give">) может возникнуть, хотя мы там можем предположить, что скорее всего там "<will give"> должно быть, если мы просто пользуемся тем, что это предложение с придаточным предложением временным.
Но поскольку у вас здесь задание было всё-таки связано с разными случаями использования сослагательного наклонения, то ваша задача была из тех форм, которые даны (а у вас даны "<will give">, "<gave">, "<would give"> и "<had given">) выбрать всё-таки форму сослагательного наклонения.
"<Had given"> и "<gave"> туда не подходят просто вообще по определению (по всем грамматическим показателям)
По сути дела вы должны были бы выбирать между "<will give"> и "<would give">, но здесь вот именно нужно идти от того, что хотел бы сказать говорящий, и то, что вы ограничены сослагательным наклонением
Вот в источнике, откуда взят этот пример (коллеги выбирали и подбирали разные примеры -- это такой сводный тест) -- там используется именно форма "<would give">, и это предложение таким образом даёт нам пример того, что авторы хотели показать, что всё-таки это предположение (это не полная уверенность).

Если мы поставим "<will give"> (вы смотрите на это одиннадцатое предположений, да?) -- если мы поставим "<will give">, то значит, что это практически наша уверенность в будущем действии.
А вот "<would give"> -- сослагательное наклонение.
Мы с вами с этого начали, когда на позапрошлом занятии начали говорить про сослагательное наклонение.
Оно привносит значение предположительности, то есть неполную уверенность.
И вот в этом предложении в источнике в исходном варианте будет ответ "<would give">
Естественно на тестах (в тестах, когда тесты грамматические, которые выносятся на экзамен) нет таких двузначных или неоднозначных заданий.
То есть здесь даже не переживайте, потому что вы сами понимаете, что я вам дала правильные ответы специально (сознательно), чтобы вы посмотрели, удивились и задумались (не думая, что там неправильная буква стоит в качестве правильного ответа).

То же самое и с 18-м.
Потому что если мы с вами задумаемся о ситуации...
Так достаточно большое предложение -- то, что я просила вас дома подумать и задуматься.
Задумывались ли вы над этим предложением?
Я надеюсь, что может кто-то подумал и постарался осознать.
И здесь тоже хоть и по отношению к прошедшему времени, но это Subjunctive II, имеющий отношение к прошлым временам -- тем более, что у нас есть указание in 1950s.
Но опять же использование формы "<would have been viewed">.
А по логике конечно только она здесь могла бы быть использована, даже если мы просто абстрагируемся от сослагательного наклонения.
Но еще можно подумать насчет "<would view"> как Future in The Past, но нам требуется здесь пассив, поэтому можно было бы до правильной формы дойти таким путем.
Но, тем не менее, использование сослагательного наклонения, а не просто какого-то там перфекта, здесь именно опять подчеркивает предположение, потому что сослагательное наклонение...
Здесь надо просто вот понять это для себя, как бы принять это как данность, что английский язык позволяет использовать определенные грамматические феномены (это не часто бывает, но тем не менее вот они такие примеры), привнося в предложение те значения грамматических феноменов, которые им присущи.
А мы еще раз с вами, еще раз повторю, мы с вами говорили о том, что сослагательное наклонение -- это значение предположения, возможности, желания.
И поэтому использование этих форм (особенно со вспомогательным глаголом would) это и передаёт.
Поэтому это вот то, что касается этих моментов.

Я может быть подгружу (просто если у вас когда-то будет время свободное) туда в курс задания, которые касаются правильного выбора Conjunctions (союзов), потому что мы с вами уже столкнулись, анализируя в табличке примеры (если вы в таблице на это обратили внимание) и вот с этими примерами, что правильное использование союзов и их подбор тоже в английском языке, в общем-то, значим.
Потому что союзы должны использоваться правильно, потому что союзы, как и другие формальные показатели в английском языке, тоже являются обязательными маркерами определенных грамматических явлений.
Поэтому я, может быть, на союзы какое-то небольшое задание тоже туда в курс погружу.
Если у вас будет время и желание, то вы можете так сказать optionally его сделать.

\newpage
\sublinksection{A summary and an abstract}

Вот я помню, что на сегодня я просила вас подумать и для себя суммировать топики 4 и 5.
Мы к ним чуть-чуть попозже перейдем, так как сейчас у меня будет для вас задание.

А сейчас мне бы хотелось...
Ну в данном случае, поскольку мы все гаджетами, собственно говоря, активно пользуемся, поэтому сейчас такое задание.
To surf the net and to give me the answer: what is a summary?
Just make sure that you understand it.
What is a summary?
And what is the difference between a summary and an abstract?
And this is the task just for some five minutes for you to find, to actually consult the sources of the internet and then to give me the correct answers.
So the first point: what is a summary?
The second point: what is the difference between a summary and an abstract?
And the third point is what are the most frequently mentioned features or specific features of a summary?
So what should we keep in mind from the point of view of the language of course?
What are the linguistic (specific linguistic) features of a summary?

Что такое summary?
Какова разница между summary и abstract?
И каковы лингвистические специфические черты, которые присущи summary?
То есть, что мы должны учитывать, когда мы готовим summary?
Информации очень много, мне просто хочется, ну, и, конечно, мы это по-английски с вами постараемся сформулировать...
\\

--- Выполнение задания ---
\\

Okay, so I hope that you are ready.
I will just open the file.
So, the first question is, so what is a summary?
How would you define this piece of either writing or speaking?
So what is it?
I think it is a brief of main points of book, article, or some sort of creative work, we can say, because we can actually treat any work, whether it is a written form of some information, a written piece, or an oral piece as, I don't know, it can be a movie, for example, because you can also summarize a movie, so it's also possible.
So it is a shortened brief and shortened story.
A shortened and brief form.

Okay, so what is the difference then of a summary and an abstract?
An abstract is maybe a reflection, 100\% reflection, of course, a shortened version of reflection of something.
Maybe it's an article or statement or something.
While the summary highlights the main points of the source.
The summary is more detailed than abstract.
Okay, what else?
The abstract is more formal.
Sorry?
The abstract is more formal, it's just keywords, maybe.
Main states, not main but all states from the...
So, I still do not...
Yes?
Summary is (let's say) a document itself.
But an abstract is a part of an article.
Of an article, right.

But if we compare it from the point of view of the addressor, of the author of the summary, in its relation with the original work.
So what is it?
What is the main difference?
The abstract was written by the author. Yes.
This is the most important thing that actually I wanted to hear from you.
That an abstract is the first person shortened form of a work.
So the author him or herself actually presents this shortened brief form of a work.
And when we speak of a summary...
So in this case we are actually the readers who after that work with this work (sorry for this much of muchness) and present this shortened form which is called a summary.

So in Russian we use different words.
So how would you actually translate (or give the Russian equivalent) of this type of work?
Аннотация, конспект, пересказ, реферат.
Да, да.
В русском языке, по сути дела, множество вот этих мини-жанров, которые, по сути дела, соотносятся с английским понятием summary.
В русском языке во многом, может быть, это зависит с original work, то есть с тем, что мы, собственно говоря, реферируем, аннотируем и так далее.
И по отношению к этой работе мы будем по-разному называть, хотя часто, конечно, аннотация это эквивалент abstract, да, то есть здесь других вариантов нет.
Если мы говорим про abstract в английском языке, то это только аннотация.
Если же это summary, то здесь возможны варианты, и чаще всего зависит от того, что мы в результате вот реферируем, реферативно аннотируем, да?
Потому что у вас (в вашем экзаменационном задании) как раз и называется аннотирование, да?
То есть вы должны briefly present the contents of the article that you will work with.
Итак, вот это для нас просто важно понимать, и чтобы вы, так сказать, на перспективу, когда вас просят что-то сделать, не перепутали бы одно с другим.
Потому что в зависимости от того, что мы с вами делаем, summarize or work out an abstract, от этого зависят и те лингвистические средства, которые используются, не только средства, тут даже не столько средства, потому что средства-то как раз будут одинаковые, а конструкции и формулы, которые используются, соответственно, в одном или во втором типе научных текстов, потому что это, конечно, научные, да, это не художественные тексты.
То есть это для нас тоже важно понимать.

Поэтому, ну, не сегодня, а уже на следующих наших с вами занятиях мы уже как раз вот обратим с вами внимание именно на те конструкции, которые, по сути дела, могут стать для вас, для вас костяком при подготовке Summaries.

Опять же, я вот, когда с аспирантами общаюсь, я вам уже об этом говорила, что, по сути дела, ваш кандидатский экзамен -- это полная 100\% подготовка, если вы действительно будете к нему готовиться.
Если же это, так сказать, в каком-то смысле пускается на самотёк, то тут уже сложнее.
Но, просто потрачу 3 минуты и проговорю, что я имею в виду.

Первое задание, да, письменный перевод.
Чем быстрее вы начнёте со мной согласовывать ваши статьи, тем быстрее вы сможете с ними начать работать, по сути дела, потому что это всё-таки не огромная монография.
Это статьи, которые можно прочитать, но при желании их можно и перевести для себя, сколько бы это у вас времени не занимало, времени достаточно.
То есть первое задание вы можете хотя бы приблизительно, но всё-таки подготовить.

Второе задание -- это аннотирование этих же статей.
То есть со статьями вы уже: а) готовы, б) мы с вами обсудим, что какие формулы (или формульные выражения) вам нужно использовать при подготовке summary.
Я обычно рекомендую каждому аспиранту в отдельности выработать для себя некий стандартный перечень вот этих формул (формульных начал): the aim of the article, the structure of the article is the following, предположим, да?
То есть вы для себя выстраиваете вот эти начала предложений, а дальше, по сути дела, вы только добавляете вторую часть из каждого отдельно взятого article.
По сути дела, если у вас есть желание и достаточно времени, то дома, ну вернее, в свободное от учёбы и работы время, вы можете продумать, а вообще даже и подготовить, и написать для себя summary по каждой статье.
Поэтому, когда вас уже на экзамене спросят, при желании можно и...
Я не люблю это выучивание наизусть, но тем не менее это можно сделать.
По каждой статье заранее summary подготовить, потому что мы не скрываем, что мы заслушиваем summary по каждой статье.
Для нас принципиально важно, чтобы вы это сделали, чтобы вы знали, как делается summary.
Потому что есть работы, даже та же самая ваша диссертация, когда действительно требуется, ну, либо какой-то кусок из научной работы summarize и вставить по-русски или в вашу статью англоязычную вы точно также можете какие-то отрывки из какой-то работы кратко излагать.
Это та же самая часть, это та же самая вернее работа summary.
То что вы и делаете.
Поэтому получается, что второе задание вы тоже можете заранее подготовить, ну а про третье я вообще не говорю уже.
Третье -- это ваши топики.
То есть по сути дела, если готовиться, ну, хотеть подготовиться, то это всё можно сделать.
Это не вступительный экзамен, где вы не знаете, какой текст вам будет дан первый, какой был дан первый, какой вам попадёт второй, и тем более эти вот предложения на перевод.
Это я вам раскрываю такие тайны мадридского двора.
Между нами.
Не передавайте никому эти такие tips.
Тем не менее...

Итак, so my third question for you...
So I hope you understand now (just for you, for your future work, for your future search work)...
It was just for you to understand what is the difference between an abstract and the summary.
So a summary is in the focus of our interest and attention.

So what are the main linguistic means that are used to correctly write or orally prepare a summary?
So what should you keep in mind?
Or linguistic approaches from the point of view of this kind of work or type of work.
Кто готов?
Если вы нашли на русском, то можно по-русски.
So, what can you say?
Have you found anything interesting?
Any ideas?
What should you use to correctly prepare a summary and not to be offended or not to be actually, not to be told by the author that your summary is a piece of plagiarism?

Well, one of the points that I stumbled upon while surfing is that you should paraphrase.
Yes.
One of the foundations of the summary is not to use exact words and...
Exactly.
So, actually, this is the case that I wanted to hear from you, or this is the means that I wanted to hear from you.
Timur, you are absolutely right and correct that will be the next point of our work but before we switch to it I would like you to maybe point out some other things.
So that was the piece that we have already discussed with you in order to correctly work out a summary, so of course the first step, and this is what I was talking about, is to actually be familiar with the content.
So that is why when preparing for your examination you should definitely, well at least read the articles that you have chosen for your examination.
As you can see, paraphrasing is one of the most important approaches and aspects and means that should be used when you prepare your summary.
As for these examples, as it is said, or some theoretical points that you should point out, that will be the next point of our discussion.
And the very definition that you have started with, that when preparing a summary, you should кратко и сжато, meaning that it is really a shortened and brief form of the original material.
It is difficult.
Of course, paraphrasing in this case is not the matter that can help you to brief and to shorten.
These are different methods and different linguistic means and this is our next step.

But for today, I would like us to work with paraphrasing.
And I have prepared, this is just because I was thinking of how to arrange it, you have prepared it from the internet (вы сами уж нашли всю необходимую информацию в интернете, когда выполняли предыдущее задание).
So, paraphrasing practice.
Итак, просто мы не работали с вами с этим типом упражнений, но он вполне понятен.
Итак, здесь собраны 13 предложений (я их вытащила из разных вариантов).
То есть, ваша задача -- прочитать предложение, дальше, ну, где-то 4 есть, где-то 3 альтернативы, и выбрать ту, которая с вашей точки зрения максимально соответствует смыслу.
Здесь либо грамматическое перефразирование, да, участвует в этом, либо смысловое.
Но ваша задача, то есть, по сути дела, это задание -- это то, чем вы должны пользоваться.
И когда вы summary делаете, но summary -- это только начало.
Почему мы берем summary для экзамена?
Просто потому, что у нас не так много времени.
Но, тем не менее, на summary вы тоже можете отработать те самые моменты, которые помогают вам избегать плагиата.
Потому что перефразирование, да?
Перефразирование -- это не рерайтинг.

Мы вот с аспирантами обсуждаем разницу между рерайтингом и перефразированием в чём?
Потому что вы же знаете, что есть услуги рерайтинга.

Итак, в чем разница между рерайтингом и перефразированием?
Переписывание и перефразирование.
Так в чём разница?
Переписывание может использовать те же слова?
Да, при рерайтинге, когда заказывают рерайтинг, рерайтингом занимаются не специалисты в области.
То есть, по сути дела, рерайтинг может сделать лингвист, который понимает, как по-разному перефразировать.
Вот у нас не везде с вами вот эти примеры (в этом упражнении), конечно, названы paraphrasing, но больше это будет рерайтинг.
То есть когда использовать разные грамматические структуры, но почти не изменяя самих слов, самих лексем.
Из актива сделать пассив, поменять структуру предложения, там, изменить придаточные предложения, тем самым поставив там разные какие-то, ну, последовательность частей предложения, то есть в данном случае -- это грамматика.
А перефразирование, конечно, может делать только специалист, потому что при перефразировании уже, что вы будете использовать при перефразировании?
Синонимы для понятий, а для этого нужно знать терминологию, либо дефиниции понятий при перефразировании, когда вы сам термин какой-то описательной фразой переписываете.
Может быть, добавлять факты.
То есть в этом весь смысл, чтобы правильно понимать...
То есть при работе с научными текстами вы должны учитывать эти возможности перефразирования (paraphrasing).

\newpage
\sublinksection{Paraphrasing}

Я на сайте PORTASP потом сделаю отдельный раздел по summarizing и всё, что касается summarizing, у меня будет сведено вместе.

Поработайте сейчас с этими 13 предложениями и соответсвенно вашими 13 вариантами, которые вы считаете будут соответствовать исходным предложениям.
\\

\hypertarget{ltask:2024-03-27}{--- Выполнение задания ---} (\hyperref[task:2024-03-27]{\color{blue}{перейти к тексту задания}})
\\

\textbfind{Выполненное задание (подчёркнуто самое близкое по смыслу с исходным предложение)}
\vspace{5pt}
\begin{enumerate}[nosep, leftmargin=*]
	\itemsep15pt
	\item \textbf{Had the announcement been made earlier, more people would have attended the lecture.}\newline
		A) Not many people came to hear the lecture because it was held so late.\newline
		B) The lecture was held earlier so that more people would attend.\newline
		C) Fewer people attended the lecture because of the early announcement.\newline
		\uline{D) Since the announcement was not made earlier, fewer people came to hear the lecture.}
	\item \textbf{No one except the graduate assistant understood the results of the experiment.}\newline
		A) All of the graduate assistants understood the experiments.\newline
		B) The experiments were not understood by any of them.\newline
		\uline{C) Only the graduate assistant understood the experiments.}\newline
		D) All but one of the graduate assistants understood the experiments.
	\item \textbf{Travelling on one's own is often more expensive than taking a guided tour.}\newline
		A) An expensive guided tour costs more than travelling on one's own.\newline
		B) Travelling on one's own costs less than taking a guided tour.\newline
		\uline{C) It costs less to take guided tour than to travel on one's own.}\newline
		D) Because guided tours are expensive, they cost more than travelling on one's own.
	\item \textbf{While attempting to smuggle drugs into the country, the criminals were apprehended by customs officials.}\newline
		A) Attempting to smuggle drugs into the country, customs officials apprehended the criminals.\newline
		B) Criminals who were attempting to smuggle drugs into the country apprehended customs officials.\newline
		\uline{C) Customs officials apprehended the criminals who were attempting to smuggle drugs into the country.}\newline
		D) Smuggling drugs into the country, customs officials attempted to apprehend the criminals.
	\item \textbf{The manager is sure to have found the right man for this position.}\newline
		A) The manager has no doubts that he has found the right man for this position.\newline
		\uline{B) No doubt, the manager has found the right man for this position.}\newline
		C) No doubt the manager will find the right roan for this position.\newline
		D) The manager is not sure that he has found the right man for this position.
	\item \textbf{The paper to be discussed at the seminar is concerned with new facts about corrosion.}\newline
		\uline{A) The paper on new facts about corrosion will be discussed at the seminar.}\newline
		B) The paper on new facts about corrosion was discussed at the seminar.\newline
		C) The paper on new facts about corrosion is being discussed at the seminar.
	\item \textbf{If the meter had not failed, we should have made all the measurements required.}\newline
		A) The meter was in order and we made all the measurements required.\newline
		\uline{B) The meter was out of order and we failed to make all the measurements required.}\newline
		C) We shall be able to make all the measurements required provided the meter doesn't fail.
	\item \textbf{Henry must have all devices checked before starting his work.}\newline
		A) It's Henry's duty to check all the devices before starting his work.\newline
		\uline{B) The lab assistant must check the devices before Henry starts his work.}\newline
		C) It's quite probable that Henry checked all the devices before starting his work.
	\item \textbf{More money was allocated for industrial research than for any other item in this year's budget.}\newline
		A) The allocation of less money for research than for industrial items occurred in this year's budget.\newline
		B) This year we allocated more money for other items in the budget than for industrial research.\newline
		\uline{C) We allocated more money for industrial research than we did for other items in the budget this year.}
	\item \textbf{Federal funds will not be made available unless the governor declares a state of emergency.}\newline
		A) There is a state of emergency because the governor has not received any federal funds.\newline
		B) Since no federal funds are available, the governor will have to declare a state of emergency.\newline
		\uline{C) If the governor declares a state of emergency, federal funds will be made available.}
	\item \textbf{Maxwell is said to have applied the ordinary laws of mechanics to molecules.}\newline
		\uline{A) They say, Maxwell applied the ordinary laws of mechanics to molecules.}\newline
		B) Maxwell said that he had applied the ordinary laws of mechanics to molecules.\newline
		C) Maxwell said that he applied the ordinary laws of mechanics to molecules.
	\item \textbf{The head of the department wishes his team had started the experiment by September.}\newline
		A) The team had started the experiment by September.\newline
		B) The team had to start the experiment by September.\newline
		\uline{C) The team had not started the experiment by September.}
	\item \textbf{Having signed the papers the manager called the secretary in.}\newline
		\uline{A) The manager called the secretary in after he had signed the papers.}\newline
		B) The manager called the secretary in before signing the papers.\newline
		C) The manger called the secretary in while signing the papers.
\end{enumerate}
\ 

So let's begin.
So let's do it just one by one.
Okay?
So, first of all, I will then ask the question.
I will clarify the way I would like us to work with the original sentence and the paraphrase (or the sentence having similar meaning).
So would you please read the original sentence and then your choice (the sentence that you have chosen) to express the meaning?

Первое. Had the announcement been made earlier, more people would have attended the lecture.
Good, so what is the sentence to choose?
(D)?
Yes, would you please read it?
Since the announcement was not made earlier, fewer people came to hear the lecture.
Good, since the announcement was not made earlier, fewer people came to hear the lecture.
So what do we have actually in the original sentence?
So what about the original sentence?
What type of sentence is it?
What is it?
Conditional.
Yes, that's conditional.
And when we speak of conditionals, we remember that we have four different types.
And this is type three.
It is what we say that's the unreal condition taking place or that took place in the past.
What was the paraphrasing mechanism or the rewriting mechanism?
So, we will actually discuss when we have the rewriting and when we have the paraphrasing result.
So what was used here in order to correctly change the sentence or in order to work out the sentence which has the same meaning?
What do we have here?
Same sequence of events?
Yes, but what?
What is the meaning, actually, of the sentences using or having conditional three?
What do they express?
Сожаление (regret) о прошлом?
Regret about what?
О прошлом.
Yes but the sentence actually presents what?
When the sentence is in conditional three, so it presents, let's say, the situation which is opposite to what had really happened.
So, in this paraphrasing (paraphrased sentence) we actually have the original situation, right?
Because fewer people came because the announcement was not made earlier, right?
So the original sentence actually presents us the situation that took place in the past.
And this is actually what we use third conditionals for.
What is the function?
You are absolutely correct when expressing, let's say, if I can say this such a word, the philosophical meaning of conditional three.
Because we really express our regret about what had happened in the past, but we express it by means of actually presenting what we would like to have happened, but we didn't have it.
So we just express it when we change the situation from top to bottom, something like that.
Okay, so this is, we can say that in this case we really have the instance or the case of paraphrasing, because if you use something like this in your scientific works, for example, that will be definitely a case of (or an example of) paraphrasing.
То есть в этом случае антиплагиат точно не считает никакого плагиата, хотя по сути дела выражено то же самое.
То есть здесь у нас перефразирование: грамматическое изменение предложения, но при этом ещё и (ну скажем так) контекстуальное, интересный такой механизм, то есть мы об этом не задумываемся.
Я уверена, что вы этим пользуетесь, но при этом, когда вот анализируешь, начинаешь понимать и начинаешь подходить к этому более, так сказать, практично ко всем вот этим моментам.

Окей, предложение number two.
No one except the graduate assistant understood the results of the experiment.
So what is the correct answer? I think it's (C).
Yes, that's C.
Only the gradient assistant understood the experiment.
Good.
So what do we have here?
So what was changed and what do we have here?
Well, if we compare the sentence, it looks like absolutely the same.
So what do we have here?
Negation?
В первом случае отрицательное предложение.
So what was changed?
What kind of changes do we have here?
At what level?
"<Никто, кроме"> и "<только">.
То есть на уровне каком?
По сути дела, на уровне лексики.
Мы поменяли фразу "<никто, кроме"> на "<только">.
Это действительно синонимичные фразы (ну, в одном случае слово, в другом случае словосочетание), которые действительно выражают одну и ту же идею.
Понятно, что, конечно, для плагиата это будет слишком, так сказать, очевидно, но тем не менее акцент немножко по-другому стоит, потому что, как вы абсолютно правильно сказали, что в первом случае у нас предложение отрицательное.
Помните, почему оно отрицательно?
Потому что "<no"> стоит при подлежащем.
Помните, мы в самом начале с вами говорили, что в английском языке синтаксическая структура и синтаксические requirements, требования английского языка таковы, что предложения имеют статус отрицательного, тогда, вернее, они квалифицируются как отрицательные, если отрицание стоит либо при подлежащем либо при сказуемом.
В нашем случае это действительно отрицательное предложение, потому что "<no one">.
Но "<no one except"> в данном случае -- это сложное подлежащее, потому что подлежащее мы должны с вами рассматривать "<no one except the graduate student assistant">.
Вот это всё сложное подлежащее, и оно отрицательное.
Но стоит нам только поменять в данном случае это сложное образование (конструкцию) на всего лишь усилительное наречие "<only">, и предложение становится утвердительным с усилительным наречием "<only">.
Но при этом мы видим здесь действительно изменения.
Да, это rewriting.
Вот это пример rewriting.
Это не paraphrasing, а рерайтинг -- это достаточно (как это сказать) зыбкая почва для научных работ, то есть это только если можно позволить себе уже в аспекте какого-то значимого процента оригинальности (ещё немножко почистить и свои авторские моменты внести).
Но тем не менее мы с вами наблюдаем, что акцент немножко по-другому уже поставлен, тип предложения другой.

Окей, предложение number three.
Travelling on one's own is often more expensive than taking a guided tour.
Good.
Вариант (C).
It costs less to take guided tour than to travel on one's own.
How would you comment upon the changes in this sentence?
So what is changed here?
Давайте проанализируем с точки зрения грамматики.
Можно по-русски?
Давайте, я ведь тоже, я поняла, что по-английски сложно.
Мы поменяли местами главное и придаточное предложения.
Здесь нет главного и придаточного!
Точнее, не главное и придаточное, а...
Ну, как бы, можно сказать, смысловые блоки, которые в главном предложении у нас выражены чем?
Английский язык, кстати, с точки зрения грамматического рерайтинга и грамматического перефразирования гораздо более богатый, если я могу так сказать, богаче, чем русский.
Потому что давайте посмотрим, что у нас используется.
И это то, что будет вашим домашним заданием (такая преамбула).
То, что у нас используется в исходном предложении.
Что у нас здесь используется?
Герундий.
Travelling и taking.
Смотрим на перефразированное предложение.
Они становятся инфинитивами (to travel and to take), хотя по сути дела передают те же мысли.
То есть герундий и инфинитив -- это разные стороны одной медали с точки зрения выражения действия.
Путешествие или путешествовать.
Все равно речь идет о путешествии, как о процессе.
Это первое.
И что еще интересно?
То что тут уже лексика включается.
Очень хорошо перефразированное...
Вот если в первом предложении (которое разобрали выше) у нас там четко антиплагиат не считает, потому что поменялась структура грамматическая предложения.
А в этом предложении (в третьем) всё чётко.
Во-первых, мы понимаем, что речь идёт о стоимости.
Expensive -- это значит, что речь о стоимости идёт.
Вводится новый глагол (costs).
Прям вообще перефразирование.
И плюс ещё акценты расставлены по-другому.
Вы как правильно сказали, поменялось местами.
В исходном предложении было "<more">.
А в перефразированном "<less">.
За счёт того, что вы поменяли то, к чему относится наша вот эта количественная оценка.
У нас то же самое было и в первом случае (в первом предложении).
Было "<more people would attend">, а там "<fewer people came">.
Но в том случае мы действительно с вами вот это вот грамматическое значение третьего conditional передали, а здесь прям вообще перефразирование, ну не подкопаешься, хотя вот это всего лишь просто по-другому расставлены акценты, то есть нам есть чему учиться.
По-русски тоже интересно это попробовать, но вот я говорю, что английский язык в силу более такой развитой вот именно аналитической грамматики позволяет вот эти вот такие моменты проводить, когда всего лишь поменяли герундий на инфинитив и всё, сразу всё стало по-другому.
В русском тяжелее построить такие предложения.

Так, пожалуйста, четвёртое (number four).
While attempting to smuggle drugs into the country, the criminals were apprehended by customs officials.
Вариант (C).
Yes, would you please read the sentence?
Customs officials apprehended the criminals who were attempting to smuggle drugs into the country.
Что мы здесь? Ну, если так, давайте уже не по-английски, понятно, что можно и по-английски все это анализировать, но вы, так сказать, не лингвисты.
Так, что вы здесь сделали?
Ну, во-первых, мы действительно в главном предложении изменили залог, тем самым, в общем-то, перефразировали, но это грамматическое перефразирование, то есть активный и пассивный залог, -- это действительно тоже, извините за повторение, но тем не менее, это две стороны одной медали, когда мы только субъект или объект ставим на первое место в предложении.
Что еще здесь интересного?
Где еще произошло такое изменение?
Поменялся порядок следования.
И ещё в первом случае у нас есть...
Здесь не совсем, конечно, это придаточное предложение...
Это причастный оборот всё-таки, но тем не менее он вводится всё равно союзом while, а в перефразированном предложении у нас уже становится придаточное предложение, определительное (attributive), потому что вводится who, то есть какие преступники -- "<who were attempting to smuggle.
Смысл тот же самый, понятно, да?
То есть им всё равно приписывается их вот это действие attempting, только по-другому расставлен акцент.
Здесь перефразирование.
Грамматические изменения тоже есть, но при этом уже перефразирование достаточно очевидно.

Так, пожалуйста, пятое предложение (number five).
The manager is sure to have found the right man for this position.
Sentence A?
Нет.
Правильный ответ (B).
Давайте проанализируем почему.
Здесь, забегая вперёд, интересная конструкция, которую мы с вами вспоминали уже.
То, что я называла Complex Subject (сложное подлежащее).
Вы достаточно часто в английском языке, я думаю, и встречались с ним, а может быть и сейчас встречаетесь.
He seems to be active.
He is thought to be a very important person.
В чем смысл этих предложений?
То, что некий взгляд со стороны, который выражается глаголами, то есть he seems to be active -- это он кому-то кажется активным.
Может, на самом деле он и не активен совсем.
Или he is thought to be active -- о нём думают, то есть отношение со стороны выражается сказуемым.
Давайте посмотрим в нашем случае, вот это the manager, ведь это то же самое.
The manager is sure to have found.
Это менеджер уверен, что он нашел?
Нет -- это по поводу него уверены.
Он-то нет.
Может быть, он так не думал, для него это, может быть, было замечательное открытие.
И поэтому, если вы выбрали (A), то тогда получается, the manager has no doubt.
А он-то как раз...
Про него мы ничего не знаем.
Может, он и уверен тоже, но мы не знаем из того предложения, поэтому правильное предложение -- это (B), потому что "<no doubt"> -- это больше разговорное, конечно, предложение, но по своей форме так немножко теряют в красоте английского вот этого выражения.
Потому что сказать the manager is sure to have found -- это, ну простите за пафос, но это очень по-английски, то есть английский язык не любит придаточные предложения, он старается всё в конструкции уводить, то есть чтобы предложение было простым.
А the manager is sure to have found -- это простое предложение, потому что по структуре предложения the manager и is sure это подлежащее и сказуемое.
Просто есть дополнительный смысл, о котором мы с вами поговорим.
Поэтому здесь правильное предложение (B).
Акцент точно так же так и остался (со стороны думают), потому что вводные слова "<no doubt">, а вводные слова в любом предложении, они выделяются всегда запятыми, да, несомненно, определенно, то есть то, что можно изъять.
Почему запятыми выделяется (даже в русском языке выделяется)?
Потому что то, что мы выделяем запятыми, мы с вами об этом говорили -- это то, что можно изъять из предложения, и глобально на содержание предложения это не повлияет на смысл.
Да, какие-то детали уйдут, но тем не менее глобально структура предложения не развалится, если мы уберем то, что в запятых.
И вот у нас тут получается, что если это выделено запятыми, то это некий дополнительный аспект, то есть "<no doubt"> -- это вот эти вводные слова такие авторские.
Поэтому мы проводим здесь вот эту параллель.
Так, ну давайте быстренько заканчиваем.
У меня вопрос.
А как тогда оригинальное предложение сказать, что менеджер уверен?
В смысле оригинальное?
The manager is sure that he has found.
Вот это будет то, что он уверен.
Разница the manager is sure that he has found, потому что the manager и he -- это одно и то же подлежащее.
А вот эта структура, Complex Subject, но более подробно про Complex Subject мы поговорим на следующем занятии, именно подразумевает, что вот это "<is sure"> -- это взгляд со стороны.
Вот пока для себя вы можете пометить; у вас будет ещё, я сейчас скажу чуть позже -- это называется Complex Subject.

Окей, предложение number six.
Тимур, давайте Вы и дальше мы пройдёмся по коллегам.
The paper to be discussed at the seminar is concerned with new facts about corrosion.
It is (A).
Yes, that's (A).
Что мы здесь имеем?
Почему (A)?
Почему не "<was discussed"> и не "<is being discussed">.
Потому что здесь у нас to be discussed подразумевает собой, что это признак, отсылающий нас к будущему.
Как мы на русский переведём исходное предложение?
Как вы переведёте по-русски?
Статья для обсуждения...
Имеется в виду, что всё равно её будут потом обсуждать.
Не сейчас обсуждают, не в прошлом обсуждали.
А статья, которая представлена к обсуждению.
И это the paper to be discussed / the book to be read / the museum to be visited -- вот это предполагает, что эта характеристика относится вот к тому к предмету, действие над котором будет произведено позже.

Так, пожалуйста, седьмое предложение (number seven).
If the meter had not failed, we should have made all the measurements required.
(B)?
Yes, that's (B).
And what do we have here?
Third conditional?
Yes, that's third conditional.
But what else?
It is somehow equal to the first example.
Для тех, кто помнит.
Мы опять имеем описание ситуации, которая уже произошла.
И плюс ещё и некоторые лексические изменения, потому что the meter -- характеристика его какая?
It has failed.
To fail -- сломаться; не только оценку плохую получить на экзамене, но и сломаться.
А в перефразированном у нас еще и to be out of order, да?
Вышел из строя.
То есть плюс ещё лексические изменения.
То есть мы с вами говорили о том, что рерайтинг -- это грамматические изменения.
Хотя в данном случае -- это такой сложный рерайтинг, потому что это грамматический рерайтинг (тоже нужно знать вот эти соответствия грамматических конструкций) и плюс у нас ещё есть и лексическое перефразирование.

Окей, предложение number eight.
Henry must have all devices checked before starting his work.
(C)?
Нет.
Правильный ответ будет (B).
Почему?
Потому что "<must have all devices checked">.
Другим человеком.
I have my hair cut.
I have the walls painted.
То есть действие было произведено третьим лицом.
Да, здесь уже некое такое авторское дополнение в перефразированном предложении (уточнение the lab assistant), но предположим, что если это equipment devices, то явно это не человек со стороны и не профессор (то есть действительно может быть the lab assistant).
В данном случае такое перефразирование с некой вольностью со стороны работающего над этим, добавлен новый некий персонаж, но тем не менее это кто-то третий (the lab assistant), кто должен был проверить эти девайсы прежде чем Генри будет их использовать.
Вот эта фраза "<to have something done">, вернее, не "<to have something done">, то есть в данном случае неправильно, вроде как правильно, но нельзя так эту конструкцию, давайте, просто каждое предложение -- это то, что я вас попрошу делать на следующее занятие, давайте, закончим.

Предложение number 9.
More money was allocated for industrial research than for any other item in this year's budget.
(C)?
Yes, that's (C).
So would you please read it?
We allocated more money for industrial research than we did for other items in the budget this year.
Да, что у нас здесь произошло?
С чем мы тут дело имеем?
"<More money was allocated"> и "<we allocated more money">.
Что здесь?
Passive and Active Voices.
Да, если в первом предложении у нас это более такое научное и официальное предложение, да, поскольку в научной литературе и в каких-то формальных документах subject не так важен, как результат, вернее object, над которым действие производится -- это то, что мы в исходном предложении имеем, но тем не менее, нам никто не запрещает перефразировать и представить себе, что это касается "<we">.
И таким образом мы выстроили активный залог предложения.

Ну и давайте, десятое предложение (number 10).
Federal funds will not be made available unless the governor declares a state of emergency.
(C)?
Да.
Would you please read it?
If the governor declares a state of emergency, federal funds will be made available.
Хорошо, что у нас здесь произошло?
Что мы, во-первых, за предложение имеем?
Conditional?
Да, и в оригинальном, и в перефразированном у нас это Conditional какой?
У нас их четыре.
Zero?
Нет, это не Zero.
Это First.
Zero -- это когда законы физики или какие-то правила, а здесь мы просто с вами размышляем по поводу того, что может произойти в будущем при каком-то условии.
И что ещё?
Что мы в результате делаем, вернее?
We replace "<unless"> with "<if">.
And?
Reverse?
No, and we get rid of negation.
So, actually, in both parts of the sentence we get rid of negation.
Because "<unless"> is negative and "<will not be made"> is negative.
So we omit negation and actually get the same sentence because we know that "<unless"> is the opposite side of "<if">.
But with the negative point.

Так, ну и давайте быстро, там уже раз пришли (пришла следующая группа на занятие).
Мы как обычно, видите, на вас я тренируюсь, а с ними быстрее всё идёт.

Итак, в одиннадцатом правильное перефразирование?
(A).
Да, очень хорошо.
Опять же, тут мы прямо с вами видим этот вопрос о том, что у нас было is sure, помните?
Здесь прям у нас чётко, если вы выбрали (A), то это значит что вы увидели, что Maxwell is said -- это значит they say, да?
Тут это предложение нам прямо подсказывает, как оно должно быть.
(B) и (C) чётко некорректны, потому что как раз ошибочно интерпретируют предложение.

В двенадцатом предложении какой вариант?
(B)?
Нет, правильный вариант (C).
Потому что здесь третий Conditional, вернее, не третий Conditional, а здесь wishes относятся к прошлому.
И поэтому мы здесь и должны с вами это выбрать.
А в (B) же это "<had to start"> -- это тоже по сути "<должны были начать">?
Нет, им пришлось.
Не должны были, а им пришлось.
(B) -- это "<had to">, а это у нас заменитель глагола "<must"> в прошедшем времени.
А (A) почему у нас не подходит?
Потому что по смыслу, потому что мы с вами знаем, что предложение с wish -- это сожаление.

Так, ну и в 13?
(A).
Да, (A).
Ну, я думаю, что здесь понятно, да?
Почему?
Давайте переведем.
Having signed the papers the manager called the secretary in.
Подписав бумаги...
Подписав -- то есть сначала бумаги были подписаны, а потом секретаря позвали.
И теперь ищем, где это предложение, которое так и передаёт, да?
То есть у нас именно в (A) сказано -- видите то, что мы с вами часто обсуждали, но никогда прямо не использовали -- использование Past Perfect, когда Past Perfect нам показывает, используясь в придаточном предложении, что действие произошло раньше.

Итак, ваше домашнее задание.
Но уже остаётся...
Всё равно мы с вами устно поговорим на темы 4 и 5 -- это остаётся пока.
Ничего не прошу, потому что хочу, чтобы мы поговорили.
Вот.

А там же в онлайн-курсе...
Только я не помню, открыто уже или нет -- теория по неличным формам глагола.
Я свела всё вместе в один большой файл, но у вас неделя на то, чтобы его просмотреть, по возможности.
Почитайте, обдумайте, вспомните, что вы знаете, обратите внимание на то, что вы не знаете.
Неличные формы глагола.
В английском языке их четыре.
Это инфинитив, герундий, причастие первое, причастие второе.
И туда же я включила...
Да много, но мы просто будем сразу на практике смотреть, как это используется, чтобы в практическом применении было понятно.
Туда же вставлены еще и конструкции, которые используются.
Причастные обороты, например, герундиальные обороты.
Вот такого плана вы будете использовать.
Всё тогда.

И тема топика остаётся -- четвёртая и пятая темы.
Пока их готовим устно.
Мы пока пообсуждаем, послушаем друг друга и, может быть, кто-то от кого-то чего-то позаимствует.

Если вы написали ваши прекрасные опусы по поводу международных ваших и российских программ, то я их сейчас соберу.

До следующей среды.
Спасибо!

\newpage
\sublinksection{Неличные формы глагола (теория для следующего занятия)}
\label{subsec:impersonal-forms-full-view}

{\parindent25pt\includegraphics[width=0.87\textwidth, page=1,trim={1in 0.8in 0.5in 0.7in},clip=true]{ImpersonalFormsFull.pdf}}\newpage

{\parindent25pt\includegraphics[width=0.87\textwidth, page=2,trim={1in 0.8in 0.5in 0.7in},clip=true]{ImpersonalFormsFull.pdf}}\newpage

{\parindent25pt\includegraphics[width=0.87\textwidth, page=3,trim={1in 0.8in 0.5in 0.7in},clip=true]{ImpersonalFormsFull.pdf}}\newpage

{\parindent25pt\includegraphics[width=0.87\textwidth, page=4,trim={1in 0.8in 0.5in 0.7in},clip=true]{ImpersonalFormsFull.pdf}}\newpage

{\parindent25pt\includegraphics[width=0.87\textwidth, page=5,trim={1in 0.8in 0.5in 0.7in},clip=true]{ImpersonalFormsFull.pdf}}\newpage

{\parindent25pt\includegraphics[width=0.87\textwidth, page=6,trim={1in 0.8in 0.5in 0.7in},clip=true]{ImpersonalFormsFull.pdf}}\newpage

{\parindent25pt\includegraphics[width=0.87\textwidth, page=7,trim={1in 0.8in 0.5in 0.7in},clip=true]{ImpersonalFormsFull.pdf}}\newpage

{\parindent25pt\includegraphics[width=0.87\textwidth, page=8,trim={1in 0.8in 0.5in 0.7in},clip=true]{ImpersonalFormsFull.pdf}}\newpage

{\parindent25pt\includegraphics[width=0.87\textwidth, page=9,trim={1in 0.8in 0.5in 0.7in},clip=true]{ImpersonalFormsFull.pdf}}\newpage

{\parindent25pt\includegraphics[width=0.87\textwidth, page=10,trim={1in 0.8in 0.5in 0.7in},clip=true]{ImpersonalFormsFull.pdf}}\newpage

{\parindent25pt\includegraphics[width=0.87\textwidth, page=11,trim={1in 0.8in 0.5in 0.7in},clip=true]{ImpersonalFormsFull.pdf}}\newpage

{\parindent25pt\includegraphics[width=0.87\textwidth, page=12,trim={1in 0.8in 0.5in 0.7in},clip=true]{ImpersonalFormsFull.pdf}}\newpage

{\parindent25pt\includegraphics[width=0.87\textwidth, page=13,trim={1in 0.8in 0.5in 0.7in},clip=true]{ImpersonalFormsFull.pdf}}\newpage



\end{document}
