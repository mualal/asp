\documentclass[main.tex]{subfiles}

\begin{document}

\linksection{Лекция 22.11.2023 (Смольская Н.Б.)}

\sublinksection{Информация о кандидатском экзамене}

Well, actually I will try to speak English during our classes all the time. 
In case only, in case that it is not necessary, I will switch into Russian.
But, so this is the case, and I also want you to use your English, I would say, to promote your level of speaking skills and listening skills of the English language.
As you understand that during the examination, so we tried during your entrance examination to speak English and we will definitely speak English during your so-called candidate examination.
So we will very quickly discuss the program of your examination that will take place in September.
This is the period when you will pass two candidate examinations.
I mean the examination in the history and philosophy of science and your candidate examination in the foreign language.
So we will discuss the program in Russian so that there won't be any problematic, any, oops, any problematic problems, any problematic issues. Is it okay or shall I enlarge the view forward? I think that's ok.

Итак, программа кандидатского экзамена.
Красным я выделила два первых пункта.
Это то, что как бы такие пререквизиты, которые вы должны будете выполнить для того, чтобы потом перейти к третьему, четвертому и пятому пунктам, которые представляют из себя (я вывешу это не в чате, а где первый курс поступления 2023 года я туда повешу это, просто мы так сказать до тех пор пока со всеми группами не проговорили это мы пока туда это не вывешивали, чтобы просто не создавать дискуссии в чате, так потому что могли бы возникнуть вопросы у тех, кто еще это не слышал и не видел; когда уже все преподаватели со своими группами со всеми обсудят, то можно будет уже туда вот этот документ повесить, чтобы уже у всех он там был и может его там сделают закрепленным сообщением).
 
 Итак, вот третий, четвертый и пятый пункты, это непосредственно сам кандидатский экзамен в аудитории, который вы будете сдавать в сентябре.
 А первый и второй пункты, это то, что вы должны будете выполнить для того, чтобы к этому вот устной части быть допущеными.
 По сути дела, пункт два, это то, что вы должны будете готовить, ну, как только мы перейдем, обсудим здесь требования по статьям, которые чуть ближе идут.
 
 Итак, первое -- самостоятельно подготовленный глоссарий.
 Я надеюсь, что вы знаете, что такое глоссарий, или, если так быть точными, в английском языке в этом случае часто используется термин thesaurus.
Что это такое?
Знаете, что это такое?
Это не словарь, потому что, обратите внимание, пункт 3, письменный перевод выполняется без использования словаря, в отличие от вашего вступительного экзамена.
Глоссарий или тезаурус -- это заранее подготовленный список слов, ну а вообще это существует такой тип словарей, который, по сути дела является англо-английским словарем, где приводится слово и дается его дефиниция.
То есть это то, что вы должны будете тоже для себя сделать, подготовить эти глоссарии.
Глоссарии обычно это терминологические словари, глоссарии свои вы будете готовить на основании тех статей, которые вы тоже самостоятельно подберете к кандидатскому экзамену.
Ну вот пока, значит, сделали для себя заметочку.
Глоссарий можно начинать делать хоть с завтрашнего дня, если вы уже статьи начнете подбирать и работать с ними, но глоссарий вы представляете перед экзаменом, в тот же день, когда и пишите первое задание, то есть лексико-грамматический тест и получаете за глоссарий зачет/не зачет, то есть если он подготовлен правильно, туда включены действительно достойные термины, а не общие какие-то (общеупотребительные лексики и так далее), то вы получаете зачет (допуск) и можете использовать этот свой глоссарий при письменном переводе на случай, если вы вдруг забыли, добрый день, Евгений Андреевич, если вы вдруг забыли, как переводить то или иное слово.
Просто единственное, что мы запрещаем пользоваться двуязычными словарями на кандидатском экзамене, но своим глоссарием, поверьте мне, 80-100 терминов это больше, чем достаточно, как показывает практика.
Некоторые аспиранты не могут набрать 60.
То есть опция у вас достаточно большая, большой range, да, of options to be included into the glossary, поэтому поверьте мне, что это не будет совсем сложно.

Лексико-грамматический тест.
Он пишется, ну, в зависимости от того, какие у нас сроки.
Мы его проводили за неделю до кандидатского экзамена.
В этом году мы его проводили на бумажном носителе, но я думаю, что со следующего года мы уже будем использовать компьютерное тестирование.
Потому что были проблемы с компьютерными классами, но тем не менее формат лексико-грамматического теста никто менять не будет.
Он не будет поменян.
То, что касается стоимости баллов, извините, пунктов и так далее, вы все это видите.
По его наполнению.
Это по сути дела, лексика академическая, academic vocabulary, и то же самое и с грамматикой.
Что это за грамматика, мы с вами будем в процессе нашей interaction, так сказать, наших практических занятий обсуждать, просто для того, чтобы и вы практиковали свой английский, и для того, чтобы мы еще и вот эти теоретические моменты тоже с вами обсудили, потому что, естественно, сдача кандидатского экзамена это не сдача экзамена по разговорному английскому или по базовому английскому, который сдают бакалавры или даже магистры.
Вы все-таки должны показать свое умение академического общения, умение академического письма, умение академического говорения.
Academic English, по-русски мы чаще называем это язык научной коммуникации.
Академический в русском языке этот термин несколько иной, значение имеет и трактовку.
То есть мы в данном случае говорим об английском языке научной коммуникации, научного общения.
Scientific English это, конечно же, неправильно.
В данном случае английский термин это academic English.
Мы с вами будем обсуждать, какие грамматические явления характерны для академического английского, и, соответственно, вот эти явления, степ-бай-степ, будем повторять, и потом вы в этих тестах покажете свое умение ими пользоваться.
Тесты несложные, поверьте мне, конечно же, определенная подготовка, определенная тренировка для этого нужна, но, тем не менее, все сдают, все проходят, показывают очень хорошие результаты, так что несмотря на то, что вот тут у нас 50 заданий, 100 баллов, ну как правило все очень хорошо.

Итак, вот эти вот два пункта это то, что вы должны будете выполнить для того, чтобы дальше прийти в обозначенную дату, чтобы сдавать устный экзамен.
Как вы видите, в целом, примерно он напоминает, кандидатский экзамен напоминает тот вступительный экзамен, который вы сдавали.
Но только что второе задание немножко отличается от пересказа текста страноведческого характера, который вы все выполняли.
Итак, письменный перевод с иностранного на русский, практически то же самое, что выполняли.
Объем обозначен, как он выполняется тоже.
Более конкретно по материалу я пониже пролистаю, как раз для нас это будет почва для разговора.

Четвертый пункт, то бишь второе задание во время кандидатского экзамена, это аннотирование.
Кто знает термин по-английски, который в этом случае мы должны будем использовать?
Что значит аннотирование?
Это не пересказ, это не retelling, это не rendering, это summarizing.
Я надеюсь, что summary и summarizing, эти термины вы слышали.
Это определенный вид пересказа, где требуется еще и высказывание собственного мнения, то есть как бы рецензирование текста, с которым вы знакомы, и как научный кадр можете оценить, высказать какую-то оценку представленному материалу в научной работе.
Понятно, да?
То есть это не пересказ слово за слово, не дословный пересказ, это уже некая evaluation того материала, с которым вы знакомы, и оценить который вы можете.
Но опять же, это устное высказывание.
Если первый пункт это письменный перевод, задание, которое по сути дела вы делаете 40 минут, и на этом подготовка к кандидатскому экзамену, я уже по собственному опыту вам говорю, на этом подготовка, вот именно ответа по кандидатскому экзамену заканчивается, потому что два остальных пункта это то, что вы готовите заранее и если вы с чувством, с толком, с расстановкой готовитесь к сдаче кандидатского, то по сути дела вы приходите вообще готовы даже к первому заданию письменного перевода.

Пятое задание.
Монологическое высказывание и беседа.
То есть точно так же есть список, темы там чуть-чуть пониже.
Мы сегодня одну тему попробуем с вами уже начать обсуждать.
Все эти, с этими темами вы знакомы, их меньше, их не 15, их уже 8.
Поэтому, как вы сами понимаете, за целый год их можно подготовить.
Более того, я буду заставлять вас их готовить, потому что я традиционно своих аспирантов, с которыми я работаю, заставляю эти темы писать и сдавать, я их у себя коллекционирую до последнего с моими пометками (до разумного последнего -- с тем, чтобы вы их не потеряли), а потом, накануне экзамена, готовясь к кандидатскому, вы достаете полученные от меня проверенные файлики, кладете под подушку, они у вас переливаются, записываются на обратной стороне лобовой кости, и дальше нам рассказываете монолог.
Монолог, а далее, как правило, мы задаем все-таки, как вы понимаете, мы все-таки communication тоже проверяем, поэтому какие-то вопросы по тому, что вы изложили, конечно, возникают у экзаменаторов, желание пообщаться с молодыми перспективными научными кадрами, поэтому какие-то вопросы задаются, проверяется ваше диалогическое умение вести диалог на иностранном языке, и на этом вы счастливые покидаете комнату, экзаменационное помещение.

Дальше то, что касается оценки.
Конечно, это всем известно.
Итак, мы оцениваем каждое экзаменационное задание.
В протоколе выставляется оценка за каждое задание.
Там отлично, 5, 4, 3, 2, 1.
Обычно, вы сами понимаете, у нас в общем-то тут двухбальная система оценивания ответа, это отлично и хорошо, но для удовлетворительно надо очень постараться, чтобы получить.
Вот.
Ну, это действительно по опыту, да, потому что, в общем-то, те, кто поступили в аспирантуру, в подавляющем своем большинстве действительно имеют достойный уровень английского.
Я уверена, что у вас есть большая практика обширного написания статей.
Ну, там дальше все написано, собственно говоря, что важно.

И вот теперь самое интересное.
Кто-то там из поступающих пытался писать мне письма, как вот там согласовать тексты, согласовать статьи и так далее.
При вступительном экзамене у нас этого нет, а вот при кандидатском это действительно есть.
Но этот отбор статей, он носит тоже своего рода квалификационный характер, имеет квалификационный характер.
Почему? Потому что вы должны подобрать статьи, определенные, соответствующие выставленным требованиям.
То есть уже показать, что вы понимаете, каким образом оценивать тот материал, с которым вы будете дальше работать.
По сути дела, вот этот подбор этих статей, вот эти критерии, мы согласовывали с нашими коллегами из института, потому что мы предполагаем, что мы ценим ваше время и свое тоже, и решили, что статьи, которые вы будете использовать на экзамене, они должны быть вам полезны и при написании ваших кандидатских диссертаций.
То есть, по сути дела, работая и подбирая статьи англоязычные, вы должны сразу же думать, что эту статью потом вы сможете использовать в списке литературы, вы ее туда включите, вы сможете ее цитировать, вы сможете ее использовать на русском языке, если она написана по-английски.
Я вам об этом не говорила, естественно, хотя антиплагиат у нас теперь считывает это тоже.
То есть это должны быть те статьи, которые будут вам полезны и которые при написании диссертации вы будете использовать.
Это первый, самый важный признак, по которому вы их подбираете.
Именно поэтому статьи, которые в первом задании, если вы внимательно прочитали, написано, что перевод отрывка текста по научной тематике диссертационного исследования.
То есть предполагается, что эти статьи должны представлять, как я уже сказала, для вас интерес либо по тематике, либо они там вам с точки зрения методологии должны быть интересны.
То есть вот таким образом вы подходите к отбору статей.

Есть, конечно, еще и формальные критерии, которые у нас работают при отборе статей.
Количество статей должно быть не более 10.
То есть нам не нужно приносить статейки по 2-3 странички, и их будет бесконечное множество.
Таких статей нам не надо.
Во-первых, мы все прекрасно понимаем, что очень тяжело столько статей найти, чтобы покрыть 250 тысяч печатных знаков.
То есть это будет бесконечное множество маленьких статей.
Мы даже рекомендуем попытаться и найти так называемые Review articles.
Не Research articles, хотя и Research тоже однозначно это интересно и вам это полезно, но если какая-нибудь одна статья у вас будет Review, то Review articles они обычно большие.
Если постранично говорить, то они от 30 страниц, и по сути дела 100 тысяч печатных знаков в результате получаются.
Последние 10 лет у нас это общее требование.
Каждый год оно, соответственно, год меняется за последние 10 лет.
Но в вашем случае я уже исправила на 2014 год.
То есть, соответственно, статьи 2010, 2008, 2005 мы не рассматриваем для сдачи кандидатского.
То есть используйте.
Я считаю, что 10 лет это тоже уже, хотя зависит от отрасли наук или научного направления, но, тем не менее, промежуток в 10 лет это действительно тот промежуток, когда мы действительно говорим о том, что это современники, которые показывают и описывают результаты, которые и в настоящий момент времени тоже значимы.
Монографии русскоязычные, англоязычные, естественно, вы можете использовать, но это для того, чтобы потом писать вашу теоретическую часть вашей диссертации.

И еще очень важный момент.
Однозначно статьи не должны быть написаны вами и вашими научными руководителями, то есть они не должны быть переводные статьи с русского языка, поэтому у нас сразу сказано что авторы статьи должны быть самое лучшее конечно если это носители иностранного языка, который вы сдаете.
То есть, соответственно, USA, Great Britain, Канада, Индия, нам не поспорить, это государственный язык, если вдруг у вас научные школы располагаются в индийских научных центрах, то, конечно, никто спорить не будет.
Но мы понимаем, что это достаточно сложно, особенно сейчас, с доступом.
Есть определенное ограничение доступа к статье.
Поэтому мы дальше говорим, что если это не носитель языка, не американец и не англичанин, то тогда мы ожидаем, что авторы статей имеют аффилиацию с англоязычными университетами.
То есть это может быть чех, португалец, но он будет в MIT, поставит себе аффилиацию в MIT.
То есть мы предполагаем, что все-таки просто так человек не поставит.
Это определенное сотрудничество, определенный уровень английского языка.
Если же уж прямо совсем плохо, то тогда самое крайнее это журналы Q1, Q2.
Потому что все-таки при отборе статей в эти журналы, статьи проходят и лицензирование на предмет уровня языка, в котором они написаны.
То есть вот или-или-или, мы уже, так сказать, постарались смягчить требования.
Это понятно, да?
Вопросов нет?
Чтобы не было потом внезапно, можно статью моего научного руководителя?

При всем уважении, да, но мы понимаем, что если Saint-Petersburg Polytechnic, это значит, что все равно статья была написана сначала по-русски.
Все равно мы думаем по-русски.
Доверяем переводчику, Яндекс-переводчик или Google-translate, но тем не менее, все-таки хочется, чтобы мы с вами работали и наслаждались оригинальным языком.
Ничего против не имею, да, статьи вы можете использовать, включать в список литературы, но мы с вами изучаем иностранный язык, конечно, мы хотели, чтобы это были лучше всего, конечно, носители языка, потому что мы можем с вами провести эксперимент, такой своего рода, на каком-нибудь занятии, вот сравнить статьи, написанные сразу же по-английски, носителем языка, и взять какого-нибудь даже с аффиляцией с англоязычным университетом и посмотреть, насколько будет отличаться язык с точки зрения грамматических структур, которые будут использоваться, вокабуляра, который там будет использоваться.
Сразу чувствуется некое усложнение и осложнение, потому что если не является человек носителем языка, то, как вы знаете, интерференция родного языка, то есть дурное влияние родного языка на то, как вы излагаете свои мысли на иностранном языке, она имеет место быть.
В разной степени, но тем не менее, имеет место быть.

Далее красненьким выделены моменты, которые нужно принять к сведению с точки зрения работы по подбору этих статей.
Статьи должны быть утверждены, то есть вы будете мне их приносить, либо между парами, либо в какой-то другой день, если вам удобно.
Если вы подобрали эти 10, 8, 7 статей, они подходят по всем формальным критериям, мы просто с вами тут или останемся, или вы подойдете, мы будем их смотреть на предмет языка, объема и соответствия вашей научной тематике.
Вы оформляете эти статьи в определенном формате, я его тоже кину туда, у нас он универсальный для всех список, мы его забираем, прикладываем к протоколу, на сдачу кандидатского экзамена.
И этот список утверждается, обратите внимание, не позднее даты начала сессии, то есть у вас там сессия начнется с 15 сентября, значит к 15 сентября у всех этот список должен быть уже согласован и заверен.
Но можно начинать хоть со следующей недели, если у вас такие статьи уже готовы, то вы можете уже начинать их предъявлять и с ними начинать работать.
На экзамен однозначно мы не допускаем студентов, которые не согласовали статьи, и списки, даже если они у вас есть, если вы не согласовали статьи и списки даже если они у вас есть если вы не согласовали, то тоже на экзамен вы не допускаетесь.
То есть определенные формальные вот эти требования, согласованные с отделом аспирантуры, у нас есть.

Теперь вот к тому моменту, что я говорила что по сути дела вы выходите готовыми к экзамену.
Чем быстрее вы подберете эти 10 статей, тем раньше вы начнете с ними работать.
Потому что именно из этих статей вы будете себе готовить свой глоссарий.
То есть вы читаете статьи, выбираете термины, которые вам не совсем понятны, вы их не знаете, и готовите из них в табличном виде глоссарий.
То есть тем самым вы прорабатываете эти статьи.
Если вы хотя бы на 50\% сознательно подходите к этому процессу, то получается, что к экзамену вы эти статьи прочитали, подобрали глоссарий, то есть вы с ними знакомы.
Это не незнакомый текст, который вы имели на вступительном экзамене.
Пусть он там был и не сильно сложный, но они были, конечно, сложные, но с точки зрения прям уж совсем лексики и грамматики они не были сложные.
Это был там уровень В2, мы специально пропускали через специальную систему и этот уровень проверяли.
То есть, таким образом, вы выходите на экзамен с готовыми статьями, которые вы сами подобрали, то есть, что подберете, то и будете на экзамене сдавать.
То есть, вся ответственность лежит на вас.
Мы с себя всю ответственность снимаем.
Что подберете, с чем придете, с тем и будете работать.
Так, значит, статьи начали переводить, читайте, знакомитесь с ними.

Второе задание, как вы помните, аннотирование, это тоже по другой статье.
Единственное, что мы вариативность стараемся все-таки сохранять.
Из другой статьи вам другая статья вам выбирается для аннотирования.
Вы тоже, опять же, с этой статьей готовы, уже заранее.
По сути, дело вас ночью разбуди, если вы ее прочитали, вы расскажете, о чем она.

Мы с вами во втором семестре будем разбирать, что такое summary, какие используются конструкции, какова структура и логика summary, поэтому и с этой точки зрения мы с вами аннотирование подготовим.
То есть второй вопрос, второе задание вы тоже уже придете готовы, то есть раз-два.

Ну а монологическое высказывание, если вы все восемь несчастных тем, которые сейчас я теперь выведу на экран, подготовите и мне сдадите, и тем более у меня будет возможность, если вы вовремя будете сдавать их проверять, то они у вас тоже готовы, но в общем вы приходите полностью готовы, если вы готовитесь.

Согласитесь, кандидатский экзамен он совсем не страшный.
Но может быть самое напряженное, чего все почему-то боятся, хотя я еще раз повторяю, не было никогда у нас совсем плохих результатов, даже на тройку, ну на грани, но всегда решается все в пользу студента, и уж тем более аспиранта.
Это лексикограмматический тест, потому что тестов боятся все.
Ну понятно, там что написано пером, того уже не вырубишь топором, тем более если нажал кнопочку на компьютере.
Совсем топор уже не поможет.
Но вот, пожалуй, это самый психологически напряженный момент.

Все остальное, согласитесь, и сама программа кандидатского совсем не такая сложная.
Мне кажется, философию сложнее сдавать.
Не кажется, а по рассказам.
Есть вопросы по кандидатскому?
Ясно?
У вас ещё будет возможность, я буду так во время наших встреч выводить всю эту программу на экран.

Можно вопрос? У нас будет какая-нибудь отчетность по окончании этого семестра?
Да, у вас будет зачет с оценкой.
А что будет в нём?
Ну, хорошо, теперь давайте об этом.
То, что будет касаться зачета с оценкой, это вот эти 8 тем, я обычно, когда я работаю с аспирантами, я делю пополам.
То есть мы 4 темы обсуждаем, разбираем, вы пишете и сдаёте мне в первом семестре, и 4 темы у нас уходят на второй семестр.
Поэтому для получения зачета с оценкой вам нужно будет написать четыре темы, которые мы просто по разному логика всегда идёт, да, то есть мы с одним годом мы вот эти темы четыре берем, а с другим годом у нас меняется range.
Вот, то есть написать, и сдать четыре темы, получить от меня их, то есть эти оценки, да, за эти темы, то есть у вас будет четыре оценки за темы написанные.
Поскольку мы будем заниматься с вами грамматикой, у вас будет небольшой, не такой тест, а задание по тем грамматическим аспектам, которые мы с вами разберем.
Соответственно тоже будет зачет, вернее тест с оценкой.
И небольшое собеседование если у меня будут серьезные вопросы или серьезные замечания по выполненному заданию.
То есть вы своего рода частично уже вы пройдете какие-то определенные этапы кандидатского экзамена, уже они будут у вас готовы.

\sublinksection{The prospects of your scientific career}

So, actually, for today, I would like us to start discussing topic number two, which is the prospects of your scientific career.
Because I think that having become the PhD students, you started thinking about it.
So, that would be nice if you now share your ideas and your plans in English, about your scientific work, about, as I have said, the prospects of your scientific career and your scientific research.
So actually, what do you think everything is done for?
So why are you here, sir, and what are you thinking about at this very moment?
Because this is the beginning of your new stage, of the new stage of your life, and I'm sure that you have new ideas, new thoughts, and I would like you to discuss it.
I would like you to say a few words about it and also to practise your spoken English.
So the prospects of your scientific career and probably let us try for our next meeting, I would like you to present your written topics.
This is the point, actually, because I'm sure that you all remember that this word became a sort of a term.
I mean, the word topic, topic.
You all remember this word from your school years.
Because that's not the, I don't want it to be an essay.
So, my very important requirement for your written works is that these written works should be written by means of using spoken English phrases.
So it shouldn't be an essay.

Я не хочу, чтобы вы мне писали эссе.
Потому что, к сожалению, у меня большой опыт, как вы сами понимаете, преподавания, и естественно, я прекрасно понимаю, что и студенты, и аспиранты может быть более сознательны и более ответственны, но все равно, как это, Google Translate в помощь, поэтому, хотя я не против, ради бога, отвечать-то вы все равно будете без него, будете отвечать устно.
Значит, sorry, опять-таки, я переключилась на русский, чтобы объяснить, что я хочу.

Я хочу, чтобы то, что вы мне будете сдавать в письменной форме, это была письменная фиксация вашей устной речи.
Мне не нужно писать эссе, потому что иногда я получаю замечательные эссе по вот этим тематикам, но я прекрасно понимаю, что если вы придёте на экзамен, то вы мне не расскажете то, что вы написали в форме эссе, потому что эссе это жанр письменной академической речи, письменной, да, её чудесно и прекрасно читать, но поверьте мне, её воспроизводить, но это нужно выучить наизусть, но тогда понятно, что это выученный текст, поэтому когда вы будете задумываться о том, что написать в топике, вспоминайте, как вы в школе писали.
Простые предложения, не сильно осложненные какими-то конструкциями.
Где нужно конструкции какие-то использовать, так это в аннотировании, потому что там все-таки предполагается, что вы уже просто устно, но тем не менее презентуете некий академический текст, а монологическое высказывание это именно проверка вашей способности, ну назовем так, ввести small talk, но на более такую глубокую тематику, нежели о погоде или о природе, потому что small talk это беседа, светская беседа, но вот у нас такая будет светско-научная беседа, но тем не менее это должна быть беседа by means of every day English, поэтому вот это как раз я и буду проверять, и если там у вас будет замечательно написано, но я буду прекрасно понимать, что это эссе, я буду возвращать такие работы на переписывание.

По мне лучше, если вы сделаете какие-то ошибки, и я подправлю вот именно с точки зрения грамматического оформления или чего-то еще, нежели я увижу, что все замечательно, хорошо поработал переводчик, но тем не менее это не устный текст, то есть это не устная высказывание.
То есть таким образом вот это требование к тому, что вы мне будете письменно представлять тем не менее должно быть в дальнейшем устным монологическим высказыванием. Это понятно? Окей. Prospects of your scientific career. Just a few sentences, I think three or four, not more, so that as many of you as possible would actually show, or would actually enjoy yourselves speaking.
So, who would like to begin?
Не стесняйтесь, я сейчас оценки не ставлю, поэтому вспоминайте, как это было.
Вы уже забыли, да, как это было в сентябре, ну, естественно, прошло больше двух месяцев, поэтому давайте вспоминать.
So, я тогда сейчас попрошу вас написать три предложения, и потом хотя бы прочитайте, чтобы все точно приложились к источнику знаний.
Ну давайте так и сделаем.
Ok, so, before we start dealing with grammar, I would like you to just to get into the English language.
So, three sentences about the prospects of your scientific career.
What would you tell your colleagues, your partners, your parties when you discuss your future scientific career?
You are welcome.
Three sentences.
I don't ask for more.
Write.
3 предложения, больше не прошу.
Пишем.
\\

---Пишем---
\\

Okay, well I think you're ready.
Let's listen to each other.
So let it be just like sort of a monologue without any questions.
Just to begin with and to listen to each other.

The reason why I decided to get a philosophy degree is to promote myself in the future.
By the future, I mean a job which is connected with my academic achievements and inventions.
In short, my research is about data analysis of sub-security problems.
Okay.
That's it, that's enough.
The only thing is, what about the degree? Philosophy degree? No, that's not philosophy degree.
What is it? What are you studying? So, PhD in.
That's not the...

Когда вы используете philosophy degree, это значит degree in philosophy.
PhD это все-таки немножко другое.
Это некое формальное обозначение степени, если вы потом будете, когда вы защитите кандидатскую диссертацию, когда, как говорил мой научный руководитель, если, а когда.
Так вот, когда вы защитите, вы будете говорить, I have a PhD in something.
So, in your case, that is why you have decided to get a PhD in what? So, what is your major? Technical sciences.
Нет.
Major это всё-таки вот эти вот всякие technical sciences, candidate of technical sciences, это не совсем, скажем так, корректный перевод.
Это прямой перевод того, что мы имеем в русском языке, да, но тем не менее, если вы хотите корректно объяснить англоговорящему иностранному специалисту, то наши конечно ученые степени кандидата и доктора наук это специфическая отечественная система научной аттестации, а что такое major? У всех разные ассоциации.
Major -- это то, что вы тоже неправильно, если кто-то использовал это, да, будете использовать в качестве, вернее, вы будете использовать слово специализация, например, специализация, да, или специальность, specialty.
Specialty используется, а specialization, пожалуйста, не используйте.
Это просто перевод с русского на английский, но такое понятие, оно существует, но совсем не в том значении, которое мы в русском языке вкладываем.
Specialty еще, может быть, пойму, а вот в нормальном академическом английском языке называть вашу специальность или специализацию в чем-то, major, я думаю, что вы знаете, ну, может быть быть кто-то из вас надеюсь что не один: в системе сейчас у нас тоже это используется в системе зарубежного образования используется то, что человек поступает на majors, но у него еще есть minors, то есть есть основное направление подготовки это major, а можно еще получать minors, у нас это в Политехе, это так называемый модуль мобильности, образовательный форсайт, карьерная адаптивность, где вы можете брать курсы, которые не связаны с вашей основной специализацией.
Так вот, your major is.

So, your major is what? Вот ваша научная специальность, по сути дела.
So, you're majoring, глагол to major, I'm majoring in.
Ваша научная специальность, по сути дела.
Поэтому мы, давайте, вот всё-таки изучая английский и планируя говорить по-английски, мы всё-таки с вами будем стараться затрагивать или стараться вспоминать, или изучать вот эти вот традиционно все-таки англоязычные формулы выражения различных мыслей.
Итак, первое, что мы с вами вспомнили, да?
So, my major is... или I am majoring in something.
Ok, thank you.


So, what about you?
Well I think there are certain periods of time when you just tought and you need to decide what to do next and in my case I had several situations when during my work I thought well why I'm doing something like this, maybe because I'm accustomed to these certain methods of dealing with my problems at work.
But I lacked a certain scientific background.
That's why I decided to apply for this certain program, educational program, and to make this solid background for my work.
For what educational program, finally? Fluid mechanics.
Fluid mechanics, okay, very good.


Thank you, so what about you?
Well, I think I have to mention that my field of knowledge is fluid mechanics.
I succeeded as an IT.
I got my master's degree in the field of fluid mechanics.
And then I decided to switch to fluid mechanics.
Because there are some tasks that require knowledge and skills in both areas.
And that's why I decided to widen my range of knowledge.
To what? You have decided to? To widen. To widen.
You've become so special for that.
to widen.
Okay, very good, thank you.
So let's switch to your colleagues.

Не бойтесь говорить, поэтому наслаждайтесь своими словами.
Все, что вы можете хотя бы хоть в каком-то виде составить по-русски, однозначно вы должны мне говорить по-английски тоже.
Пусть это будет с ошибками, главное пока говорите, я пока не исправляю.
Окей, so what is your major? So what are you, what will you be majoring in? Итак, ваша вот специальность какая? 
Thank you so, you're welcome.

I think in the future it is planned to use this knowledge for the development of rotary blade engines in science groups.

Итак, как произносится слово двигатель? Engine.
А инженер? Engineer.
На самом деле, да, слово инженер от слова engine образовалось.
Производное как раз с тем самым суффиксом er, то есть тот, который engines, двигает что-то.
Он инженер.
Окей, thank you.

You're welcome.
In general, I don't really think to become a professor or get a degree.
The main reason why I continue studying is simply I don't have enough collaborating activities or scientific activities at my present work and just want to get some additional incentive or motivation to figure out some difficult theory in areas of my interest.
Okay, well, for the beginning, it's okay.
Thank you very much.
Well, we all start with just clearing up the ideas of our future.
So for you, that's the stage.
Thank you.

So gentlemen, let's move this way.
So yes, you're right.
My scientific career as postgraduate student would help me and my colleagues complete current project.
That's why I don't want to continue my scientific career after post-graduate and I think it will be only four years just for education.
So what is your major?
Well at least in any case, so what are you going to deal with and to conduct the research in? Mechanical engineering I think.
Mechanical engineering.
Okay, I think.
Mechanical engineering.
Okay, thank you.
You're welcome.
My major is theoretical mechanics.
In my scientific career, I would like to create new numerical methods or modify already existing methods for solving differential equations numerically.
These improvements can help to integrate equations in certain complex scenarios and to avoid instability in numerical simulations.
Если я попрошу закрыть, вы повторите все то же самое?
Это к вопросу о том, что вы использовали, сами писали?
Вот сейчас, да? Сможете вот то же самое сказать? Нет, вы замечательно, терминология, все хорошо.
Но сможете, это просто вопрос, сможете то же самое сказать, не глядя?
Ну нет, ну то есть у вас эти слова в активном словаре?
То есть в активном, не в пассивном, не то, что вы там подглядывали в переводчик и что-то делали.
В активном.
Хорошо, хорошо.
And I would like to discuss these improvements with colleagues all over the world.
Okay, thank you very much.

My research area is oil production industry.
I think my first prospect is scientific conferences in different countries.
And second is to develop new things for oil production.
And third is to create a startup.
Okay, thank you.

Yes? Now my job is connected to an automotive engineering.
I hope that I will be able to implement my knowledge from the university at my work.
I just want to become a good engineer, so I do not only push prescribed buttons, but also understand what is happening.
I do not want to become a scientist.
Okay, thank you.

I think that my prospect of scientific career is to have a candidate degree of technical science.
I will write as much as I could in short and then I will write a dissertation in four years.
So I can help to solve some problems of my work company.
Хорошо, спасибо.

I'm engineer.
А сейчас буду читать по листу.
The applications of polymeric composite material is one of the dynamical development areas in engineering.
In this area numerical methods are widely used for design research.
There is still a problem to develop effective methods to analise this complex mechanism of modern composite structures behaviour.
Окей, ну, учитесь говорить от души.
Не только писать от души, но и говорить тоже.
Ничего, будем потихонечку практиковаться.
У нас целый год.
Yes, please.

OK.
My prospects of scientific career.
I am majoring in mechanical engineering, as known as in Russia, mechanics of deformable solid bodies.
I think that after my graduation, I can research some important things.
It helped me to get a lot of respect from scientific society and a lot of money.
In another case, I can become a scientific advisor, some wise man who can share my knowledge with the students.
Okay, Thank you.

So I'm majoring in Mechanics of Deformable Solid Bodies.
I'm researching the modular connections.
And so I'm trying to develop modular construction, which is in my opinion, the perspective field of civil engineering.
And also I want to try teaching and sharing my knowledge with students.
To try to share, okay, thank you.
Yes, please.

My decision to study on this program is forced mainly by solving the problems in my work.
So I'm going to research the dynamic of systems of solid bodies.
In my paper I will obtain the problem of the decision of nuclear waste storage.
I hope to develop the method for determination of the seismic resistance for these constructions.
Okay, thank you.

I finished my master degree in this university.
So my major is Mechanics of Deformable Solid Bodies.
And my master's work was connected with ferroelectric materials.
So I just want to continue this.
I want to continue my scientific research in this area.
And maybe one day become real.
The Nobel Prize winner.
Okay, thank you.
You're welcome.

You have decided to use this chance? Yeah, of course.
Okay, so that's it.
Okay, thank you.
Some small notes and then we proceed.
Как вы скажете, какой глагол вы будете использовать, когда мы говорим про окончание университета?
To graduate, да?
Не to finish.
Запоминайте, да? Когда мы говорим to finish, это to finish school.
Поэтому сразу для себя запомните, если вы там во время кандидатского вас будут спрашивать, ну, мало ли, вашу предыдущую тему, будут спрашивать о вашей предыдущей жизни, то мы отчетливо с вами должны использовать корректный термин.
Что еще интересно? Вы можете сказать, я закончил университет.
И в этом случае как будет звучать фраза? Закончил университет?
Graduated from.
Мы это запоминаем обязательно с этим предлогом.
To graduate from.
To graduate from.
Что? Закончил, да.
Именно закончить университет.
Обратите внимание на этот корень.
Что это такое? To get a grade.
То есть получить некую степень.
Понятно, что это слово degree, но тем не менее, grade это некую градацию получить в иерархии.
И вот как раз это именно к университету и относится.
По поводу произношения.
Этот глагол произносится с двумя ударениями.
Как раз про инженера, когда я вам сказала.
Но это просто для того, чтобы мы правильно с вами говорили.
To graduate.
Есть вторичное и первичное ударение у этого глагола.
А если вы будете про себя где-нибудь выступая, говорить, что я, как выпускник Политехнического университета, горжусь своим университетом, какое будет слово выпускник?
Ну, во-первых, слово будет писаться точно так же, да?
Какое будет слово выпускник?
Во-первых, слово будет писаться точно так же, да?
Это исчисляемое, поэтому я здесь написала пока "<э">.
Но как существительное это слово будет иметь только одно ударение (на первый слог).
Graduate.
Graduate.
Graduate, but I'm as a graduate of Saint Petersburg Polytechnic, it's proud of being here, giving you the report, etc.
Обратите внимание, в английском языке очень много таких пар слов, которые отличаются только, в зависимости от части речи, отличаются только ударением.
Это ещё достаточно сложно.
Я думаю, что самые простые слова вы мне, наверное, приведете сейчас.
Вот экономические, например, термины, да? Импорт и импортировать.
Как будет импорт, так и будет import.
А импортировать будет to imp\'ort.
A report -- отчет, доклад.
To rep\'ort (сообщать; давать отчёт) -- в зависимости от того, куда вы ставите в английском языке ударение.
Но, как правило, эти вот пары слов, они заимствованы из латинского языка.
Мы как раз сегодня с вами уже начнём об этом говорить, потому что как раз академический английский, а в данном случае терминологический аппарат, это признак академического дискурса, академического языка.
Он имеет свои специфические особенности.
И самой главной особенностью является первое, то, что он наполнен терминами, терминологическим аппаратом, и то, что очень большая часть этих терминов является заимствованной из латинского языка.
Не только в английском, но и в русском тоже.
Именно поэтому я говорила, как будет импорт, так и будет по-английски.
Потому что это слово пришло, в общем-то, однозначно во все индоевропейские языки, но, наверное, и не только в индоевропейские, в таком виде, в каком оно было создано в латинском языке.

А вы все помните, как меня зовут?
На всякий случай, потому что я-то вас всех послушала, я знаю, да?
Ну, на всякий случай, да?
Вы помните, что я Наталья Борисовна.
Нам с вами целый год вместе жить.
Почему я так люблю, вот, и буду, к сожалению, вам иногда, как я шучу, шучу обычно, рассказывать сказки про англосаксов, потому что my major is the history of the English language, and I'm a PhD in philology, and my PhD thesis or PhD dissertation, which is more correct to say, is about the epic poems of Anglo-Saxons.
And I really love speaking about the history of the English language, and when we speak about the English language, everything that you are probably used to as naming as exclusive cases is not actually exclusions.
These are the normal rules that are explained by the history of the English language.
Поэтому я иногда буду стараться вам объяснить, что все не так страшно в английском языке.
Я очень люблю говорить про английский.
Меня надо останавливать иногда.

Итак, to graduate from, a graduate.
И ещё тоже было использовано слово, когда? A graduate of, да? Да.
Ну, естественно, если вы хотите сказать, что вы выпускник чего-то, то родительный падеж в английском языке, вы знаете, что в английском языке нет падежей, даже если кто-то из ваших учителей или преподавателей говорил, что есть какой-нибудь там common case, говорят общий падеж, нет, в английском языке он утрачен, падежей нет, падежных окончаний тоже.
Если нет самого концепта, то и соответственно нет и тех элементов, которые выражают эту категорию, да, не концепт, вернее, категорию.
По поводу предлогов мы будем с вами говорить, это тоже очень интересный момент.

Как вы скажете, исследование? Research.
Что интересно по поводу этого слова с точки зрения грамматики?
Слово research никогда не используется во множественном числе.
Researchs сказать нельзя.
My researchs.
В английском языке, в академическом английском, есть слово, которое является по сути дела синонимом.
Если вам надо сказать, что множественные, многочисленные исследования этого феномена.
Если вы хотите подчеркнуть, что это именно исчисляемые, что их там 10 или 15 или 20, какое слово вы будете использовать в этом случае?
A lot of studies.
A lot of studies.
Studies.
Studies.
Это практически прямой синоним, который будет использовать.
A lot of studies dealing with the problem of something are conducted by the Russian scientists, researchers.
Запомните, пожалуйста, мы с вами все-таки должны использовать корректно, учитывать правильные грамматические правила и знания, поэтому для себя тоже это берем на заметку.
Это такие моменты маленькие, потому что я хочу, чтобы мы все-таки с вами еще поделали небольшое грамматическое задание.

\sublinksection{Выполнение заданий по лексике и грамматике}

Я вам сейчас раздам похожие задания из тех тестов, которые у вас будут в компьютере.
Единственное, что я, наверное, не буду вас просить в этот раз, я буду в следующие разы уже распечатывать, чтобы у вас у всех было, вы могли бы печатать, но сейчас пока, чтобы для ваших коллег осталось, чтобы у нас хватило времени, я попрошу вас сделать первое.
Да, я дам, дам, дам.
\textbf{Первое} и \textbf{четвёртое}, но только в своих тетрадочках.
Второе и третье задания делать пока не надо, это будет ваше домашнее задание, но только я очень вас прошу, поскольку это не на оценку.
То, что мы с вами делаем на парах, на наших встречах, это не на оценку, это для проверки ваших знаний.
Поэтому не нужно пользоваться гаджетами.
У них корень такой, как раз, пожалуйста, не пользуйтесь ими.
Все, что вы делаете здесь, это вы делаете для себя и для того, чтобы себя проверять.
Поэтому мы все равно с вами проверим, вы будете знать правильный ответ, вы его себе запишите.
Даже если вы в своем ответе будете давать неправильный ответ, то это ни на что не повлияет.
Вы просто будете знать, что вы здесь неправы, а дальше мы обсудим правильный ответ.
Итак, сейчас я хочу, чтобы вы вот..
Ну, минут 10 больше на это не уйдёт.
Первое задание и четвёртое.
Посмотрите, пожалуйста, первое задание.
Это то, что мы будем с вами обсуждать в течение первого семестра, как я уже сказала.
Это, по сути дела, заимствование из греческого и латинского языка простых слов, которые в большинстве случаев похожи даже и на русский язык.
Итак, в первом случае вам нужно будет выстроить цепочку.
Вот, например, первое слово у нас Pater.
Все знают, что это однозначно.
Не знаете, что такое Patter? Кто знает? Father.
Это отец по-латински.
Таким образом, во второй колоночке.
Вторая колоночка -- это значение.
Вы смотрите, patter, находите слово father, буквочка (c).
А третья колоночка -- это слова современного английского языка, где латинский корень, вот этот pater используется.
Давайте найдем это слово.
Expatriate.
И таким образом мы с вами выстраиваем цепочку.
Остальные семь, пожалуйста, сделайте самостоятельно, и мы проверим, насколько вы...
Начинайте работать с тем, что вы знаете.
А дальше тесты, как вы знаете, всегда нужно решать не по порядку, а по степени понимания, что вы это знаете.
То есть, никогда не нужно тратить время на раздумывание в тесте.
Нужно всегда сначала выполнять то, что вы знаете.

\sublinksection{Греческие и латинские заимствования}

--- Выполнение задания ---
\\

\textbf{Выполненное первое задание.}

patter -- father -- expatriate

cord, cour -- heart -- encourage

alter -- another -- alternative

medi -- middle -- medieval

mod -- measure -- immoderate

scrib -- write -- prescription

tract -- pull, draw -- subtract

urb -- city -- suburban
\\

So, let's begin.
It's a lot for you to spend more time than you are supposed to, though I'm ready.
Okay, so as you know, as we have already discussed, for the first.

In this case корень is root, not only for the tree, but also for the word, right?
So, as for the first root, pater, from Latin, we have this chain of letters and words which is father and expatriate.
By the way, what is expatriate? How do you translate this word? That's the term.
Вы понимаете, что это термин, по сути дела.
То есть мы как раз с вами будем сталкиваться все равно сейчас с определенного рода терминологическими словами.
Это не те, которые в повседневной речи, in everyday English мы используем.
Итак, это экспатриат.
Или же в зависимости от вектора действия, это либо эмигрант, либо иммигрант.
Тот, который каким-то образом расстается с родиной, либо обратно в нее возвращается, тогда он эмигрант, а если он во вне, то тогда имигрант, а экспатриат это некий такой общий термин, который, ну, вероятно, в какой-то миграционной политике, и так далее, используется.
Но в любом случае мы понимаем, что в данном случае, кстати, как английском языке будет родина? Fatherland.
В английском языке это будет Fatherland.
Как ни странно.
В русском языке мы используем родина мать.
В английском языке вот такого рода такой, да, рода, рода, полоразличительный термин.
Так, окей, so, number two.

So, cord or cour.
What will be the explanation, what will be the definition?
В Париже кто-нибудь был? Когда-нибудь? Нет еще. Еще это звучит оптимистично, но там знаменитый собор такой белый есть, Сакре-Кёр, знаете, да, собор Святое Сердце, такой красивый на Монмартре стоит, вот, Cour это, сказали правильно?
That is heart.
And what is the word? Encourage.
How do you translate the word, the verb to encourage?
Воодушевлять, поощрять, ободрять.
Да, то есть привносить положительные эмоции.
В английском языке вот этот вот корень, он как раз сохраняется только в слове encourage или courage.
Что такое courage? Отвага, храбрость, смелость.
Абсолютно верно.
И в данном случае, несмотря на то, что значение действительно сердце, но во французском языке вот у этого "<cour"> -- да, это как место хранения эмоций, да, это не орган физиологический, а именно вот как некое такое метафорическое обозначение, потому что во всех языках индоевропейских для сердца используется, ну понятно, в русском языке я имею в виду сердце это термин, но тем не менее, все что связано с обслуживанием сердечной системы, там другой немножко корень, какой? Кардиолог, да, и кардия это греческий глагол, извините, греческий корень, кардия это сердце, именно как орган человеческого тела.
То есть в данном случае во французском языке, и французский язык вот подарил английскому языку после нормандского завоевания и большого тесного взаимодействия с французским языком английского языка вот это слово и глагол глагол соответственно кураж который мы тоже используем в русском языке некий такой подъем эмоций да кураж это опять же эмоции, да, несмотря на то, что тот же самый глагол.
Ну и глагол to encourage в английском языке поощрять, то есть помогать подняться на бой с драконами.

Так, number three.
По-латински мы прочитаем латинский язык просто, как читается, хотя аудиозаписей, к сожалению, нет на латинском языке, как древние латиняни говорили, но считается, что, в общем-то, максимально приближено чтение было к графику, графика максимально фиксировала произношение.
Поэтому alter, хотя в современном английском мы, естественно, по правилам современного английского языка будем читать.
So what is the definition? Another. Absolutely correct!
And what is the word? Alternative.

Number four.
Medi. That is middle. Good
And word is medieval.
How do you translate it? Средневековый.
Очень хорошо, medieval.

Number five.
Mod.
Mod is what?
That's measure.
Да.
Мод это мера.
And what is the word? Immoderate.
How do you translate it? Несдержанный, неумеренный.
What is interesting about this word?
From the morphological structure of the word.
What do we have in this word? Im.
Im is what? Negative prefix.
Negative prefix.
And in this case, it is interesting because I'm sure you know the rule that in the English language, if the word, it can be a verb or an adjective, if it starts with the letters m or p so the negative prefix sounds like im so there are some other instances and they are interesting but the the most easiest ones are similar to im will be what? I mean from the point of view of its uniqueness.
In? No, in has particular cases of usage like un.
Un is the most universal one, then goes in, but there are three prefixes.
Im and two similar.
Im, il, ir.
So the rule is the same.
If we have, let it be an adjective, if it starts with l, so the negative prefix is il.
Legal or illegal.
Since the majority of adjectives starts with r (with letter r), so in this case we have prefix ir.
Regular -- irregular.

If it starts with labials, губных звуков, да, m or p, so then we have im.
Остальные случаи, они, есть определенная система; при возможности мы может быть про эту систему проговорим там слишком много instances, чтобы прямо сразу начать вспоминать; не всегда они отчетливо систематизируются, но по крайней мере определенную какую-то систему можно вывести и в этом случае moderate -- это умеренный, то есть тот который можно посчитать, да не зря мы с вами говорим умеренный, то есть с определенной мерой, да.
В английском языке латинский суффикс mod, в русском языке перевод этого слова, да, мера у нас, корень в русском языке.
Ну, и мы имеем отрицательное прилагательное.

Number six. Scrib?
That is mean to write.
And the example of the word is prescription.
What is it? Предписание.
Или рецепт, например, если мы про нечто материальное будем говорить.

Number seven. Tract?
И тут мы вспоминаем русское слово, которое прямо от этого корня образовано, и сразу все становится понятным.
Тот, который тянет.
И по этому значение как вы сами понимаете pull or draw; русское слово трактор у нас совсем не русско; прямо 100 процентов на 500 процентов латинское, хотя мы даже вот даже задуматься невозможно теперь мы с вами знаем что, а нет оно не русское.
Ламборджини тоже сначала спроектировал трактор.
Вот, вот латиняне, всё они, всё они, от них весь вред.
Итак, пример, итак, to pull, to draw, а пример? Subtract.
How would you translate it? What is to subtract? Вычитать.
Вычитать, вычитать.
Математический глагол вычитать.
То есть вытягивать, откуда-то убирать лишнее, по сути дела.
Если задуматься о значении этого глагола.

Ну и восьмой. Urb?
City. Урбанист.
Не зря мы говорим с этим же корнем в русском языке.
And the example is suburban.
What is it? Пригородный.

\sublinksection{Задание на повторение страдательного залога (Passive Voice)}

Okay, and so we have seven more minutes, and I think that we will very quickly...
\\

--- Выполнение задания ---
\\

\textbf{Выполненное четвёртое задание.}

29. Guide dogs \textbf{were} first \textbf{used} by soldiers who had been blinded during World War One.

30. I regret to inform you that your application for visa \textbf{has been turned down}.

31. Leonardo's sketchbooks, with notes often \textbf{written} in mirror form, were full of ideas for his inventions.

32. We \textbf{are supposed} to be at the presentation.

33. If the bill \textbf{is passed} by both parliamentary houses, then it becomes law.

34. The quarterly accounts not \textbf{having been finalised}, the auditors were unable to present their report.

35. This area \textbf{is being monitored} by closed circuit cameras.

36. We require that all receipts \textbf{be submitted} to the committee for approval.
\\

Number four. Почему я выбрала это? Потому что здесь как вы уже прочитали -- это страдательный залог или passive voice -- как наверняка вы знаете это в общем-то одна из традиционных конструкций, которая используются при написании статей.
Каков общий смысл passive voice?
Для чего он используется?
Вернее почему он появился и почему он используется?
Чтобы что?
Тут даже не обезличивание, а тут необходимость обойтись без указания на деятеля.
Потому что что такое активный залог? Когда у нас подлежащее соответствуют деятелю, тому, кто это действие выполняет.
Если же мы не хотим, не знаем, не можем обозначить того, кто это действие выполняет, глагол в активном залоге "<я читаю книгу">, когда нам не важно или не нужно, то тогда в языке есть конструкция, которая с ног на голову переворачивается, с хвоста в начало перестраивает предложение, мы можем таким образом опустить того, кто это действие выполняет, поставив в грамматическую позицию подлежащего тот объект, над которым действие производится.
А поскольку в научных статьях понятно, что все то, что описывает, это описывает, как правило, все-таки автор или деятель, то само это понятие можно, не понятие, а самого этого, сам этот субъект, самого этого субъекта можно, так сказать, отодвинуть, просто априори осознав, что он есть, и описывать уже объективную реальность, не концентрируясь на субъекте этого действия.
Именно поэтому в научных текстах вы очень часто, особенно в англоязычных, будете сталкиваться с предложениями и, наверное, будете использовать именно предложения в пассивном залоге.
Поэтому в течение нашего первого семестра мы будем с вами обращать особое внимание на пассивный залог.
Активный залог, естественно, вспомним, потому что без воспоминаний приятных о временах английского глагола невозможно про пассивный залог говорить.
Мы все-таки должны будем их вспомнить.
Но тем не менее, уделим особое внимание пассивному залогу, потому что, тем более вы должны были обратить внимание, что пассивный залог это не только сказуемое.
В пассивном залоге могут выражаться и другие члены предложения, или другие грамматические формы глагола.
И посмотрим, смогли ли вы их определить, и смогли ли вы их правильно использовать.
Ну и дальше это будет у нас почва для дальнейших наших обсуждений.

Окей, so sentence number 29.
So who would like to read it? To give a correct passive form.
Guide dogs were first used by soldiers who had been blinded during World War One.
Абсолютно верно.
Guide dogs это кто? Это собаки-поводыри.
So, the grammar form is absolutely correct.
Were used.
Обратите внимание, что это сложное предложение.
У нас здесь две грамматических глагольных формы.
Were used and had been blinded.
Здесь мы с вами видим, что в одном случае were used -- это какая форма глагола?
Мы с вами научимся, я надеюсь, к концу певого курса.
Забудем про то, что есть первая форма глагола, вторая форма глагола и третья форма глагола.
Точно так же, как общие вопросы и специальные вопросы.
Да, потому что, знаете, вы так по-английски, по-английски говорите.
Would you please ask me a special question? Не вопрос, сейчас задам тебе special question.
И будешь мне долго отвечать.
Вот эти термины, это изобретение русскоязычной грамматики.
В английском языке таких понятий нет.
Вторая форма глагола так называется метафорически, потому что это столбик у неправильного глагола, который идет вторым по счету.
А кто-нибудь когда-нибудь в заголовок этого столбика заглядывал, как называется этот столбик? А называется он как раз по названию вот этой грамматической формы.
Вторая форма глагола, вот этот столбик, используется только в каком.
Ну, не совсем правильно, но тем не менее скажу, в каком времени?
В Past Simple.
Только для Past Simple.
To go -- went.
Went -- это тот самый второй столбик.
Если вы потом, ну, у кого есть дома словарики, придите, найдите эту вот большую таблицу, где там 100, 150 этих неправильных глаголов, и посмотрите, как она называется.
Именно так она и будет называться -- Past Simple.
Потому что больше ни в какой другой форме эта форма использоваться не будет.
В данном случае мы с вами столкнулись с Passive Form of Past Simple.
Но это все-таки все равно Past Simple.
Что такое Past Simple, какова его особенность, мы будем с вами постепенно повторять.
А во втором случае Had Been Blinded, то что у нас написано, это какая форма?
Это Past Perfect.
И вы помните, да, наверное, из своих предыдущих знаний, что Past Simple и Past Perfect относятся друг к другу как предшествующие и последующие, да?
Но оба говорят о событиях, которые имели место в прошлом.

Так, следующий, тридцатый, кто хочет? Пожалуйста.
I regret to inform you that your application for visa has been turned down.
Здесь мы должны использовать Present Perfect Passive.
Почему? Потому что I regret, я сейчас, сожалею, поэтому, но, тем не менее, и вот это вот я сейчас постараюсь объяснить, откуда у нас, русскоговорящих, возникает вот эта проблема.
Я думаю, что вы тоже понимаете, для русскоговорящих правильно определить, какую форму использовать, Past Simple, Present Perfect или Past Perfect, очень тяжело.
Почему? Потому что перевод на русский язык будет в одной той же форме.
Ваша виза не оформлена или там я прочитал эту книгу.
Прочитал.
Я прочитала её вчера или я уже её прочитал.
В английском языке вот от этих маленьких слов "<вчера"> или "<уже"> будет зависеть та форма, которую мы будем использовать в английском языке.
А в русском языке мы используем одну и ту же форму прошедшего времени совершенного вида, и поэтому для русскоговорящих очень важно как раз вот на эти словечки обращать внимание, поскольку, и это вот правило, на которое не обращают внимание в школах, и тоже, так сказать, мы тоже к этому достаточно долго все приходим, Past Simple используется только тогда, когда у вас в предложении есть хоть какое-то слово, которое четко отвечает на вопрос когда.
Вчера, год назад, на прошлой неделе.
В нашем предложении этого нет.
Значит, Past Simple мы сразу отметаем.
Дальше, естественно, как в предыдущем предложении, может у нас быть сомнение использовать нам Present Perfect или Past Perfect.
Но Past Perfect, как я только что специально вам обратила ваше внимание, используется только в том случае, если он показывает предшествование по отношению к действию, которое тоже было в прошлом.
Вот were used и have been blinded -- это два действия, которые имели место во время, вернее, после Первой мировой войны.
А в нашем случае I regret сейчас.
И это не и вчера, вот если бы у нас было yesterday I regretted that your, что там, visa или application form, вот тогда было бы had been и так далее, тогда бы у нас был Past Perfect, а поскольку у нас regret стоит в Present Simple, то тогда, соответственно, мы в придаточном предложении будем тоже Present использовать, только Perfect.
Понятно? То есть вот это вот очень важно для русскоговорящих, для себя сделать, вот как бы искать эту лакмусовую бумажку.
Есть слово, которое отвечает на вопрос когда, или слово, или часть предложения, тогда Past Simple однозначно.
Нет этого слова, потому что слово уже, already, на вопрос когда не отвечает.
Это не момент времени, уже.
Это стадия завершенности, но не конкретный момент времени и поэтому если нет у нас указания на часах мы не найдем уже.
Здесь нет такой точечки значит мы однозначно Past Simple использовать не будем.

Давайте еще чуть-чуть вот как раз следующее предложение мне очень даже интересно.
Пожалуйста, кто попробует?
Leonardo's sketchbooks, with notes often written in mirror form, were full of ideas for his inventions.

Итак, как мы раскроем скобочки от глагола write? Просто формой written!
Сейчас еще вы мне скажете, что это третья форма глагола.
И я опять буду ругаться.
Ну, я так понимаю, что никто не смотрел заголовочек этой таблицы у неправильных глаголов.
Как называется эта форма? Совершенная.
Нет, это не Perfect.
Потому что Perfect это глагольная форма.
А для того, чтобы форма стала глагольной, должен быть вспомогательный глагол.
А вспомогательный глагол у нас появляется, если это сказуемое.
В нашем случае это не придаточное предложение.
Это конструкция -- это причастный оборот.
В причастном обороте нет смыслового глагола.

Кстати, в русском языке что такое причастие? Причастие как грамматическая форма?
Как по-английски причастие? Participle.
А второе причастие, которое у нас в русском языке переводится? Past participle.
А третье причастие, которое церковная церемония.
Я про нее.
В русском языке тоже причастие переводится.
То же слово.
Просто на всякий случай.
Кто не знает?
Если мы знаем это по-русски, мы должны, ну нет, не должны, но тем не менее, хорошо знать все то, что ты знаешь по-русски, еще и на иностранном языке, который ты изучаешь.
У меня была такая, ну мы уже заканчиваем, у меня может быть еще одно предложение.
Меня когда-то попросили перевести юридическую статью (вернее не перевести, а проверить перевод).
Ну, я читаю сначала русский, abstract, да, и там, значит, каждое законченное убийство, ну, там, чего-то там расследуется и так далее, страшного уголовного кодекса и так далее.
Перевод.
Every perfect crime.
Реально, понимаете, это Яндекс Транслейт.
Завершённое, законченное, всё.
Perfect, present perfect.
Понятно, что потом, естественно, и с Яндекс переводчиком и с Google Translate уже очень хорошо всё отработано.
Такие, такие там уже очень редко встречаются, но тем не менее это было очень так, every perfect crime.

Итак, это у нас причастный оборот, в котором будет использоваться Participle Two, то, что вы помните со школы как третью форму глагола, но мы с вами будем стараться называть вещи своими именами, и поэтому, поскольку это не придаточное предложение, а мы знаем, что основой каждого предложения является пара подлежащее-сказуемое, а сказуемое в английском языке в большинстве случаев будет образовываться с помощью вспомогательного глагола и неличной формы глагола, то есть либо Participle One, либо Participle Two, то раз это у нас причастный оборот, то есть "<с заметками, часто сделанными в зеркальном отражении">, даже по-русски мы таким образом это переведем, то как было правильно сказано, в этом случае у нас появится пассивная форма Participle Two -- "<with notes often written in mirror form">.
И это как раз то, о чем я вам говорила, что страдательный залог -- это не только сказуемое в страдательном залоге, но еще и причастие в страдательном залоге, а еще в страдательном залоге может стоять инфинитив to be done (for the works to be done in time, да, например) -- это тоже страдательный залог, поэтому страдательный залог это не только сказуемое.

Так, тридцать два.
We are supposed, абсолютно верно, we are supposed to be at the presentation.
Здесь, слава Богу, мы имеем дело с простым сказуемым.
Что это за форма, кстати? Are supposed.
Какая это грамматическая форма?
Это не Perfect, это Present Simple, потому что начнем с того, что обсудим с вами на следующем занятии, прямо вот форму, универсальную формулу пассива.
Английский язык это математика в лингвистике, то есть английский язык четко раскладывается на формулу, и если вы знаете эти формулы, то выстраивать предложение становится очень легко.
Это не как в школе вас учили вопросительное слово, вспомогательный глагол, подлежащее, потом там смысловой глагол, нет, в английском языке есть гораздо более аналитически выстроенные, вернее, аналитические формулы, и если эти формулы постепенно для себя прояснять, то английский язык становится понятным как математический, это математический язык по сути дела.
Поэтому у кого хорошо с математикой, у того всегда хорошо с английским.
У вас все хорошо с математикой? Я однозначно в этом уверена.
Поэтому мы с вами вспомним всё то, что вы знаете, и правильно скомпонуем эти формулы.

Итак, следующее. 33.
А, это же интересно.
Ну как, 33? Пассивный залог у нас (страдательный).
If the bill is passed by both parliamentary houses, then it becomes law.
Обратите внимание.
Очень хочется сказать, пользуясь переводом на русский язык, if the bill passes, да? Но, во-первых, нам сразу же это не даёт с вами сделать что? Предлог by, а вы все знаете, да, ещё и со школы, что by всегда требует пассивного залога, потому что он указывает нам на деятеля, того, кто это сделал.
Но и даже если бы не было by, поскольку bill сам ножками не может пройти, поэтому, несмотря на то, что перевод на русский язык мы используем активный залог, если законопроект проходит обе парламентские палаты, но тем не менее мы понимаем, что законопроект это неодушевленный предмет и в данном случае это метафорическое описание действия обсуждения этого законопроекта.
Поэтому английский язык здесь более честен с читателями и он по-честному пишет пассивную форму, потому что все-таки законопроект проводят через обе палаты парламента.
А русский язык описывает образно; в английском языке тоже это используется, но это более характерно для разговорной речи, нежели для письменной, тем более академической.
Поэтому if the bill is passed by both parliamentary houses.

34. Что вы здесь предложите?
The quarterly accounts not having been finalized, the auditors were unable to present their report.
Абсолютно верно.
The quarterly accounts not having been finalized.
Это что за форма?
Ну, во-первых, это однозначно пассив, да, потому что у вас задание -- использовать в пассиве.
А вот попробуйте мне сказать having been finalized.
Если у вас есть ing, это что?
Какая форма у нас использует ing?
Continuous? Past Perfect Continuous?
У вас там нет вспомогательного глагола.
Если это Past Perfect Continuous, то у вас будет вспомогательный глагол have.
А у вас having!
Здесь нет личной формы глагола!
Это Participle One Passive.
Если written -- это Participle Two, это причастие страдательное или второе причастие, то когда мы имеем ing-овое окончание, вернее, если быть правильным из истории языка, то ing -- это суффикс.
Ну, просто привычно для всех называть его окончанием, потому что в конце слова стоит.
Если в английском языке ing, то два варианта может быть: либо герундий, либо Participle One.
Герундий это что за форма такая?
Она аналогична чему?
Это существительное.
Герундий -- это отглагольное существительное, и соответственно будет занимать оно позиции, которые может занимать только существительное, а у нас это позиция занята, это the quarterly accounts, поэтому это не герундий, значит это Participle One, ну а дальше having been finalized мы ставим в Perfect и плюс еще в пассивную форму.
То есть, опять же говорю, английский язык, он формульный.
Если вы видите вот эти А, В, С, или там вот как формула силы, да, или там формула, я не знаю, я уже забыла физику совсем, но тем не менее вы знаете, что за каждой буквочкой стоит определенная категория, определенная, как этот вопрос, что такое М, что такое В, что это? Как обозвать вот эти понятия? Это не понятие, да? Категория величины.
Точно даже в английском языке за каждым вот этим ing или been и так далее стоит определенное понятие, определенная грамматическая величина.
Назовем ее так.
И если вы знаете, что за этим стоит, то вы будете правильно это все образовывать.
Ну давайте, чуть-чуть осталось, два предложения, и я вас отпущу.

35.
Вот представьте себе, вот этот вот крутится и всё видит.
Это я про форму глагола показываю.
Вот сейчас он крутится и всё видит.
Is monitored -- это обычно, каждый день.
А сейчас?
This area is being monitored by closed circuit cameras.
Нам в данном случае is monitored тоже хорошо, но у нас здесь нет вот этого вот указания, а хочется сказать, что вот что у нас тут территория охраняется камерами, да, вот висит на дверях, вы знаете, что сейчас вас видят.
То есть смысл вот такой в этом предложении.
Если у нас добавят everyday или usually, тогда однозначно будет is monitored.
А если мы хотим вот именно вот эту надпись, то это форма пассивного залога Present Continuous.

Ну и последнее, 36-е, выдох будет.
А это особая форма, это использование subjunctive mood в английском языке.
Это не conditional, потому что у всех subjunctive mood, сослагательное наклонение, почему-то сразу ассоциируется с conditional 1, conditional 2, conditional 3.
А subjunctive mood в английском языке шире, чем только условные предложения.
Так вот это вот subjunctive mood.
У кого-нибудь есть предположения? Have been submitted? Нет.
Be submitted? Be submitted.
Абсолютно верно.
Если брать более такую грамматически правильную формулу (традиционную, консервативную) в британском языке, то будет should be submitted.
Но тенденция современного английского языка заключается в том что вспомогательный глагол should -- это не долженствование, а именно вспомогательный глагол subjunctive mood он опускается и остается только be submitted.
Что осталось то и остается.
Should просто опускается -- никаких is, никаких was, никаких were быть не должно -- просто be submitted.
Но это тоже subjunctive mood у нас с вами, я надеюсь, все-таки мы во втором семестре будем затрагивать, потому что это тоже такая достаточно мощная тема в английском языке, грамматическая.
Но тем не менее, вот такие вот некоторые мазки по грамматике английского языка.

Мы еще не начинали эту пару.
Если вы к 19:40, то еще пока нет, но можете уже заходить.
Вот, в общем, вот так.
Я собираю у вас эти листочки.
Ваши домашние задания на следующую нашу встречу, чтобы вы пока сразу не совсем прямо погружались во все прелести английского языка, это, как вы поняли, я от вас жду -- это написать к следующему занятию просто my scientific career хорошим разговорным английским языком
Я у вас это соберу, это будет так сказать первый ваш задел на получение зачёта по английскому языку в конце первого семестра.
Объем у нас там был написан, 12-15 предложений, ну давайте на 12 ориентироваться.
Потому что все равно, вы знаете, что мы не звери на экзамене, мы так долго не слушаем, но где-то все равно 5-6-7 предложений вы обычно высказываете, а дальше уже становится понятно, как хорошо вы выучили эту тему и вам задаются дополнительные вопросы.

\end{document}
