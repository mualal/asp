\documentclass[main.tex]{subfiles}

\begin{document}

\section{Лекция 17.01.2024 (Смольская Н.Б.)}

I hope that you are ready after this New Year/Christmas period of (I don't know) entertaining and holidays.
So our plan for today...
Итак, наш план на сегодня.
We will discuss the last (for this semester) topic -- number four, which will be also a task for you to prepare in written form.
And also to discuss a phenomenon which does exist as a phenomenon in the Russian language, but does not demonstrate such serious grammar changes that it has in the English language.
And this phenomenon is Reported Speech.
Which is what?
Reported Speech -- косвенная речь.
Итак, на сегодня разговорная тема -- это наш с вами первый, так сказать, этап сегодняшнего занятия.
И второе -- это грамматика; я думаю, что вы поймёте, почему именно Reported Speech (косвенная речь) становится таким неким аккумулирующим моментом после того, как мы с вами вспоминали времена.
Потому что если вы не знаете времён, то вряд ли вы сможете корректно построить предложение в косвенной речи в английском языке.
Поэтому логично как раз попробовать применить ваши освежённые знания грамматики в аспекте косвенной речи.

So but the topic that we are going to somehow discuss because we will definitely continue doing it at our next class because the topic is really serious (it is not like my department or my scientific interests that you have prepared).
But it is the topic which actually is enumerated as number eight in that list, and it is "<Вопросы научной этики и гражданской ответственности ученых">.
So in English, we presented in a more simplified form, I would say, so "<Ethical problems of modern science"> ("<Этические проблемы современной науки">).
У нас так серьёзно достаточно в программе кандидатского экзамена это прописано, но английский вариант однозначно не снижает серьёзности этой темы, но, тем не менее, звучит мне кажется немножко лайтово (извините за сленг).

\subsection{Ethical problems of modern science}

Итак, первое задание.
The first thing that I would like you to do is to make a list of problems that you think do exist nowadays in modern science and let's say lead to some ethical problems.
Итак, список небольшой, скажем так, 3-4 проблемы, которые вы считаете действительно такими burning questions, burning problems.
Просто запишите себе, конечно, меня интересует, как вы скажете это по-английски, да, поэтому попробуйте назвать их по-английски, либо словосочетанием, либо неким описательным предложением, штучки 3-4.
Дальше я специально освободила доску для того, чтобы я дальше буду записывать, мы посмотрим, насколько вы либо разнообразны, либо наоборот, мыслите и оцениваете ситуацию в современной науке с точки зрения этических норм в одном направлении.
So some three or four problems of modern science that you think really influence modern science from the point of view of ethical approaches, ethical behaviour.
Take your time.
Просто пока для себя, вот что бы вы обозначили как такие проблемы?
In English, of course.

Можно вопрос? Это же третья тема, да? Нет, по-моему, у нас это четвертая уже.
У нас было только две.
Ну, ничего страшного.
Значит, третья.
Тогда четвертая будет совсем такая лёгенькая из тех оставшихся, совсем лёгких.

Вот вам выпадает эта тема в билете.
Ну не в билете, а вас начинают спрашивать: would you please say a few words about the ethical problems of modern science?
Вот что бы вы начали обозначать для экзаменаторов.
Пока для меня просто назовите, о чем бы вы внезапно подумали.

Если вам нравится более такое серьезное название по-русски, "<Вопросы научной этики и гражданской ответственности учёных">, может быть, это вам как-то поможет.

If you want you can use your user-friendly gadgets in order to find some ideas.
\\

--Выполнение задания--
\\

\textbf{Выполненное задание}

1. Usage of nuclear power

2. Plagiarism

3. AI and copyright

4. Usage of false and unproven statements

5. IVF (in vitro fertilisation) and cloning / genetic engineering

6. Humanoid robots

7. Bioengineering

8. Origin of humankind

9. Animal testing

10. ET (extraterrestrial intelligence) searching

11. Modern military issues

12. 3 g-s problem (ghost, guest, gift)
\\

So, ready?
Well, that's not difficult at all because that's not the task for you to write something correctly.
These are just the ideas.
So who would like to begin?
So let's do it just one by one, those who are ready.
And I will try to fix them on the blackboard, on the green blackboard.
So, you're welcome.
Yes.

First of all, the first thing, the first problem that came into my mind is the problem of the usage of nuclear energy.
Because... Thank you. That's it. Without exposes. Right. Just to name and state the problem.

So, what would you suggest being fixed?
Maybe I'm wrong, but I can't understand something.
Maybe because I didn't understand the topic correctly, but I thought it would be about problems in the field of... Interpersonal relationship?
Yes, relationships between scientists about works.
That's also in the same direction.
Ah, so it's clear.
Yes.
So, okay, then maybe using different information from other works (sources, better to say), okay, from other sources to prove your own studies, because these days everyone have an opportunity to...
Yes, so what is the problem?
Using or usage or the use is not a problem.
You should use it but what is the problem?
Without agreement, maybe?
Okay, what is the legal name of this illegal act?
Что это такое, о чём мы говорим?
Как это называется одним словом?
Это что?
Что это, когда вы incorrectly use the results of other person's intelligent.
Плагиат?
Да, плагиат.
So what's the English for плагиат?
Plagiarism.
Absolutely.
So this is really one of the most important and actually one of the most frequently discussed during the examination problems.
То есть обычно все, как правило, начинают, аспиранты отвечать именно, вернее, рассуждать на тему плагиата.
Это самое такое простое, с чего можно начать.
А дальше уже перейти на более серьезные проблемы.
Хорошо, так, еще какие предложения, пожалуйста.
В современной науке существуют этические...

I think it is design of artificial intelligence, because it is a problem about a person.
What does it mean?
Well, maybe not the design of artificial intelligence.
Well, I'm not sure whether we can say that artificial intelligence is the problem.
But how would you give the correct statement of the problem?
Artificial intelligence is definitely that's not a problem.
That is a scientific breakthrough, I would say.
But what? 
Artificial intelligence is a subject.
Subject of what?
Okay, let's find the correct description of this...
If you think that it is a problem so let's (maybe even in Russian) find Russian equivalent.
What is the problem?
Я имею в виду субъектность, какие-то права, может быть, личность ли это.
Ну более простая проблема -- это использование того, что она сгенерировала -- не понятно чьё это.
Именно в плане коммерческого использования.
Ну может быть это кстати, так скажем, следующий виток плагиата по сути дела.
Да?
Потому что мы под плагиатом понимаем действительно двух авторов: один написал, второй некорректно использовал.
А здесь тоже по сути дела плагиат не с точки зрения stealing, а с точки зрения correct addressing and correct authorship.
Хорошо.
Давайте запишем AI and what? Copyright.
Let it be copyright.
Because I think that's really what we have nowadays when we do not deal with physical persons.
But now it is AI which can also lead to some plagiaristic issues.
Okay, interesting.
That's a new thing because that's the first time actually that postgraduate students mention such a problem.
What else?

I think it's mainly the usage of false statements or statements that are not proven in the works of the scientists.
Especially in cases when they want to modify results to make them in that way that they expected.
Okay, so now let's find the term or the word expression that would define this problem because you gave the explanation now; let us try and name this problem.
Maybe, usage of false or unproven  statements.
Okay, so that's interesting, how would you speak and deal on this topic?
So, let it be usage of false and unproven statements.
What about the word proven?
What is the infinitive of this verb?
To prove.
So what about the past simple?
Proved.
And Participle 2 is proven.
If you remember, we have the same instance with the verb to show.
To show -- showed -- shown.
To prove -- proved -- proven.
Это такие mix, да, то есть это неправильный глагол, хотя вроде Past Simple от этого глагола наводит нас на мысль, что и Participle 2 тоже будет в такой же форме.
Однако нет.
Participle 2, то есть причастие, будет иметь суффикс -en.
Это бывшее окончание, форма proven с финальным исходом (как бы окончанием, но сейчас это называется суффикс) -en.
Окей.
А как переводится unproven?
Unproven -- недоказанные или недоказуемые.
То есть недоказуемые и недоказанные в полной мере?
Здесь именно идёт речь об утверждениях, которые не доказаны в полной мере или не доказаны корректно допустим.
В данном случае unproven.
Я слово дала, я имею в виду английское слово.
В данном случае либо unproven, либо false.
Вот как раз unproven -- те, которые недоказанные, то есть они, может быть, какая-то идея была, но они не доказаны, а false -- это неправильно однозначно, да, может, их доказывали, но некорректно, некорректно доказаны.
So what else?
We have four problems.
But there are too many of you here.
I am sure that you have other problems to speak about.
Что ещё мы можем сюда включить в этот список?
Кошечки, собачки?
Намёк, пожалуйста.

In my opinion one of the ethical problems is in-vitro fertilisation (IVF, ЭКО = экстракорпоральное оплодотворение). And cloning.
Да, можно их связать в одно.
What's the English for аббревиатура?
Short form?
Abbreviation?
Да, abbreviation.
Ну, на всякий случай спросила.
Good, very good.
So it's really one of the ethical, modern ethical, scientific ethical problems.
Anything else?
What else?

I think that the close problem is genetic design.
Well, maybe.
I think it is somehow related to cloning in this case.
So you can mention, if we discuss it, so you can mention it here in this issue.
Good, what else?
Maybe some problems that are closer to your field of science, because well, as far as I understand, somehow nuclear power is somehow related to your mathematics and physics experiments and substances and nowadays life.
What else?

Development of humanoid robots.
Okay, I will fix it.
And we will try (maybe) to discuss it later.
What's wrong with the Yandex robot?
Well, I've never seen it, but I think that's great.

I think it's maybe genetic engineering.
So, well, again, I think somehow it is here (in IVF).
I will probably slash it and then add it here (with IVF).
So, genetic engineering.
Okay.

Maybe some sort of problem is bioengineering when we integrate some electric stuff in human brains or something like that.
Okay, so probably that's not like that.
In this matter that's something different.
So let it be bioengineering.
Okay.
What else?
Anything else that we can including to ...

Maybe scientific discussions about God.
Well, from the point of view of ethical problems.
Who is the maker of the Earth?
The origin of humankind, or what?
Or maybe pseudo-scientific...
Would you suggest something how to name this problem, if you agree, so what can it be?
What can be the name of this problem?
Pseudo-scientific discussions on the origin of nature?
Maybe just the origin of everything.
Okay, so let me fix it, and then we'll see.
So, origin of humankind, actually.
To narrow the problem, probably origin of humankind.
Monkeys or non-monkeys?
Extraterrestrials or not?
Okay, origin of humankind.
And actually, discussion around that.
So, anything else?

Maybe tests on animals and ... .
I actually think about it.
Animal testing.
Anything else?

Maybe signals sending into space to find some Earth-like life.
Well, that's interesting, but what is the problem in this?
Alien searching.
Well, I see.
Is it a problem? Yes.
Okay, maybe.
ET (extraterrestrial) searching, okay.
We'll see.
So, anything else?

I think about some engineering in military industry.
Military science.
Military industry.
So modern military issues.
Let's fix it and we will see what is within it.
Anything else?
Что-нибудь ещё?
Конечно у вас уже много, но может быть что-то ещё появилось в ваших головах.

Я могу сказать, но это прямо на фоне уже названных проблем будет мелкой такой проблемой.
Ну давайте.
Я не знаю, как это сформулировать кратко и правильно.
Можно по-русски сказать всё-таки?
Учитывая, что у нас иностранный язык; вы сами понимаете, что мне не важно что, мне важно как.
Чтобы вы пробовали.
The problem about when some person who didn't take part in research is in authorship list.
Well, you have stolen my ideas and actually your home task for the next class.
Give me the opportunity to give you the home task for the next class for you to think about because that is what I wanted you to think about.
The problem, actually the problem of three g's as we called it with your let's say ancestors.
Three g's problem.
Ghost author, gift author, and guest author.
That will be a home task to think about it because it is somehow related to plagiarism, but plagiarism and as we treat it and as we consider it is like a sort of a broader phenomenon but the thing that we usually come across is this problem of three g's (guest, gift and ghost).
Ну, я просто думала, что бы такое вам вот с точки зрения обдумывания дать, и как раз вот это была моя идея.
В следующий раз мы с этого начнем.
Ну чтобы вы каким-то образом.
Okay, so eleven points and the home task.

Окей, so well, I think that we clearly understand that some of the problems are really ethical and some problems are related through ethics to modern science or to scientific issues.
Now I would like each of you to think and to choose one and apply it to your field of research.
What is really important to think about and either to overcome or to get rid of within your field of science or your field of research.
What is really broadly come across or frequently discussed in your field of research?
Think about it and then I would like you to say a few words about that problem.
Which one of those enlisted here?
Some two or three minutes, and then that will be an opportunity for you to speak a little bit.
\\

--Выполнение задания--
\\

So, that's the chance for you now to say a few words, a few phrases or sentences on the matter.
So imagine that you were asked to dwell upon this problem, this topic during your examination.
So that's the chance for you to speak on one topic and then to listen to the others and maybe to collect some information for them for your written topic.
So, who would like to begin?
Who would like to begin the first?
I can try.
Okay.
So let's listen to each other.
Let's practice it, because this is the chance for you to have an eye contact.
It is also important for you to have an eye contact.

Okay, it was somehow difficult to choose a topic because, well, we can talk about plagiarism or something connected with it.
It's something about science in general.
So I think the closest among them is the topic of the usage of nuclear power because I'm studying fluid mechanics and my thesis is more about heating and cooling, and it is also important when we speak about nuclear power.
So when I mentioned this topic at the very beginning of our class, I was thinking about the issue of choice.
Because on the one hand we have scarce resources, oh, sorry, yes, we have scarce resources that are limited and we have to find some unconventional means of powering our economy.
And on the other hand, we have limitless, almost limitless power of nuclear reaction, but it has its own price, its own price, because it is dangerous, as some recent events (for example, relatively recent events in Japan) showed us.
And as well as another part of this big issue is how we recycle nuclear waste, because we need to put them somewhere.
It is very expensive to build special boundaries to keep it away from people, from nature.
So, that's what I think about this problem.
Okay.
Good.
So that's the problem that exists in modern ecosystem.
Ecosystem, yes.
Okay.
Thank you very much.

So who would like to be the next to speak on other issues mentioned?
You're welcome.
That's not difficult at all, as you can see.
And even our last class also demonstrated it.
It is even quite an entertainment, I would say, for you.
So.
Третий класс, вторая четверть.
Особенно, когда глаза опускают.
Это вообще потрясающе.
Давайте, давайте попробуйте, конечно...
I think that the problem of plagiarism is universal; поэтому если нет идей, то тогда к ней.


Throughout most of these topics I think more comprehensive problem is 

I have some particular research.
My scientific problem related to fluid dynamics, computational fluid dynamics, and particularly to improve some models by neural networks.
And what about acceptance of, for example, computational fluid dynamics?
It is just an instrument and now it of course can be used for example for creating artificial heart and something more related to improve level of people health.
And I think I don't need to take a lot of concern at current level of my research to this ethics problems because I just elaborate instrument for creation something maybe bad, maybe not, but it could be regulated maybe by governments or sort of these discussions.
I think that's all.
Okay, well, somehow you dwelled upon some problematic questions.
Okay, thank you.

So, if there is no direct problem that is directly related to your field of research, so then maybe there is a problem that disturbs you, makes you feel anxious about, and you would like to speak on this topic.
So, who would like to practice speaking?
На экзамене всё равно не уйдёте от этого, поэтому всё равно придётся.

So I was thinking, which of these problems is in my field of studying and I decided to tell you about another problem.
Still another problem.
I remembered that in building sphere there is one branch named Energy Efficiency.
And now there is a problem with this branch because it is important, it is connected with ecology and energy saving and some other issues.
But a year ago, our government decided that the reason so important that our country has a big amount of resources, gas or oil, and now this chapter of project documentation is deleted as a chapter.
Now specialists and scientists on this area have problem because their branch and problems are not actual and not important nowadays.
And maybe, I don't know exactly, I didn't study this properly.
Maybe they have problems with funding, for example.
So, that's all.
Okay, okay.
So, somehow that's ethical and professional, but really important for the human being, for the humankind.
Okay, thank you.

Anyone else who would like to share some ideas about this?
If you are asked during your examination to speak on this topic, so what would you speak about?
So this is just a chance for you to practice.
Ну, странно, когда аспирантов заставляешь.
Вы должны рваться сюда все, чтобы попробовать и показать, что все нормально, я не боюсь.
Хоть не умею, но не боюсь.
Ну что, будет кто-нибудь еще?
No?
Плохо.
Чувствуется, что брейк был большим.
Привычка потерялась.
Когда мы каждую неделю с вами встречались, вы были как-то более такими активными.

Так, буквально несколько таких заметочек, просто чтобы вы для себя отметили и внимательнее относились.
Итак, как будет диссертация по-английски?
Слово, которое мы используем?
Thesis.
Фисис -- да, принципиально.
Это не тезис, как в русском языке у нас озвончение происходит.
Это фисис.
И в первом, и во втором случае мы произносим глухой звук.
Только если вы говорите, что вы пишете две диссертации, одну по математике, вторую по философии, тогда у вас будет, как вы помните, фисииз.
Тогда у нас происходит озвончение последнего звука и удлинение предшествующего ии.

Так, как у нас будет строительная сфера?
Строительная, строительство.
Construction, construction, не building.
Construction.
Or civil engineering, но не building.
Потому что building -- это нечто материальное.
The building on the opposite side of the street.
Мы просто лишний раз с вами вспоминаем.

Так, и это то, что я уже говорила, так называемая ложная интерференция, то есть неправильное влияние одного на другого, в нашем случае русского языка на английский или vice versa.
What's the Russian for actual?
Как переводится на русский?
Кстати, два последних занятия с вами уже в этом семестре будем практиковать перевод, потому что вы помните, что самое первое ваше основное задание -- это перевод.
Вот как раз такие моменты, пока просто на уровне отдельных слов, но тем не менее.
Actual это что?
Это действующий или действительный.
А вот то, что актуально, важно, потому что в русском языке мы слово актуально произносим, когда это для нас важно ещё.
То есть занимает наши мысли и является важным для нас.
В английском в этом случае будет, вот когда вы будете описывать актуальность проведённого исследования.
Какое слово будет?
Importance, significance.
Если же как прилагательное, то ещё urgent можно использовать, да?
Важный, вот именно в текущий момент.
То есть в данном случае слово actual в английском языке не описывает актуальность, то бишь важность.
Actual documents, actual documentation, да, которые действуют в настоящий момент.
Поэтому в данном случае вы помните, что есть такие слова, типа magazine, когда нужно просто чисто на уровне, чаще всего просто как-то на таком неосознанном, то есть потом сам думаешь, как я такое сказал, не мог такое сказать, но произошло.
Поэтому просто, когда возникают такие моменты, мы с вами на них обращаем внимание.
Хорошо.
So.
Я просто думала, чтобы вас немножко взбодрить, потому что как-то вы там все ушли в себя.
Пойдем путем (знаете как называется) flipped classroom (перевернутый класс), когда начнём сначала с реальности, а потом будем её обсуждать.

\subsection{Reported Speech}

So, reported speech.
Раздаю вам маленькие листики, и у вас есть прекрасная возможность почувствовать себя педагогом.
Сейчас работаете, и там прямо сказано, что в этом задании ваша задача найти around 21 ошибки.
То есть был нерадивый студент, который написал рассказ про то, как у них была a college lecture, и некто (какой-то college lecturer) прочитал её.
Почувствуйте себя преподом.
Задание как раз на тему Reported Speech.
Заодно вспомните.
Потому что в школе все проходили Reported Speech (косвенную речь), и в университете на занятиях тоже всегда эта тема затрагивалась.
Найдите все возможные ошибки в этом задании.
И дальше мы с вами вот путем обсуждения каждой из допущенных ошибок как раз и вспомним, что такое Reported Speech и что нужно принимать во внимание, когда вы строите предложение в косвенной речи.
Прямо здесь, да, отмечайте и почувствуйте себя преподавателем.
\\

--Выполнение задания--
\\

\textbf{Исправьте ошибки, связанные с нарушением правила согласования времён и пунктуации, в следующем тексте. В общей сложности Вы должны сделать 21 исправление.}
\\

\textbf{ПРЕДСТАВЛЕН ТЕКСТ \textcolor{red}{С ОШИБКАМИ}}

\centerline{\textbf{A College Lecture}}

$^1$Professor Sanchez gave a lecture on transistors last Tuesday.
$^2$First, he explained what \textcolor{red}{are} transistors.
$^3$He said\textcolor{red}{,} that they \textcolor{red}{are} very small electronic devices used in telephones, automobiles, radios, and so on.
$^4$He further explained that transistors \textcolor{red}{control} the flow of electronic current in electronic equipment.
$^5$He wanted to know which popular technological invention \textcolor{red}{cannot} operate without transistors.
$^6$Most students agreed\textcolor{red}{,} it \textcolor{red}{is} the personal computer.
$^7$Professor Sanchez then asked if the students \textcolor{red}{know} how \textcolor{red}{do} transistors \textcolor{red}{function} in computers.
$^8$He said that the transistors were etched into tiny silicon microchips and that these transistors \textcolor{red}{increase} computers' speed and data storage capacity.
$^9$Then he asked the class when \textcolor{red}{had} transistors been invented\textcolor{red}{?}
$^{10}$Sergei guessed that they \textcolor{red}{were} invented in 1947.
$^{11}$The professor said that he \textcolor{red}{is} correct.
$^{12}$Professor Sanchez then asked what \textcolor{red}{was} the importance of this invention\textcolor{red}{?}
$^{13}$Many students answered that it \textcolor{red}{is} the beginning of the information age.
$^{14}$At the end of the lecture, the professor assigned a paper on transistors.
$^{15}$He requested that each student \textcolor{red}{chooses} a topic by next Monday.
$^{16}$He suggested that the papers \textcolor{red}{are} typed.
\\

\textbf{ДАЛЕЕ ПРЕДСТАВЛЕН \textcolor{blue}{ИСПРАВЛЕННЫЙ ТЕКСТ}}

\centerline{\textbf{A College Lecture}}

$^1$Professor Sanchez gave a lecture on transistors last Tuesday.
$^2$First, he explained what transistors \textcolor{blue}{were}.
$^3$He said that they \textcolor{blue}{were} very small electronic devices used in telephones, automobiles, radios, and so on.
$^4$He further explained that transistors \textcolor{blue}{controlled} the flow of electronic current in electronic equipment.
$^5$He wanted to know which popular technological invention \textcolor{blue}{could not} operate without transistors.
$^6$Most students agreed \textcolor{blue}{that} it \textcolor{blue}{was} the personal computer.
$^7$Professor Sanchez then asked if the students \textcolor{blue}{knew} how transistors \textcolor{blue}{functioned} in computers.
$^8$He said that the transistors were etched into tiny silicon microchips and that these transistors \textcolor{blue}{increased} computers' speed and data storage capacity.
$^9$Then he asked the class when \textcolor{blue}{transistors had been} invented\textcolor{blue}{.}
$^{10}$Sergei guessed that they \textcolor{blue}{had been} invented in 1947.
$^{11}$The professor said that he \textcolor{blue}{was} correct.
$^{12}$Professor Sanchez then asked what the importance of this invention \textcolor{blue}{was.}
$^{13}$Many students answered that it \textcolor{blue}{was} the beginning of the information age.
$^{14}$At the end of the lecture, the professor assigned a paper on transistors.
$^{15}$He requested that each student \textcolor{blue}{should choose} a topic by next Monday.
$^{16}$He suggested that the papers \textcolor{blue}{be} typed.
\\

Ну что, всё?
Знаете, как в школе, кто выполнил, отложите ручку в сторону, чтобы я видела, что работа закончена.
Ну ладно.
Немножко вводной теории и дальше будем по ходу обсуждать, в том числе и на английском языке, чтобы привлекать к нашему общению.

Итак, что такое косвенная речь, для чего она используется?
Это первый вам вопрос.
Для чего мы используем косвенную речь?
Принципиально.
Для того, чтобы что?
Прагматичнее надо быть; там как раз вот проблема одна из них была как раз...
Снять ответственность?
Чтобы не слово за слово, я имею ввиду, не пословно представлять.
Когда вы используете косвенную речь, то вы все равно предложение меняете, по крайней мере адресуете к автору, то есть, чтобы избежать цитирования прямого.
Для нас это с вами важно, почему я просто обращаю ваше внимание, потому что мы понимаем, что мы с вами на уровне аспирантуры, и ваша основная задача при обучении в аспирантуре -- это вести научную работу, которая потом выльется в конкретно материализованном результате, то бишь в вашей диссертации, которая, конечно же, должна быть самостоятельным научным исследованием.
И если цитировать или приводить цитаты, то приводить цитаты нужно грамотно, и в этом случае как раз вам  и помогает косвенная речь, когда вы либо кавычками, либо путем переформулирования предложений это делаете, по крайней мере указывая, кто это сказал.
Как раз в этом случае косвенная речь и используется.

С грамматической точки зрения, вот вы сейчас как раз работали с этим текстом, практически в каждом предложении здесь имеется косвенная речь.
С грамматической точки зрения, что такое предложение с косвенной речью, с точки зрения грамматики, структуры предложения, это какое предложение?
Это сложное предложение, сложноподчиненное, абсолютно верно, ну и теперь уж, так сказать, двигаясь ближе к английскому языку, если это сложное предложение значит оно состоит больше чем из одной части.
Сложноподчиненное предложение состоит из каких частей?
Из главной (вернее из главного предложения) и придаточного предложения.
Я вам говорила, что в английском языке слово sentence определяет единицу текста от заглавной буквы до пунктуационного знака, который обозначает конец предложения, а если мы говорим про части этого предложения, то в английском языке используется другой термин, используется термин clause.
Вот как такового эквивалента в русском языке нет, просто потому что в русском языке мы используем слово предложение, как более, ну, обозначающее еще и часть сложного предложения.
Clause, то есть некая значимая часть чего-то, имеющая самостоятельное значение.
Это латинское слово.

В главном предложении при косвенной речи мы с вами будем обязательно наблюдать, какие слова, то есть что нам сразу же показывает, что это косвенная речь?
Told, asked.
Да, абсолютно верно.
То есть это глаголы речи, обозначающие речь.
Абсолютно верно.
To say, to tell, to record, to shout и так далее.
Или, и это как раз ближе уже к вашей научной деятельности, это могут быть ещё глаголы умственной деятельности.
To know, to believe, to wonder.
То есть не обязательно то, что звуками выражается, но ещё и идеями.
Он знал или он знает, что это единственное возможное решение.
Это же не глагол речи, да?
Это глагол умственной деятельности.
То есть набор вот этих глаголов обязательно показывает нам, что дальше речь пойдет о предложении, которое будет иметь статус косвенной речи.

То, что касается придаточного предложения, что это с точки зрения их роли в предложении, что это за предложения такие?
Для кого-то это было откровение на предыдущих занятиях, да?
Придаточные предложения и в русском, и в английском языке выполняют ту же функцию, эквивалентную функцию простому слову второстепенному члену предложения.
Так вот, придаточные предложения в косвенной речи, они эквивалентны какому второстепенному члену предложения?
Дополнению.
Дополнению, абсолютно верно.
Дополнительные придаточные предложения.
Ну, в русской грамматике вы их называли изъяснительные, но они нам немножко, как бы так сказать, немножко нас заморочат, поэтому мы понимаем, что, по сути дела, придаточное предложение в косвенной речи эквивалентно по своей функции второстепенному члену предложения, которое называется дополнением.

Каким образом в английском языке, вот опять же из того, что вы сейчас читали и может быть исправляли или не исправляли, каким образом присоединяется это дополнительное предложение к главному с помощью чего?
С помощью союзов.
Каких?
С помощью союза that, абсолютно верно, с помощью чего еще?
С помощью вопросительных слов, которые являются обязательным элементом в так называемых, помните, специальных вопросах, или если быть более правильным с английской точки зрения WH-questions.
И какой третий вариант в косвенной речи?
С помощью ещё двух союзов.
Вернее, это не союзы, простите, но по сути дела это частицы.
If или whether.
На русский язык, если мы будем переводить, если вы будете мне переводить это предложения, то переводится вот эти два слова будут каким словом?
Это будет частица ЛИ.
Съел ли ты свой ужин, спросила мама, позвонив мне на работу.
Нам принципиально важно вот эти три способа запомнить, потому что эти три способы будут присоединять различные типы предложений, о чём мы чуть позже с вами скажем.

Ну и раз кто-то тут упомянул важное слово пунктуация, это тоже важный момент, который мы в английском языке должны с вами принимать во внимание.
Для косвенной речи в английском языке в отличие от русского языка принципиально то, что главное предложение от придаточного не отделяется запятой.
В отличие от русского языка и вы все это правило помните в русском языке что он сказал, что пойдет в кино -- перед что вы не знаю там с какого класса всегда ставите уже неосознанно подсознательно может и бессознательно уже запятую.
В английском языке в предложениях, имеющих статус косвенной речи, никакой запятой не ставится между главным и придаточным.
Это вот одно из таких вот действительно важных различий, которое нам нужно принять как данность и запомнить.

Можно вопрос?
Без связующих слов может быть?
Да, может, но чаще всего без связующих слов в этом случае мы можем опустить только союз that.
Which, what, when и так далее, или же if и whether опускать нельзя.
Это важные слова, которые показывают, что мы присоединяем предложения особого типа.
То бишь, какого коммуникативного типа?
Это будут какие предложения?
Вопросительные.
Вопросительные предложения, они либо сохраняют свои слова, вот эти специальные вопросительные, which, what, when, where и так далее, либо...
В каком случае у вас появляются слова if и whether?
Частица ЛИ в каком случае появляется? 
При yes or no questions или alternative questions.
Читал ли я книгу?
Ну то есть вот по статусу по своему, если вопросительное предложение начинается просто со вспомогательного глагола, а это не только yes or no question, но еще и alternative question, то тогда в этом случае присоединение будет идти с помощью слов if и whether.
В русском языке они переводятся, еще раз повторяю, частицей ЛИ.

Что еще важно?
Ну и окончательный знак препинания.
Так, вот мы с вами говорим.
Он спросил, когда я приду.
Это какое предложение?
Вот в русском языке это какое коммуникативное предложение?
Утвердительное или вопросительное?
Утвердительное.
Почему мы определяем, что оно утвердительное?
По главному.
По статусу главного предложения.
То же самое правило будет работать, это просто вам напоминаю, то же самое правило будет работать в английском языке.
Он спросил, когда я приду.
He asked when I would come.
Точка. Утвердительное предложение.
Do you know when he will come?
Это будет какое предложение?
Это будет вопросительное, потому что мы определяем статус всего предложения по статусу главного предложения.
А do you know по своей структуре нам сразу же в английском языке показывают, что это вопрос.
Значит, в конце мы поставим вопросительный знак.

Так, ну что, давайте посмотрим, что мы будем иметь здесь, и уже какие-то тонкости по ходу будем вспоминать на примере того, что вы сделали.
Давайте по очереди, каждому будет по одному предложению.
Предложений у нас 16, ну, в общем-то на всех хватит, если читать каждое предложение.
So, who would like to read the first sentence?
Because this is the starting point and it's worth paying attention.
So.

Я понимаю, что всем хочется предложение, где нужно что-то исправлять.
Да?
Я понимаю, что вы в первом предложении что-то исправили?
Да.
Что вы исправили в первом предложении?
Мне показалось, что на личном "<lecture on"> нужно исправить на "<lecture about">
Нет, как раз "<lecture on"> абсолютно правильно, это английская preposition, но вам здесь нужно было, конечно, исправлять только нарушение правил согласования времён.
Больше здесь ошибок нет.
Поэтому, в любом случае, would you please read the first sentence?
Теперь все вы хоть что-то мне по-английски произнесёте.
Пожалуйста.

Professor Sunchez gave a lecture on transistors last Tuesday.
Для нас это предложение важно.
Почему?
Потому что оно нам указывает last Tuesday.
Значит, для нас основным временным регулятором будет то, что это произошло в прошлый вторник, и значит по правилам английского языка, если мы имеем last, то есть, по сути дела, ситуация, с которой мы дальше будем иметь дело, будет максимально описана с помощью каких видовременных форм?
Past Simple или Past Continuous.
То есть это для нас тоже важно.
Present у нас появляться не будет.
То есть первое предложение для нас тоже принципиально важно.
На него надо тоже было сразу обратить внимание.
Почему я обращаю внимание на Last Tuesday?
Потому что мы снова своего рода адресуем наши с вами знания к знаниям косвенной речи.
И самое главное, вот то, с чего я и начала, что косвенная речь в русском языке, она тоже существует.
Примером, вот моё предложение.
Он спросил, когда я приду.
Когда ты придёшь?
Это прямой вопрос.
Он спросил, когда я приду -- это косвенный вопрос.
Но какова особенность, которая отличает полностью правила косвенной речи в английском языке от русского?
Когда в предложении, в главном предложении присутствует определенная особенность, которая дальше рулит всё остальное.
Когда мы с вами наблюдаем в главном предложении, что?
Глагол в одном из прошедших времен (назовем их так), в одной из прошедших видовременных форм, то тогда в придаточном предложении (то есть в той самой косвенной речи) не может использоваться никакая форма настоящего или будущего времени.
То есть здесь нужно себя контролировать с точки зрения грамматики.
При этом на уровне перевода мы остаёмся всё равно в той ситуации, которая должна быть без формального изменения.
В чём смысл вот этого правила?
Вот как раз то, что я сказала.
When will you come?
Вопрос относится к будущему.
Мы здесь с вами наблюдаем форму будущего времени.
Когда ты придёшь? Вопрос.
Если мы хотим построить этот вопрос, передать его в косвенной речи, he asked...
Я передаю там своему другу.
Он спросил про то, когда ты придёшь.
Суффикс -ed указывает нам на то, что это прошедшее время. 
Значит, в любом придаточном предложении, которое будет следовать после ask, не может использоваться future.
В этом случае мы с вами на прошлом занятии вспоминали, есть особая искусственно созданная форма, которая не существует, не может описывать никакое реальное действие в нашем философском или физическом времени.
Да, времени (на английский язык мы переводим это словом time).
Что мы в этом случае должны использовать?
Future in the past, как раз формально созданное грамматическое явление, подчеркиваю, формально созданное.
Оно существует на уровне формы, но на уровне содержания его нет.
На уровне содержания у нас остаётся будущее время.
He asked when I would come.
Вот мы с вами формальную трансформацию провели, will поменяли на would.
Но если вы сейчас на русский язык переведёте это предложение, что у нас получится?
Мы переводим на русский, вам достался перевод.
Он спросил, когда я приду.
То есть будущее время на уровне содержания и на уровне перевода на русский язык, оно всё равно будущее время осталось.
То есть вот это несоответствие формальности и реального содержания для английского языка очевидно.
Это просто грамматическая форма -- показатель того, что это косвенная речь.
Это просто показатель, а на уровне содержания мы остаемся как бы в той же ситуации, которая была в прямой речи.
Или I like coffee.
She said that she liked coffee.
При этом, если мы переводим, мы всё равно с вами остаемся в ситуации Present Simple, и переводим это с помощью настоящего.
Для русскоговорящих это нужно осознать, что это и форма, и содержание, они между собой никаким образом не связаны.
Вот то, что в школе говорили, нужно опуститься на ступеньку вниз.
Так вот это опускание на ступеньку вниз, это только на уровне формы.
На уровне содержания мы остаемся там же, где и были.
И так далее, двигаемся дальше.

Можно вопрос? Да.
Если в главном предложении будет Future, например.
Всё, мы ничего в этом случае не производим.
Как есть, так и есть.
When will you come?
She will ask you when you will come.
На вот это вот изменение, на формальное изменение влияет только наличие глагола в главном предложении в прошедшем, в любом из прошедших времён.
Если нет прошедшего, то тогда, слава Богу, все остаётся как было.
Если прошедшее, то тогда происходит грамматическая трансформация.
Грамматическая, не содержательная.
Только грамматика меняется.

Можно вопрос?
А мы меняем порядок слов?
Сейчас мы как раз дойдём до этого и вспомним.

Итак, двигаемся дальше, пожалуйста, предложение под цифрой 2.
Прочитайте, что у вас получилось и мы дальше посмотрим.
First, he explained what were transistors.
Тут Present Simple, а справил на Past.
На Past Simple.
Почему?
Потому что в главном предложении у нас Past Simple.
Потому что у нас глагол explained.
Past simple стоит в главном предложении.
Итак, ошибка по неправильному использованию времени.
Есть ли ещё какие-то ошибки в этом предложении?
Может быть, were в конец поставить?
Абсолютно верно.
Were нужно поменять местами с transistors.
И правильно предложение будет звучать.
First he explained what transistors were.

И вот теперь как раз мы приходим к вашему вопросу, потому что в этом предложении мы как раз и столкнулись с этой самой ситуацией.
Если с предложениями утвердительными, I like coffee, she said she liked coffee, всё понятно.
Здесь было утвердительное предложение и стало оно точно так же сохранило свой порядок слов для утвердительных предложений.
А вот с вопросами немножко сложнее, вернее, проще, но сложнее, потому что это нужно знать, что происходит с вопросами, когда мы их трансформируем в косвенную речь.
В вопросах, вот если здесь я написала, здесь мы с вами наблюдаем wh-question, мы видим, что у нас на первом месте стоит вопросительное слово и частично обратный порядок слов -- когда вспомогательный глагол в вопросе должен занять место перед подлежащим.
Если же мы вопрос прямой переводим в косвенный вопрос, то тогда в качестве союза-присоединителя сохраняется вопросительное слово, а дальше восстанавливается прямой порядок слов.
То есть вспомогательный глагол занимает свое место после подлежащего.
Will you come?
Здесь у нас when I would come.
Здесь у нас как звучал вопрос, который задали профессору Санчесу?
Его спросили, а после этого он explained.
Кто-то поднял руку и что у него спросил?
What are transistors?
Спросил Иван.
И he (профессор Санчес) explained, если мы его делаем косвенным вопросом, то, во-первых, глагол are встает на свое место, то бишь после подлежащего what transistors are, но поскольку explained у нас в прошедшем времени, то are должно опуститься на ступеньку вниз и стать were.
И таким образом мы с вами получаем вот это вот как раз предложение.
First he explained what transistors were.

Number three, пожалуйста.
He said that they were very small electronic devices used in telephones, automobiles, radios, and so on.
Замечательно.
Сколько ошибок в этом предложении?
Ну, я две нашёл.
Это лишняя запятая.
Больше и не надо, Слава Богу.
Обратите внимание, после said по правилам русского языка, чувствуется, русскоязычный студент писал, да, поставил запятую.
В английском варианте её быть не должно.
Абсолютно верно.
И вторая ошибка.
Вот временная форма.
Are мы поменяли но were.
Почему?
Потому что у нас глагол said в главном предложении.
Абсолютно верно.
Обращаю ваше внимание еще.
Это не связано с косвенной речи, но просто мы здесь сталкиваемся с вами.
У нас идет telephones, automobiles, radios, and so on.
Здесь как раз с точностью до наоборот по сравнению с русским языком.
В русском языке мы знаем, что когда перечисляются однородные слова, в данном случае однородные дополнения, то у нас ставятся запятые до тех пор, пока мы не доходим до союза И, перед которым запятая не ставится.
В английском языке если у нас два однородных члена, то тогда запятая перед союзом И не ставится.
Если их больше, чем два, то есть вот как в нашем случае, telephones, automobiles, radios, and so on.
So on -- это тоже, по сути дела, такое однородное, но без уточнения, то тогда перед союзом И в английском языке запятая ставится.
Вот в нашем предложении она как раз и стоит.
Если бы было used in radios and so on или in telephones and automobiles, то запятой бы не было.
А поскольку у нас их больше, чем два, то перед союзом И запятая тоже ставится.

Пожалуйста, четвёртое.
He further explained that transistors controlled the flow of electronic current in electronic equipment.
Да, controlled мы поставили в Past Simple, because we have explained in the main clause.
Very good, thank you.

Number five.
He wanted to know which popular technological invention  cannot operate without transistors.
Are there any mistakes here?
Where is the mistake?
Maybe, cannot.
First of all it should be separated.
And it should be could.
И у меня тогда огромная просьба, не читайте с ошибкой.
Вы прочитали предложение как оно есть, не надо такое делать.
Вы мне читайте сразу, как вы исправляете.
А то я сразу напряглась, думаю, ну всё, не исправил ошибку.
Итак, что у вас получается?
Пожалуйста, правильно прочитайте предложение.
He wanted to know which popular technological invention  could not operate without transistors.
Good. Could not. Очень хорошо. Okay.

Number six.
Most students agreed, it was the personal computer.
What mistakes have you found?
I changed is into was because there is a Past Simple.
Agreed in Past Simple.
Because in the main clause there is a verb agreed in Past Simple.
Okay, is that all?
Any other mistakes in this sentence?
I can propose alternative variant.
Most students agreed that it was the personal computer.
That's not an alternative variant, that's the only possible variant.
We should insert that.
Итак, мы вставляем союз that, который должен здесь быть после agreed.
Most students agreed that it was the personal computer.
Абсолютно верно.
And what else?
Delete comma (запятая).
А dot -- это точка в конце.
So, we delete comma, add that and change is into was.
Три ошибки в одном предложении.
Причем это именно вот то, что себя надо заставить делать.
Так как раз вот при косвенной речи.
И в этом предложении у нас три ошибки.

Number seven.
Professor Sanchez then asked if the students knew how do transistors function in computers.
Aha.
So, how many mistakes?
I think that one, but maybe more.
Well, one mistake is correct, is correctly corrected, but there are two more mistakes.
So let's find them.
Well, first of all we should change word order, because, потому что если у нас есть how, это у нас намёк на что?
На то, что когда-то в прямой речи был вопрос.
А если был вопрос, то из вопроса он становится вопросом с прямым порядком слов.
То есть вспомогательный глагол после вопросительного слова how больше использоваться не может.
Правильно?
Это значит, что мы его оттуда изымаем.
Он не должен там стоять.
Таким образом, у нас уже получается.
Professor Sanchez then asked if students knew how transistors...
Эту ошибку мы уже исправили.
А дальше?
Что ещё мы исправляем?
Functioned?
Почему?
Потому что это согласование с knew.
Потому что если мы теперь посмотрим на структуру этого предложения, то у нас в этом предложении аж два косвенных вопроса.
Один косвенный вопрос у нас присоединяется к частице if (вот этим ЛИ).
Do you know? – спросил он у студентов.
А в косвенном вопросе мы получили if the students knew.
А дальше -- do you know how transistors function?
Спросил он у них.
Но поскольку knew у нас образовалось из know, потому что asked стоит в самом главном предложении, да?
В самом главном предложении у нас есть asked.
Поэтому после него никакого настоящего времени больше быть не может.
Главное предложение рулит всем.
И полностью убирает любую возможность использования настоящего времени после себя.
И таким образом мы с вами вынуждены все глаголы, которые у нас встречаются во всех оставшихся придаточных предложениях, исправлять/изменять в прошедшее время.
Итак, professor Sanchez then asked if the students knew how transistors functioned in computers.
И мы ставим точку, поскольку professor Sanchez asked.
Это утвердительное предложение.

Так, number eight.
Кто следующий?
He said that the transistors were etched into tiny silicon microchips and that these transistors increased computers' speed and data storage capacity.
Good.
Increased computers' speed and data storage capacity.
Абсолютно верно.
Increased мы исправили.

Девятое.
Сможете? Давайте.
Then he asked the class when had transistor been invented?
Есть тут ошибки?
Какие-то ошибки нашли в этом предложении?
Что-нибудь бросается в глаза?
Раздражает что-нибудь глаз?
Нет пока?
Ну ладно, ничего для первого раза.
Итак, давайте поможем.
Итак, какие ошибки?
Сколько в этом предложении ошибок?
Все нашли две?
Все две нашли?
Итак, давайте, самая главная, первая ошибка какая, которую вы должны были исправить сразу?
В конце предложения что?
Стоит question mark.
А здесь question mark не должен стоять.
Почему?
Потому что у нас then he asked.
He asked, вопроса нет.
Он вопрос задал, а само то предложение утвердительное.
Поэтому question mark исправляем на точку.
Это на внимательность.
Как в любом деле.
Ну и плюс, конечно же, мы должны здесь еще исправить грамматическую ошибку, которая касается чего?
Порядка слов.
Word order.
И если мы исправляем word order, то где у нас (в каком месте) должен быть исправлен порядок слов?
When transistors had been invented.
Поскольку это косвенный вопрос, то вот как у меня там и записано, и мы говорили, восстанавливается прямой порядок слов, да, прямой порядок слов.
When transistors had been invented. Dot.

Кстати, dot ставится, вернее, используется когда?
Когда адрес почты пишешь.
Я сознательно сказала.
Итак, мы используем в английском языке слово dot.
В каких случаях?
Когда мы называем адреса электронных почт (dot com и так далее).
Какие ещё слова в английском языке есть, которые обозначают слово точка?
Point.
Когда используется point в числах?
Когда мы отделяем дробную десятичную часть, ну и там сотую и так далее, от целой.
Six point zero как в оценках фигурного катания.
Знак умножения multiplied.
Или times.
Dot не называется.
Это dot product, может быть, только если скалярное произведение.
Да.
А вот с точки зрения пунктуации какие слова используются?
Full stop или period.
Full stop -- это British English, period -- это American English.
Вот так вот, вот американский вариант всё равно находит для себя хоть какую-то возможность, тем не менее, дистанцироваться от британского английского.
Поэтому в нашем случае это будет full stop, ну или те, кто американского варианта придерживаются, period.

А можно вопрос? Да.
Можно внести немножко хаоса?
Я боюсь хаос под конец дня, но давайте.
Ну тогда давайте я отдельно потом спрошу, чтобы всех не задерживать.
Хорошо, давайте мы проверим, и дальше уже я вас отпущу, напомню домашнее задание и отпущу.

Так, десятое.
Давайте с вас.
И сюда вернемся.
Как раз на всех хватит.
Читайте.
Sergei guessed that they had been invented in 1947.
Very good.
Очень замечательно, всё правильно.
Что вы здесь сделали?
Заменил пассивный залог Past Simple на пассивный залог Past Perfect.
На Past Perfect, всё правильно, потому что мы должны, пользуясь этой метафорой, опуститься на ступеньку вниз.
До этого вы с настоящим временем имели дело, да?
Изменяли его всё время на Past Simple, а здесь мы с вами видим обязательное условия для того, чтобы на Past Perfect изменить.
Had been invented в предложении номер 10.

Так, пожалуйста, 11-е.
The professor said that he is correct.
Нет, он Сергею сказал.
Молодец, садись, пять.
Нет.
Профессор said обратите внимание.
He was.
That he was correct.
The professor said that he was correct.
А здесь the не потеряно.
Здесь всё правильно с артиклями.
Мы артикли не трогаем.
С артиклями здесь хорошо всё, в этом тексте/задании всё.
Здесь только косвенная речь.

Так, пожалуйста, кто из вас возьмёт 12-е в порядке очереди.
Professor Sanchez then asked that...
Нет, он спросил.
Если он спросил, то он мог спросить либо if или whether, либо what, поэтому what у вас остается.
Он же спросил, он же задал какой-то вопрос.
Если у вас это вопрос, то тогда что?
То тогда восстанавливается прямой порядок слов.
Правильно? Восстановите прямой порядок слов.
Прямой порядок слов -- это когда подлежащее, а потом сказуемое.
Professor Sanchez then asked what the importance of this invention was.
И что ещё? И full stop.
И full stop мы опять задаем.
Здесь вернее ставим.
Точку (full stop) вместо вопроса (question mark).

Так, тринадцатое предложение.
Many students answered that it was the beginning of the information age.
Абсолютно верно.
That it was the beginning of the information age.

Так, пожалуйста, кто из вас возьмёт на себя 14-е, отдаст право 15-го предложения кому-то другому.
Ну, пожалуйста, кто будет джентельменом и отставит себе 15-е.
Ну, давайте.
Так.
Ну давайте, давайте, вам передали.
At the end of the lecture, the professor assigned a paper on transistors.
В 14-м ошибок нет.
In the sentence number 14 there are no mistakes.

Так, ну что, пожалуйста, вам 15-е.
А вам 16-е будет для справедливости.
Так, слушаем 15-е.
Потому что мне даже интересно, давайте так, кто сколько ошибок нашел в 15-м?
He requested...
Вот здесь на самом деле у меня сомнения, мне кажется, здесь that вообще не нужно.
Ну that можно опускать...
Но это не относится в данном случае к прямой и косвенной речи.
Ладно.
Как вы понимаете, нас интересуют глаголы, порядки слов.
He requested that each student chose a topic by next Monday. 
У кого есть другие варианты?
Would choose, но только would это не Future in the Past, а would это что?
Модальность?
Здесь не модальность.
Это subjunctive mood, это сослагательное наклонение.
Либо по правилам сослагательного наклонения должно быть he requested that each student should choose a topic, либо, и я вам это один раз уже говорила, по правилам современного английского языка это частица "<БЫ">.
Он попросил, чтобы...
Ведь разница, да?
Он сказал, что пойдёт в кино.
И он попросил, чтобы студенты...
Вот как только в русском языке "<бы"> появляется, хоть в каком-то виде, значит, дело нечисто.
«Бы» как правило, это наш отсыл в русском языке к сослагательному наклонению.
То же самое.
Вот здесь русский язык нам на уровне интуиции подсказывает, вернее помогает, что здесь не косвенная речь.
Хотя очень похоже по конструкции предложения на косвенную речь.
Сразу же забегая вперед.
16-е это тоже не косвенная речь.
Это тоже сослагательное наклонение и аналогично будет выстраиваться.
Мы сейчас заканчиваем, и вы будете заходить.
Точно так же будет строиться это предложение.
Итак, я просто уже как бы прочитаю вам так, как это должно быть.
И сослагательное наклонение мы будем с вами разбирать во втором семестре.
Но это такая вот затравочка.
Итак, he requested that each student should choose a topic by next Monday.

И в 16-ом.
He suggested that the papers be typed.
Be typed.
Без всяких should, would и так далее.
Be typed.
Это два предложения, не содержащие, еще раз повторяю, косвенную речь, а наоборот, демонстрирующие нам сослагательное наклонение.
Один из способов использования, вернее, один из пунктов использования сослагательного наклонения.
Так, хорошо.

Я записала вас всех, кто у меня был в списках.
Те, кто сегодня новые пришли, просто сейчас мне напомните, я вас добавлю в свою группу на портал аспирантуры, и может вам должно было прийти оповещение, но где-то оно затерялось.
И просто я ничего пока туда не грузила, но все проверьте, что у вас есть доступ, соответственно, у вас появится, если вы зайдёте на PORTASP.

Если зайдёте под своим логином и паролем, то в вашем личном кабинете появится иностранный язык и в скобочках Смольская Н.Б.
Для вас сейчас, вы сейчас там пока ничего не увидите, у меня все закрыты задания.
Но либо сегодня вечером, либо завтра я открою и подгружу для следующего нашего занятия достаточно такое challenging упражнение по использованию времен глагола.
Мы с него начнем проверять, но это не какие-то простые очевидные будут случаи.
А я постараюсь где-то, может быть, штук 15 предложений, даже 15 будет достаточно, чтобы они у вас, ну, нужно будет подумать, что-то поискать, либо ответы, либо объяснения.
Выбор правильного ответа из четырех.
То есть примерно это одно из тех типовых заданий, которое в тесте у вас тоже будет.
То есть выбрать правильную форму, которая будет соответствовать.
Я его просто туда подгружу, соответственно мы начнем с его проверки.
Вы для себя выберете ответы, я выведу потом само упражнение на экран.
У вас будут ваши домашние ответы -- что вы выбрали, что вы подобрали и так далее.
Мы обсудим использование этих времен.

Значит, дальше ваше домашнее задание, вот это вот, the problem of three g's (ghost author, guest author, gift author), мы послушаем и посмотрим.
Кто-то знаком с этими терминами, для кого-то это будет новым, ну, может быть, вы сам феномен этот понимаете, но, тем не менее, вот термин, может быть, никогда не использовали.

Ну, и потом уже, на следующем занятии я выдам всё, что там проверила, выдам вам ваши топики, потому что у вас останутся еще две темы.
Соответственно, будет то, что мы в следующий раз с вами обсудим, и какую-нибудь простенькую, типа University as a Science Centres.
Это совсем простая тема, и я думаю, что вы тоже её выполните, чтобы 4 топика к тесту (вернее к зачёту) у вас были сделаны.
Всё.
С двух сторон уже очередь.


\end{document}
