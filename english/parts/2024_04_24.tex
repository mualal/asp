\documentclass[main.tex]{subfiles}

\begin{document}

\linksection{Лекция 24.04.2024 (Смольская Н.Б.)}

Today I would like you and us (you in the majority of cases) to actually practise your speaking and at the same time we will discuss some grammar aspects and I would like you to also practise some of those skills that will be important for you during your examination.

And as you remember, there is one topic left that I actually mentioned several times.
I hope you remember what topic is it.
So what is the topic that is left and that is really important and dear for me?
So this is the Role of Foreign Languages in Modern Scientific Communication.
So this is the topic that we haven't discussed yet.
And actually I would like you and us to begin with the very first question, which is quite obvious.
So do you think that foreign languages are really important nowadays for scientific communication?
And what is this role that you would describe or present in some particular aspect?
Because then we will begin discussing this topic in a more precise manner, in a more precise way.
And maybe what languages?
So that's up to you, just to begin discussing.
The role of foreign languages.

We can communicate in different languages, maybe in one special, for example, English.
And also...
Okay.
Of course you understand that my main interest is, of course, English, but we do understand that our modern world is multilingual (многоязычный).
But when we speak and when we think of English as the language of international communication and communication in the sphere of science and scientific collaboration and cooperation.
In this case, we actually find ourselves in what way of language communication?
It is actually bilingual communication because in this case we speak and we think of our native language and one more language, which is (in the majority of cases) English.
I would say this is like a sort of a conflict between a bilingual international and scientific communication and multilingual world.

That's not by chance that I actually mentioned this very thing, because I have prepared a short text for you, so, and of course, the beginning of our work with this text will be the task for you to read this text, and then I will give you some tasks and we will work somehow with this text and then we will move forward further.
So, well, I hope that I have enough copies.
If not then I will prepare.
So, I would like you first to read this text, to first start reading and then after you are ready with the text we will start discussing it.
Yes, that's for you.
Yes, first read this text.
This is just a starting point for us.
One minute, I will bring the text for you.

\newpage
\sublinksection{The Role of English in the Field of Science}

\hypertarget{ltask:2024-04-24-1}{--- Выполнение задания ---} (\hyperref[task:2024-04-24-1]{\color{blue}{перейти к тексту задания}})
\\

\textbfind{Выполненное задание (чтение текста и перевод на русский язык)}

\textbf{The Role of English in the Field of Science.}

``English has become the de facto language of science.''

Written by Mike Brotherton, he elaborated the increasing role of English in our day-to-day lives. Every published paper is written in English; all movies will always include English; Most conferences are using the most common idiom known to man as English... life goes on and on with the people’s responsibility to \textbf{learn English language} throughout our lifestyle and socialization.

How would we picture the world speaking the only spoken language known today? Imagine the business of \textbf{foreign language courses} beginning to slope down. Would this be beneficial to all people around the world?

Added by Brotherton, the advantages of a single language are self-evident. ``In the past papers published in other languages often went unnoticed and incited, resulting in people reinventing the wheel. Furthermore, at most people must learn a second language, not 3 or 4 or 5 or whatever to read journals in multiple languages.''

Do you think this is good or bad news to you? How would we respond with the other foreign languages we have been using today?

But in the field of science, sticking to English alone makes such quandary. It’s not like there are laws that science must be done in English, just as what Brotherton said. ``This is cultural evolution and while minor regulation is often needed to prevent problems and keep things running smoothly, wholesale imposition of major changes is almost always a terrible idea that destroys the institution it’s meant to improve. We’re going to have to live with English as the language of science, and find ways to let everyone operate within that standard.''

Such state of principle has been supported by a site where a person named Jacob wrote an article online about \textit{Science is Human Right}. Here is its highlights:

\textit{``Over the past 80 years, there has been a steady increase in the proportion of scientific publications written in English and a corresponding decrease in the use of all other languages for global scientific communication. At first sight, this appears to offer significant gains in efficiency and universality, as scientists everywhere are required to learn, at most, one language in addition to their own.}

\textit{``There are, however, drawbacks to this system that may be significantly affecting both the development of global science and its social effects. \textbf{Issues of concern include the poor correlation between ability in a second language and scientific ability; the disproportionate burden of language training and technology borne by non-English-speaking countries; and a widening gap between the language of science and the languages of public discourse, government policy, and community life.}''}
\begin{itemize}
	\item What do you think about this?
	\item Do you possibly assume that the rising of English language in the field of Science slowly eradicates the role of other foreign languages?
\end{itemize}

\textbf{Роль английского языка в науке.}

...
\\

So, ready?
Yes?
Well, I hope, because we will discuss the language of this text and the text itself.
My first question is, of course, well, you all know that I love English and I speak about English in many directions of actually studying the text.
So how would you define the type or the genre of this text?
What is it?
You can use the Russian word, that's okay.
So what is it?
How would you define this text?
Newspaper article.
Yes...
But that's not exactly newspaper, I would say, because it develops some specific topic.
So in this case, that would be what kind of article?
Рассуждение?
Yes, but if we generalize it.
That's not a scientific article, definitely not, because we do know the structure of any scientific article, but what is it?
This is what is called научно-популярный текст.
What's the English for научно-популярный?
Popular science.
Popular science, exactly.
Popular science, well, actually, our Russian term, научно-популярный, actually, that's the back translation of the English equivalent.

How do you define popular science text or popular science article?
So what is their aim?
And what is the audience of such texts?
That's not for people actually being specialists in a particular scientific area of studies, but still to give them information about specific point or specific area.
So what, and now my next question is, so how would you define the main idea of this text?
The main idea, not the main topic, because we do understand that the main topic is what?
What is the topic of this text?
The Role of English in the Field of Science.
Right.
Usually when we speak about the topic of any text, in terms of or in the sphere of science, this is actually the title of the article.
This is the topic of the text.
But what about the idea?
How would you define the idea of this text?

The last two questions are included in this articles.
Yes, the questions will be our next step, because I will definitely then ask you to...
I think that in such type of articles the main idea are summed up in the last paragraphs, and we can see the questions that are asked.
But how would you define this idea?
So would you please present it in your sentence, in the sentence of your own?
So for this sentence not to be a plagiarism.
So would you please paraphrase and give your version of the main idea of this statement?
I think that the main idea of this article is to make the audience think about these two question.
What?
What do you think about the role of English in modern science?
Okay, so that's your opinion.
Any other variants of how to present the idea?

Pros and cons of implementing English as the language of science.
Good.
Any other variants?
So try to verbalize the idea that you all, I am sure, you definitely understand what is the idea, but I would like you now to use your skills, you remember, your skills of paraphrasing, which is important when we deal with science.

This article describes increasing role of English in modern science, in contemporary science.
Good, good, thank you.
Any other variants?

What benefits of using English language as the main language in science?
Good, of using English as the main language of modern science.
Good.
Any other variants?
Think about synonyms or paraphrasing of the text given.
Is there any?
No.
Okay, because, well, we still have two questions, but that will be our next step.

So, and before we continue speaking about the text as a text, as a problem, I would say.
I would like us to quickly look through the language of this text, because there are some specific things that I would like you to comment upon and to give me your explanation or translation when necessary.
So it is interesting and it is actually in the direction of our grammar discussion -- the beginning of the first and of the third абзац.
Абзац?
Как перевести на английский?
Paragraph.
Проверяю, что вы помните это слово.
Of the first and of the third paragraphs.
And the constructions are interesting.
So it is "<written by Mike Brotherton"> and "<added by Brotherton">.
How would you translate these constructions into Russian?
So what would be your variants of translation?
Read the whole sentence and try to think how would you introduce or how would you present these constructions in Russian.
"<Было сказано кем-то">.
Good.
Хороший вариант, абсолютно верно.
Как сказал ... .
By the way, do you know Mike Brotherton?
Have you ever heard about this person?
Well, he is quite a popular popular scientific writer.
He is an astronomer.
But at the same time presenting popular scientific novels, which are quite popular nowadays in the United States.
So, and somehow, Timur, you are right when speaking about, I mean, the newspaper, so like newspaper article, but at the same time, it is popular science article because it discusses the scientific topic.

Okay, "<added by Brotherton">?
How it could be translated into Russian?
Literally it is "<дополнено">.
Ну, может быть, опять же, да?
Или "<как далее отмечает">, да, как далее отмечает Brotherton, да?
Просто вот к вопросу о том, что, опять же, so, when we use the constructions or the translation of the Russian language, we use a subordinate clause, да, то есть мы целое предложение с вами выстраиваем, когда у нас есть подлежащее и сказуемое, а в этом случае английский язык, опять же, от предложения избавляется и использует причастный оборот.
И там, и там, просто вот вы сами видите, тут практически параллельная конструкция "<written by">, "<added by">, то есть через предлог "<by"> мы показываем деятеля, а не выстраиваем, как в русском языке, целое предложение.

Okay, just to make sure that you know how to translate or you know how to explain the phrases.
So what is "<to slope down">?
Замедлиться?
Схлопнуться?
Снижаться?
Да, снижаться, уменьшаться по количеству, но в данном случае снижаться, то есть если мы представляем с вами графики.
Абсолютно верно.

Well, I hope that you have paid attention to the term?
Потому что это действительно термин специфический, and I would like you to explain it.
This is paragraph number three.
Furthermore, at most people must learn a second language, not third, fourth or fifth, or whatever, to read journals in multiple languages.
Have you paid attention to this term?
And what is interesting here?
A second language.
What is interesting about it?
Мы с вами знаем, что second, third, fourth, fifth -- это что такое?
Числительные.
Какие?
Порядковые (ordinal).
What about grammar in the English language?
So when we use ordinal numerals, we should remember what?
That it should be accompanied, almost obligatory, by the definite article.
Да, определенным артиклем "<the">.
А здесь мы с вами видим "<a">, a second language.
Неопределенный артикль.
Я думаю, because этот артикль to "<language">, not to "<second">.
So how would you explain what kind of language or what type of language is it?
Some.
That's not some.
No, that's not some.
This is the term.
Мы с вами помним, что это научно-популярная статья.
Не научная, то есть это не для методистов, преподающих иностранные языки, но тем не менее это в рамках лингвистики и лингводидактики.
Так назовём, готовьтесь, дальше будет продолжение.
Тоже вас, так сказать, намекаю вам.
И поэтому, что такое будет?
Как вы на русский переведёте?
На русский язык, что это за термин, как вы думаете?
Другой?
Нет, это не другой.
Мы переводим это как второй иностранный язык.
Как правило, иностранный опускают, а это второй язык.
Или же, что стоит за этим вторым языком?
Это неродной язык.
То есть это не второй, третий, который я знаю.
Потому что вы можете сказать, so I know English, and this is the first foreign language I know, then I know French, it is the second foreign language.
В этом случае будет the first, the second, the third.
А если мы используем вот это вот a second foreign language, то просто имеем в виду, что это мой неродной язык.
English is a second language for me.
Интересное такое вот, да, то есть мы с вами знаем правила и научены определенному грамматическому правилу, но, тем не менее, это не исключение из правил, а это именно определённое значение.
Так.

Just again some point.
How would you translate the sentence in the next paragraph which is "<How would we respond with the other foreign languages we have been using today">?
How would you translate this sentence.
Well, of course I'm interested in the verb "<to respond"> with the preposition "<with">.
Because we usually use the verb "<to respond to"> -- "<отвечать кому-то">.
Мы помним, что предлог "<to"> в английском языке обозначает направление.
Здесь мы с вами наблюдаем "<with">.
Как вы переведете?
Попробуйте.
Как бы мы реагировали, если бы использовали...
Ну, у нас здесь нет, вот видите, уже русский добавляете креатив.
Что же нам делать со всеми другими иностранными языками, с которыми мы имеем дело сейчас?
Что же нам с ними делать?
Если дословно переводить, то "<отвечать чем-то или кем-то">, да, ответить сарказмом в ответ на шутку, то есть вот это прямое значение, но в вашем случае абсолютно правильный перевод.
А что же нам в таком случае делать с другими оставшимися языками?
Грамматический момент: как раз вы правы, что мы добавляем здесь со всеми ДРУГИМИ оставшимися языками.
Где грамматический показатель в английском языке, который именно так вынуждает нас это перевести?
The other.

Я надеюсь, что вы помните, что в английском языке есть пары прилагательных: other и another.
Но почему у нас есть another, а есть other?
Когда у нас появляется одно, а когда другое?
Какие есть варианты?
Another для единственного числа, а other -- для множественного.
Так, а если я вам скажу, so I have two files, one is transparent and the other is blue.
Тогда ваш вариант не работает и another не для единственного!
В чем разница?
Когда вы будете использовать форму another, а когда the other, если говорим про единственное число?
Когда у нас есть выбор только двух, говорим один и другой (и третьего не дано), то тогда мы используем one and the other.
Это один и другой из двух.
Если же у нас широкий выбор, мы просто говорим, другой вариант может быть тем, то тогда мы используем another, потому что вы помните, если the other, все-таки они существуют, эти два слова отдельно, а мы помним, что the -- это указание на определенность, то another, который уже слился с этим словом other, тем не менее показывает нам, что это один другой из множества.
То есть этот есть другой, есть ещё какой-то другой, another and another and another.

Если у нас такая же ситуация, но во множественном числе, как вот здесь, например, с languages, то тогда мы можем использовать пару либо the other, имея в виду, что есть английский, а есть все остальные, собранные в кучу.
И больше третьего не дано.
English and the other foreign languages.
Если просто мы говорим, что другие иностранные языки, не имея в виду, что мы как бы ограничиваем их (строго там собираем в пакетик и в кулёчек), а просто имеем в виду, что другие иностранные языки, то тогда мы просто "<the"> опускаем.

Итак, another и the other -- это уровень единственного числа, other и the other -- это уровень множественного числа.

Так.
Just an interesting.
Well, not an interesting, but just a new word, maybe not all of you know it and it is in the next paragraph: "<But in the field of science, sticking to English alone makes such quandary">.
Как вы это переводите?
Quandary -- затруднительное положение.
Problematic situation.
Actually quandary is a problematic situation.
Но если вы, опять же, даже на английский язык, это не совсем неразбериха, да, это просто вот...
Неразберихи нет, просто трудности.
Если в русском языке, мы даже с вами переводим это как словосочетание, то вот английский язык предлагает нам замечательное слово, хотя, если мы посмотрим на это слово, то мы скажем однозначно, что это не англосаксонское, это не английское слово, потому что всё то, что с буквы "<Q"> в современном английском начинается -- это слова, заимствованные из латинского языка, из романских языков.
Так что опять английский здесь немножко не в себе, не с собой.

Окей, "<wholesale">.
Do you know what the word "<wholesale"> means?
Well, I'm sure you know.
Да, это опт.
Wholesale -- это оптовое что-то (там оптовая продажа, например).

And two more aspects that I would like you to comment upon.
This is the very end of the text where we have жирный шрифт.
Bold.
Typed (printed) in bold.
We have the form.
I'm sure that you all know it.
You have this form.
Five letters.
\textbf{BORNE.}
What about this form?
What is this form?
Что это за форма?
Пассивная?
Нет пассивного.
Пассивный залог у нас -- это аналитическая форма.
Мы помним, что пассивный залог, мы с вами обсуждали, образуется с помощью глагола to be и вот этого самого, то, что вы сейчас неправильно назвали.
Что это такое?
Как форма называется?
Participle II -- абсолютно верно, или Past Partiсiple.
Итак, это причастие прошедшего времени или второе причастие.
От какого глагола?
\textbf{Bear.}
На самом деле, первое значение этого глагола – это "<нести">.
Между ними здесь есть какая форма?
\textbf{Bore.}

Bear -- Bore -- Borne (или Born).

Что это за глагол?
Неправильный глагол.
Потому что мы помним вот эти две крупные группы глаголов современного английского языка.
Правильные и неправильные.
Начиная сразу же отсюда, мы не увидели показатель правильных глаголов.
Какой для правильных глаголов показатель?
"<-ed.">
Здесь его нет, значит это точно не относится к правильным глаголам.
Но вы, я думаю, все знаете, и, наверное, чаще все-таки встречаетесь вот с этой формой, без "<e"> (формой born, а не borne).
На самом деле эти две формы возможны, то есть для обеих этих форм инфинитивом является to bear -- первое значение это нести, а потом уже, так сказать, рождать и так далее.
Bore -- это что за форма, вспоминаем с вами?
Как она называется?
Это форма называется Past Simple.
Эта форма используется только при образовании Past Simple, потому так и называется.
Borne -- это Participle II -- достаточно активно используемая форма, потому что используется она для образования чего?
При образовании каких аналитических форм современного английского языка мы используем эту форму?
Сама эта форма -- это причастие.
Мы её когда используем?
Для образования чего?
Для образования каких аналитических форм?
Пассива: to be плюс причастие второе.
И ещё?
Для образования каких форм?
Целый столбичек в таблице Tenses and Aspects требует эту форму.
Это что за формы?
Это формы Perfect, абсолютно верно.
И все формы Perfect.
Has borne.
Had borne.
Will have borne.
Везде мы будем использовать форму borne/born, которая является причастным вторым.

Ну и теперь вопрос: задумывались ли вы когда-нибудь о том, что существует два варианта, и причем они четко разделяются по способу использования, вернее, по месту их использования.
Вы все привыкли, наверное, вот к форме born, да?
Ну и вспомните, когда мы её используем.
Как правило, дата рождения или место рождения, да?
I was born in Saint Petersburg или in Leningrad.
I was born...
Дата?
On.
I was born on (дата).
А если в году, то абсолютно верно, будет in.
I was born in (год).
А если мы говорим дату, то мы все помним, что когда мы называем в английском языке дату, то это будет on the first of May.
И это на самом деле всё, когда используется форма born.
Потому что во всех остальных случаях будет использоваться форма borne.
Как правило она используется в перфекте.
Она родила ребёнка.
Родила мальчика.
Будет использоваться форма has borne.
Или же, если говорится, что Anna has borne a baby, a baby boy.
Или a baby boy was borne by Anna.
Если есть предлог by (я специально вас спрашивала про предлоги: in или on), то тогда вот эта форма borne тоже будет использоваться.
Ну и, соответственно, в нашем случае я надеюсь, что вы запомнили для себя, что если есть фиксированное использование места или даты рождения, то используется вот эта форма born без "<-e"> на конце.
В остальных случаях (грубо говоря, вынашивать и в прямом значении, и в фигуративном / переносном смысле) будет использоваться форма borne с этим немым "<e">.

Теперь давайте посмотрим на наш текст.
Что у нас там за фраза и how would you translate it?
So, the disproportionate burden of language training and technology borne by non-English speaking countries.
Borne by non-English speaking countries.
Как вы это переведёте?
Вот здесь borne by.
Видите, здесь by.
Понятно, что это не Анна.
Как вы переведете?
Появившиеся в не англоговорящих странах.
Абсолютно верно, да.
Это очень замечательно, без какого-либо соотношения вот с этим вот значением рождения, да, как бы давания жизни.
Но тем не менее появляться -- это давать начало чему-то.
Потому что что такое нести?
Нести -- to bear -- нести и предъявить потом.

Ну и последний грамматический момент.
Это просто обратить ваше внимание на выражение типа, вернее на слово "<a widening gap"> (between the language of science and the languages of public discourse).
Widening -- что это такое?
Расширающийся.
Да, абсолютно верно, это не широкий, а именно расширяющийся.
Но прежде всего, что это за форма?
Continuous?
НЕТ, ЭТО НЕ CONTINUOUS.
Я надеюсь, что ещё за оставшиеся несколько наших редких встреч мы с вами отучимся называть эти формы так, и будем называть их правильно!
Так widening -- это что?
Это Participle I.
Это форма, образуемая с помощью суффикса "<-ing">.
Это причастие первое, которое показывает нам признак предмета по действию сейчас.
Итак, от какого глагола образовано это причастие?
От глагола to widen, абсолютно верно.
И в английском языке достаточно большое количество, я просто хочу обратить ваше внимание, и это именно исконно англосаксонские глаголы, то есть те глаголы, которые образовались и зародились внутри английского языка, ещё до эпохи нормандского завоевания, образовывались они с помощью вот этого суффикса "<-n">, и, как правило, они образуются от прилагательных, которые обозначают качество, признак.
Это либо признак по цвету.
Как будет побелить?
To whiten.
Вот он покраснел или она покраснела.
Redden.
Да.
Небо потемнело.
Почернело.
Blacken.
Darken или blacken.
Абсолютно верно.

Я думаю, что вот теперь по аналогии с to widen, вы, наверное, вспомните мне ещё достаточно большую группу глаголов и поймете, что вы их знаете.
To widen -- переводится как?
Расширять.
Widen -- это обычно на плоскости.
На плоскости, когда растягивают.
Есть другой глагол или другое тоже прилагательное, обозначающие широкий, но это вот как бы такое либо в переносном значении, либо, как это сказать, в пространственном значении.
Это какое прилагательное?
Тоже переводящееся "<широкий">.
Broad.
Да, broad, абсолютно верно.
Расширять в этом случае -- to broaden.
Ещё какие-то глаголы знаете такие?
To shorten -- сокращать.
А как будет он удлинять?
To elongate?
Это здесь два других варианта, и в данном случае это действительно одно...
To lengthen.
И в данном случае -- это вот меньшее количество глаголов.
Что такое length?
Это длина.
То есть чувствуете, мы в основном с вами до этого образовывали от прилагательных, качественных, но есть некоторые глаголы, которые от существительных образованы.
Итак, to lengthen.
А еще?
Знаете еще один?
To strengthen.
Абсолютно верно.
Усиливать.
Я думаю, что вы этот глагол встречается достаточно часто в своих научных статьях.
Что там ещё?
To deepen.
Я просто тоже вспоминала, какие еще там можно образовать...

Ну, их не огромное количество, то есть уже просто в языке не все эти качественные прилагательные образуют глаголы таким образом.
Но, тем не менее, это вполне продуктивный способ образования глаголов.
И если вы, грубо говоря, образуете от прилагательного, обычно это односложное всё-таки прилагательное, потому что односложные прилагательные -- это исконно англосаксонские прилагательные.
Те, которые многосложные, типа beautiful, important -- это прилагательные не англосаксонские, заимствованные из других языков, и они таким образом глаголы не образуют, потому что английский язык не любит очень сильно многосложных слов.
Если слова многосложные, то это однозначно можно искать похожие слова и в русском языке.
Это значит, что они могут быть транслитерированы.
И это нам помогает при переводе.
Если же слова односложные, то, как правило, это англосаксонские слова.
И поэтому вполне себе от прилагательных, которые качество образуют, можно попробовать и окказионализм образовать с помощью суффикса "<-en">, и в общем-то носитель языка вас поймёт.
Я просто пыталась там образовать to brownen, но просто не встречала такого слова, но я думаю, что если мы таким образом глагол образуем, то будет понятно, что это значит сильно закрасить коричневой краской.
Так, хорошо, двигаемся дальше.
Время у нас: достаточно хорошо мы с вами идём.
So that's not by chance that I asked you to discuss some of the aspects of how to translate different words, because we have to paragraphs in brackets (quotes) and I would like you to very quickly use your written translation skills.
Would you please translate these two italicized paragraphs into Russian in a written form?
So take your time.
And then we will compare what we have.
Там есть ещё интересные моменты, которые мне хочется, чтобы мы потом сравнили, проверьте себя.
So the text is not difficult.
Take your time.
\\

--- Выполнение задания ---
\\

So, two minutes and we start discussing.
Okay, so let's compare what you have worked out.
Well, I hope that you agree that this text is not really difficult from the point of view of its vocabulary, because it's quite easy to translate, though there are several constructions that I would like us to actually compare your variants of translating into Russian.
Because they look simple, and you definitely understand what they mean, but what would be the Russian equivalent that you would present, for example, today or during your examination.
Well, and it is actually, it's like a sort of a paradox, but the simplest construction in the English language from the point of view of its construction, sorry for much of muchness, which you all have got used to, which is "<there is"> and "<there are">.
And in this case, "<there has been">.
But it actually sometimes, and I would say quite often, presents some difficulties, especially in the scientific texts.

So, the first sentence.
Let's begin with the variants, and then we will...
Just I would be glad if you give your variants or add some other aspects of translation.
So who would like to start with the translation of the first (quite a long) sentence?
Three lines long.
But still...
Who would like to begin?
Yes, please.

Последние 80 лет происходит устойчивый рост доли научных публикаций, написанных на английском языке, и соответствующее снижение использования всех других языков в глобальном научном общении.
Замечательно, мы посмотрим, есть ли какие-то другие варианты.
Так ещё раз ваш вариант вот для этого "<there has been a steady increase">.
Хороший вариант.
Происходит устойчивый рост.
Замечательно!
Какие ещё варианты для вот этого максимально простого предложения "<there has been a steady increase">.
Может наблюдается?
Наблюдается, замечательно.
Просто для того, чтобы вы себе это в копилку поместили, потому что в статьях достаточно часто эта конструкция встречается, потому что она распространена, она лишена каких-то дополнительных коннотаций (т.е. дополнительных значений), но при этом перевести ее на русский язык дословно нельзя.
Мы должны продумать эти варианты.
Итак, "<наблюдается">.
Еще какие-то у кого-то варианты были интересные, которые вы хотите озвучить, предложить?
Нет пока?
Так, то, что касается остального перевода, есть какие-то еще варианты или коррективы, может быть?
Пожалуйста, ваши, я надеюсь, что все попробовали, да?
Потому что вы все зрите уже в недалекое будущее, всем придётся письменно работать на экзамене, поэтому я и прошу вас пока на простом варианте.
Есть какие-то еще варианты?
Yes, no?
Ну, во второй части.
Да, пожалуйста.
И соответствующее уменьшение -- я так перевёл.
Corresponding decrease.
Соответствующее уменьшение.
Да, хорошо.
Ещё какие-то варианты?
Нет.

Так, двигаемся дальше.
Тоже достаточно длинное предложение.
Это вот к вопросу о том, что это однозначно не газетная статья, потому что для газетных статей не характерны длинные предложения.
Это просто требования языка медиа.
Наоборот, говори кратко, уходи быстро.
А в данном случае это научно-популярная всё-таки статья.
Научно!
Здесь ключевое слово popular SCIENCE, поэтому позволительно вот такие вот длинные с дополнительными какими-то элементами и конструкциями вставленными в предложения.
Пожалуйста, для этого предложения, которое тоже практически на три строки, и вот эти два предложения образуют у нас целый абзац.
Кто хочет предложить вариант?
Второе предложение, потому что там тоже есть интересная конструкция, и я думаю, что вы тоже обратили внимание на неё и задумались, наверное, как её перевести "<this appears to offer significant gains">.
Не "<this offers">, да?
Мы с вами помним, что английский язык за счёт вставки вот этих "<it seems">, "<it looks like">, "<it appears to be">, "<it is thought to be"> нас вынуждает задуматься о том, как перевести это предложение на русский.
Пожалуйста.
Какие варианты?

На первый взгляд, это как-будто предоставляет существенные выгоды в отношении эффективности и универсальности, так как учёным повсеместно необходимо учить только один язык, помимо родного.
Хорошо, только я бы конечно от этого "<как-будто"> в научно-популярном стиле избавилась бы.
Вы так перевели как раз вот это "<appears to be">, да?
Да.
Давайте попробуем поднять это всё-таки вот в стиль научно-популярный, научный.
"<Казалось бы">?
Да, "<оказывается"> может быть.
Чтобы снять условность, да, вот это "<бы"> (частицу "<бы">) мы убираем, потому что appears -- это оказывается, что...
"<Видимо">?
Да, "<видимо">, замечательно, да.
Если мы глагол не будет в данном случае, да, то, "<видимо">, да, можно притянуть, да, то есть это не совсем разговорный язык.
"<Видится">?
Ну, "<at first sight"> здесь ещё есть.
Как тогда вы переведёте целиком это предложение?
На первый взгляд, это видится...
Ну да, да, я понял, понял, да так, не совсем, да, и взгляд видится, то есть у нас всё тут тоже связано.
Так, какие еще есть варианты?
Предлагайте свои варианты, вы же взрослые мальчики.
И девочки.
"<Представляется">?
"<Представляется">, да.
Я, несмотря на то, что это научно-популярная статья, "<кажется"> бы здесь оставил, потому что оно здесь подходит.
"<Кажется"> -- это нейтральный глагол.
Да, да.
Да, "<кажется"> -- нормальный вариант тоже.
Я просто послушать вас, да, то есть убрать всё-таки такой разговорный стиль, но при этом здесь глагол "<оказывается"> или "<кажется"> сюда тоже подходит.

Так, хорошо, следующее опять нам даёт эту вот вполне распространённую и широко используемую конструкцию, которая каким образом у вас в русском языке нашла отражение.
There are, however, drawbacks to this system...
Вам предоставляют слово, пожалуйста.
Однако существуют недостатки этой системы, которые могут значительно повлиять как в принципе на научное развитие, так и на его общественное влияние.
Так, хорошо.
Итак, давайте ещё раз.
Drawbacks to this system.
Как там переведём?
Недоставки системы.
Хорошо.
Дальше у меня вопрос.
May be significantly affecting.
В данном случае у вас в переводе было "<могут существенно повлиять">.
Здесь это не "<повлиять">.
Просто вот, например, на экзамене на это обратят внимание, просто ваше внимание на это обратят.
Что это у нас здесь используется?
Be affecting -- это у нас Continuous Infinitive, показывающий длительность, поэтому не "<повлиять">, а "<влиять"> -- "<которые могут существенно влиять">.
Вы должны выдерживать вот эту вот...
Автор мог бы написать may significantly affect, но он же написал be affecting, подчеркивая именно длительность вот этого процесса.
Так, абсолютно верно было переведено, и мы обратили с вами внимание на это "<как на ..., так и на ...">, и вот это "<both ... and ..."> -- это именно для письменной речи характерно использование вот этого сложного союза.
В разговорной речи, ну, только если вы очень хорошо его знаете и любите, вы будете его использовать, но в основном, конечно, and, and, перечисление за счёт and идет в разговорной речи.
Это для научного письменного языка характерно.

Так, следующее, теперь уже такой просто серьезный, такой терминологический набор, но тем не менее, пожалуйста.
Да, пожалуйста.

Можно вопрос?
Если использовать не "<недостатки">, а "<подводные камни"> -- это будет как отхождение от научного стиля или нет?
Нет, потому что мне кажется, что это метафора такая, можно назвать ее мёртвой метафорой, да, потому что, потому что уже вошедшая в обиход, но "<подводные камни">, в общем-то, мы, конечно, и в разговорной речи в русском языке используем, но мы считаем, что нет, вполне подойдет и для этого перевода тоже, замечаний здесь нет, если вы такую, как я сказала death metaphor используете, да?
Можно!
То есть это некие препятствия, да, в этом случае, потому что подводные камни -- это все-таки препятствия, это не недостатки.
Вот как мнение других?
Недостатки и препятствия -- это разные вещи, правильно?
Ну это красиво звучит.
Ну, немножко, хоть и другое, но поскольку drawbacks, понятно, что чаще всего мы как недостатки переводим, но всё-таки это некие такие вот проблемные моменты.
А проблемный момент -- это дальше "<подводные камни">.
Поэтому спасибо за предложение.
У нас же нет с вами русского варианта, поэтому сравнивать нам не с чем.
Мы просто с вами обсуждаем, у кого какой вариант есть.

Ну и теперь, пожалуйста, может быть, конечно, поделимся здесь, вы между собой поделитесь, я имею в виду частями перевода.
So, who would like to begin translating the next sentence?
Опять же, да, вроде предложение простое.
Issues of concern include, etc., etc.
Но по-русски мы все равно немножко трансформируем это предложение.
Дословный перевод в русском языке не пойдёт, будет очень резать и слух, и глаза.
Пожалуйста, вариант.
Начинаем.
Проблемы, которые стоит принять во внимание...
Да, хорошо.
Но только не...
Я бы начала: к проблемам, которые стоит принять во внимание, относятся...
То есть всё равно...
Можно сказать "<включают">?
Нет, проблемы, которые включают...
Нет, в смысле проблемы, которые стоит принять во внимание, включают в себя бла-бла-бла.
Проблемы включают в себя не очень хорошо звучит.
Даже если вы по-русски напишите, и даже не перевод.
Это всё равно такой стилистический, немножко такой нарушенный стиль.
Поэтому здесь лучше пойти неопределенно личным предложением в русском языке -- к проблемам относятся следующие...
Так, ну и давайте.

The poor correlation between ability in a second language and scientific ability.
Ну тут уже терминология специфического, но не сильно узкопрофессионального плана.
Пожалуйста, как переведете?
Все понятно, просто красиво назвать.
Итак, the poor correlation between abilities...
Слабая корреляция...
Да.
Мы помним, что poor в данном случае это слабая, а не бедная.
Хотя понятно, что здесь метафорически.
Но тем не менее слабая корреляция...
Между навыком владения вторым языком и навыками в профессиональной (научной) области.
А, ну что значит навыки?
Компетентность.
Научное достижение.
Может быть, и научной компетентностью.
Способности?
Но scientific ability.
Ability -- это способность, действительно.
Вот если в первом случае способность...
Просто не скажешь способность к науке, поэтому научные достижения.
Да, ну вот поэтому я и спрашиваю.
Научное достижение, у кого какие-то варианты еще были, которые вы записали?
Ну можно сказать научная компетентность, компетентность в сфере научной деятельности.
Можно так описать вот эту часть, потому что компетенции наши, вернее, компетентность, как правило, в русском языке описывается через способен что-то делать, способность что-то делать.
Поэтому в данном случае слова способность или способен здесь подойдут.
Здесь навыки владения вторым (неродным) языком и компетентность в научной области (в профессиональной области).

Так, the disproportionate burden of language training and technology и так далее, мы уже переводили.
Пожалуйста.
Какие есть варианты?
Disproportionate Burden?
Ваши предложения.
Неравноценный, несоразмерный, ...
А burden?
Смысл-то в чем здесь имеется?
Что слишком много сил тратится на одно и на второе.
"<Несоответствие усилий">?
Может быть, да, вот несоответствие усилий, затрачиваемых на изучение языка и технологий?
Я бы "<диспропорция усилий"> оставил бы на самом деле. 
Можно, да, диспропорция усилий, да.
То есть мы здесь меняем с вами прилагательное на существительное, но тем не менее тоже хороший вариант.
Хорошо.
Нам главное постараться понять, что имеется в виду, да?
И найти русский вариант, эквивалент.

Так, ну и последнее достаточно большое тоже вот это перечисление.
Пожалуйста.
Увеличивающийся разрыв между языком науки и языком публичных дискуссий, государственной политики и общественной жизни.
Да, хорошо.
Хорошо.

Так ну и вопрос первый: What do you think about this?
We have already discussed this question.
Мы начали с этого.

И второй вопрос: Do you possibly assume that the rising of English language in the field of Science slowly eradicates the role of other foreign languages?
Just to finish discussing this text.
So what is your opinion.
Eradicate -- снижает, размазывает, разрушает, уменьшает.
Коррозия -- разрушает структуру и жизнь других иностранных языков.
So what do you think about it?
Do you actually agree with this statement or you think that the situation is not really like this?
Yes, no?
Yes!
No!
Well, of course, that would be nice if you give some arguments to prove your ideas.

Okay, so let it be one of the questions that you would then continue and maybe somehow cover and discuss in your topic, which is not for the next meeting of us, but still that can be the discussion.

Well, actually, what I wanted us to continue with, but I will probably share it in between our class and your home class.
It is summarizing.
So I have prepared an article which is in the same field of research, I would say in the same topic field, which is of interest for us and of our discussion for this class.
And it is in the course, but it is closed at the moment, so I will then open it for you.
So what I would like you to do...
Давайте, у нас есть же ещё 15 минут?
Выдержите ещё 15 минут?
Ладно, давайте перейдём к домашнему заданию.

Дело в том, что я 15 мая уезжаю в командировку, меня не будет, меня отправляют на три дня, мы с делегацией едем на конференцию.
Поэтому у меня к вам вопрос, а дальше мы уже будем решать, как мы с вами сейчас поступим.
У меня есть предложение: либо мы перенесем наши пары на 14-е на вторник, ну просто возьмем и перенесем, либо на 29.
То же самое: вот как у нас стоит одни в 18:00, другие в 19:40, но во вторник 14 мая.
Просто вас здесь...
Вы почти все здесь, поэтому я могу с вами обсудить...
Две пары сразу или как?
Нет, 17 апреля (в прошлую среду) занятия не было, поэтому эту пару мы полностью с вами проводим 22-го мая.
То есть 22-го у нас с вами встреча.
Но я просто думала, что всё нормально у нас будет 15-ого мая, но я уезжаю на конференцию.

Когда у вас пары по ОДИ?
У нас пары по организации диссертационного исследования 15 мая и 29 мая.

А всё, тогда 29-ое мая снимается.
Тогда давайте мы просто переносим пару с 15 мая, я попрошу расписание изменить, переносим на 14-ое мая.
Соответственно на 18:00 и на 19:40.
Кому как удобно, кто как приходит.
Итак, перенесли на 14 мая в 18:00 и 19:40.
Нормально?
Всё, тогда я попрошу изменить расписание.

Пары будут ещё 14-ого и 22-ого, да?
Да, 14-ого и 22-ого мая.
Так, тогда давайте я дам задание, а то чувствуется, что тяжело вам немножко уже.
Значит, объясняю.
У меня в курсе (я открою сегодня вечером) подгружен article, который...

Он называется "<Science Communication in Multiple Languages Is Critical to Its Effectiveness">, то есть опять как бы в сторону Foreign Languages.
Уж извините немножко, да?
Потому что после этого, я надеюсь, вы сможете написать такую классную тему мне The Role of Foreign Languages.
Итак, ваша задача будет...
Я в этой же теме создам задание.
И к следующему, вот к 14-му числу, правильно, у вас будет много времени, это же даже не неделя, да, это три недели, ваша задача будет подгрузить туда два файла.
Я сняла то, что вы сделали по переводу и подготовила презентацию.
Просто сегодня я решила не делать, потому что вас много.
Мы сделаем сравнительный анализ и уже с вами обсудим на наших более узких встречах.
Точно так же там я сделаю задание, куда вам нужно будет подгрузить, одним файлом можете сделать, но тем не менее, этот article я специально подобрала, там нет abstract.
Значит, ваша задача будет написать abstract к этой статье.

И там же в этой же теме я открою, там мы уже начинали с вами обсуждать, но тем не менее, там два файла.
Один называется...
И я просто попрошу вас его изучить и просто использовать.
How to summarize a text.
Simplified explanation.
То есть это, ну, это описательный, конечно, текст, но тем не менее здесь собраны такие тоже советы о том, как писать, как готовить summary.
То есть то, к чему мы с вами должны подойти и стремиться.

И второй файл, который туда тоже я уже подгрузила, это list of useful phrases.
Просто показываю, двухстраничный, просто вам в качестве подсказки.
List of useful phrases for making summaries and preparing reports.
То есть здесь достаточно много собрано всяких фраз, про которые я вам говорила, что вы, по сути дела, готовя свои summary, можете эти вот различные фразы использовать, но как правило я аспирантам советую выбрать для себя те фразы, которые вам больше всего нравятся, и во все summary по любой статье эти фразочки использовать.
То есть это своего рода такой план построить.
Первое предложение я начинаю с этой фразы всегда, второе с этой и так далее.
И вот соответственно такой первый summary научной статьи, но по теме multilingualism, требуется составить.

Итак, ваша задача будет написать abstract, как если бы вы были автором, вы помните, мы обсуждали разницу, да, между abstract и summary.
Абстракт пишет автор статьи, summary пишет читатель или критик (peer review).
Abstract -- это если вы автор статьи пишете мне.
И тоже небольшой summary (не больше 7-8 предложений) summary этой статьи и подгружаете мне одним файлом.
Первая часть, вторая часть.
На одном листочке делайте и туда подгружайте.
Я посмотрю соответственно, что у вас получилось и тоже мы с вами это обсудим и дальше ещё попробуем на другом материале это делать.

Можно вопрос?
Да.
Вы до этого сказали, что ещё открывали какое-то задание по переводу?
Да, было задание, которое на предыдущее занятие было.
Было задание.
Оно там уже есть в курсе, где один кусочек на английском, один на русском.
Там содержатся разные причастные обороты, и задача была -- ваш вариант с определенными сложностями перевести на русский, а дальше попробовать перевести с русского на английский, тоже используя вот эти причастные обороты.
Да, туда подгружайте, и я посмотрю...
Потому что я же вижу, что у меня не проверены.
Если я что-то проверяла, то я ставила там.

Итак, значит, в это задание...
Вы читаете эту статью, она тоже не сложная с точки зрения терминологии, тему я вам прочитала, и за эти три недели готовите abstract и пробное summary.
Посмотрим, что у вас получится из summary, исходя из того что я туда подгрузила, и дальше продолжим обсуждать summary.

Всё, до 14 мая.

\newpage
\sublinksection{A summary and an abstract. Домашнее задание}

\hypertarget{ltask:2024-04-24-2}{--- Выполнение задания ---} (\hyperref[task:2024-04-24-2]{\color{blue}{перейти к тексту задания}})
\\

Требуется составить abstract и summary для данной статьи.
Задание для самостоятельного выполнения.
Результат требуется прикрепить на сайте портала аспирантуры \href{https://portasp.spbstu.ru/login/index.php}{PORTASP SPBSTU}.


\newpage
\sublinksection{Tips on Summarizing}
\label{subsec:tips-on-summarizing}

{\parindent25pt\includegraphics[width=0.88\textwidth, page=1,trim={1in 0.85in 0.5in 0.77in},clip=true]{TipsOnSummarizing.pdf}}\newpage

{\parindent25pt\includegraphics[width=0.88\textwidth, page=2,trim={1in 0.85in 0.5in 0.77in},clip=true]{TipsOnSummarizing.pdf}}\newpage

{\parindent25pt\includegraphics[width=0.88\textwidth, page=3,trim={1in 0.85in 0.5in 0.77in},clip=true]{TipsOnSummarizing.pdf}}\newpage

{\parindent25pt\includegraphics[width=0.88\textwidth, page=4,trim={1in 0.85in 0.5in 0.77in},clip=true]{TipsOnSummarizing.pdf}}\newpage

{\parindent25pt\includegraphics[width=0.88\textwidth, page=5,trim={1in 0.85in 0.5in 0.77in},clip=true]{TipsOnSummarizing.pdf}}


\newpage
\sublinksection{List of Useful Phrases for Making Summaries}
\label{subsec:list-of-useful-phrases-for-making-summaries}

{\parindent25pt\includegraphics[width=0.87\textwidth, page=1,trim={1in 0.9in 0.5in 0.7in},clip=true]{ListOfUsefulPhrasesForMakingSummaries.pdf}}\newpage

{\parindent25pt\includegraphics[width=0.87\textwidth, page=2,trim={1in 0.9in 0.5in 0.7in},clip=true]{ListOfUsefulPhrasesForMakingSummaries.pdf}}



\end{document}
