\documentclass[main.tex]{subfiles}

\begin{document}

\linksection{Лекция 03.04.2024 (Смольская Н.Б.)}

\sublinksection{Наука в исторической перспективе и современное состояние науки}

\newpage
\sublinksection{Задание на перефразирование}

\hypertarget{ltask:2024-04-03-1}{--- Выполнение задания ---} (\hyperref[task:2024-04-03-1]{\color{blue}{перейти к тексту задания}})
\\

\textbfind{Выполненное задание (подчёркнуто самое близкое по смыслу с исходным предложение)}
\vspace{5pt}
\begin{enumerate}[nosep, leftmargin=*]
	\itemsep15pt
	\item If the new project is to be adopted, it must be approved by the Board.\newline
		A) No doubt the, project has been approved by the Board, so it we'll be adopted.\newline
		B) If the project is adopted it must be approved by the Board.\newline
		\uline{C) For the project to be adopted it must be approved by the Board.}
	\item In order to protect their position on the market, the company should have developed new technology.\newline
		A) The company has developed an upgraded technology and protected its position on the market.\newline
		B) In order to protect their position on the market the company is developing a new technology.\newline
		\uline{C) Using an obsolete technology, the company was not able to keep up with its competitors.}
	\item New bonds are likely to be issued before long, the finance director says.\newline
		A) New bonds are similar to those issued long time ago.\newline
		\uline{B) New bonds are expected to appear very soon.}\newline
		C) The finance director likes the idea of issuing new bonds.
	\item The Marketing Director wanted the Sales Agent to be instructed by the Secretary as to filling out invoice forms.\newline
		A) The Secretary did not know how to fill out invoice forms.\newline
		B) The Marketing Director wanted to tell the Sales Agent how to fill out invoice forms.\newline
		\uline{C) The Secretary knew how to fill out invoice forms, but the Sales Agent did not.}
	\item Nissan is to announce on Monday a further modest rise in jobs at its plant in the north east England city.\newline
		A) The aim of Nissan is to announce on Monday further modest increase in jobs at its plant.\newline
		\uline{B) Nissan must announce on Monday a further modest rise in jobs at its plant.}\newline
		C) Nissan announced on Monday a further modest rise in jobs at its plant.
	\item The method is not economical enough to be applied.\newline
		A) The method is not economical but it can be applied.\newline
		\uline{B) The method is not economical and it can't be used in practice.}\newline
		C) The method is not economical and it was not applied.
	\item Commonwealth Bank of Australia yesterday failed to announce a jump in profit for the year.\newline
		A) Commonwealth Bank of Australia was forced to announce a jump in profits.\newline
		B) Commonwealth Bank of Australia managed to announce a jump in profits.\newline
		\uline{C) Commonwealth Bank of Australia was not able to increase its profits.}
	\item The auditors are sure to have found fraud in the company's transactions.\newline
		\uline{A) No doubt the auditors have found fraud.}\newline
		B) We are sure the fraud will be found.\newline
		C) The auditors are sure they will find the fraud.
	\item We had all the accounts checked through, but the error was not spotted.\newline
		A) We checked all the accounts in order to find the error, but we did not succeed.\newline
		\uline{B) We asked the bank clerk to check the accounts, and so he did, but the error was not found.}\newline
		C) We were forced to check the accounts because the error was not spotted.
	\item On settling the price problem the partners discussed the terms of delivery.\newline
		A) The terms of delivery were discussed prior to the price problem.\newline
		\uline{B) The partners discussed the terms of delivery after the price problem.}\newline
		C) The problems of price and delivery were discussed and settled at once.
\end{enumerate}
\ 

\newpage
\sublinksection{Задание по грамматике. Неличные формы глагола}

\hypertarget{ltask:2024-04-03-2}{--- Выполнение задания ---} (\hyperref[task:2024-04-03-2]{\color{blue}{перейти к тексту задания}})
\\

\textbfind{Выполненное задание (глагол в скобках поставлен в требуемую по контексту форму)}
\vspace{5pt}
\begin{enumerate}[nosep, leftmargin=*]
	\itemsep\eitsp
	\item Why do you object to \uline{\textbf{following}} the directions? (follow)
	\item I hear someone \uline{\textbf{knocking}} at the door. (knock)
	\item ``Why don't you like Jack?''\newline ``I dislike \uline{\textbf{hearing}} him tell old jokes.'' (hear)
	\item The Ford Theatre where Lincoln was shot \uline{\textbf{must be restored}}. (must/restore)
	\item The examiner made us \uline{\textbf{show}} our identification in order to be admitted to the test centre. (show)
	\item I want to finish \uline{\textbf{writing}} my report tonight. (write)
	\item Experiments show that certain foods can have their useful life \uline{\textbf{extended}} by treating them with doses of radiation. (extend)
	\item Silicon doped with phosphorous or another pentavalent element is called an n-type semiconductor, the n-type \uline{\textbf{representing}} the negative electric charge of the conduction electrons. (represent)
	\item Pierre and Marie Curie showed that beta rays consist of a stream of electrons \uline{\textbf{moving}} at terrific speed. (move)
	\item A competitive product and a receptive market are not enough: the product must somehow \uline{\textbf{be transported}} from the factory to the export market. (transport)
\end{enumerate}


\end{document}