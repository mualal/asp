\documentclass[main.tex]{subfiles}

\begin{document}

\linksection{Лекция 03.04.2024 (Смольская Н.Б.)}

So, well, the period of your examination is slowly but still approaching.
And in any case I would like us to continue practising your monologues and practising answering the questions.
But, of course, well, I do ask questions when you speak on the topics, but I would like you also to practise asking the questions.
So that is why we have already practised this way of communicating.
So it means that I would like some of you today to say a few words on the topic of the history and present state (current state) of your field of studies.
And after that, I do insist on your asking the questions to your colleagues, just practising.
Because, well, I would say I'm totally sure that you do not have any other, almost, no practice of English, only in case you have some discussions (maybe online discussions) with your international colleagues, but not so many.

So that is why practise your English.
Well.
At least for, as it is nowadays called edutainment.
Есть entertainment, а вот есть edutainment.
Это как раз подразумевается процесс education and entertainment mixed together, not mixed together, but joined together.
So like education through entertainment.
So that is why just entertain yourselves and practise English and also practise grammar and vocabulary.
So who would like to say a few words?
Because that's also the case for us to discuss this problem and maybe to discuss what can be said if you are asked to speak on this topic during your examination.
So, what points to highlight, what actually to say, just for you not to feel lost if you are asked to speak on this topic during your examination.
So who would like to say a few words (just to practise speaking)?

I am almost sure that Timur is ready.
That's up to you.
Just, I think, two or three of you.
Давайте.
На экзамене всё равно придётся отвечать за всё.
До экзамена осталось уже меньше полугода.
А когда точно, ещё неизвестно?
Ну точно не скажу, конечно, нет.
В сентябре.
Минимум на второй неделе сентября, максимум на последней неделе сентября.
На первой неделе (первые 10 дней) у нас вступительные.
Принимают те же самые, что и кандидатский принимают.
Поэтому точно не в первой половине сентября.
В прошлом году было, по-моему, с 15 сентября.
Но там у нас, поскольку было наслоение, второй и первый курс сдавали, поскольку по разным планам учились.
Я думаю, что у вас, наверное, будет где-то начиная с 18 сентября неделю.
Просто будет расписано по научным специальностям.
Наверное, как-то так.
То есть вторая половина сентября.

А может быть в теории попозже, если уехать надо будет?
Не знаю.
Смотря куда.
Если приедете ещё и загоревшие, или загорелые, как там, разница в русском языке.
Уставший оптимист или усталый оптимист, да?
Вот сразу думаешь.
Не смотрите этого блогера?
Уставший оптимист, такой очень прикольный питерский.
Ну, он по поводу current state and current situation очень интересно и прикольно рассказывает.
Вот он уставший оптимист, и я всё время задумываюсь, почему уставший, а не усталый?
В русском языке, да?
Загорелый и загоревший.
Результат действия или уже как результат действия?
Вот если подумать, то у нас два способа образования.
Это русский язык.
Ну что, давайте, вы начнёте.
Можно с места отвечать?
That's up to you.
You can actually stay here and speak.
So, then, not the challenge, but the task is to ask questions.
Something just to get some detail or some continuation of some ideas.
So, you're welcome.

%\newpage
\sublinksection{Наука в исторической перспективе и современное состояние науки}

Okay, as far as I remember, the current topic is the current state of science in my field of knowledge.
So, when I think about this current state, the first thing that comes to mind is that there are two things common not only for my field, the fluid mechanics, but the mechanics in general.
First of all, in the last 10 years if we look in such retrospective way, the significant increase in computational advancements enabled to use many methods for calculations that were known just theoretical but have not been used due to some limitations by weak computers or computers being not so widespread.
That is the first thing that comes to mind, as I told you before.
The second one is, well, as you can guess, today I attended the Science Week of PhysMech, as you can see from this stuff, from this banner.
Yes, and I noticed, and I noticed it before, but today I just became stronger in my thoughts, in my opinion, that not only the current state of mechanics, including fluid mechanics, but the future for the next several years will be determined by such phenomenon, I can't say in other words, as machine learning.
Because machine learning and neural networks enable to increase computational speed and their applications can be really various and this will determine the phase of mechanics, especially mechanics and of course all other adjustments and not only just fields of knowledge in the next years.
And perhaps even make a real breakthrough.

Okay, thank you very much.
So questions or opinions and asking to make something clear.
So, imagine that you are the examining person and you definitely, you should ask the questions.
So, make yourselves ask questions.

How, in your opinion, neural networks can increase the speed of computations?
Well, today I was talking to one of the participants of the Science Week.
And unfortunately, I didn't have so much time to talk to him properly, because I had to stand by my own banner.
But he presented a new numerical method, as far as I remember, enhanced by the use of machine learning.
And it enabled to solve tasks with much higher speed.
Just in general.
Unfortunately I can't explain any details but if you want I can find the topic so we can read it in the thesis book I don't know how it's called, when the short descriptions of...
The collection of extended...
Yes, yes, the collection of extended articles so you can read and learn what I meant.
Okay, thank you.

Okay, some more questions.
You told about neural methods, neural networks.
So this is, it can be used not for only increasing the speed of calculating, but for...
Я хотел как раз добавить, но забыл, как будет "<обрабатывать"> по-английски?
To process big data.
Yes, of course, you are right.
I think that this technology is increasing just for that by the way.
Not only for increasing speed.
Just optional.
Also, I can add that another viable field of application is complex optimization for almost everything.
I mean structural optimization processes.
Neural networks can also be used there.
Yes, of course.

Okay, so questions.
Maybe I'm going to ask a close question, but how much time you need to learn your neural model?
And maybe this time will be more than calculating something?
Yes, yes, currently it's one of the main obstacles that lies before people trying to implement it but as I told you before the rapid increase in computational capacities will also help to overcome this problems.
Besides that in some cases, for example, you made your model (neural network) and after the process of teaching you can use it multiple times for solving some adjacent problems so you teach it only once to use many times.
That's what I meant.
But you need to spend much time to teach your model.
Yes, you're right.

Okay, anything else?
Well, actually, for me as a philologist, because, well, I'm not totally humanitarian oriented, but still, machine learning.
So that's the question, that's not because of you, it's just my inquiry, so would you please explain to me this very phenomenon of machine learning, why it is machine learning?
Why learning?
That's not the question as of the examining person, just the question of a person who wants to understand.
Why learning?
In Russian, we use машинное обучение.
In Russian, this is like a sort of a double-oriented or two-faceted word, обучение.
So you are обучаетесь, и вас обучают.
So this Russian word.
In the English language, the word learning is also very specific what is meant under the term of machine learning?
So how would you explain it to me just to me for me to finally understand what is meant under this?
That is the question.
If you want to participate or to add something you're welcome just for me to finally understand.
In general terms you have let's call it a program and you feed it with a a lot of data in general words.
So that this program finds correlation between these several pieces of data and by this learns to solve some similar programs.
For example, you feed it a picture of an apple, different pictures of apples, and say, well, in its own language, this is an apple.
And finally, after certain amount of iterations, when you show it another picture of apple (not the one that you show previous but a new one) it says "<it's an apple">.
That's what usually meant by the word "<learning"> because you give it some data.
So it's artificial intelligence?
No, it's just learning.
Why learning?
I'm still, I would say, crazy about using this word.

But that's machine that learns.
Yes.
So, it has nothing actually to help specialists in different fields of science to make their life easier.
It is used instead so it helps specialists just to do nothing, well, it gives them the opportunity to do nothing because of solving the problem that the specialists can't solve or what?
Not nothing.
You have some base functions of your brain so when you can to compare then you can choose the right one or the wrong one.
So as for human we need much more time than machine.
So we can delegate some functions, some problems, some maybe...
So make our life easier.
Yes, of course.
Just because for me as a philologist, it is just because it is necessary for us to understand the terminology, why this very term is used, just to understand the semantic relations between the words that are used to nominate different aspects.
And for me the word "<learning"> is always somehow troublesome.
So that is why it's not the process.
I don't understand what it means, but I'm just always thinking and struggling with understanding why "<learning">.
Yeah, I'm told about making mistakes, so I think this is the main reason why it's called learning.
To learn how to play maybe, that's why.
How can we name it without learning?
I don't know.
I don't know.
So it's just the case to think.
Because why not teaching, machine teaching?
Training.
So why learning?
Because "<to learn"> is both very wide and very narrow.
So this very term (ML) makes me feel somehow uncomfortable in understanding.
Для меня, как для филолога, вот что-то вот до конца я не могу понять.

Training doesn't mean any mistakes.
It is I think so repetitive.
It is just when you are doing same things every day or every week.
But learning means to make something new with mistakes but then after many attempts make it better (with fewer mistakes).

Okay, thank you very much to those who were active enough.
I think training is a word that we can use in this case.
Yes, instead of learning.
I don't know how learning of neural networks in science works, but I know a few about the process of learning of neural networks that makes pictures.
This is quite interesting and not obvious process.
Okay.
Thank you very much.
So who else would like to practise speaking and then answering the questions?
That's not difficult at all; that is like a sort of edutainment.
Пробуйте, чтобы не только на экзамене отвечать.
Всё равно всем придётся.
So about your field of studies.
So what is new?
What would you like to add?
What do you know?
What are you not sure about?
What is the basis of your field of knowledge?
For example, the theory of Isaac Newton was the starting point of developing...
Either names or theories or scientific schools.
Давайте, пробуйте.
Уже экватор первого курса магистратуры пережили.
Заставлять не могу, но могу сказать shame on you!
Тогда домашнее задание будет в следующий раз, тогда письменно.
Буду проверять досконально каждого.

How vast is scientific area in this topic?
That's up to you.
You can start in general with physics and then come to something on mechanics and then come to the current state and mention your field of studies or you can directly start with your field of knowledge saying that was the nineteen century when scientists started discussing or developing...
That's up to you.
I do continue repeating that for us it is important how you speak and not what you speak.


\newpage
\sublinksection{Задание на перефразирование предложений с неличными формами глагола}

Okay, so then you remember that I asked you to look through the theory on infinitive, gerunds and participles (неличные формы глагола) of the English language in order to understand actually this specific feature in general of the English language, which is quite different from what we have in the Russian language.
And in this case, we really need to step aside from our knowledge in the Russian language and to clearly understand it as a sort of absolute data that is given to us.
So I would like us to start refreshing your knowledge of this theory and practising your knowledge by using the sentences and tasks.

So first of all I would like you to begin with the easy task.
There are three different tasks.
Start with the first one.
\\

\hypertarget{ltask:2024-04-03-1}{--- Выполнение задания ---} (\hyperref[task:2024-04-03-1]{\color{blue}{перейти к тексту задания}})
\\

\textbfind{Выполненное задание (подчёркнуто самое близкое по смыслу с исходным предложение)}
\vspace{5pt}
\begin{enumerate}[nosep, leftmargin=*]
	\itemsep15pt
	\item If the new project is to be adopted, it must be approved by the Board.\newline
		A) No doubt the, project has been approved by the Board, so it we'll be adopted.\newline
		B) If the project is adopted it must be approved by the Board.\newline
		\uline{C) For the project to be adopted it must be approved by the Board.}
	\item In order to protect their position on the market, the company should have developed new technology.\newline
		A) The company has developed an upgraded technology and protected its position on the market.\newline
		B) In order to protect their position on the market the company is developing a new technology.\newline
		\uline{C) Using an obsolete technology, the company was not able to keep up with its competitors.}
	\item New bonds are likely to be issued before long, the finance director says.\newline
		A) New bonds are similar to those issued long time ago.\newline
		\uline{B) New bonds are expected to appear very soon.}\newline
		C) The finance director likes the idea of issuing new bonds.
	\item The Marketing Director wanted the Sales Agent to be instructed by the Secretary as to filling out invoice forms.\newline
		A) The Secretary did not know how to fill out invoice forms.\newline
		B) The Marketing Director wanted to tell the Sales Agent how to fill out invoice forms.\newline
		\uline{C) The Secretary knew how to fill out invoice forms, but the Sales Agent did not.}
	\item Nissan is to announce on Monday a further modest rise in jobs at its plant in the north east England city.\newline
		A) The aim of Nissan is to announce on Monday further modest increase in jobs at its plant.\newline
		\uline{B) Nissan must announce on Monday a further modest rise in jobs at its plant.}\newline
		C) Nissan announced on Monday a further modest rise in jobs at its plant.
	\item The method is not economical enough to be applied.\newline
		A) The method is not economical but it can be applied.\newline
		\uline{B) The method is not economical and it can't be used in practice.}\newline
		C) The method is not economical and it was not applied.
	\item Commonwealth Bank of Australia yesterday failed to announce a jump in profit for the year.\newline
		A) Commonwealth Bank of Australia was forced to announce a jump in profits.\newline
		B) Commonwealth Bank of Australia managed to announce a jump in profits.\newline
		\uline{C) Commonwealth Bank of Australia was not able to increase its profits.}
	\item The auditors are sure to have found fraud in the company's transactions.\newline
		\uline{A) No doubt the auditors have found fraud.}\newline
		B) We are sure the fraud will be found.\newline
		C) The auditors are sure they will find the fraud.
	\item We had all the accounts checked through, but the error was not spotted.\newline
		A) We checked all the accounts in order to find the error, but we did not succeed.\newline
		\uline{B) We asked the bank clerk to check the accounts, and so he did, but the error was not found.}\newline
		C) We were forced to check the accounts because the error was not spotted.
	\item On settling the price problem the partners discussed the terms of delivery.\newline
		A) The terms of delivery were discussed prior to the price problem.\newline
		\uline{B) The partners discussed the terms of delivery after the price problem.}\newline
		C) The problems of price and delivery were discussed and settled at once.
\end{enumerate}
\ 

\newpage
\sublinksection{Неличные формы глагола. Задание по грамматике и использованию неличных форм глагола}

\hypertarget{ltask:2024-04-03-2}{--- Выполнение задания ---} (\hyperref[task:2024-04-03-2]{\color{blue}{перейти к тексту задания}})
\\

\textbfind{Выполненное задание (глагол в скобках поставлен в требуемую по контексту форму)}
\vspace{5pt}
\begin{enumerate}[nosep, leftmargin=*]
	\itemsep\eitsp
	\item Why do you object to \uline{\textbf{following}} the directions? (follow)
	\item I hear someone \uline{\textbf{knocking}} at the door. (knock)
	\item ``Why don't you like Jack?''\newline ``I dislike \uline{\textbf{hearing}} him tell old jokes.'' (hear)
	\item The Ford Theatre where Lincoln was shot \uline{\textbf{must be restored}}. (must/restore)
	\item The examiner made us \uline{\textbf{show}} our identification in order to be admitted to the test centre. (show)
	\item I want to finish \uline{\textbf{writing}} my report tonight. (write)
	\item Experiments show that certain foods can have their useful life \uline{\textbf{extended}} by treating them with doses of radiation. (extend)
	\item Silicon doped with phosphorous or another pentavalent element is called an n-type semiconductor, the n-type \uline{\textbf{representing}} the negative electric charge of the conduction electrons. (represent)
	\item Pierre and Marie Curie showed that beta rays consist of a stream of electrons \uline{\textbf{moving}} at terrific speed. (move)
	\item A competitive product and a receptive market are not enough: the product must somehow \uline{\textbf{be transported}} from the factory to the export market. (transport)
\end{enumerate}


\end{document}