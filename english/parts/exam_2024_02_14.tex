\documentclass[main.tex]{subfiles}

\setcounter{secnumdepth}{0}

\begin{document}

\setcounter{section}{100}

\linksectionold{Февральский зачёт 14.02.2024 (Смольская Н.Б.)}

\sublinksectionold{Выполненное первое задание}

Из предложенного ряда выберите слово, которое является синонимом подчёркнутого слова.
\vspace{3mm}

\begin{enumerate}[nosep,leftmargin=*]
	\item One recent advance \underline{confirmed} that Alzheimer's disease is sometimes inherited.
	
	\underline{\textbf{(A) made valid}} \quad (B) questioned \quad (C) diagnosed \quad (D) made doubtful
	\item Video compression is sending not a complete colour portrait for each frame, but rather a shorthand version that describes the difference between the current frame and the \underline{previous} one.
	
	\underline{\textbf{(A) preceding}} \quad (B) already seen \quad (C) viewed \quad (D) following
	\item The International Union of Biological Sciences met in Amsterdam to discuss how many species there are, and how many there will be if the environment \underline{is altered} in various ways by man.
	
	\underline{\textbf{(A) is changed}} \quad (B) is polluted \quad (C) is made worse \quad (D) is affected
	\item Almost half of U.S. newspaper editors say that dinosaurs and humans lived \underline{contemporaneously}.
	
	(A) peacefully \quad (B) for a short period of time \quad \underline{\textbf{(C) at the same time}} \quad (D) destroying each other
	\item Establishing an \underline{appropriate} correspondence between time and path position parameter is an important condition in controlling the path of the robot arm.
	
	(A) approximate \quad \underline{\textbf{(B) proper}} \quad (C) precise \quad (D) the closest
\end{enumerate}

\sublinksectionold{Выполненное второе задание}

Выберите правильный вариант из каждой пары. Выбранный вариант подчёркнут.
\vspace{3mm}

\begin{enumerate}[nosep,leftmargin=*]
	\setcounter{enumi}{5}
	\item Everything is going well. We \textbf{didn't have / \underline{haven't had}} any problems so far.
	\item Margaret \textbf{\underline{didn't go} / hasn't gone} to work yesterday. She wasn't feeling well.
	\item Look! That man over there \textbf{wears / \underline{is wearing}} the some sweater as you.
	\item Your son is much taller than when I last saw him. He \textbf{grew / \underline{has grown}} a lot.
	\item I still don't know what to do. I \textbf{didn't decide / \underline{haven't decided}} yet.
	\item I wonder why Jim \textbf{is / \underline{is being}} so nice to me today. He isn't usually like that.
	\item Jane had a book open in front of her but she \textbf{didn't read / \underline{wasn't reading}} it.
	\item I wasn't very busy. I \textbf{\underline{didn't have} / wasn't having} much to do.
	\item Mary wasn't happy in her new job at first but she \textbf{begins / \underline{is beginning}} to enjoy it now.
	\item After leaving school, Tim \textbf{\underline{found} / has found} it very difficult to get a job.
	\item When Sue heard the news, she \textbf{\underline{wasn't} / hasn't been} very pleased.
	\item This is a nice restaurant, isn't it? Is this the first time you \textbf{are / \underline{you've been}} here?
	\item I need a new job. I\textbf{'m doing /} \underline{I\textbf{'ve been doing}} the same job for too long.
	\item I'd like to see Tina again. It's a long time \underline{\textbf{since} I \textbf{saw} her} \textbf{/ that} I \textbf{didn't see} her.
	\item Bob and Alice have been married \textbf{since} 20 years / \underline{\textbf{for} 20 years}.
\end{enumerate}


\end{document}
