\documentclass[main.tex]{subfiles}

\begin{document}

\linksection{Лекция 31.01.2024 часть 2 (Смольская Н.Б.)}

\sublinksection{Перевод текста. Задание}

--- Выполнение задания на перевод текста ---
\\

\textbf{The ultimate quest.}

The elevator doors opened into a cavernous room in an underground tunnel outside Geneva.
Out came the eminent British astrophysicist Stephen Hawking, in a wheelchair as always.
He was there to behold a wondrous sight.
Before him loomed a giant device called a particle detector, a component of an incredible machine whose job is to accelerate tiny fragments of matter to nearly the speed of light, then smash them together with a fury far greater than any natural collision on earth.

Paralysed by a degenerative nerve disease, Hawking is one of the world's most accomplished physicists, renowned for his breakthroughs in the study of gravitation and cosmology.
Yet the man who holds the prestigious Cambridge University professorship once occupied by Sir Isaac Newton was overwhelmed by the sheer size and complexity of the machine before him.
Joked Hawking: "<This reminds me of one of those James Bond movies, where some mad scientist is plotting to take over the world.">
It is easy to understand why even Hawking was awed: he was looking at just a portion of the largest scientific instrument ever built.
Known as the large electron-positron collider, this new particle accelerator is the centrepiece of CERN, the European Organisation for Nuclear Research and one of Europe's proudest achievements.
LEP is a mammoth particle racetrack residing in a ring-shaped tunnel 27kms (16.8 miles) in circumference and an average of 110 meters (360 ft.) underground.
The hardware, including nearly 5,000 electromagnets, four particle detectors weighing more than 3,000 tons each, 160 computers, and 6,600 km (4,000 miles) of electrical cables.
Tangles of brightly coloured wires sprout everywhere, linking equipment together in a pattern so complicated, it seems that no one could possibly understand or operate the device.
In fact, it takes the combined efforts of literally hundreds of Ph.Ds to run a single experiment.
\\

\textbfind{Выполненное задание}

\end{document}
