\documentclass[main.tex]{subfiles}

\begin{document}

\linksection{Лекция 29.11.2023 (Смольская Н.Б.)}

So let us begin.
The earlier we begin, the earlier we finish.
But I'm not sure, because I always make myself stop.
But in any case, let's begin.
So our today's meeting will be more over grammar discussion.
But I do remember that your hometask for today was to think and to work out and now to present for me your materialized ideas on the topic we were trying to somehow discuss during our previous class, which was prospects of your future scientific career.
So I hope that I will enjoy your topics, not the essays, as you remember that we discussed.
So I will collect them just by thinking several minutes, because the thing, or actually two things I would like us to discuss today, in terms of grammar, because we will like change the aspects of our discussion.
So today that will be grammar, and again for the next class, I will again choose somehow of conversation, and again grammar and some discussion on the matters of your, I think, scientific and academic English interactions.

So, but today, I, well, I was thinking what to do, so for you to feel somehow strongly impressed by the grammar that you were studying during quite a long period of your previous life, being undergraduates and then graduates in your master's school.
So we will refresh your knowledge of, as I have said, of passive voice, but that will be not a separate passive voice, but the system of tenses in the English language which is also very important.

But we will begin with task that you had as the first task during our previous class test, which is borrowings from Latin and Greek, which if I hope you remember was the thing which is really important for academic English taking into account that Latin and Greek words are actually the words that gave life to a great amount of terms in modern English, in modern academic English, and not only in academic English, but also, if I say, academic Russian.
Because the basis for the terminological vocabulary of all the, especially Indo-European languages, is of course Greek and Latin as the starting points for science in the European or Eurasian part of our planet.
So that is why our first discussion and your first training exercises will be, again, related to these borrowings.
So what's the Russian for borrowing? What is a borrowing? To borrow money.
Заимствование.
To borrow -- взять взаймы.
So that is why not only in our materialistic life, but even in our linguistic life, we also borrow something and within the languages these are borrowings (заимствования) from other languages.

\newpage
\sublinksection{Задания на греческие и латинские заимствования}

So, I have prepared, collected together several exercises which are not actually difficult, but still they will be the basis for our discussion.
So, the first two are actually the exercises with the same approach that you practised during our previous class.
The first one is just the beginning to actually make sure that you know all the words and you understand what is written in English.
So to match together the words from modern English which actually originated either from Greek or from Latin and to match them to their definitions in the right column.
And the second exercise is maybe a little bit more difficult, but still it is familiar for you.
So the same to match the meaning of the root from either Latin or Greek with its meaning in the second column and then to find the word worked out from this root in the third column, so to match them together.
And when we discuss the first two, we will switch to the second task, oh sorry, to the third task, which is also quite interesting because it also develops your vocabulary.
But that will be another approach to practise your Greek and Latin models.
So let me enlarge, enlarge probably the screen.
So take your time.
So I don't think that it is difficult, but it's still, it's just the case for us to also discuss the words and to make sure that you remember what the words mean and what actually to do with them.
So take your time.
Five, not more than seven minutes, I think.
So I hope that you will write down the words and maybe some of them will be clear for you and familiar after that.
\\

--- Выполнение заданий ---
\\

\textbfind{Выполненное первое задание}

\begin{itemize}[nosep,leftmargin=*,label={}]
	\itemsep\eitsp
	\item immobilize -- \textbf{to} prevent from moving
	\item finite -- limited or bordered by time or any measurements
	\item fortify -- \textbf{to} make stronger
	\item commemorate -- \textbf{to} honour the memory of smth or someone
	\item memorandum -- \textit{a} note or record of events written as a reminder
	\item pedestrian -- \textit{a} person who is walking
	\item revitalize -- \textbf{to} put new life into
	\item autonomous -- functioning independently
	\item dogmatic -- controlled by a single teaching or doctrine
	\item amorphous -- having no specific or recognisable shape or form
	\item sophisticated -- not naive; complex, as a piece of machinery
\end{itemize}
\ 

It's like a puzzle.
When you join the correct things together, parts together, everything becomes clear.
So one minute and we start our discussion.

So let's begin.
We will actually begin not just with the straight and direct matching the Latin and the Greek words with their definitions, but we will also refresh your knowledge somehow of grammar and morphology of the English words, of the English language, because it is also important when you deal with translating texts.
And okay, I will try to explain what I mean, and I hope that it will somehow help you in the future.
So, as you can see, if you analyze what you have, I think it is normal for those who deal with mathematics and physics to apply your analytical skills to everything you do, because it really helps a lot.
So if you pay attention to the column of definition, you definitely can subdivide the definitions given to some clear elements or clear, let it be some definite words that will at least help us understand that they mean something grammatically clear.
Sorry for, for, for, масло масляное, как будет по-английски, who knows?

Have you ever thought of it?
Because, well, actually, what I usually say, upon which I usually insist when I deal with students of linguistics, is actually if you know something in Russian, you should definitely, well at least you should have a strong desire to know it in English.
So just to compensate and to have this equilibrium.
Well, and just, well it is interesting, just to know how to say it in another language.
So one of the variants, there can be different variants, but the variant that I like, and it is very simple, is much of muchness.
So it is масло масляное, when you repeat something by means of repeating the words, actually.

So if we now, okay, if we now pay attention to this column, which actually I would like us to pay attention to, so what element, morphological and grammatical element, and this is not the element but a separate word, so what element is repeated several times, just to make it clear for you that you should think of and actually match this definition to some clear and definite words.
What element is repeated? It is very obvious.
\textbf{To.}
Right? What is it? \textbf{To.}
\textbf{To} honour, \textbf{to} put, \textbf{to} make, \textbf{to} prevent.
So what is \textbf{to}?
What is it?
Well if we join \textbf{to} together with the verb that's an infinitive but if we speak about \textbf{to} that's not an infinitive.
That's an infinitive particle (приинфинитивная частица).
It is like a sort of a marker that shows that if we have the verb after it, it is an infinitive, and we now how translate it into Russian.
It will definitely have the ending "<ть"> or "<ти">, sometimes.
То есть, если вы видите, to read, to write, вы точно будете переводить это писать, читать, а не пишет, будет писать и так далее.

Это некий формальный показатель.
Английский язык наполнен и насыщен этими маркерами.
Потому что, ещё раз говорю, английский язык -- это математика в лингвистике, никакой другой язык настолько не аналитизирован (я позволю себе неологизм такой, даже не неологизм, а окказионализм называется, знаете что такое окказионализм? Это авторский неологизм, он может не быть зафиксирован в языке, но он вот сиюминутно появился, может какое-то время жить еще), что я хотела сказать, вот что значит окказионализм, да, он у меня появился в голове, я его не озвучила, и он...
Аналитизирован, вот что я хотела сказать.
Никогда просто не слышала этого, может это и существует, но тем не менее, по правилам русского языка слово может быть образовано.

So in any case, back to this \textbf{to}. 
So if we have this \textbf{to} with the verb in the English language, so we definitely understand that will be the definition of the verb in the left column.
So in our case, it can help to solve the problem of matching the words together.
If we now view this left column as also the collection of words, some of them are actually verbs, it will also help us to define some other morphological elements that will help us to understand that this is the verb.
So what actually I lead to.
So now let's match the words, not like step by step, word by word, but let us work with those that start with the definitions of which start with \textbf{to}.

So what is actually the left column choice to the definition to prevent from moving? What is it?
Immobilize.
That's immobilize.
Absolutely correct.
So we mark it for us.

So then we have to make stronger it is what?
Fortify. Absolutely correct.

To honor the memory of something or someone.
To commemorate.

And last one, to put new life into.
And it is to revitalize.
According to the rules of the English language, to the reading rules of the English language.

So now if we pay attention to these words, immobilize, fortify, commemorate and revitalize, especially the first and the second.
So what are actually the things that you should know and what are the things that attract your attention?
Endings?
These are not the endings, these are the suffixes that are used to form these verbs.
All these verbs are borrowings from other languages.
They are not Anglo-Saxon in their origin.
And if we see, if we find out that some word contains one of these suffixes, we should definitely understand and know that this word is a verb and this is a borrowed verb.
So I would like actually to think and to write down at least one example each of you with your own examples of the verbs containing the same suffixes.
Because we have immobilize and revitalize.
They have the same suffix.
So think of any verb that you have in your minds that also contains the same suffix.
So then we have fortify with suffix "<fy"> and commemorate with suffix "<ate">.
This is also the suffix for the verbs.
And we shall compare.
And it will also somehow help us to enrich and refresh your vocabulary.
Some verbs that ends with "<ize">, "<fy"> and "<ate">.
I also have three verbs.
We shall then compare.
Because it's quite easy, actually, to translate the English text into Russian because you are calm when you translate or to do vice versa, or to write something.
But to find somewhere in the depth of your mind the words that you should say, it is quite a training.
Okay, so just one by one, any, anyone, so let's begin again, so not to be disturbed by, so what is your example? Any that you have thought of?

Realize, generate, verify, commercialize, simplify, analyse, initialize, crystalize, create, qualify, optimize, modify, clarify, notify, satisfy, activate.

Okay, so it means that actually there is a great amount of such verbs, that you actually come across in the text and when you see this word you definitely now that it is a verb by all means.
For the English language it is really very important to know that this or that word is a verb, because the verb is the center of the sentence in the English language.
Everything is circled around the verb.
And when you deal with the sentence first of all you should find the verb or to be more precise the predicate.
What is a predicate? Predicate?
Это сказуемое.
Predicate in the sentence.
Это сказуемое.
В английском языке, когда вы переводите предложение, строите предложение, самое главное всегда определиться с глаголом.
Потому что глагол в английском языке...
If we discuss the sentence in the English language what is important when we speak about the sentences in the English language?
What is the most important principle for the English language sentence?
The word order.
And we definitely know that the word order in the English sentence is of what type?
The direct word order.
What does it mean?
What does it mean that the sentence has the direct word order?
Подлежащее, сказуемое, ...
Yes, it means that the members of the sentence have the obvious obligatory order in which they follow each other.
The subject, the predicate, ... And it is not the end.

Почему я использую the subject?
Потому что подлежащие, как член предложения, не количество подлежащих, а подлежащее, может быть только одно.
У глагола не может быть много подлежащих в разных категориях.
Если мы говорим об однородных подлежащих, они потому и называются однородные, что они одинаковые.
Это не какие-то разные по категории: если говорим мальчик и девочка играли, то мальчик и девочка подразумевается, что это не объекты разных категорий, а их должен объединять какой-то признак.
Поэтому подлежащее в предложении одно, оно может быть представлено несколькими словами, но оно одно, и точно также the predicate, сказуемое тоже может быть одно, или же однородные, это значит, что они однородные, одинаковые.
Но, как я уже только что сказала, подлежащее, сказуемое и ...
Вот это обязательно, потому что очень часто почему-то останавливаются студенты вот на этой конструкции, дальше не принимая во внимание, что обязательным условием еще будет вот этот третий элемент.
Он может быть представлен двумя вариантами.
Это либо здесь будет либо object, либо complement.

Что такое object? Дополнение?
Это не простое дополнение в английском языке, это конкретное дополнение.
Вспоминайте русскую грамматику.
Это прямое дополнение.
Прямое дополнение.
Что значит прямое дополнение?
Если мы будем говорить про формальные признаки, то это существительное, которое отвечает на вопрос, кого что стоит в винительном падеже, и, что важно, присоединяется без предлога.
Это прямое дополнение.
Я люблю кофе.
Я вижу кошку.
Кого? Что?

Что такое complement?
Complement -- это to complete.
Завершать что-то.
Дополнять -- завершать.
В этом случае мы тоже говорим о любых элементах, которые завершают высказывание.
Понятно, что высказывание может и на подлежащем, сказуемом заканчиваться.
It is raining, например.
Но в этом случае это не переходный глагол.
Но тем не менее высказывание предполагает собой наличие третьего элемента.
И в этом случае complement -- это будет непрямое дополнение или любой другой второстепенный член предложения, который может стоять после глагола.
Поэтому я и говорю, что глагол в английском языке в предложении это главное.
Чтобы избегать сложности с пониманием предложения, нужно всегда найти глагол.
Вы найдете глагол, вы какими-то различными, так сказать, ассоциативными способами вычленяете его значение, и уже более-менее становится понятна ситуация, выраженная английским предложением.
Значит это может быть непрямое дополнение, то, что мы имеем в русском языке, просто аналогия русского языка нам здесь важна.
В этом случае под непрямым дополнением, под косвенным, мы понимаем остальные способы присоединения с существительным, то есть через подлог.
Я иду в школу, я направляюсь к дому и так далее, то есть через какой-то дополнительный элемент присоединяются существительные.
В этом случае оно действительно из косвенного дополнения приобретает статус и роль иного члена предложения.
Обстоятельства места, обстоятельства времени, манеры (of manner, способа действия) и так далее.
Либо же мы здесь на этом месте точно так же эти же обстоятельства можем иметь в качестве выражения их после глагола.
Что может у нас стоять?
Глагол что к себе может присоединять?
Или существительное или наречие.

Какие части речи вы ещё знаете?
Заметка: причастие и деепричастие это не части речи, а формы глагола.
А еще у нас осталось?
Итак, существительное, глагол, наречие. И прилагательное!

Мы с вами говорим о разных уровнях языка, язык это multi-layered system, очень многоуровневая система, в которой начинается фонетики, фонологический уровень, морфологический, грамматический, лексический, синтаксический.
Они между собой связаны, очень тесно всегда переплетаются, но тем не менее они отчетливо выражаются и оперируют определенными понятиями, понятийными категориями.
Так вот, прилагательное, оно не просто так ведь называется в русском языке.
Русский язык очень метафоричен даже с точки зрения терминологического аппарата, если пользуется своими собственными источниками, а не заимствованиями.
Если оно прилагательное, значит оно к чему-то прилагается.
И оно обычно прилагается к существительному.
Прилагательное, это не зря называется определение, определяет именно существительное.
Больше ни к чему прилагательное вы не приложите.
Но больше ни к чему не прилагается.
Поэтому прилагательное, если мы дальше будем говорить с вами про те места в предложении, где могут стоять другие части речи в качестве других членов предложения, то определение тех членов предложения, которые могут быть выраженными существительными, предполагает, что в препозиции к ним будут стоять прилагательные.
Прилагательные в постпозиции к существительному возможны только в исключительных случаях.
Петр Великий, но в этом случае это даже не прилагательное, а приложение.
Это даже такое, сейчас я постараюсь, но это будет терминологически уже, адъективированное имя собственное.
То есть это прилагательное, которое стало сначала существительным, а потом опять стало прилагательным.
У него такой особый статус, у таких вот определений, которые стоят в постпозиции.
А нормальная позиция прилагательного -- это в препозиции к существительному, то бишь к субъекту и к объекту.

А вот эта конструкция (S -- V -- O (or C)) в английском языке неизменна.
И это как раз и есть прямой порядок слов.
Есть инверсия в английском языке, но в крайне исключительных случаях.
Может быть вы мне скажете, что такое инверсия?
Это обратный порядок слов.
То есть когда подлежащее и сказуемое меняются местами.
Есть прямая инверсия.
Кто-нибудь знает какие-то способы?
Вот, вернее, не способы, а случаи, когда инверсия в английском языке проявляется.

Never saw he such an interesting person.
Полная инверсия в английском языке на настоящем уровне развития практически никогда не присутствует, только за исключением эмфатических конструкций, то есть усилительных конструкций.
Они вызываются определенными словами и, грубо говоря, определенной ситуацией.
Вот один из примеров -- это наречие "<never">, которое может быть.
"<Rarely"> -- есть отдельные слова, может быть, мы с вами постараемся, я какой-то список соберу, чтобы вы приняли это к сведению.
Но это, я еще раз подчеркиваю, это так называемые эмфатические конструкции.
Эмфазы, я думаю, все знаете, что такое.
Это позиция усиления, грубо говоря, привлечения внимания к чему-то.
В английском языке только в определенных случаях такое может быть.
Never -- это что?
Это наречие.
Это никогда.
И мы знаем, что положение наречия чаще всего вокруг глагола.
Отличаются предложения?
He never met such a wonderful person.
Или Never met he such a wonderful person.
Чувствуете разницу? Эмоциональная вовлечённость.
Правильно.
Вот та самая эмфаза, то самое привлечение внимания, либо просто констатация, факт, как в первом случае, либо вот это вот эмфатическое усиление ситуации, английский язык это пока еще сохраняет, слава богу, и даже выражает это на уровне синтаксиса.
В древние времена, когда были живы англосаксы, достаточно свободный порядок слов тоже был в то время, и тоже можно было менять местами, потому что были падежные окончания.
Сейчас этого нет, и кстати, и не кстати, вернее, для нас не кстати, но для них кстати.
Порядок слов, это был именно самый основной механизм компенсации утраты падежных окончаний.
Потому что что такое падежные окончания? Для чего вернее они нужны.
Чтобы связывать слова.
Да, чтобы указывать, чтобы передавать отношение между словами в предложении.
Да, в русском языке мама мыла раму.
Мыла мама раму.
Раму мыла мама.
Независимо от того, как мы будем переставлять слова в предложении, отношение между ними будет всё равно то же самое.
Мама -- это субъект действия, мыло -- это действие, а всё действие нацелено на раму, и куда бы мы ни поставили это слово, мы всё равно будем понимать, что раму -- это объект.
В английском языке, если мы примерно что-то такое переведём, я встретил кошку.
I met a cat.
Вы чётко все понимаете.
I met или The girl met a cat.
Девочка встретила кошку.
Кошку встретила девочка.
Переводим.
The cat met the girl.
И в английском языке сразу же, как только мы поменяли местами два существительных, они заняли разные позиции в предложении, ситуация стала совсем другой.
Почему?
Потому что девочка окончание "<а"> указывает нам на то, что это именительный падеж, это деятель, это субъект действия, а кошку, винительный падеж окончание "<у"> выражает нам, что это не именительный падеж, а это "<у"> окончание винительного падежа, выражающего объект, на который нацелено действие.
В английском языке the girl met a cat никаких указаний падежных не имеет.
И поэтому английский язык, дойдя до такой стадии утраты любых падежных окончаний, это было на рубеже 17-18 веков, пришел к тому, постепенно приходил к фиксации в английском языке, постепенно происходила фиксация слов в определенных позициях, но как раз по 17 веку объективно сложилась информация, что порядок слов в английском языке стал обязательным признаком для выражения отношений между словами в предложении.

И вот я теперь прихожу к тому, что глагол -- это центр.
То, что стоит от него слева, то, что он держит.
Или если мы так будем говорить, что нет, он на нас смотрит лицом, то, что он держит правой рукой, это тот, кто его выполняет.
А тот, кого он держит левой, куда тянется левая, это тот, на кого это действие нацелено или на что.
Именно таким образом положение вот в пространстве в английском языке выражает вот эти отношения.
И окончания по сути дела не нужны.
Да, такой нельзя назвать атавизм, потому что это очень важный второй признак выражения значения, выражения отношения между словами, который активно используется в русском языке и сохранился в английском как второй признак, это что еще выражает отношение между словами в предложении?
Предлоги, абсолютно верно.
Мы точно знаем в английском языке, что предлог to выражает что? 
Итак, to это направление действия.
Это остаток дательного падежа.
"<In"> -- это место внутри.
Это остаток предложного падежа, хотя в письменных памятниках, которые нашли на древнеанглийском языке, предложного падежа уже не было.
"<On"> на поверхности, "<for"> родительный падеж, цель, "<of"> всем известный -- выражение родительного падежа притяжательности, принадлежности, то есть в английском языке при фиксации слов в предложении обязательный поэтому когда говорят русские студенты русскоговорящие наверное славянскоговорящие но наверное больше восточнославянскоговорящие, потому что запададнославянские языки, они всё равно уже тяготеют, достаточно давно тяготеют в сторону выстраивания вот этого аналитизма, но ещё не так сильно, но тем не менее русский язык менее аналитичен по сравнению с западнославянскими или южнославянскими языками.

Так, ну, в любом случае, мы с вами вот определили, что порядок слов -- это тот важный элемент, который для нас принципиальный, когда мы выстраиваем английское предложение.
Давайте дальше.
Все-таки вернемся теперь к вашему упражнению.
Я обратно все включаю.

So before, while it is working to be self-activated, I would say, we will continue discussing the same approach to analyzing the words because it is also important.
So now we know that when we deal with the tests or when you deal with the texts, when you come across the words with some elements, you actually, you can guess what part of speech is it.

Кстати, дайте мне определение слова "<слово">?
Самое простое, вот как вы там своему младшему брату объясните, или маленькому сыну, что такое слово?
Слово -- это набор букв от пробела до пробела.
Вот ребёнку, чтобы понять, да.
И этот набор букв, это слово, которое можно произнести вслух.
И это слово будет называть те предметы, которые вокруг тебя.
Вот.
Машина.
Вот она, твоя машина.
Так? Ну, как детей учат.
Готовьтесь. У нас ещё вся жизнь впереди.

Окей, so, verbs.
Какие ещё части речи есть?
A noun.
How many nouns do we have in this list?
Now in the left corner.
Two.
And how many definitions show us that we actually have two nouns?
So we have two definitions, and they are, as we just had when discussing the verbs, we also have an indefinite article.
It also helps us to actually understand and to solve the task.
If you see two articles, there will be definitely two words that will be nouns.

So, a person who is walking.
A pedestrian.
Of what origin is this word?
Pedestrian, это что за слово?
Откуда?
Это Греция.
Это греческое слово.
Что такое pedi? Это ноги.
Во множественном числе из греческого языка.
Педали, педикюр и педагог.
И встаёт вопрос почему?
Знаете, кто такой педагог?
Это раб, который ребёнка в школу водил.

Поэтому я филолог.
Кандидат филологических наук.
Хотя очень уважаю педагогов.
Я ученый секретарь педагогического диссовета.
Но филолог.

And now a note or record of events written as a reminder.
A memorandum.
What can we say about the word memorandum?
I hope you know that the origin of this word is what?
It's definitely Latin.
What makes you definitely now that it is Latin?
Here is the ending "<um">, which is in the English language, so it is the ending that shows us that this word originated from Latin but in Latin this word was of neutral gender.

Because if we think of other Latin words.
So this is the word which is a note so it means that it is in what number?
Single.
Single and its ending is "<um">.
So now we know that in Greek that was neutral
We also have male and female.
Do you know the markers in the English words of the Latin words that show us that this word is Latin but it is of male gender?
"<Us">.
Absolutely correct.

And what about female?
That is "<a">.

So let us give the examples.
Ursus/ursa.
Lexus/lexa.

Vita -- жизнь.

So, but what is important is that these words in their majority, almost in all the cases, they preserve the form of their plural.
What will be the plural for memorandum if we have several notes uh stickers on the um on the freeze on the freezer door so what will be the plural?
A memorandum but a lot of...
And if I write the word for you, you will definitely remember that you know how to say it.
So what about this word? Memoranda.
Because data, you all know that data is plural, from the singular of datum.
And well, actually, according to the rules, so nowadays, as you know, English is quickly developing, and we can even come across the word in singular after the word data, and nowadays it is accepted, though maybe even some 20 years ago or even 15, that was a mistake, because data is in plural, and it means that the verb that stands after it should be in plural.
Эти данные важны.
These data are important.
К сожалению сейчас даже в научных статьях уже встречается и принято к сожалению да вот это на уровне поскольку все-таки терминологическое слово это некая особая вот это академический английский уже встречается the data is important.
Как некое собирательное существительное, но тем не менее его природа и вот эта форма -- это форма множественного числа.
Сейчас мы будем, я не буду забегать вперед, пока что мы вспомнили вот это, давайте сейчас дальше пойдем, и мы сегодня все равно дойдём до вот этих примеров.

So, таким образом, что у нас здесь?
Это тоже важно, да, вот слова типа memorandum, datum.
Итак, слова, которые заканчиваются на нестандартные для английского языка окончания, типа "<um">, "<us">, "<a">, потому что в английском языке исконно английских слов, заканчивающихся на "<a">, практически нет.
Это не англосаксонские слова.
Это будут существительные.
Слово pedestrian тоже нам демонстрирует интересный суффикс, "<ian">.
И я думаю, что для вас это всех известно, но просто лишний раз акцентировать ваше внимание, что этот суффикс присущ для существительного.
Если вы таким образом historian, politician, то есть это как в исконно английских словах суффикс "<er">, деятель чего-то.
Так вот этот суффикс характерен для заимствованных слов, которые тоже выражают человека, который выполняет какое-то действие или некую деятельность.
Значит, возвращаемся дальше.

Здесь мы больше не видим таких вот намеков на какие-то конкретные части речи, но у нас остались с вами еще следующие определения, и факт у нас следующий, что оставшиеся слова, это какие части речи?
Это прилагательные.
Да, это уже прилагательные.
Вот как раз вы обратили внимание здесь на два слова, которые тоже имеют определённый, отчётливый суффикс "<ous">.
Autonomous и Amorphous.
Это, причем чаще всего, вот этот суффикс характерен именно для заимствованных из греческого языка, но распространился, конечно, с течением времени и на остальные.
Это всё прилагательные.
Давайте их тоже определим и тоже буквально там несколько слов тоже еще скажем.

Итак, у нас осталось number two.
Finite -- limited or bordered by time or any measurement

So, what about autonomous, that's what?
Functioning independently
Good.

Dogmatic.
Controlled by a single teaching or doctrine.

Then amorphous.
Having no specific or recognisable shape or form

And the last one is sophisticated.
Not naive, complex as a piece of machinery.

So what is machinery, by the way?
How would you translate it?
Механизм и чаще всего механизм такой достаточно серьезный крупный, потому что в экономике machinery это что?
Да, это тяжелая индустрия, тяжелая промышленность, а если мы будем говорить о том что нас окружает человеком в более в бытовом смысле, в бытовом, in a more sophisticated manner, not the freezers or something else.

Как мы назовем оборудование? Это будет что? Какое слово? Собирательное оборудование.
Вот если machinery это оборудование с точки зрения тяжелой промышленности, станки.
То нечто утончённое будет equipment.
\\

\textbfind{Выполненное второе задание}

\begin{itemize}[nosep,leftmargin=*,label={}]
	\itemsep\eitsp
	\item bio -- life -- bionics
	\item fort -- strong -- comfortable
	\item chron -- time -- synchronize \textit{(греческое заимствование)}
	\item port -- carry -- disport
	\item ced -- go -- procession \textit{(латинское заимствование)}
	\item scrib -- write -- transcription
	\item ali, alter -- another -- alibi
\end{itemize}
\ 

Второе задание.
Окей, соотнесите латинский и греческий корень с его значением.
Давайте просто очень быстренько.
Можно вспомнить молодость, поднимать руку, предлагать варианты.
Bio -- life -- bionics.
Very good.
Что по поводу bionics можете сказать?
Что это за часть речи?
Это существительное.
Это тоже одно из правил.
Вот это "<cs">.
Обозначение отраслей, областей науки.
Bionics, economics, mathematics, politics.
Если мы видим вот такое окончание, это значит, мы говорим про научные области исследования.
Потому что economy -- это что?
Это экономика как некая характеристика жизни общества.
Это не научное исследование, а именно характеристика жизни общества.

Fort -- strong -- comfortable.
Comfortable, ну, здесь такое очень интересное слово.
Тут прямо столько его можно с точки зрения морфологии разбирать, что будет интересно.
Но мы пока этим с вами не занимаемся.

Chron -- time -- synchronize.
Обратите внимание, опять наш вот этот с вами суффикс здесь встретился.
Слово однозначно производное от греческого корня.
И вот мы, пожалуйста, с вами видим эту самую характеристику.

Port -- carry -- disport.
Классический фильм "<Ночной портье"> не смотрели, нет?
Портье -- французское слово, прямо нам показывает значение этого корня.

Ced -- go -- procession.
Процесс от этого же слова.
Это чередование d и s, которое регулярно происходят в заимствованных латинских словах.
В глаголе, глагол какой будет? От Procession.
Proceed.
А существительное будет иметь чередование с согласным "<s">.
Exceed, accession, да, четко будет везде просматриваться вот это чередование согласных.


Так, шестое, мы с вами в прошлый раз с этим сталкивались.
Scrib -- write -- transcription.
Но тем не менее всё равно, scrib, p и b чередование в зависимости...
Здесь происходит чередование с p, потому что дальше идёт глухой согласный t, происходит оглушение, которое в романских языках имеет место быть.

Кстати, как по-английски романский?
Давайте начнём, как германский будет?
Вот германский -- Germanic.
А романский -- Romans.
Ни в коем случае не Roman, потому что Roman это римский.
А мы говорим о собирательном вот этой языковой группе романских языков.

Ну и последнее.
Ali,alter -- another -- alibi.
То есть это нечто другое.
Alibi, значение по сути слова, это другой.

\newpage
\sublinksection{Задание подобрать синоним (навык для составления summary статьи)}

Окей, очень быстро, но буквально только дам посмотреть и проверить.
Here is well actually it is one of my favourite tasks.
I like it very much.
When you are given the sentence, in this case you just need to choose the correct word.
But we will also practise it in terms of grammar.
Очень быстро, просто для себя, чтобы себя проверить, да?
Я не проверяю, вы себя просто на оценку.
Подчеркнутое слово -- это слово заимствованное, соответственно, здесь у вас немножко расширен выбор возможных значений этого слова.
Вам четыре дано, и вам здесь нужно выбрать одно правильное из данных.
\\

--- Выполнение задания ---
\\

\textbfind{Выполненное третье задание}

\begin{enumerate}[nosep,leftmargin=8mm]
	\itemsep\eitsp
	\item One recent advance \textit{\uline{confirmed}} (\textbf{made valid}; questioned; diagnosed; made doubtful) that Alzheimer's disease is sometimes inherited.
	\item Video compression is sending not a complete colour portrait for each frame, but rather a shorthand version that describes the difference between the current frame and the \textit{\uline{previous}} (\textbf{preceding}; already seen; viewed; following) one.
	\item The International Union of Biological Sciences met in Amsterdam to discuss how many species there are, and how many there will be if the environment \textit{\uline{is altered}} (\textbf{is changed}; is polluted; is made worse; is affected) in various ways by man.
	\item Almost half of U.S. newspaper editors say that dinosaurs and humans lived \textit{\uline{countemporaneously}} (\textbf{at the same time}; peacefully; for a short period of time; destroying each other).
	\item Establishing an \textit{\uline{appropriate}} (\textbf{proper}; approximate; precise; the closest) correspondence between time and the path position parameter is an important condition in controlling the path of the robot arm.
	\item It wasn't until language researches began computer programs that the importance of lexical \textit{\uline{ambiguity}} (\textbf{having two or more possible meanings}; having an emotional component; having a pictorial component; having an idiomatic character) came to be understood.
	\item The negative charge of an electron slightly \textit{\uline{distorts}} (\textbf{changes the usual form}; expands; make weaker; contracts) the lattice of the metal.
\end{enumerate}
\ 

Сделали?
Для чего, например, полезны такие упражнения?
Это, по сути дела, ваше второе задание на кандидатском экзамене, когда вы даёте summary статьи.
Есть определенные формальные требования, которые мы ещё будем с вами проговаривать, но тем не менее я проговорю.
Когда вы отвечаете на экзамен, или же если вас просят, может быть в будущем вы когда-нибудь будете выступать в качестве peer review, а кто это такой?
Это рецензент, к которому обращаются для того, чтобы дали рецензию на статью.
Кстати, что значит слово peer в английском языке?
Не зря слово пэр в русском языке теперь существует.
Перевод это как раз транслитерация слова peer.
Но как раз в среднеанглийский период оно звучало как пэр.

A peer -- это кто?
Как вы переведете?
Это что за человек?
Значение, что это некто равный вам.
Да?
Ну, в зависимости от того, про что мы говорим.
Это может быть сверстник, ровесник, коллега.
А пэры, это были, помните, были палаты пэров, да?
Это некие равные благородного происхождения заседали, они должны были быть равными, чтобы принимать законы при управлении страной.
То есть это идея парламента такая была при короле в монархии.

А дальше возвращаемся обратно.
Тут я себя поймала за хвост.
A peer review -- это рецензент, который является равным к тому, кто написал статью, либо занимается исследованием в аналогичных областях, примерно занимает аналогичное положение, человек, которому обращаются для того, чтобы дать оценку, так сказать валидность, современность, актуальность написанного научного произведения.
И просто так нельзя, просто так человек не имеет права написать какую-то отписку, есть определенные требования.
Одно из них это действительно письменная работа, в которой разрешаются цитаты, но определенного количества и для того, чтобы подтвердить свое высказывание, а вообще это пересказ, но другими словами.
То есть мы называем это нехорошим словом rewriting, но идея в рерайтинге, если её утилизировать, то это плохо, а вообще рерайтинг -- это как раз то, чему нужно учиться.
То есть когда можно написать ту же самую идею, но другими словами.
И вот по сути дела здесь мы с вами такую попытку уже к поиску того, чтобы выразить вроде как похожую идею, но за счет других слов, подбирая какие-то отдельные слова, подбирая к ним синонимы или же описательные конструкции, как в некоторых случаях, это как раз вот развитие этого навыка или по крайней мере понимания, что от вас требуется.
Поэтому для научного дискурса очень важно расширять свой словарь не только новыми словами, но еще и синонимами и антонимами к тем словам, которые вы знаете.
Итак, здесь возможность уже прочитать целиком предложение, а не только корни или отдельные слова.
Поэтому у вас есть возможность сразу же предложить свой вариант вместо подчеркнутого слова.
Кто начнёт, кто возьмёт на себя смелость начать произносить в голос по-английски то, что написано, пока что читать?
Ну, пожалуйста, это не страшно.

One recent advance \textit{confirmed} that Alzheimer's disease is sometimes inherited.
So what is the variant that you would choose as a synonymous one?
For confirmed, it's made valid.
Made valid.
Absolutely correct.
The correct answer is A.

Video compression is sending not a complete colour portrait for each frame, but rather a shorthand version that describes the difference between the current frame and the \textit{previous} one.
В данном случае мы знаем, что current -- это современная, текущая ситуация и так далее.
А если мы поменяем букву e на a, какое слово получится в английском языке? 
Currant -- это смородина, на всякий случай, не опишитесь никогда.
На всякий случай.
Что это за ягода? Это чёрная смородина.
Почему она красная? Потому что она зелёная.
Продолжаем.
So what will be the variant?
Preceding.
Absolutely correct.
Здесь нам даже подсказывает, что?
Приставка pre, которая предшествующая, имеет значение нечто предшествующее.

The International Union of Biological Sciences met in Amsterdam to discuss how many species there are, and how many there will be if the environment \textit{is altered} in various ways by man. 
В английском языке мы помним правило, буква a перед l плюс любой другой согласной или же l читается как o.
So what will be the variant?
Is changed.
Absolutely correct.
And Species.
What do you know about this word?
What is interesting about this word from the point of your grammar?
Who knows?
Who remembers?
Who remembers this rule?
For several words.
This word has the same form in plural and in singular.
A species or a lot of species.
Вот это одно из тех небольших слов, как news, например, an interesting news, a lot of interesting news.
Когда слово вроде бы во множественом, trousers, что у нас там ещё, тоже надо внезапно вспоминать, но тем не менее, quite a number of words in English.

Almost half of U.S. newspaper editors say that dinosaurs and humans lived \textit{countemporaneously}.
Countemporaneously = at the same time.
Coun -- это с.
Tempo -- показатель времени (more).
А ly -- это что такое?
Это суффикс, который показывает нам, что в подавляющем большинстве случаев это наречие, за исключением некоторых слов, которые, например, кто-нибудь мне приведет слово, которое в английском языке будет иметь одинаковую форму и в наречии, и как прилагательное.
Friendly, например.
Friendly это и adjective, и adverb.
Это слово, которое исконно англосаксонское.
К нему просто нельзя второй раз lyly добавить.
Но сразу, таким образом, с помощью этого суффикса образовалось от friend, стало friendly, дружелюбный.
И дальше уже в зависимости от позиции в предложении, оно становится либо прилагательным, либо наречием.

Establishing an \textit{appropriate} correspondence between time and the path position parameter is an important condition in controlling the path of the robot arm.
Appropriate = proper.

Number six.
Здесь у нас слово ambiguity.
What is ambiguous? How would you translate it?
Двойственный, двусмысленный, ambiguity, двойственность в значении, да?
Ну и всё, собственно говоря.
Пожалуйста.
What will be the word? What will be the explanation? Пожалуйста.
B) having two or more possible meanings
Imbuguity -- двойственный, ну и дальше ищем two в ответах.
Мы еще с вами таким образом пытаемся вычленить какие-то подходы к выполнению тестов, которые тоже все-таки вас ждут.
Но, тем не менее, еще и что-то вспоминаем.



The negative charge of an electron slightly \textit{distorts} the lattice of the metal.
Distorts = changes the usual form.
Ну, и в отличие от остальных слов, которые вам знакомы, хотя я думаю, что to distort тоже вам знаком, but does it mean to distort, искажать, деформировать?
Абсолютно верно.
Ну ладно, дальше тогда пока не будем.
Я думаю о том, чтобы тоже использовать для иностранного языка как раз вот в случае, если вы что-то такое вам дать порешать тоже курс на портале аспирантуры
Я просто проверю, что все работает.
Там существует еще и самозапись.
Поэтому просто в следующую нашу встречу я уже к этому времени проверю, что все нормально.
И просто вам дам ссылку, вы туда сами запишитесь.
И я буду просто по возможности для тренировки, кому будет хотеться, это не будет в качестве обязаловки, пока.
Может быть во втором семестре я что-нибудь и придумаю.
Но, тем не менее, буду туда какое-то домашнее задание, какие-то упражнения, чтобы вы могли, когда вы скучно, сидя там на работе, туда зайти порешать.

Сегодня я планировала разобрать другую тему, но тем не менее порядок слов мы с вами обсудили, тогда я немножко поменяю свои планы на следующее занятие и ваше домашнее задание будет освежить, то есть ничего такого дополнительного, каких-то задумок на креатив не будет, будет просто задание вспомнить типы вопросов, которые существуют в английском языке.
Вспомнить, на что эти вопросы нацелены, когда они задаются.
Соответственно, поскольку вопросы это предложение вспомнить их структуру, но мы на нашей встрече постараемся с вами аналитизировать это.
Не следовать тому подходу, когда пословно в школе говорили вопросительное слово, вспомогательный глагол, подлежащее и так далее.
Нет, на самом деле опять же всё это сводится к более таким аналитическим формулам, нежели просто какое-то условное выстраивание.
Вот мне бы хотелось, чтобы вы тоже это осознавали, потому что это помогает в дальнейшем, когда ты понимаешь, как строится предложение, строить предложение в английском языке.
Поэтому вот просто, когда у вас будет время, посмотрите где-то там, будете ехать, в телефоне задайте какие типы вопросов в английском языке.
А мы с вами как раз вот эти вот упражнения.
Третье, я просто возьму его за основу, вы уже с ним знакомы, это будут предложения для вас не совсем незнакомые.
Мы к этим ответам потренируемся задавать вопросы, потому что вы помните, что в третьем задании вашего кандидатского экзамена вопросы присутствовать будут.
Для того, чтобы правильно отвечать на вопросы, нужно понимать.
Чтобы победить врага, нужно его понимать.
Поэтому мы будем заходить с этой стороны, с точки зрения обсуждения вопросов, ну естественно правильно на них ответы выстраивать.



\end{document}
